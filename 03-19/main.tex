
\documentclass[a4paper,11pt]{article}
\usepackage[a4paper, margin=8em]{geometry}

% usa i pacchetti per la scrittura in italiano
\usepackage[french,italian]{babel}
\usepackage[T1]{fontenc}
\usepackage[utf8]{inputenc}
\frenchspacing 

% usa i pacchetti per la formattazione matematica
\usepackage{amsmath, amssymb, amsthm, amsfonts}

% usa altri pacchetti
\usepackage{gensymb}
\usepackage{hyperref}
\usepackage{standalone}

% imposta il titolo
\title{Appunti Economia ed Organizzazione Aziendale}
\author{Luca Seggiani}
\date{2025}

% disegni
\usepackage{pgfplots}
\pgfplotsset{width=10cm,compat=1.9}

% imposta lo stile
% usa helvetica
\usepackage[scaled]{helvet}
% usa palatino
\usepackage{palatino}
% usa un font monospazio guardabile
\usepackage{lmodern}

\renewcommand{\rmdefault}{ppl}
\renewcommand{\sfdefault}{phv}
\renewcommand{\ttdefault}{lmtt}

% disponi il titolo
\makeatletter
\renewcommand{\maketitle} {
	\begin{center} 
		\begin{minipage}[t]{.8\textwidth}
			\textsf{\huge\bfseries \@title} 
		\end{minipage}%
		\begin{minipage}[t]{.2\textwidth}
			\raggedleft \vspace{-1.65em}
			\textsf{\small \@author} \vfill
			\textsf{\small \@date}
		\end{minipage}
		\par
	\end{center}

	\thispagestyle{empty}
	\pagestyle{fancy}
}
\makeatother

% disponi teoremi
\usepackage{tcolorbox}
\newtcolorbox[auto counter, number within=section]{theorem}[2][]{%
	colback=blue!10, 
	colframe=blue!40!black, 
	sharp corners=northwest,
	fonttitle=\sffamily\bfseries, 
	title=Teorema~\thetcbcounter: #2, 
	#1
}

% disponi definizioni
\newtcolorbox[auto counter, number within=section]{definition}[2][]{%
	colback=red!10,
	colframe=red!40!black,
	sharp corners=northwest,
	fonttitle=\sffamily\bfseries,
	title=Definizione~\thetcbcounter: #2,
	#1
}

% disponi problemi
\newtcolorbox[auto counter, number within=section]{problem}[2][]{%
	colback=green!10,
	colframe=green!40!black,
	sharp corners=northwest,
	fonttitle=\sffamily\bfseries,
	title=Problema~\thetcbcounter: #2,
	#1
}

% disponi codice
\usepackage{listings}
\usepackage[table]{xcolor}

\lstdefinestyle{codestyle}{
		backgroundcolor=\color{black!5}, 
		commentstyle=\color{codegreen},
		keywordstyle=\bfseries\color{magenta},
		numberstyle=\sffamily\tiny\color{black!60},
		stringstyle=\color{green!50!black},
		basicstyle=\ttfamily\footnotesize,
		breakatwhitespace=false,         
		breaklines=true,                 
		captionpos=b,                    
		keepspaces=true,                 
		numbers=left,                    
		numbersep=5pt,                  
		showspaces=false,                
		showstringspaces=false,
		showtabs=false,                  
		tabsize=2
}

\lstdefinestyle{shellstyle}{
		backgroundcolor=\color{black!5}, 
		basicstyle=\ttfamily\footnotesize\color{black}, 
		commentstyle=\color{black}, 
		keywordstyle=\color{black},
		numberstyle=\color{black!5},
		stringstyle=\color{black}, 
		showspaces=false,
		showstringspaces=false, 
		showtabs=false, 
		tabsize=2, 
		numbers=none, 
		breaklines=true
}

\lstdefinelanguage{javascript}{
	keywords={typeof, new, true, false, catch, function, return, null, catch, switch, var, if, in, while, do, else, case, break},
	keywordstyle=\color{blue}\bfseries,
	ndkeywords={class, export, boolean, throw, implements, import, this},
	ndkeywordstyle=\color{darkgray}\bfseries,
	identifierstyle=\color{black},
	sensitive=false,
	comment=[l]{//},
	morecomment=[s]{/*}{*/},
	commentstyle=\color{purple}\ttfamily,
	stringstyle=\color{red}\ttfamily,
	morestring=[b]',
	morestring=[b]"
}

% disponi sezioni
\usepackage{titlesec}

\titleformat{\section}
	{\sffamily\Large\bfseries} 
	{\thesection}{1em}{} 
\titleformat{\subsection}
	{\sffamily\large\bfseries}   
	{\thesubsection}{1em}{} 
\titleformat{\subsubsection}
	{\sffamily\normalsize\bfseries} 
	{\thesubsubsection}{1em}{}

% disponi alberi
\usepackage{forest}

\forestset{
	rectstyle/.style={
		for tree={rectangle,draw,font=\large\sffamily}
	},
	roundstyle/.style={
		for tree={circle,draw,font=\large}
	}
}

% disponi algoritmi
\usepackage{algorithm}
\usepackage{algorithmic}
\makeatletter
\renewcommand{\ALG@name}{Algoritmo}
\makeatother

% disponi numeri di pagina
\usepackage{fancyhdr}
\fancyhf{} 
\fancyfoot[L]{\sffamily{\thepage}}

\makeatletter
\fancyhead[L]{\raisebox{1ex}[0pt][0pt]{\sffamily{\@title \ \@date}}} 
\fancyhead[R]{\raisebox{1ex}[0pt][0pt]{\sffamily{\@author}}}
\makeatother

\begin{document}

% sezione (data)
\section{Lezione del 19-03-25}

% stili pagina
\thispagestyle{empty}
\pagestyle{fancy}

% testo
Riprendiamo la discussione delle società di capitali, concluse le S.p.a.:

\subsection{Società a responsabilità limitata}
Abbiamo visto come nelle S.r.l. non parliamo di azioni, ma di \textbf{quote di partecipazione} (non sono previste azioni).

Il capitale sociale minimo è di 10.000 €, e non è permessa l'emissione di obbligazioni.

Non è obbligatoria la presenza di un collegio sindacale, e le regole di funzionamento dell'assemblea sono semplificate.
Gli amministratori, inoltre, devono essere obbligatoriamente soci.

Esistono una variante dell'S.r.l., l \textit{S.r.l. a un euro}, disciplina di riferimento \textbf{S.r.l. semplificata} (S.r.l.s.).
Questa richiede minori costi di costituzione ed è stata pensata per favorire l'imprenditorialità giovanile.

\subsection{Società in accomandita}
Le società in accomandita rappresentano la via di mezzo fra società di persone e società di capitali.
Sono rappresentate dalle S.a.s. e dalle S.a.p.a.

Abbiamo visto come si distinguono per la differenza fra soci \textit{accomandatari} e soci \textit{accomandanti}, che dispongono rispettivamente di responsabilità illimitata e limitata.

Una problematica delle società in accomandita può essere quella della presenza di accomandatari di "paglia", cioè soci accomandatari nullatenenti messi nella loro posizione per favorire in qualche modo gli accomandanti. 
Per mitigare tale problematica si richiede infatti che il nome di almeno uno dei soci accomandatari figuri nella ragione sociale, così che se ne possa verificare la consistenza del patrimonio.

\subsection{Titoli di credito}
Iniziamo a parlare dei \textbf{titoli di credito} partendo dalle azioni.

\subsubsection{Azioni}
I \textit{titoli azionari}, comunemente \textbf{azioni}, sono documenti che rappresentano le quote di partecipazione nelle S.p.a. (S.a.s. e S.r.l. non prevedono azioni).

Le azioni possono essere di 2 tipi rispetto alla proprietà:
\begin{itemize}
	\item \textbf{Azioni nominative:} intestate a nome di una persona fisica o giuridica, il cui nome è riportato sull'azione, e su un registro tenuto dalla società emittente (il cosiddetto \textbf{libro soci});
	\item \textbf{Azioni al portatore:} il trasferimento di questo tipo di azione avviene mediante la semplice consegna del titolo all'acquirente. In questo caso il possessore del titolo è legittimato all'esercizio dei suoi diritti di socio previa la sola presentazione del titolo della società. 
\end{itemize}

Si può poi distinguere fra diverse categorie di azione (quasi tutte nominative tranne le \textit{azioni di risparmio}), divise in 2 macrocategorie:
\begin{itemize}
	\item \textbf{Azioni ordinarie}, caratterizzate dai:
	\begin{itemize}
		\item \textit{Diritti di partecipazione alla vita della società}: di partecipazione alle assemblee (ordinarie e straordinarie) e contestualmente di voto nelle assemblee;
		\item \textit{Diritti patrimoniali}: diritto al \textbf{dividendo}, cioè alla ricezione di una parte di utile della società (\textit{se} l'assemblea che approva il bilancio, cioè l'ordinaria, approva la distribuzione utili), e alla restituzione del capitale in caso di scioglimento della società o di riduzione del capitale sociale.
			Sempre dal punto di vista patrimoniale, chi detiene azioni ordinarie ha l'obbligo di concorrere alle perdite della società. 
	\end{itemize}
	\item \textbf{Azioni speciali}, che si dividono in diverse categorie:
		\begin{itemize}
			\item \textbf{Azioni privilegiate:} hanno il diritto di preferenza nella distribuzione e nelr imborso, ma sono disposte in questo al di sotto delle azioni di risparmio (le ordinarie stanno sotto, e tutte le altre categorie di azione stanno ancora sotto); 
			\item \textbf{Azioni di godimento:} vengono attribuite come forma di rimborso agli azionisti ordinari che hanno visto ridotto il loro capitale sociale in seguito a perdite. Patrimonialmente sono come le azioni di risparmio, ma senza i privilegi in termini di distribuzione dell'utile. In termini di diritti, sono uguali alle azioni di risparmio (non c'è partecipazione alla vita della società). Vengono per ultime in termini di distribuzione utili e ripartizione del capitale sociale;
			\item \textbf{Azioni assegnate ai prestatori di lavoro};
			\item \textbf{Azioni con prestazioni accessorie} Sono nominative e impongono al socio, oltre all'obbligo del conferimento, prestazioni non consitenti in denari;
			\item \textbf{Azioni a voto limitato} non hanno diritto di voto, o lo limitano (o condizionano) significativamente (solitamente permettono il voto solo nell'assemblea straordinaria). Non possono superare (insieme alle azioni di risparmio) la metà del capitale sociale;
			\item \textbf{Azioni di risparmio:} sono le sole che possono essere intestate \textit{al portatore}, e quindi che non sono nominative.
				Hanno il diritto di partecipazione e intervento in assemblea, ma non il diritto di voto.
				Il diritto di partecipazione alla vita della società è quindi limitato, mentre è rafforzato quello patrimoniale: infatti assicurano un dividendo annuo minimo pari al 5\% del valore nominale dell'azione, e l'eventuale distribuzione degli utili residui deve essere effettuata in modo che allezioni di risparmio corrisponda il 2\% in più rispetto alle azioni ordinarie.

				Riguardo al \textbf{valore} dell'azione, possiamo fare una parentesi.
				\begin{enumerate}
					\item Un azione ha un valore \textbf{nominale} che rappresenta la frazione di capitale sociale che questa rappresenta (cioè il numero di \textit{azioni} in senso stretto che sono erogate in un azione-documento);
					\item C'è poi il valore di \textbf{emissione}, cioè il prezzo di emisione dell'azione stessa dalla società al momento dell'emissione nel mercato mobiliare primario;
					\item Infine, c'è il valore di \textbf{mercato}, cioè il prezzo che l'azione ha sul mercato mobiliare secondario, cioè la quotazione del giorno in borsa, determinata dai meccanismi della domanda/offerta.
				\end{enumerate}

				Tornando alle azioni di risparmio, queste hanno altri due privilegi: sono privilegiate (sopra alle privilegiate stesse) nella distribuzione del'utile e nella restituzione del capitale, e vengono dopo alle azioni ordinarie per la concorrenza alle perdite.

				Le azioni di risparmio sono quelle che si comprano e vendono più spesso nel mercato della borsa, in quanto l'interesse dei possessori è principalmente quello patrimoniale (per cui potrebbero decidere di lucrare sulla differenza in caso di crescita del valore di mercato).
				Di contro, chi acquista azioni ordinarie cerca solitamente una qualche partecipazione, e quindi \textit{controllo}, sulla società. 
		\end{itemize}
\end{itemize}

La distribuzione degli utili, effettuata dall'assemblea ordinaria dei soci, ha quindi la seguente scaletta di priorità:
\begin{enumerate}
	\item Azioni di risparmio;
	\item Azioni privilegiate;
	\item Azioni ordinarie;
	\item Tutte le altre, con priorità sulle azioni di godimento. 
\end{enumerate}

\subsection{Obbligazioni}
Le obbligazioni sono titoli di credito nominativi o al portatore che rappresentano frazioni di uguale valore nominale e con uguali diritti di un operazione di finanziamento unitaria a titolo di mutuo.
In questo, chi detiene le obbligazioni è per le società un creditore, e non un socio.

\par\medskip

La differenza fra azioni e obbigazioni si può schematizzare nella seguente tabella:

\begin{table}[H]
	\center \rowcolors{2}{white}{black!10}
	\begin{tabular} {p{7cm} | p{7cm}}
		\bfseries Azione & \bfseries Obbligazione \\
		\hline 
		Qualità di socio & Qualità di creditore della società \\
		Diritto di compartecipare ai risultati & Remunerazione periodica fissa (\textit{interesse}) svincolata dai risultati \\
		Rimborso del capitale conferito solo in sede di liquidazione, e residuale (cioè se ne rimane un attivo netto). Inoltre la quota di liquidazione è svincolata al valore nominale di conferimento. & Diritto al rimborso del valore nominale del capitale prestato alla scadenza, nella sua totalità. \\
	\end{tabular}
\end{table}

Facciamo una nota sulle \textbf{obbligazione convertibili}, cioè obbligazioni che possono essere convertite in azione.

\end{document}
