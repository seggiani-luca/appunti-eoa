
\documentclass[a4paper,11pt]{article}
\usepackage[a4paper, margin=8em]{geometry}

% usa i pacchetti per la scrittura in italiano
\usepackage[french,italian]{babel}
\usepackage[T1]{fontenc}
\usepackage[utf8]{inputenc}
\frenchspacing 

% usa i pacchetti per la formattazione matematica
\usepackage{amsmath, amssymb, amsthm, amsfonts}

% usa altri pacchetti
\usepackage{gensymb}
\usepackage{hyperref}
\usepackage{standalone}

% imposta il titolo
\title{Appunti Economia ed Organizzazione Aziendale}
\author{Luca Seggiani}
\date{2025}

% disegni
\usepackage{pgfplots}
\pgfplotsset{width=10cm,compat=1.9}

% imposta lo stile
% usa helvetica
\usepackage[scaled]{helvet}
% usa palatino
\usepackage{palatino}
% usa un font monospazio guardabile
\usepackage{lmodern}

\renewcommand{\rmdefault}{ppl}
\renewcommand{\sfdefault}{phv}
\renewcommand{\ttdefault}{lmtt}

% disponi il titolo
\makeatletter
\renewcommand{\maketitle} {
	\begin{center} 
		\begin{minipage}[t]{.8\textwidth}
			\textsf{\huge\bfseries \@title} 
		\end{minipage}%
		\begin{minipage}[t]{.2\textwidth}
			\raggedleft \vspace{-1.65em}
			\textsf{\small \@author} \vfill
			\textsf{\small \@date}
		\end{minipage}
		\par
	\end{center}

	\thispagestyle{empty}
	\pagestyle{fancy}
}
\makeatother

% disponi teoremi
\usepackage{tcolorbox}
\newtcolorbox[auto counter, number within=section]{theorem}[2][]{%
	colback=blue!10, 
	colframe=blue!40!black, 
	sharp corners=northwest,
	fonttitle=\sffamily\bfseries, 
	title=Teorema~\thetcbcounter: #2, 
	#1
}

% disponi definizioni
\newtcolorbox[auto counter, number within=section]{definition}[2][]{%
	colback=red!10,
	colframe=red!40!black,
	sharp corners=northwest,
	fonttitle=\sffamily\bfseries,
	title=Definizione~\thetcbcounter: #2,
	#1
}

% disponi problemi
\newtcolorbox[auto counter, number within=section]{problem}[2][]{%
	colback=green!10,
	colframe=green!40!black,
	sharp corners=northwest,
	fonttitle=\sffamily\bfseries,
	title=Problema~\thetcbcounter: #2,
	#1
}

% disponi codice
\usepackage{listings}
\usepackage[table]{xcolor}

\lstdefinestyle{codestyle}{
		backgroundcolor=\color{black!5}, 
		commentstyle=\color{codegreen},
		keywordstyle=\bfseries\color{magenta},
		numberstyle=\sffamily\tiny\color{black!60},
		stringstyle=\color{green!50!black},
		basicstyle=\ttfamily\footnotesize,
		breakatwhitespace=false,         
		breaklines=true,                 
		captionpos=b,                    
		keepspaces=true,                 
		numbers=left,                    
		numbersep=5pt,                  
		showspaces=false,                
		showstringspaces=false,
		showtabs=false,                  
		tabsize=2
}

\lstdefinestyle{shellstyle}{
		backgroundcolor=\color{black!5}, 
		basicstyle=\ttfamily\footnotesize\color{black}, 
		commentstyle=\color{black}, 
		keywordstyle=\color{black},
		numberstyle=\color{black!5},
		stringstyle=\color{black}, 
		showspaces=false,
		showstringspaces=false, 
		showtabs=false, 
		tabsize=2, 
		numbers=none, 
		breaklines=true
}

\lstdefinelanguage{javascript}{
	keywords={typeof, new, true, false, catch, function, return, null, catch, switch, var, if, in, while, do, else, case, break},
	keywordstyle=\color{blue}\bfseries,
	ndkeywords={class, export, boolean, throw, implements, import, this},
	ndkeywordstyle=\color{darkgray}\bfseries,
	identifierstyle=\color{black},
	sensitive=false,
	comment=[l]{//},
	morecomment=[s]{/*}{*/},
	commentstyle=\color{purple}\ttfamily,
	stringstyle=\color{red}\ttfamily,
	morestring=[b]',
	morestring=[b]"
}

% disponi sezioni
\usepackage{titlesec}

\titleformat{\section}
	{\sffamily\Large\bfseries} 
	{\thesection}{1em}{} 
\titleformat{\subsection}
	{\sffamily\large\bfseries}   
	{\thesubsection}{1em}{} 
\titleformat{\subsubsection}
	{\sffamily\normalsize\bfseries} 
	{\thesubsubsection}{1em}{}

% disponi alberi
\usepackage{forest}

\forestset{
	rectstyle/.style={
		for tree={rectangle,draw,font=\large\sffamily}
	},
	roundstyle/.style={
		for tree={circle,draw,font=\large}
	}
}

% disponi algoritmi
\usepackage{algorithm}
\usepackage{algorithmic}
\makeatletter
\renewcommand{\ALG@name}{Algoritmo}
\makeatother

% disponi numeri di pagina
\usepackage{fancyhdr}
\fancyhf{} 
\fancyfoot[L]{\sffamily{\thepage}}

\makeatletter
\fancyhead[L]{\raisebox{1ex}[0pt][0pt]{\sffamily{\@title \ \@date}}} 
\fancyhead[R]{\raisebox{1ex}[0pt][0pt]{\sffamily{\@author}}}
\makeatother

\begin{document}

% sezione (data)
\section{Lezione del 27-03-25}

% stili pagina
\thispagestyle{empty}
\pagestyle{fancy}

% testo
Riprendiamo la discussione delle strutture organizzative.

\subsubsection{Diversifizazione e differenziazione}
I termini \textbf{diversificazione} e \textbf{differenziazione} fanno riferimento alla \textbf{gamma} di prodotti di un'impresa.
La gamma di prodotti di un'impresa è definita dalle \textbf{linee} di prodotto.
Ad esempio, nel caso di un impresa che si occupa di abbigliamento, si potrebbe avere una linea di jeans, una linea di camicie, una linea di intimo, ecc... fra di loro distinte.
Questo tipo di sviluppo in \textbf{verticale} (aumento del numero di linee) rappresenta una \textit{diversificazione}, quindi più linee di prodotto sono disponibili, più l'impresa diversifica.

Chiaramente, aumentando le linee di prodotto le imprese entrano in nuove \textit{linee di business}.
Ad esempio, l'impresa di abbigliamento di prima potrebbe diversificare entrando nell'arredo casa, ecc...

Guardando alle singole linee di prodotto, invece, ci rendiamo conto che questa può essere più o meno \textbf{profonda}: per \textit{profondità} ci riferiamo al numero di \textbf{varianti} di prodotto all'interno di quella linea.
Quindi nel caso dei jeans si potrebbe dividere in più tipologie (regular, skinny, ecc...).
A loro volta, queste tipologie si distinguono in taglie, colore, ecc...
Si ha quindi che più una linea si divide in \textbf{varianti} di prodotto, più questa è \textit{differenziata}.

\par\smallskip

Tornando al discorso delle \textit{strutture organizzative}, direzioni tipiche di sviluppo delle imprese, man di mano che crescono in dimensioni e complessità, sono quelle di diversificazione e differenziazione. 

\subsubsection{Progettazione della microstruttura}
Avevamo parlato di \textbf{macro} e \textbf{micro}, e riguardo alla \textit{micro}-struttura, che si occupa di stabilire i meccanismi di coordinazione fra individui e gruppi e di definirne la \textit{specializzazione}, avevamo parlato di specializzazione \textbf{orizzontale} e \textbf{verticale}.

Iniziamo quindi a vedere più nel dettaglio i concetti fondamentali di \textbf{compito}, \textbf{mansione} e \textbf{ruolo}.
\begin{itemize}
	\item \textbf{Compito:} l'elemento più piccolo di lavoro che andiamo a considerare, inteso come un insieme di attività o operazioni necessariamente collegate in funzione di proprietà/capacità del lavoro umano;
	\item \textbf{Mansione:} l'insieme di compiti assegnabili ad una certa \textbf{posizione};
	\item \textbf{Ruolo:} termine che deriva dal mondo del teatro, che identifica l'insieme delle aspettative sul \textit{comportamento} di una persona in riferimento agli obiettivi dell'intera organizzazione.

		Si parla quindi non solo della \textit{mansione} che un individuo deve svolgere, ma anche sul tipo di \textit{comportamento} che questo deve mantenere nello svolgimento di tale mansione.
		In questo potremmo dire che la mansione è fondamentalmente \textit{asettica}, mentre il ruolo copre anche le aspettative di comportamento che si hanno sull'individuo che la svolge.
\end{itemize}

La diversificazione dei compiti assegnabili ad una posizione (la mansione) rappresentava quella che avevamo definito \textbf{specializzazione orizzontale}.
Veniamo quindi alla \textbf{specializzazione verticale}.
Questa riguarda la quantità di autonomia che un individuo ha nello svolgimento della sua mansione, e quindi alla quantità di \textit{controllo} che ha sul \textit{come} viene svolta la mansione.

Possiamo individuare la seguente tabella che distingue fra specializzazioni orizzontali e verticali, alte e basse:
\begin{table}[h!]
	\center 
	\begin{tabular} { c | c c }
		& \bfseries Orizzontale alta & \bfseries Orizzontale bassa \\
		\hline
		\bfseries Verticale alta & Mansioni non qualificate & Manager di basso livello \\
		\bfseries Verticale bassa & Mansioni professionali & Manager \\
	\end{tabular}
\end{table}

Abbiamo quindi che i \textbf{manager} hanno bassa specializzazione sia verticale che orizzontale: si occupano quindi di una vasta gamma di compiti con una grande autonomia, con il caso specifico dei manager \textit{di basso livello}, che hanno minore controllo (specializzazione verticale).

Le \textbf{mansioni professionali} (cioè che riguardano i \textit{professionisti}), invece, hanno alta specializzazione orizzontale e bassa specializzazione verticale, cioè si occupano di ambiti specifici ma con un grande livello di autonomia.
Le mansioni \textit{non qualificate} invece, hanno alta specializzazione sia orizzontale che verticale, cioè compiti specifici con bassa autonomia (in questo non richiedono \textit{qualifiche}).

\subsubsection{Riprogettazione delle mansioni}
Abbiamo diverse modalità di riprogettazione delle mansioni durante lo sviluppo dell'impresa:
\begin{itemize}
	\item \textbf{Job enlargement:} l'ampliamento dei compiti elementari assegnati a ciascuna posizione;
	\item \textbf{Job enrichement:} aumento della discrezionalità riaccorpando compiti di programmazione e controllo dei risultati, cioè aumento del grado di autonomia;
	\item \textbf{Job rotation:} rotazione dei compiti ftra gli individui per ottenere minore ripetitività e rigidità.
\end{itemize}

Abbiamo quindi che se partiamo da mansioni \textit{tayloristiche} (ad alta sepcializzazione orizzontale e verticale), il job enlargement e la job rotation abbassano la specializzazione orizzontale (\textit{mansioni polivalenti}); il job enlargement e il job enrichment abbassano la specializzazione orizzontale e verticale, e infine il job enrichement di per sé abbassa la specializzazione verticale (maggiore autonomia).

\subsubsection{Formalizzazione della mansione}
Il \textbf{mansionario} è la descrizione verbale dei compiti assegnati ad una \textbf{posizione} od \textbf{unità organizzativo}.
IN questo il mansionario garantisce il prestatore d'opera ed esplicita nero su bianco le dipendenze e i compiti assegnate alle diverse mansioni.
Di contro, può rappresentare uno strumento molto rigido, con basso livello di libertà nella diversifiazione delle mansioni. 

\subsubsection{Matrice di responsabilità}
Un'altro strumento per la formalizzazione delle mansioni è la \textbf{matrice delle responsabilità}, che rappresenta le responsabilità (più propriamente i tipi di coinvolgimento) dei diversi attori riguardo ai diversi compiti.
La struttura di una matrice di responsabilità è la seguente:
\begin{table}[h!]
	\center \rowcolors{2}{white}{black!10}
	\begin{tabular} {  c | c c }
		\bfseries Attori / Compiti & A & B \\
		\hline 
		1 & Modalità A-1 & Modalità B-1 \\ 
		2 & Modalità A-2 & Modalià B-2
	\end{tabular}
\end{table}

\subsection{Progettazione della macrostruttura}
La \textbf{macrostruttura} tiene conto delle \textbf{unità organizzative}, cioè l'integrazione delle posizioni e delle unità stesse.
Quindi, in breve, un unità organizzativa si può intendere come un insieme di posizioni, o di altre unità.

Possiamo definire riguardo alle unità tre parametri:
\begin{itemize}
	\item I \textbf{criteri di raggruppamento};
	\item I \textbf{meccanismi} e i \textbf{ruoli di collegamento};
	\item La \textbf{formalizzazione} della struttura (\textbf{organigramma}).
\end{itemize}

\subsubsection{Criteri di raggruppamento}
I criteri possono essere \textbf{funzionali} o \textbf{divisionali}.
\begin{itemize}
	\item \textbf{Criteri funzionali:} in questi la base di raggruppamento per la formazione dell'unita è orientata agli \textit{input}: quindi fa riferimento  alle competenze necessarie, o al tipo di processo di cui si occupa l'unità;
	\item \textbf{Criteri divisionali:} in questi la base di raggruppamento è invece l'\textit{output}: si guarda al prodotto, al cliente e all'area geografica.
\end{itemize}
Notiamo che il problema della scelta dei criteri di raggruppamento si ripete a ciascun livello della struttura.



\end{document}
