
\documentclass[a4paper,11pt]{article}
\usepackage[a4paper, margin=8em]{geometry}

% usa i pacchetti per la scrittura in italiano
\usepackage[french,italian]{babel}
\usepackage[T1]{fontenc}
\usepackage[utf8]{inputenc}
\frenchspacing 

% usa i pacchetti per la formattazione matematica
\usepackage{amsmath, amssymb, amsthm, amsfonts}

% usa altri pacchetti
\usepackage{gensymb}
\usepackage{hyperref}
\usepackage{standalone}

% imposta il titolo
\title{Appunti Economia ed Organizzazione Aziendale}
\author{Luca Seggiani}
\date{2025}

% disegni
\usepackage{pgfplots}
\pgfplotsset{width=10cm,compat=1.9}

% imposta lo stile
% usa helvetica
\usepackage[scaled]{helvet}
% usa palatino
\usepackage{palatino}
% usa un font monospazio guardabile
\usepackage{lmodern}

\renewcommand{\rmdefault}{ppl}
\renewcommand{\sfdefault}{phv}
\renewcommand{\ttdefault}{lmtt}

% disponi il titolo
\makeatletter
\renewcommand{\maketitle} {
	\begin{center} 
		\begin{minipage}[t]{.8\textwidth}
			\textsf{\huge\bfseries \@title} 
		\end{minipage}%
		\begin{minipage}[t]{.2\textwidth}
			\raggedleft \vspace{-1.65em}
			\textsf{\small \@author} \vfill
			\textsf{\small \@date}
		\end{minipage}
		\par
	\end{center}

	\thispagestyle{empty}
	\pagestyle{fancy}
}
\makeatother

% disponi teoremi
\usepackage{tcolorbox}
\newtcolorbox[auto counter, number within=section]{theorem}[2][]{%
	colback=blue!10, 
	colframe=blue!40!black, 
	sharp corners=northwest,
	fonttitle=\sffamily\bfseries, 
	title=Teorema~\thetcbcounter: #2, 
	#1
}

% disponi definizioni
\newtcolorbox[auto counter, number within=section]{definition}[2][]{%
	colback=red!10,
	colframe=red!40!black,
	sharp corners=northwest,
	fonttitle=\sffamily\bfseries,
	title=Definizione~\thetcbcounter: #2,
	#1
}

% disponi problemi
\newtcolorbox[auto counter, number within=section]{problem}[2][]{%
	colback=green!10,
	colframe=green!40!black,
	sharp corners=northwest,
	fonttitle=\sffamily\bfseries,
	title=Problema~\thetcbcounter: #2,
	#1
}

% disponi codice
\usepackage{listings}
\usepackage[table]{xcolor}

\lstdefinestyle{codestyle}{
		backgroundcolor=\color{black!5}, 
		commentstyle=\color{codegreen},
		keywordstyle=\bfseries\color{magenta},
		numberstyle=\sffamily\tiny\color{black!60},
		stringstyle=\color{green!50!black},
		basicstyle=\ttfamily\footnotesize,
		breakatwhitespace=false,         
		breaklines=true,                 
		captionpos=b,                    
		keepspaces=true,                 
		numbers=left,                    
		numbersep=5pt,                  
		showspaces=false,                
		showstringspaces=false,
		showtabs=false,                  
		tabsize=2
}

\lstdefinestyle{shellstyle}{
		backgroundcolor=\color{black!5}, 
		basicstyle=\ttfamily\footnotesize\color{black}, 
		commentstyle=\color{black}, 
		keywordstyle=\color{black},
		numberstyle=\color{black!5},
		stringstyle=\color{black}, 
		showspaces=false,
		showstringspaces=false, 
		showtabs=false, 
		tabsize=2, 
		numbers=none, 
		breaklines=true
}

\lstdefinelanguage{javascript}{
	keywords={typeof, new, true, false, catch, function, return, null, catch, switch, var, if, in, while, do, else, case, break},
	keywordstyle=\color{blue}\bfseries,
	ndkeywords={class, export, boolean, throw, implements, import, this},
	ndkeywordstyle=\color{darkgray}\bfseries,
	identifierstyle=\color{black},
	sensitive=false,
	comment=[l]{//},
	morecomment=[s]{/*}{*/},
	commentstyle=\color{purple}\ttfamily,
	stringstyle=\color{red}\ttfamily,
	morestring=[b]',
	morestring=[b]"
}

% disponi sezioni
\usepackage{titlesec}

\titleformat{\section}
	{\sffamily\Large\bfseries} 
	{\thesection}{1em}{} 
\titleformat{\subsection}
	{\sffamily\large\bfseries}   
	{\thesubsection}{1em}{} 
\titleformat{\subsubsection}
	{\sffamily\normalsize\bfseries} 
	{\thesubsubsection}{1em}{}

% disponi alberi
\usepackage{forest}

\forestset{
	rectstyle/.style={
		for tree={rectangle,draw,font=\large\sffamily}
	},
	roundstyle/.style={
		for tree={circle,draw,font=\large}
	}
}

% disponi algoritmi
\usepackage{algorithm}
\usepackage{algorithmic}
\makeatletter
\renewcommand{\ALG@name}{Algoritmo}
\makeatother

% disponi numeri di pagina
\usepackage{fancyhdr}
\fancyhf{} 
\fancyfoot[L]{\sffamily{\thepage}}

\makeatletter
\fancyhead[L]{\raisebox{1ex}[0pt][0pt]{\sffamily{\@title \ \@date}}} 
\fancyhead[R]{\raisebox{1ex}[0pt][0pt]{\sffamily{\@author}}}
\makeatother

\begin{document}

% sezione (data)
\section{Lezione del 14-05-25}

% stili pagina
\thispagestyle{empty}
\pagestyle{fancy}

% testo
Riprendiamo la trattazione dei costi per destinazione, introducendo il concetto di \textbf{oggetto di costo}.

\subsubsection{Oggetto di costo}
L'oggetto di costo è ciò in relazione a cui valutiamo i costi, cioè lo scopo per il quale i costi sono misurati.
Nel caso più semplice, questo è un \textbf{prodotto}.
Il \textbf{costo pieno} (\textit{full cost}) di un oggetto di costo dipende da tutte le risorse utilizzate per tale oggetto di costo.

Il costo può essere:
\begin{itemize}
	\item \textbf{Diretto:} riconducibile a un singolo oggetto di costo;
	\item \textbf{Indiretto:} sempre legato alla produzione, ma non necessariamente riconducibile a un singolo oggetto di costo.
\end{itemize}

Abbiamo quindi che la produzione genera chiaramente dei \textbf{costi di produzione}, che possono essere diretti o indiretti, fra sostanzialmente 3 componenti:
\begin{itemize}
	\item Materiali diretti;
	\item Manodopera diretta;
	\item Costi generali di produzione: questi sono indiretti, in quanto non riguardano direttamente il singolo prodotto (ma sono connessi al funzionamento della fabbrica, cioè ad eesmpio riscaldamento, ammortamento strutture produttive, manutenzione macchinari, ecc...).
		Possono classificare questi in:
		\begin{itemize}
			\item Materiali indiretti;
			\item Manodopera indiretta;
			\item Altre risorse consumate in produzione non legate al prodotto.
		\end{itemize}
\end{itemize}

Restano quindi fuori i \textbf{costi non di produzione}, che classifichiamo in:
\begin{itemize}
	\item Costi di marketing/vendita: costi necessari per ottenere l'ordine e consegnare il prodotto:
	\item Costi amministrativi: costi generali per mantenere gli uffici amministrativi;
	\item Costi generali, ad esempio interessi, ricerca e sviluppo, eccetera.
\end{itemize}

\par\smallskip

Vediamo che l'idea di oggetto di costo si può estendere oltre i singoli prodotti: ad esempio possiamo prendere come oggetti di costo:
\begin{itemize}
	\item Prodotti (il caso visto adesso);
	\item Linee di prodotti;
	\item Marchi;
	\item Agenti (venditori);
	\item Canali di distribuzione;
	\item Servizi;
	\item Progetti;
	\item Attività;
	\item Unità organizzative.
\end{itemize}

\subsubsection{Classificazione dei costi}

Abbiamo quindi che il costo si classifica come:
\begin{itemize}
	\begin{itemize}
		\item[] Materie prime;
		\item[$+$] Costo del lavoro diretto;
	\end{itemize}
	$=$ \textbf{Costo primo} (o diretto) di prodotto;
	\begin{itemize}
		\item[$+$] Costi di produzione indiretti;
	\end{itemize}
	$=$ Costo pieno industriale;
	\begin{itemize}
		\item[$+$] costi non di produzione;
	\end{itemize}
	$=$ Costo pieno aziendale.
\end{itemize}

Dove notiamo che:
\begin{itemize}
	\item Il \textbf{costo pieno industriale}, o \textit{costo inventariabile} o \textit{costo di prodotto} è il valore delle risorse associabili, in modo diretto o indiretto, a un prodotto, perciò valorizza le rimanenze;
	\item Il \textbf{costo di periodo} (costi non di produzione) comprende attività non sostenute allo scopo diretto di produzione, cioè associabili alla realizzazione di un prodotto (sono fra queste amministrazione, ricerca e sviluppo, ecc...).
\end{itemize}

\subsubsection{Classificazione quantitativa dei costi}

Abbiamo quindi che vale, in linea generale, la seguente classificazione dal punto di vista quantitativo:
\begin{itemize}
	\item \textbf{Costi diretti:} sono ricondotti specificamente all'oggetto di costo in quanto sono da questo causati:
		\begin{itemize}
			\item Quantità del fattore effettivamente impiegata dall'oggetto $\times$ il suo prezzo;
			\item Valore di fattori produttivi i cui servizi sono impiegati in modo esclusivo dall'oggetto di costo.
		\end{itemize}
	\item \textbf{Costi indiretti:} sono causati da 2 o più oggetti di costo, e quindi non sono direttamente riconducibili a nessun oggetto di costo singolo.
		In questo caso la quantificazione rispetto ai singoli è impossibile, o economicamente non conveniente.
\end{itemize}

\subsubsection{Cost driver}
Riguardo alle variazioni di livello di attività, i costi possono essere:
\begin{itemize}
	\item \textbf{Fissi}, cioè che rimangono inalterati in un intervallo significativo di variazione del livello di attività.
		Questi possono essere:
		\begin{itemize}
			\item Costi fissi \textbf{impegnati}: importanti sul lungo termine, non si possono tagliare sul breve termine senza danneggiare gravemente la redditività;
			\item Costi fissi \textbf{discrezionali}: importanti sul breve termine, possono essere tagliati per brevi periodi con danni minimi alla redditività.
		\end{itemize}
	\item \textbf{Variabili}, cioè che si modificano assieme al livello di attività.
		Fra questi possiamo distinguere:
		\begin{itemize}
			\item Costi variabili \textbf{lineari}: che scalano come il cost driver;
			\item Costi variabili \textbf{progressivi}: che scalano sempre più al crescere del cost driver, ad esempio manodopera (grazie a straordinari, ecc...);
			\item Costi variabili \textbf{degressivi}: che scalano sempre meno al crescere del cost driver, ad esempio materia prima (grazie a sconti di quantità, acquisto all'ingrosso, ecc...).
		\end{itemize}
	\item \textbf{Misti}, cioè semivariabili o a scalini (semifissi o variabili al gradino, insomma dati da combinazioni di costi variabili e costi fissi);
\end{itemize}

A spiegare tale variazioni di livello sono i cosiddetti \textbf{cost driver}, presi all'interno di un intervallo di variazione (\textbf{area di rilevanza}), su un certo \textbf{periodo} di tempo.

\end{document}
