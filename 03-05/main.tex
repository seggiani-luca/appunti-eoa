
\documentclass[a4paper,11pt]{article}
\usepackage[a4paper, margin=8em]{geometry}

% usa i pacchetti per la scrittura in italiano
\usepackage[french,italian]{babel}
\usepackage[T1]{fontenc}
\usepackage[utf8]{inputenc}
\frenchspacing 

% usa i pacchetti per la formattazione matematica
\usepackage{amsmath, amssymb, amsthm, amsfonts}

% usa altri pacchetti
\usepackage{gensymb}
\usepackage{hyperref}
\usepackage{standalone}

% imposta il titolo
\title{Appunti Economia ed Organizzazione Aziendale}
\author{Luca Seggiani}
\date{2025}

% disegni
\usepackage{pgfplots}
\pgfplotsset{width=10cm,compat=1.9}

% imposta lo stile
% usa helvetica
\usepackage[scaled]{helvet}
% usa palatino
\usepackage{palatino}
% usa un font monospazio guardabile
\usepackage{lmodern}

\renewcommand{\rmdefault}{ppl}
\renewcommand{\sfdefault}{phv}
\renewcommand{\ttdefault}{lmtt}

% disponi il titolo
\makeatletter
\renewcommand{\maketitle} {
	\begin{center} 
		\begin{minipage}[t]{.8\textwidth}
			\textsf{\huge\bfseries \@title} 
		\end{minipage}%
		\begin{minipage}[t]{.2\textwidth}
			\raggedleft \vspace{-1.65em}
			\textsf{\small \@author} \vfill
			\textsf{\small \@date}
		\end{minipage}
		\par
	\end{center}

	\thispagestyle{empty}
	\pagestyle{fancy}
}
\makeatother

% disponi teoremi
\usepackage{tcolorbox}
\newtcolorbox[auto counter, number within=section]{theorem}[2][]{%
	colback=blue!10, 
	colframe=blue!40!black, 
	sharp corners=northwest,
	fonttitle=\sffamily\bfseries, 
	title=Teorema~\thetcbcounter: #2, 
	#1
}

% disponi definizioni
\newtcolorbox[auto counter, number within=section]{definition}[2][]{%
	colback=red!10,
	colframe=red!40!black,
	sharp corners=northwest,
	fonttitle=\sffamily\bfseries,
	title=Definizione~\thetcbcounter: #2,
	#1
}

% disponi problemi
\newtcolorbox[auto counter, number within=section]{problem}[2][]{%
	colback=green!10,
	colframe=green!40!black,
	sharp corners=northwest,
	fonttitle=\sffamily\bfseries,
	title=Problema~\thetcbcounter: #2,
	#1
}

% disponi codice
\usepackage{listings}
\usepackage[table]{xcolor}

\lstdefinestyle{codestyle}{
		backgroundcolor=\color{black!5}, 
		commentstyle=\color{codegreen},
		keywordstyle=\bfseries\color{magenta},
		numberstyle=\sffamily\tiny\color{black!60},
		stringstyle=\color{green!50!black},
		basicstyle=\ttfamily\footnotesize,
		breakatwhitespace=false,         
		breaklines=true,                 
		captionpos=b,                    
		keepspaces=true,                 
		numbers=left,                    
		numbersep=5pt,                  
		showspaces=false,                
		showstringspaces=false,
		showtabs=false,                  
		tabsize=2
}

\lstdefinestyle{shellstyle}{
		backgroundcolor=\color{black!5}, 
		basicstyle=\ttfamily\footnotesize\color{black}, 
		commentstyle=\color{black}, 
		keywordstyle=\color{black},
		numberstyle=\color{black!5},
		stringstyle=\color{black}, 
		showspaces=false,
		showstringspaces=false, 
		showtabs=false, 
		tabsize=2, 
		numbers=none, 
		breaklines=true
}

\lstdefinelanguage{javascript}{
	keywords={typeof, new, true, false, catch, function, return, null, catch, switch, var, if, in, while, do, else, case, break},
	keywordstyle=\color{blue}\bfseries,
	ndkeywords={class, export, boolean, throw, implements, import, this},
	ndkeywordstyle=\color{darkgray}\bfseries,
	identifierstyle=\color{black},
	sensitive=false,
	comment=[l]{//},
	morecomment=[s]{/*}{*/},
	commentstyle=\color{purple}\ttfamily,
	stringstyle=\color{red}\ttfamily,
	morestring=[b]',
	morestring=[b]"
}

% disponi sezioni
\usepackage{titlesec}

\titleformat{\section}
	{\sffamily\Large\bfseries} 
	{\thesection}{1em}{} 
\titleformat{\subsection}
	{\sffamily\large\bfseries}   
	{\thesubsection}{1em}{} 
\titleformat{\subsubsection}
	{\sffamily\normalsize\bfseries} 
	{\thesubsubsection}{1em}{}

% disponi alberi
\usepackage{forest}

\forestset{
	rectstyle/.style={
		for tree={rectangle,draw,font=\large\sffamily}
	},
	roundstyle/.style={
		for tree={circle,draw,font=\large}
	}
}

% disponi algoritmi
\usepackage{algorithm}
\usepackage{algorithmic}
\makeatletter
\renewcommand{\ALG@name}{Algoritmo}
\makeatother

% disponi numeri di pagina
\usepackage{fancyhdr}
\fancyhf{} 
\fancyfoot[L]{\sffamily{\thepage}}

\makeatletter
\fancyhead[L]{\raisebox{1ex}[0pt][0pt]{\sffamily{\@title \ \@date}}} 
\fancyhead[R]{\raisebox{1ex}[0pt][0pt]{\sffamily{\@author}}}
\makeatother

\begin{document}

% sezione (data)
\section{Lezione del 05-03-25}

% stili pagina
\thispagestyle{empty}
\pagestyle{fancy}

% testo
\subsection{Modello di business}
L'agente principale che prendiamo in osservazione durante il corso è l'\textbf{impresa}, nella sua organizzazione e nella sua attività commerciale e finanziaria.
L'impresa rappesenta a tutti gli effetti un \textbf{sistema}, che interagisce con l'esterno ottenendo qualcosa, e restituendo da parte loro qualcos'altro.

L'impresa interagisce quindi coi \textbf{mercati} scambiando merci (coi fornitori), prodotti (coi clienti), titoli di vario tipo e in generale denaro.
Avrà bisogno di \textbf{capitale}, che si procurerà tramite \textit{scelte finanziarie} (mezzi propri e finanziamenti).
Sfrutterà il capitale così ottenuto per cimentarsi nella \textbf{produzione} di un bene (un \textbf{prodotto}) da metter in commercio.

Nella \textit{costituzione} dell'impresa sarà necessaria un \textit{idea} di impresa, nonchè decisioni riguardo alla merce prodotta, quindi al \textit{mercato target}, al \textit{sistema di offerta} e alla struttura interna (\textbf{organizzazione aziendale}). 

Dall'esterno agiranno sull'impresa varie \textit{forze macroeconomiche}, nonchè l'attività dello \textbf{stato} (imposte, ecc...), le \textit{forze di mercato} e i \textit{trend socio-economici}.
Le imprese saranno poi in \textbf{concorrenza}, sia questa diretta, indiretta o potenziale, fra di loro.

\subsubsection{Mercati finanziari}
Una delle categorie di mercati con cui interagisce l'impresa è rappresentata dai \textbf{mercati finanziari}.
Questa interazione viene fatta attraverso gli \textbf{strumenti finanziari} (obbligazioni, azioni, mutui, ecc...).
In particolare, azioni e obbligazioni vengono comprate e vendute sul \textbf{mercato mobiliare}, solitamente detto \textit{borsa}.

\subsection{Capitale}
L'obiettivo principale dell'impresa è quello dell'ottenimento del \textbf{capitale}.
Questo può derivare da:
\begin{itemize}
	\item \textbf{Capitale proprio} dell'imprenditore o di eventuali \textbf{soci} disposti ad unirsi al supporto dell'idea di business, viene detto anche \textbf{capitale di rischio}, in quanto non ha alcuna garanzia di essere recuperato nel caso del fallimento dell'impresa. Di contro, i soci hanno \textbf{diritto residuale}, cioè di suddividersi ciò che residua a soddisfacimento di tutti gli altri debiti.

		In termini contabili un sottoinsieme del capitale proprio viene definito \textbf{capitale sociale}, cioè l'insieme dei \textit{conferimenti} effettuati dai soci durante la costituzione o in momenti successivi della vita dell'impresa.

		Altre fonti di capitale proprio sono rappresentate anche dall'\textbf{utile}, cioè dal risultato delle attività dell'impresa.
	\item \textbf{Capitale di terzi} dei \textit{finanziatori}, viene detto anche \textbf{capitale di credito}, in quanto i finanziatori hanno diritto alla sua restituzione anche nel caso di fallimento, oltre che ad un \textbf{interesse} che l'impresa paga per il capitale che il finanziatore non spende personalmente, ma gli mette a disposizione.
		I finanziatori rappresentano per noi dei \textbf{creditori}, cioè abbiamo per loro un \textbf{debito}.
	\item \textbf{Crowdsourcing}.
\end{itemize}

\subsubsection{Obbligazioni e azioni come fonti di capitale}

Le \textbf{obbligazioni} rappresentano quindi un tipo di \textit{capitale di credito}, cioè dei prestiti che l'impresa si impegna a rimborsare al finanziatore con un certo interesse.
Le obbligazioni sono poi \textbf{titoli di credito} (cambiali, assegni, ecc...), con le loro regole proprie che determinano le modalità secondo le quali possono essere immesse nel mercato.

Le \textbf{azioni} sono invece un tipo di \textit{capitale di rischio}.
L'azione consiste per gli acquirenti in un versamento da parte dell'impresa di \textbf{dividendi}, ricavati dall'attività della stessa, e legati alla relazione dell'azionista con la stessa.

\subsubsection{Note sui tipi di società}
Notiamo che solo alcuni tipi di società (S.p.A., \textit{Società per Azioni} o S.a.p.a. (\textit{Società in accomandita per azioni})) possono emanare azioni.
Le S.r.l. \textit{Società a responsabilità limitata} e le \textit{società di persone} detengono invece \textbf{quote di capitale}.

In particolare, le S.p.A., le S.a.p.a. e le S.r.l. rappresentano \textbf{società di capitali}, mentre le società di persone sono ulteriormente divise in S.n.C. (\textit{Società in nome Collettivo}), S.s. (\textit{Società semplice}) e S.a.s (\textit{Società in accomandita semplice}).

Si ha quindi la divisione:
\begin{itemize}
	\item Società di capitali:
		\begin{itemize}
			\item S.p.A. - Società per Azioni;
			\item S.a.p.a. - Società in accomandita per azioni;
			\item S.r.l. - Società a responsabilità limitata.
		\end{itemize}
	\item Società di persone:
		\begin{itemize}
			\item S.n.c. - Società in nome collettivo;
			\item S.s. - Società semplici;
			\item S.a.s. - Società in accomandita semplice.
		\end{itemize}
\end{itemize}

\subsection{Produzione}
La società in disponibilità di liquido (capitale proprio e di terzi) dovrà effettuare \textbf{operazioni di acquisto} dei cosiddetti \textbf{fattori produttivi}, che distinguiamo in:
\begin{itemize}
	\item \textbf{Fattori produttivi pluriennali}, stabilimenti, automezzi, strumentazioni, ecc...
	\item \textbf{Manodopera}, cioè il \textit{lavoro} dei dipendenti;
		\item \textbf{Fattori produttivi di esercizio}, cioè \textit{merci}, \textit{materia prima}, o ancora \textit{materia sussidiaria}, ecc...
\end{itemize}

Notiamo che i fattori produttivi di esercizio contribuiscono totalmente al valore di un prodotto, mentre i fattori produttivi pluriennali vi contribuiscono solo parzialmente (la stessa fabbrica produce più di un prodotto all'anno).
Il processo di ripartizione di un costo pluriennale sul consumo fatto in un periodo di tempo (solitamente un anno) viene detto \textbf{ammortamento}.

Nell'operazione di acquisto l'impresa crea un \textbf{debito} dei fornitori dei fattori produttivi.
Questo debito si distingue dal debito che abbiamo con i finanziatori in quanto rappresenta \textbf{debito di funzionamento} o \textbf{debito commerciale}.
La presenza di debiti commerciali è fisiologica all'attività dell'impresa, in quanto è necessaria alla produzione e vendita del prodotto.

\par\smallskip

Una volta che si è in possesso delle liquidità, e quindi dei fattori di produzione, si procede con la produzione vera e proprio del prodotto da mettere in commercio.

Nelle \textbf{imprese manufatturiere} è tipica una \textit{trasformazione fisica} dei fattori produttivi di esercizio.

Le \textbf{imprese commerciali} effettuano invece \textit{trasformazioni economiche} (e.g. di denaro o merci) nel \textit{tempo} e nello \textit{spazio}, traendo guadagno da fattori come l'interesse o la rivendita di merci in veste di intermediari fra produttori e consumatori.

\subsection{Vendita}
Il prodotto ultimato verrà venduto nei \textbf{mercati di sbocco}, che possono essere:
\begin{itemize}
	\item \textbf{B2B}, \textit{Business-to-Business}: da imprese ad altre imprese, ad esempio componenti (\textbf{prodotti intermedi}) che verranno assemblati da altre imprese;
	\item \textbf{B2C}, \textit{Business-to-Consumer}: dall'impresa al consumatore, di \textbf{prodotti finali}.
\end{itemize}

Durante la fase di vendita si può creare un'altro tipo di debito, cio il \textbf{credito commerciale} dei clienti nei confronti dell'impresa.

\subsection{Gestione}
L'insieme di passaggi che abbiamo visto finora (finanziamento, produzione e vendita) formano il \textbf{ciclo operativo della gestione} dell'impresa.
La \textbf{gestione} d'impresa e l'insieme delle operazioni che le persone operanti nell'impresa compiono (sia decisioni che azioni) tramite i fattori produttivi a disposizione per svolgere le attività che l'impresa ha definito.

Le operazioni di gestione \textbf{ordinaria} sono ciò che l'impresa fa su base quotidiana: \textbf{acquisto} $\rightarrow$ \textbf{produzione} $\rightarrow$ \textbf{vendita}, supportate da finanziamento \textbf{attinto} (mercati finanziari) e finanziamento \textbf{concesso}.

In particolare, chiamiamo operazioni di gestione ordinaria \textbf{esterna} quelle operazioni che ci mettono in contatto con terzi.
Sono di questo tipo tutte le operazioni tranne la produzione.
La produzione è quindi l'unica operazione che, dal magazzino di ingresso al magazzino di uscita, non mette l'impresa in contatto con i terzi.
Le operazioni di gestione ordinaria esterna sono quindi quelle che riguardano anche la \textbf{contabilità} e quindi il \textbf{bilancio}, cioè la valutazione dei flussi di denaro in entranta e in uscita.

\subsection{Soggetti dell'impresa}
	L'impresa coinvolge una vasta gamma di soggetti, primi fra tutti gli \textbf{stakeholder} (portatori di \textit{interesse}), cioè coloro che costituiscono il \textbf{soggetto economico} di influenza sull'impresa.
Questi possono essere:
\begin{itemize}
	\item I soci;
	\item I finanziatori;
	\item Manager e dipendenti;
	\item I clienti;
	\item Lo stato;
	\item I concorrenti;
	\item La comunità sociale.
\end{itemize}

In particolare, gli \textbf{shareholder} sono rappresentati dai soci e dagli azionisti (di maggioranza/minoranza), cioè da chi ha fornito capitale di rischio all'impresa.

Il \textbf{soggetto giuridico} è invece rappresentato dalla persona fisica o giuridica che assume la titolarità dell'impresa.

I soggetti dell'impresa hanno fra di loro responsabilità diverse per quanto riguarda l'attività e le responsabilità cui deve render conto l'impresa.
Ad esempio, come avevamo detto, i titolari di obbligazioni avranno diritti a restituzioni, mentre gli azionisti non avranno lo stesso diritto.
Di contro, il capitale proprio degli shareholder, essendo loro privato, non potrà essere usato per il rimborso di rischi andati in fallo (mentre lo stato potrà procedere per quanto riguarda titoli di credito regolamentati).

\subsection{Struttura dell'impresa, organigramma}
A livello formale la struttura dell'impresa è solitamente data dal suo \textbf{organigramma}, che ci da informazioni di livello strutturale e formale, ma non delle interdipendenze informali fra i suoi elementi.
	
Di base, consultando l'organigramma potremo stabilire chi comanda e chi è sottoposto, ma ad esempio non potremo comprendere relazioni fra elementi costruite all'esterno della struttura propria dell'impresa.

\end{document}
