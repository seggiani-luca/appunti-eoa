
\documentclass[a4paper,11pt]{article}
\usepackage[a4paper, margin=8em]{geometry}

% usa i pacchetti per la scrittura in italiano
\usepackage[french,italian]{babel}
\usepackage[T1]{fontenc}
\usepackage[utf8]{inputenc}
\frenchspacing 

% usa i pacchetti per la formattazione matematica
\usepackage{amsmath, amssymb, amsthm, amsfonts}

% usa altri pacchetti
\usepackage{gensymb}
\usepackage{hyperref}
\usepackage{standalone}

% imposta il titolo
\title{Appunti Economia ed Organizzazione Aziendale}
\author{Luca Seggiani}
\date{2025}

% disegni
\usepackage{pgfplots}
\pgfplotsset{width=10cm,compat=1.9}

% imposta lo stile
% usa helvetica
\usepackage[scaled]{helvet}
% usa palatino
\usepackage{palatino}
% usa un font monospazio guardabile
\usepackage{lmodern}

\renewcommand{\rmdefault}{ppl}
\renewcommand{\sfdefault}{phv}
\renewcommand{\ttdefault}{lmtt}

% disponi il titolo
\makeatletter
\renewcommand{\maketitle} {
	\begin{center} 
		\begin{minipage}[t]{.8\textwidth}
			\textsf{\huge\bfseries \@title} 
		\end{minipage}%
		\begin{minipage}[t]{.2\textwidth}
			\raggedleft \vspace{-1.65em}
			\textsf{\small \@author} \vfill
			\textsf{\small \@date}
		\end{minipage}
		\par
	\end{center}

	\thispagestyle{empty}
	\pagestyle{fancy}
}
\makeatother

% disponi teoremi
\usepackage{tcolorbox}
\newtcolorbox[auto counter, number within=section]{theorem}[2][]{%
	colback=blue!10, 
	colframe=blue!40!black, 
	sharp corners=northwest,
	fonttitle=\sffamily\bfseries, 
	title=Teorema~\thetcbcounter: #2, 
	#1
}

% disponi definizioni
\newtcolorbox[auto counter, number within=section]{definition}[2][]{%
	colback=red!10,
	colframe=red!40!black,
	sharp corners=northwest,
	fonttitle=\sffamily\bfseries,
	title=Definizione~\thetcbcounter: #2,
	#1
}

% disponi problemi
\newtcolorbox[auto counter, number within=section]{problem}[2][]{%
	colback=green!10,
	colframe=green!40!black,
	sharp corners=northwest,
	fonttitle=\sffamily\bfseries,
	title=Problema~\thetcbcounter: #2,
	#1
}

% disponi codice
\usepackage{listings}
\usepackage[table]{xcolor}

\lstdefinestyle{codestyle}{
		backgroundcolor=\color{black!5}, 
		commentstyle=\color{codegreen},
		keywordstyle=\bfseries\color{magenta},
		numberstyle=\sffamily\tiny\color{black!60},
		stringstyle=\color{green!50!black},
		basicstyle=\ttfamily\footnotesize,
		breakatwhitespace=false,         
		breaklines=true,                 
		captionpos=b,                    
		keepspaces=true,                 
		numbers=left,                    
		numbersep=5pt,                  
		showspaces=false,                
		showstringspaces=false,
		showtabs=false,                  
		tabsize=2
}

\lstdefinestyle{shellstyle}{
		backgroundcolor=\color{black!5}, 
		basicstyle=\ttfamily\footnotesize\color{black}, 
		commentstyle=\color{black}, 
		keywordstyle=\color{black},
		numberstyle=\color{black!5},
		stringstyle=\color{black}, 
		showspaces=false,
		showstringspaces=false, 
		showtabs=false, 
		tabsize=2, 
		numbers=none, 
		breaklines=true
}

\lstdefinelanguage{javascript}{
	keywords={typeof, new, true, false, catch, function, return, null, catch, switch, var, if, in, while, do, else, case, break},
	keywordstyle=\color{blue}\bfseries,
	ndkeywords={class, export, boolean, throw, implements, import, this},
	ndkeywordstyle=\color{darkgray}\bfseries,
	identifierstyle=\color{black},
	sensitive=false,
	comment=[l]{//},
	morecomment=[s]{/*}{*/},
	commentstyle=\color{purple}\ttfamily,
	stringstyle=\color{red}\ttfamily,
	morestring=[b]',
	morestring=[b]"
}

% disponi sezioni
\usepackage{titlesec}

\titleformat{\section}
	{\sffamily\Large\bfseries} 
	{\thesection}{1em}{} 
\titleformat{\subsection}
	{\sffamily\large\bfseries}   
	{\thesubsection}{1em}{} 
\titleformat{\subsubsection}
	{\sffamily\normalsize\bfseries} 
	{\thesubsubsection}{1em}{}

% disponi alberi
\usepackage{forest}

\forestset{
	rectstyle/.style={
		for tree={rectangle,draw,font=\large\sffamily}
	},
	roundstyle/.style={
		for tree={circle,draw,font=\large}
	}
}

% disponi algoritmi
\usepackage{algorithm}
\usepackage{algorithmic}
\makeatletter
\renewcommand{\ALG@name}{Algoritmo}
\makeatother

% disponi numeri di pagina
\usepackage{fancyhdr}
\fancyhf{} 
\fancyfoot[L]{\sffamily{\thepage}}

\makeatletter
\fancyhead[L]{\raisebox{1ex}[0pt][0pt]{\sffamily{\@title \ \@date}}} 
\fancyhead[R]{\raisebox{1ex}[0pt][0pt]{\sffamily{\@author}}}
\makeatother

\begin{document}

% sezione (data)
\section{Lezione del 10-04-25}

% stili pagina 
\thispagestyle{empty}
\pagestyle{fancy}

% testo
\subsection{Nota integrativa}
La nota integrativa è un documento di testo che funge in qualche modo da \textbf{legenda} dello \textit{stato patrimoniale} e del \textit{conto economico}, e contiene quindi informazioni quali la data di acquisto di immobili, il modo in cui è valutato l'ammortamento, verso chi abbiamo crediti o debiti, la possibilità che i debitori hanno di pagarci o meno, ecc... 

\subsection{Prudenza}
Uno dei principi fondamentali del bilancio è il \textbf{principio di prudenza}: i \textbf{ricavi} di esercizio vanno nel conto economico \textit{solo} se sono certi, mentre i \textbf{costi} ci vanno anche se \textit{presunti}.

Questo principio ha lo scopo di salvaguardare gli \textbf{stakeholder} (clienti, dipendenti, stato, ecc...), in quanto impedisce all'impresa di \textit{"gonfiare"} il proprio capitale. 

\subsubsection{Fondi spesa e rischio}
A proposito dei crediti, distinguiamo il \textbf{valore nominale dei crediti} dal prezzo dopo la \textbf{svalutazione crediti}.
Questo va a finire nel cosiddetto \textbf{fondo svalutazione crediti}, che è una voce fra le passività dello stato patrimoniale, di valore uguale a cioè che viene accantonato del credito nominale.
Visto che non c'è un'altro movimento finanziario, la svalutazione crediti rappresenta un costo dal punto di vista del conto economico d'esercizio.

Il fondo svalutazione crediti fa parte della categoria più ampia dei \textbf{fondi rischi}: esistono es esempio:
\begin{itemize}
	\item \textbf{Fondi titoli}, riguardanti i titoli finanziari più o meno recuperabili;
	\item \textbf{Fondi cambi}, riguardanti i crediti in valute straniere, il cui valore è variabile nel tempo.
\end{itemize}

Il fondo è quindi uno strumento che ci permette di ammortizzare, nell'anno contabile, i rischi \textit{previsti} d'esercizio: al momento del pagamento \textit{parziale} del credito, ad esempio, si può usare il fondo svalutazione crediti per pareggiare la differenza non pagata, quindi non registra un costo.
Il costo invece c'è, chiaramente, quando il fondo è stato valutato in difetto, cioè il rischio è di entità maggiore di quanto previsto, anche se in questo caso si paga solo la differenza rispetto alla valutazione fatta e non il valore assoluto del costo.

Nel caso il fondo svalutazione crediti sia stato invece valutato in eccesso, questo in una normale situazione di esercizio andrà a coprire altri crediti arrivati nel frattempo.
Se invece ci si trova in una situazione di cessazione di esercizio, e non esiste più l'ipotesi di rischio, si potrà rmiuovere valore dal fondo, portando ad una variazione economica di esercizio in positivo detta \textbf{sopravvenienza attiva}, che è un ricavo di natura \textit{straordinaria}.

A far parte di un altro tipo di fondo, che è il \textbf{fondo spese}, è il \textbf{fondo TFR}.
Il fondo spese si riferisce ad eventi \textit{certi} di cui però è incerta la data e l'importo.
In questo si contrappone ai fondi rischi, che si riferisce ad eventi fondamentalmente incerti, oltre che incerti in data e in importo.
Il fondo TFR fa quindi riferimento al \textbf{TFR} (\textit{Trattamento Fine Rapporto}), cioè al \textbf{licenziamento}, sia da parte del datore di lavoro che da parte del lavoratore, o al \textbf{pensionamento}.
Il fondo TFR matura quindi nel corso del rapporto del lavoratore, assieme allo stipendio.
Quest'ultimo viene pagato periodicamente, mentre il primo viene rilasciato al termine del rapporto lavorativo.
Visto che questo finanziamento rappresenterebbe una grande uscita di cassa al momento della fine del rapporto, si dedica un fondo apposito, appunto il fondo TFR, per coprirlo. 
Questo viene accumulato, su base annuale, come un debito nei confronti di ogni dipendente, di ammontare pari alla cosiddetta \textbf{quota TFR}, che tendenzialmente varia ogni anno su base dello stipendio e del costo della vita.
Chiaramente questo è un \textit{costo d'esercizio} che va nel \textit{conto economico}.
Al versamento del TFR, quindi, non si hanno costi in quanto il TFR è stato contabilizzato anno per anno per ogni dipendente.

\subsection{Competenza}
Un'altro fattore da considerare nel bilancio è la \textbf{competenza}.
Costi e ricavi \textit{di competanza} sono quei costi e ricavi che vengono registrati prima di eventuali movimenti di cassa, cioè flussi di denaro da o nella cassa.


\subsection{Motivazione di bilancio}
Il bilancio viene redatto per più motivi, che possono essere interni od esterni:
\begin{itemize}
	\item \textbf{Motivi interni:} sicuramente il bilancio aiuta a migliorare il funzionamento dell'impresa, anche se oggi conviene avere informazioni su base ancora più frequente di quella fornita dal bilancio;

	\item \textbf{Motivi esterni:} questi si possono dividere in:
		\begin{itemize}
			\item \textit{Civilistici}: il legislatore \textit{civilistico} vuole assicurarsi che tutti i soggetti (gli \textit{stakeholder}) possano avere rapporti con l'impresa affinché questi non siano tratti in inganno da incrementi patrimoniali o reddituali fittizi;

			\item \textit{Fiscali:} il legislatore \textit{fiscale} vuole invece evitare che l'impresa attui manovre elusive in modo tale da ridurre i versamenti tributari (il pagamento delle imposte);
		\end{itemize}
\end{itemize}

Non esiste un unicio bilancio, allo stesso modo in cui non esiste un unico risultato d'esercizio.
Motivazioni diverse alla base della redazione danno vita a bilanci diversi: ad esempio, il bilancio che possimo vedere noi in qualità di \textit{stakeholder} è il bilancio \textit{civilistico}, mentre al legislatore fiscale interesserà il bilancio \textit{fiscale}, ecc...

\subsection{Fattori di esercizio}
 Modellizziamo la situazione con due magazzini, \textbf{IN} e \textbf{OUT}, che stanno agli estremi della catena di produzione.
Nel magazzino IN entrano le materie prime, e nel magazzino OUT vanno a finire i prodotti finiti.

Vediamo cosa accade nei due magazzini nel corso dell'anno contabile.

\begin{itemize}
	\item \textbf{Magazzino IN:}
nel corso dell'anno contabile si effettuano \textbf{acquisti di materie}, che vanno a finire nel magazzino IN.
Al 1/1, inoltre, il magazzino IN potrebbe contenere \textbf{esistenze iniziali} di materie, comprate in anni contabili precedenti.
Le esistenze iniziali più gli acquisti di materie ci danno le \textbf{materie disponibili} alla produzione.

Si ha quindi:
$$
\text{esistenze iniziali} + \text{acquisti di materie} = \text{materie disponibili}
$$

Al 31/12 il magazzino potrebbe non essere vuoto.
Quelle che rimangono vengono dette \textbf{rimanenze finali}.
La presenza di rimanenze finali rappresenta il fatto che una parte della materie prime non è andata in produzione, e quindi l'esistenza di risorse non consumate.

Si ha quindi:
$$
\text{materie disponibili} - \text{rimanenze finali} = \text{materie consumate}
$$

Le \textbf{materie consumate} vanno nei costi di esercizio, mentre le risorse non consumate vanno nelle attività dello stato patrimoniale.

\item \textbf{Magazzino OUT:}
	dal punto di vista della produzione, si fanno le stesse considerazioni per il magazzino OUT: se l'1/1 il magazzino OUT non è vuoto, si hanno delle \textbf{esistenze iniziali} che non sono state prodotte quest'anno.

	Allo stesso modo, quello che resta nel magazzino OUT alla fine dell'anno contabile, cioè le \textbf{rimanenze finali}, rappresentano qualcosa che si è prodotto nella \textit{durata} dell'anno contabile.

\end{itemize}

\end{document}
