
\documentclass[a4paper,11pt]{article}
\usepackage[a4paper, margin=8em]{geometry}

% usa i pacchetti per la scrittura in italiano
\usepackage[french,italian]{babel}
\usepackage[T1]{fontenc}
\usepackage[utf8]{inputenc}
\frenchspacing 

% usa i pacchetti per la formattazione matematica
\usepackage{amsmath, amssymb, amsthm, amsfonts}

% usa altri pacchetti
\usepackage{gensymb}
\usepackage{hyperref}
\usepackage{standalone}

% imposta il titolo
\title{Appunti Economia ed Organizzazione Aziendale}
\author{Luca Seggiani}
\date{2025}

% disegni
\usepackage{pgfplots}
\pgfplotsset{width=10cm,compat=1.9}

% imposta lo stile
% usa helvetica
\usepackage[scaled]{helvet}
% usa palatino
\usepackage{palatino}
% usa un font monospazio guardabile
\usepackage{lmodern}

\renewcommand{\rmdefault}{ppl}
\renewcommand{\sfdefault}{phv}
\renewcommand{\ttdefault}{lmtt}

% disponi il titolo
\makeatletter
\renewcommand{\maketitle} {
	\begin{center} 
		\begin{minipage}[t]{.8\textwidth}
			\textsf{\huge\bfseries \@title} 
		\end{minipage}%
		\begin{minipage}[t]{.2\textwidth}
			\raggedleft \vspace{-1.65em}
			\textsf{\small \@author} \vfill
			\textsf{\small \@date}
		\end{minipage}
		\par
	\end{center}

	\thispagestyle{empty}
	\pagestyle{fancy}
}
\makeatother

% disponi teoremi
\usepackage{tcolorbox}
\newtcolorbox[auto counter, number within=section]{theorem}[2][]{%
	colback=blue!10, 
	colframe=blue!40!black, 
	sharp corners=northwest,
	fonttitle=\sffamily\bfseries, 
	title=Teorema~\thetcbcounter: #2, 
	#1
}

% disponi definizioni
\newtcolorbox[auto counter, number within=section]{definition}[2][]{%
	colback=red!10,
	colframe=red!40!black,
	sharp corners=northwest,
	fonttitle=\sffamily\bfseries,
	title=Definizione~\thetcbcounter: #2,
	#1
}

% disponi problemi
\newtcolorbox[auto counter, number within=section]{problem}[2][]{%
	colback=green!10,
	colframe=green!40!black,
	sharp corners=northwest,
	fonttitle=\sffamily\bfseries,
	title=Problema~\thetcbcounter: #2,
	#1
}

% disponi codice
\usepackage{listings}
\usepackage[table]{xcolor}

\lstdefinestyle{codestyle}{
	backgroundcolor=\color{black!5}, 
	commentstyle=\color{codegreen},
	keywordstyle=\bfseries\color{magenta},
	numberstyle=\sffamily\tiny\color{black!60},
	stringstyle=\color{green!50!black},
	basicstyle=\ttfamily\footnotesize,
	breakatwhitespace=false,         
	breaklines=true,                 
	captionpos=b,                    
	keepspaces=true,                 
	numbers=left,                    
	numbersep=5pt,                  
	showspaces=false,                
	showstringspaces=false,
	showtabs=false,                  
	tabsize=2
}

\lstdefinestyle{shellstyle}{
	backgroundcolor=\color{black!5}, 
	basicstyle=\ttfamily\footnotesize\color{black}, 
	commentstyle=\color{black}, 
	keywordstyle=\color{black},
	numberstyle=\color{black!5},
	stringstyle=\color{black}, 
	showspaces=false,
	showstringspaces=false, 
	showtabs=false, 
	tabsize=2, 
	numbers=none, 
	breaklines=true
}

\lstdefinelanguage{javascript}{
	keywords={typeof, new, true, false, catch, function, return, null, catch, switch, var, if, in, while, do, else, case, break},
	keywordstyle=\color{blue}\bfseries,
	ndkeywords={class, export, boolean, throw, implements, import, this},
	ndkeywordstyle=\color{darkgray}\bfseries,
	identifierstyle=\color{black},
	sensitive=false,
	comment=[l]{//},
	morecomment=[s]{/*}{*/},
	commentstyle=\color{purple}\ttfamily,
	stringstyle=\color{red}\ttfamily,
	morestring=[b]',
	morestring=[b]"
}

% disponi sezioni
\usepackage{titlesec}

\titleformat{\section}
{\sffamily\Large\bfseries} 
{\thesection}{1em}{} 
\titleformat{\subsection}
{\sffamily\large\bfseries}   
{\thesubsection}{1em}{} 
\titleformat{\subsubsection}
{\sffamily\normalsize\bfseries} 
{\thesubsubsection}{1em}{}

% disponi alberi
\usepackage{forest}

\forestset{
	rectstyle/.style={
		for tree={rectangle,draw,font=\large\sffamily}
	},
	roundstyle/.style={
		for tree={circle,draw,font=\large}
	}
}

% disponi algoritmi
\usepackage{algorithm}
\usepackage{algorithmic}
\makeatletter
\renewcommand{\ALG@name}{Algoritmo}
\makeatother

% disponi numeri di pagina
\usepackage{fancyhdr}
\fancyhf{} 
\fancyfoot[L]{\sffamily{\thepage}}

\makeatletter
\fancyhead[L]{\raisebox{1ex}[0pt][0pt]{\sffamily{\@title \ \@date}}} 
\fancyhead[R]{\raisebox{1ex}[0pt][0pt]{\sffamily{\@author}}}
\makeatother

\begin{document}

% sezione (data)
\section{Lezione del 30-04-25}

% stili pagina
\thispagestyle{empty}
\pagestyle{fancy}

% testo
\subsection{Modelli di stato patrimoniale}
Abbiamo quindi visto il modello contabile dello stato patrimoniale che le aziende devono tenere.
Vediamo adesso due modelli per la redazione dello stato patrimoniale, quello descritto dallo standard internazionale \textbf{IAS-IFRS} (\textit{International Accounting Standards} e \textit{International Financial Reporting Standards}), e quello descritto dal codice civile (schema art. 2424 c.c.).

\subsubsection{Codice civile}
Nello schema del codice civile abbiamo sezioni individuate da lettere nel conto attivo e passivo, che sono:
\begin{itemize}
	\item \textbf{Attivo:}
		\begin{itemize}
			\item[A)] Crediti verso soci; 
			\item[B)] Immobilizzazioni:
				\begin{enumerate}
					\item Immobilizzazioni immateriali;
					\item Immobilizzazioni materiali;
					\item Immobilizzazioni finanziarie.
				\end{enumerate}
			\item[C)] Attivo circolante:
				\begin{enumerate}
					\item Rimanenze;
					\item Crediti;
					\item Attività finanziarie;
					\item Disponibilità liquide.
				\end{enumerate}
			\item[D)] Ratei e risconti attivi.
		\end{itemize}
	\item \textbf{Passivo:}
		\begin{itemize}
			\item[A)] Patrimonio netto;
			\item[B)] Fondi per rischi ed oneri;
			\item[C)] Trattamento di fine rapporto;
			\item[D)] Debiti;
			\item[E)] Ratei e risconti passivi.
		\end{itemize}
\end{itemize}

\subsubsection{IAS-IFRS}
Nel modello IAS-IFRS l'attivo fra asset \textit{correnti} e \textit{non correnti}, cioè:
\begin{itemize}
	\item Attività \textbf{correnti}:
		\begin{itemize}
			\item Crediti commerciali e altri;
			\item Rimanenze;
			\item Lavori in corso su ordinazione;
			\item Attività finanziarie correnti;
			\item Disponibiltà liquide.
		\end{itemize}
	\item Attività \textbf{non correnti}:
		\begin{itemize}
			\item Immobili, impianti e macchinari;
			\item Investimenti immobiliari;
			\item Avviamento e attività immateriali a vita non definita;
			\item Altre attività immateriali;
			\item Partecipazioni;
			\item Altre attività finanziarie;
			\item Imposte differite attive.
		\end{itemize}
\end{itemize}

Anche per il passivo si distingue fra corrente e non corrente, come:
\begin{itemize}
	\item Passività \textbf{correnti}:
		\begin{itemize}
			\item Passività finanziare correnti;
			\item Debiti commerciali;
			\item Debiti per imposte;
			\item Debiti vari e altre passività correnti.
		\end{itemize}
	\item Passività \textbf{non correnti}:
		\begin{itemize}
			\item Passività finanziarie non correnti;
			\item TFR e altri fondi relativi al personale;
			\item Fondo imposte differite;
			\item Fondo per rischi e oneri futuri;
			\item Debiti vari e altre passività non correnti.
		\end{itemize}
\end{itemize}

Infine, nel passivo si riporta anche il patrimonio netto:
\begin{itemize}
	\item Capitale emesso;
	\item Riserve;
	\item Utili/perdite d'esercizio;
	\item Utili/perdite portate a nuovo.
\end{itemize}

\subsection{Modelli di conto economico}
Anche per quanto riguarda il conto economico esiste uno schema \textbf{IAS} e uno schema dettato dal \textbf{codice civile}. 

Riprendiamo innanzitutti le aree di stratificazione che avevamo già visto per valutare il \textbf{valore della produzione}:
\begin{itemize}
	\item Area \textbf{caratteristica}:
		\begin{itemize}
			\item Valore della produzione:
				\begin{itemize}
					\item Ricavi di vendita;
					\item \textbf{Costi in economia}, cioè di macchinari/immobili che non vengono acquistati ma alla cui costruzione si delega la macchina produttiva, che non sono di competenza alla produzione e vanno quindi messi qui per annullare i costi di produzioni che vi si impiegano;
					\item Incremento magazzino OUT.
				\end{itemize}
			\item Costo della produzione:
				\begin{itemize}
					\item Materie prime;
					\item Immobilizzazioni ammortate;
					\item Salari e stipendi, più quote TFR (cioè il \textit{costo lavoro});
					\item Decremento magazzino IN.
				\end{itemize}
		\end{itemize}
		$\rightarrow$ \textbf{Margine Operativo Lordo}.
	\item Area \textbf{accessoria}:
		\begin{itemize}
			\item Proventi ed oneri accessori.
		\end{itemize}
		$\rightarrow$ \textbf{Margine Operativo Netto} (\textbf{EBIT}).
	\item Area \textbf{finanziaria}:
		\begin{itemize}
			\item Proventi ed oneri finanziari;
			\item Rettifiche di valore da attività finanziarie.
		\end{itemize}
		$\rightarrow$ risultato lordo da attività in funzionamento.
	\item Area \textbf{fiscale}:
		\begin{itemize}
			\item Imposte sul reddito di esercizio.
		\end{itemize}
		$\rightarrow$ Utile o perdita di esercizio.
	\item Area \textbf{straordinaria}:
		\begin{itemize}
			\item Utile o perdita su attività destinate a cessare.
		\end{itemize}
\end{itemize}
ricordando anche che area caratteristica e accessoria formano insieme l'area \textbf{operativa}.

Questo è il modello adottato dal codice civile.

Vediamo quindi il conto fatto sul \textbf{costo del venduto}:
\begin{itemize}
	\item Area \textbf{caratteristica}:
		\begin{itemize}
			\item Ricavi di vendita;
			\item Costo del venduto:
				\begin{itemize}
					\item Costi per acquisti di produzione;
					\item Costo del personale di produzione;
					\item Costi per investimenti produttivi;
					\item La variazione delle rimanenze.
				\end{itemize}
		\end{itemize}
		$\rightarrow$ \textbf{Margine Lordo Industriale}, che equivale al margine operativo lordo depurato delle aree non industriali (amministrazione, ricerca e sviluppo, vendite ecc...).
		\begin{itemize}
			\item Altri costi di area caratteristica.
		\end{itemize}
		$\rightarrow$ \textbf{Margine Operativo Lordo}.

\end{itemize}

Questo è il modello adottato dallo standard IAS.

\subsection{Modello di business}
Un \textbf{modello di business} definisce la logica secondo la quale un'organizzazione \textbf{crea}, \textbf{distribuisce} e \textbf{cattura} valore.

Nel business model possiamo individuare diversi elementi:
\begin{itemize}
	\item I \textbf{segmenti di clientela};
	\item La \textbf{proposta di valore};
	\item I \textbf{canali};
	\item Le \textbf{relazioni coi clienti};
	\item I \textbf{flussi di ricavi};
	\item Le \textbf{risorse chiave};
	\item Le \textbf{attività chiave};
	\item Le \textbf{partnership chiave};
	\item La \textbf{struttura dei costi}.
\end{itemize}

\subsubsection{Segmenti di clientela}
Prima di tutto potremmo chiederci per \textit{chi} stiamo creando valore.
Possiamo individuare alcune macrocategorie:
\begin{itemize}
	\item Mercato di massa;
	\item Mercato di nicchia;
	\item Mercato segmentato;
	\item Mercato diversificato;
	\item Piattaforme (o mercati) multi-sided.
\end{itemize}

\subsubsection{Proposta di valore}
La proposta di valore è ciò che l'azienda offre in \textit{più} rispetto a un altra, quindi il motivo per cui i potenziali clienti dovrebbero acquistarne i prodotti.

\subsubsection{Canali}
I canali sono quelli attraverso i quali l'azienda fornisce i suoi prodotti/servizi al cliente.
Possiamo individuare ad esempio:
\begin{itemize}
	\item Forza di vendita;
	\item Vendita su web;
	\item Negozi propri;
	\item Negozi di partner;
	\item Grossisti.
\end{itemize}

\subsubsection{Relazioni coi clienti}
Le relazioni coi clienti riguardano:
\begin{itemize}
	\item L'\textit{acquisizione} di clienti;
	\item La \textit{fidelizzazione} di clienti;
	\item L'\textit{upselling}, cioè l'aumento della frequenza di acquisto dei clienti.
\end{itemize}

Possiamo individuare diversi meccanismi:
\begin{itemize}
	\item Assistenza personale;
	\item Assistenza personale dedicata;
	\item Self-service;
	\item Servizi automatici;
	\item Community;
	\item Co-creazione.
\end{itemize}

Di base, la relazione può distinguersi su \textbf{intensità}, \textbf{frequenza} e \textbf{intimità}:
\begin{itemize}
	\item \textbf{Intensità:}
		\begin{itemize}
			\item Una relazione a bassa intensità (o \textit{indiretta}) è quella dove il cliente non interagisce con l'azienda, ma con intermediari;
			\item Di contro una relazione ad altà intensità (o \textit{diretta}) è quella dove il cliente interagisce direttamente con l'azienda.
		\end{itemize}
	\item \textbf{Frequenza:}
		\begin{itemize}
			\item In una relazione a bassa frequenza si parla principalmente di \textit{transazioni}, cioè acquisti che non creano nessun legame futuro cliente;
			\item In una relazione ad alta frequenza si svilupppono invece relazioni di \textit{lungo termine} che perdurano nel tempo (ne sono un esempio aziende che producono beni pluriennali, che magari richiedono manutenzione, ecc...).
		\end{itemize}
	\item \textbf{Intimità:}
		\begin{itemize}
			\item L'intimità riguarda la natura della relazione: alta intimità (o intimità \textit{personale}) significa relazionarsi col personale, quindi con esseri umani;
			\item Bassa intimità significa invece relazioni automatiche, quindi con operatori automatici, computer, ecc... (ne sono esempi i servizi digitali).
		\end{itemize}
\end{itemize}

\subsubsection{Flussi di ricavi}
Un modello di business può prevedere due tipi di ricavi:
\begin{itemize}
	\item Ricavi correnti;
	\item Ricavi continui.
\end{itemize}

Questi possono essere:
\begin{itemize}
	\item Vendita di beni;
	\item Canone d'uso;
	\item Abbonamenti, che non corrispondono ai canoni d'uso (questi ultimi hanno solitamente natura occasionale, contro la continuità di un abbonamento);
	\item Quote di iscrizione;
	\item Prestito/noleggio/leasing;
	\item Licenza;
	\item Commissioni di intermediazione;
	\item Pubblicità;
	\item Servizi gratuiti: si pensi ad esempio al motore di ricerca Google.
\end{itemize}

\subsubsection{Risorse chiave}
Le risorse chiave sono i beni più importanti affinché un modello di business funzioni, e possono includere:
\begin{itemize}
	\item Materie prime;
	\item Particolari servizi;
	\item Personale specializzato;
	\item Ecc...
\end{itemize}

\subsubsection{Attività chiave}
Le attività chiave sono quelle attività di progettazione, creazione e distribuzione essenziali al funzionamento dell'azienda.
Ad esempio possiamo notare:
\begin{itemize}
	\item Le attività di \textbf{produzione}, fondamentali alle aziende manufatturiere;
	\item Le attività di \textbf{problem solving}, orientate alla risoluzione di problemi per i singoli clienti, tipiche di attività di consulenza e di servizi;
	\item Le attività riguardanti la \textbf{piattaforma}, per aziende che forniscono appunto \textit{piattaforme}: si pensi ai servizi online che richiedono costante manutenzione e moderazione. 
\end{itemize}

\subsubsection{Partnership chiave}
Le partnership chiave sono quelle relazioni con fornitori, intermediari se non addirittura concorrenti fondamentali all'attività dell'azienda. 

Possiamo distinguere:
\begin{itemize}
	\item Partnership fra concorrenti, le cosiddette \textbf{coopetition};
	\item Partnership per creare nuove proposte di valore, ad esempio fra due aziende di settori diversi che creano un prodotto combinato;
	\item Partnership \textit{buyer-supplier}, fra fornitori di lunga data;
	\item Partnership per ridurre i costi o il rischio.
\end{itemize}

\subsubsection{Struttura dei costi}
La sruttura dei costi riguarda i costi che l'azienda deve sopportare nell'ipotesi di funzionamento.

Questi possono essere:
\begin{itemize}
	\item \textbf{Variabili} sulla base della produzione;
	\item \textbf{Fissi}, cioè slegati dalla produzione.
\end{itemize}

Potremo dire che la parte di \textit{destra} del modello di business, cioè quella contenente:
\begin{itemize}
	\item I egmenti di clientela;
	\item La proposta di valore;
	\item I canali;
	\item Le relazioni coi clienti;
	\item I flussi di ricavi.
\end{itemize}
è detta parte del \textbf{valore}, mentre la parte si \textit{sinistra}, cioè quella contenente:
\begin{itemize}
	\item Le risorse chiave;
	\item Le attività chiave;
	\item Le partnership chiave;
	\item La struttura dei costi.
\end{itemize}
è detta parte dell'\textbf{efficienza}.

Di aziende incentrate sulla parte del valore si dice che sono \textbf{value-driven}, mentre di aziende incentrate sulla parte dell'efficienza si dice che sono \textbf{cost-driven}.

\end{document}
