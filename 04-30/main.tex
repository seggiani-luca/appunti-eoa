
\documentclass[a4paper,11pt]{article}
\usepackage[a4paper, margin=8em]{geometry}

% usa i pacchetti per la scrittura in italiano
\usepackage[french,italian]{babel}
\usepackage[T1]{fontenc}
\usepackage[utf8]{inputenc}
\frenchspacing 

% usa i pacchetti per la formattazione matematica
\usepackage{amsmath, amssymb, amsthm, amsfonts}

% usa altri pacchetti
\usepackage{gensymb}
\usepackage{hyperref}
\usepackage{standalone}

% imposta il titolo
\title{Appunti Economia ed Organizzazione Aziendale}
\author{Luca Seggiani}
\date{2025}

% disegni
\usepackage{pgfplots}
\pgfplotsset{width=10cm,compat=1.9}

% imposta lo stile
% usa helvetica
\usepackage[scaled]{helvet}
% usa palatino
\usepackage{palatino}
% usa un font monospazio guardabile
\usepackage{lmodern}

\renewcommand{\rmdefault}{ppl}
\renewcommand{\sfdefault}{phv}
\renewcommand{\ttdefault}{lmtt}

% disponi il titolo
\makeatletter
\renewcommand{\maketitle} {
	\begin{center} 
		\begin{minipage}[t]{.8\textwidth}
			\textsf{\huge\bfseries \@title} 
		\end{minipage}%
		\begin{minipage}[t]{.2\textwidth}
			\raggedleft \vspace{-1.65em}
			\textsf{\small \@author} \vfill
			\textsf{\small \@date}
		\end{minipage}
		\par
	\end{center}

	\thispagestyle{empty}
	\pagestyle{fancy}
}
\makeatother

% disponi teoremi
\usepackage{tcolorbox}
\newtcolorbox[auto counter, number within=section]{theorem}[2][]{%
	colback=blue!10, 
	colframe=blue!40!black, 
	sharp corners=northwest,
	fonttitle=\sffamily\bfseries, 
	title=Teorema~\thetcbcounter: #2, 
	#1
}

% disponi definizioni
\newtcolorbox[auto counter, number within=section]{definition}[2][]{%
	colback=red!10,
	colframe=red!40!black,
	sharp corners=northwest,
	fonttitle=\sffamily\bfseries,
	title=Definizione~\thetcbcounter: #2,
	#1
}

% disponi problemi
\newtcolorbox[auto counter, number within=section]{problem}[2][]{%
	colback=green!10,
	colframe=green!40!black,
	sharp corners=northwest,
	fonttitle=\sffamily\bfseries,
	title=Problema~\thetcbcounter: #2,
	#1
}

% disponi codice
\usepackage{listings}
\usepackage[table]{xcolor}

\lstdefinestyle{codestyle}{
		backgroundcolor=\color{black!5}, 
		commentstyle=\color{codegreen},
		keywordstyle=\bfseries\color{magenta},
		numberstyle=\sffamily\tiny\color{black!60},
		stringstyle=\color{green!50!black},
		basicstyle=\ttfamily\footnotesize,
		breakatwhitespace=false,         
		breaklines=true,                 
		captionpos=b,                    
		keepspaces=true,                 
		numbers=left,                    
		numbersep=5pt,                  
		showspaces=false,                
		showstringspaces=false,
		showtabs=false,                  
		tabsize=2
}

\lstdefinestyle{shellstyle}{
		backgroundcolor=\color{black!5}, 
		basicstyle=\ttfamily\footnotesize\color{black}, 
		commentstyle=\color{black}, 
		keywordstyle=\color{black},
		numberstyle=\color{black!5},
		stringstyle=\color{black}, 
		showspaces=false,
		showstringspaces=false, 
		showtabs=false, 
		tabsize=2, 
		numbers=none, 
		breaklines=true
}

\lstdefinelanguage{javascript}{
	keywords={typeof, new, true, false, catch, function, return, null, catch, switch, var, if, in, while, do, else, case, break},
	keywordstyle=\color{blue}\bfseries,
	ndkeywords={class, export, boolean, throw, implements, import, this},
	ndkeywordstyle=\color{darkgray}\bfseries,
	identifierstyle=\color{black},
	sensitive=false,
	comment=[l]{//},
	morecomment=[s]{/*}{*/},
	commentstyle=\color{purple}\ttfamily,
	stringstyle=\color{red}\ttfamily,
	morestring=[b]',
	morestring=[b]"
}

% disponi sezioni
\usepackage{titlesec}

\titleformat{\section}
	{\sffamily\Large\bfseries} 
	{\thesection}{1em}{} 
\titleformat{\subsection}
	{\sffamily\large\bfseries}   
	{\thesubsection}{1em}{} 
\titleformat{\subsubsection}
	{\sffamily\normalsize\bfseries} 
	{\thesubsubsection}{1em}{}

% disponi alberi
\usepackage{forest}

\forestset{
	rectstyle/.style={
		for tree={rectangle,draw,font=\large\sffamily}
	},
	roundstyle/.style={
		for tree={circle,draw,font=\large}
	}
}

% disponi algoritmi
\usepackage{algorithm}
\usepackage{algorithmic}
\makeatletter
\renewcommand{\ALG@name}{Algoritmo}
\makeatother

% disponi numeri di pagina
\usepackage{fancyhdr}
\fancyhf{} 
\fancyfoot[L]{\sffamily{\thepage}}

\makeatletter
\fancyhead[L]{\raisebox{1ex}[0pt][0pt]{\sffamily{\@title \ \@date}}} 
\fancyhead[R]{\raisebox{1ex}[0pt][0pt]{\sffamily{\@author}}}
\makeatother

\begin{document}

% sezione (data)
\section{Lezione del 30-04-25}

% stili pagina
\thispagestyle{empty}
\pagestyle{fancy}

% testo
\subsection{Modelli di stato patrimoniale}

\subsubsection{IAS-IFRS}

\subsubsection{Codice civile}

# confronto codice civile e IAS/IFRS

\subsection{Modelli di conto economico}
Anche per quanto riguarda il conto economico esiste uno schema \textbf{IAS} e uno schema dettato dal \textbf{codice civile}. 

Riprendiamo le aree di stratificazione che avevamo già visto per valutare il \textbf{valore della produzione}:
\begin{itemize}
	\item Area \textbf{caratteristica}:
		\begin{itemize}
			\item Valore della produzione:
				\begin{itemize}
					\item Ricavi di vendita;
					\item \textbf{Costi in economia}, cioè di macchinari/immobili che non vengono acquistati ma alla cui costruzione si delega la macchina produttiva, che non sono di competenza alla produzione e vanno quindi messi qui per annullare i costi di produzioni che vi si impiegano;
					\item Incremento magazzino OUT.
				\end{itemize}
			\item Costo della produzione:
				\begin{itemize}
					\item Materie prime;
					\item Immobilizzazioni ammortate;
					\item Salari e stipendi, più quote TFR (cioè il \textit{costo lavoro});
					\item Decremento magazzino IN.
				\end{itemize}
		\end{itemize}
		$\rightarrow$ \textbf{Margine Operativo Lordo}.
	\item Area \textbf{accessoria}:
		\begin{itemize}
			\item Proventi ed oneri accessori.
		\end{itemize}
		$\rightarrow$ \textbf{Margine Operativo Netto} (\textbf{EBIT}).
	\item Area \textbf{finanziaria}:
		\begin{itemize}
			\item Proventi ed oneri finanziari;
			\item Rettifiche di valore da attività finanziarie.
		\end{itemize}
		$\rightarrow$ risultato lordo da attività in funzionamento.
	\item Area \textbf{fiscale}:
		\begin{itemize}
			\item Imposte sul reddito di esercizio.
		\end{itemize}
		$\rightarrow$ Utile o perdita di esercizio.
	\item Area \textbf{straordinaria}:
		\begin{itemize}
			\item Utile o perdita su attività destinate a cessare.
		\end{itemize}
\end{itemize}
ricordando anche che area caratteristica e accessoria formano insieme l'area \textbf{operativa}.

Vediamo quindi il conto fatto sul \textbf{costo del venduto}:
\begin{itemize}
	\item Area \textbf{caratteristica}:
		\begin{itemize}
			\item Ricavi di vendita;
			\item Costo del venduto:
				\begin{itemize}
					\item Costi per acquisti di produzione;
					\item Costo del personale di produzione;
					\item Costi per investimenti produttivi;
					\item La variazione delle rimanenze.
				\end{itemize}
		\end{itemize}
		$\rightarrow$ \textbf{Margine Lordo Industriale}, che equivale al margine operativo lordo depurato delle aree non industriali (amministrazione, ricerca e sviluppo, vendite ecc...).
		\begin{itemize}
			\item Altri costi di area caratteristica.
		\end{itemize}
		$\rightarrow$ \textbf{Margine Operativo Lordo}.

\end{itemize}

\subsubsection{IAS}
Costo del venduto solo IAS

\subsubsection{Codice civile}

\subsection{Modello di business}
Un \textbf{modello di business} definisce la logica secondo la quale un'organizzazione \textbf{crea}, \textbf{distribuisce} e \textbf{cattura} valore.

Nel business model possiamo individuare diversi elementi:
\begin{itemize}
	\item I \textbf{segmenti di clientela};
	\item La \textbf{proposta di valore};
	\item I \textbf{canali};
	\item Le \textbf{relazioni coi clienti};
	\item I \textbf{flussi di ricavi}.
\end{itemize}

\subsubsection{Segmenti di clientela}
Prima di tutto potremmo chiederci per \textit{chi} stiamo creando valore.
Possiamo individuare alcune macrocategorie:
\begin{itemize}
	\item Mercato di massa;
	\item Mercato di nicchia;
	\item Mercato segmentato;
	\item Mercato diversificato;
	\item Piattaforme (o mercati) multi-sided.
\end{itemize}

\subsubsection{Proposta di valore}
La proposta di valore è ciò che l'azienda offre in \textit{più} rispetto a un altra, quindi il motivo per cui i potenziali clienti dovrebbero acquistarne i prodotti.

\subsubsection{Canali}
I canali sono quelli attraverso i quali l'azienda fornisce i suoi prodotti/servizi al cliente.
Possiamo individuare ad esempio:
\begin{itemize}
	\item Forza di vendita;
	\item Vendita su web;
	\item Negozi propri;
	\item Negozi di partner;
	\item Grossisti.
\end{itemize}

\subsubsection{Relazioni coi clienti}
Le relazioni coi clienti riguardano:
\begin{itemize}
	\item L'\textit{acquisizione} di clienti;
	\item La \textit{fidelizzazione} di clienti;
	\item L'\textit{upselling}, cioè l'aumento della frequenza di acquisto dei clienti.
\end{itemize}

Possiamo individuare diversi meccanismi:
\begin{itemize}
	\item Assistenza personale;
	\item Assistenza personale dedicata;
	\item Self-service;
	\item Servizi automatici;
	\item Community;
	\item Co-creazione.
\end{itemize}

\subsubsection{Flussi di ricavi}
Un modello di business può prevedere due tipi di ricavi:
\begin{itemize}
	\item Ricavi correnti;
	\item Ricavi continui.
\end{itemize}

Questi possono essere:
\begin{itemize}
	\item Vendita di beni;
	\item Canone d'uso;
	\item Quote di iscrizione;
	\item Prestito/noleggio/leasing;
	\item Licenza;
	\item Commissioni di intermediazione;
	\item Pubblicità.
\end{itemize}
\end{document}
