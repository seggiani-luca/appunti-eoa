
\documentclass[a4paper,11pt]{article}
\usepackage[a4paper, margin=8em]{geometry}

% usa i pacchetti per la scrittura in italiano
\usepackage[french,italian]{babel}
\usepackage[T1]{fontenc}
\usepackage[utf8]{inputenc}
\frenchspacing 

% usa i pacchetti per la formattazione matematica
\usepackage{amsmath, amssymb, amsthm, amsfonts}

% usa altri pacchetti
\usepackage{gensymb}
\usepackage{hyperref}
\usepackage{standalone}

% imposta il titolo
\title{Appunti Economia ed Organizzazione Aziendale}
\author{Luca Seggiani}
\date{2025}

% disegni
\usepackage{pgfplots}
\pgfplotsset{width=10cm,compat=1.9}

% imposta lo stile
% usa helvetica
\usepackage[scaled]{helvet}
% usa palatino
\usepackage{palatino}
% usa un font monospazio guardabile
\usepackage{lmodern}

\renewcommand{\rmdefault}{ppl}
\renewcommand{\sfdefault}{phv}
\renewcommand{\ttdefault}{lmtt}

% disponi il titolo
\makeatletter
\renewcommand{\maketitle} {
	\begin{center} 
		\begin{minipage}[t]{.8\textwidth}
			\textsf{\huge\bfseries \@title} 
		\end{minipage}%
		\begin{minipage}[t]{.2\textwidth}
			\raggedleft \vspace{-1.65em}
			\textsf{\small \@author} \vfill
			\textsf{\small \@date}
		\end{minipage}
		\par
	\end{center}

	\thispagestyle{empty}
	\pagestyle{fancy}
}
\makeatother

% disponi teoremi
\usepackage{tcolorbox}
\newtcolorbox[auto counter, number within=section]{theorem}[2][]{%
	colback=blue!10, 
	colframe=blue!40!black, 
	sharp corners=northwest,
	fonttitle=\sffamily\bfseries, 
	title=Teorema~\thetcbcounter: #2, 
	#1
}

% disponi definizioni
\newtcolorbox[auto counter, number within=section]{definition}[2][]{%
	colback=red!10,
	colframe=red!40!black,
	sharp corners=northwest,
	fonttitle=\sffamily\bfseries,
	title=Definizione~\thetcbcounter: #2,
	#1
}

% disponi problemi
\newtcolorbox[auto counter, number within=section]{problem}[2][]{%
	colback=green!10,
	colframe=green!40!black,
	sharp corners=northwest,
	fonttitle=\sffamily\bfseries,
	title=Problema~\thetcbcounter: #2,
	#1
}

% disponi codice
\usepackage{listings}
\usepackage[table]{xcolor}

\lstdefinestyle{codestyle}{
		backgroundcolor=\color{black!5}, 
		commentstyle=\color{codegreen},
		keywordstyle=\bfseries\color{magenta},
		numberstyle=\sffamily\tiny\color{black!60},
		stringstyle=\color{green!50!black},
		basicstyle=\ttfamily\footnotesize,
		breakatwhitespace=false,         
		breaklines=true,                 
		captionpos=b,                    
		keepspaces=true,                 
		numbers=left,                    
		numbersep=5pt,                  
		showspaces=false,                
		showstringspaces=false,
		showtabs=false,                  
		tabsize=2
}

\lstdefinestyle{shellstyle}{
		backgroundcolor=\color{black!5}, 
		basicstyle=\ttfamily\footnotesize\color{black}, 
		commentstyle=\color{black}, 
		keywordstyle=\color{black},
		numberstyle=\color{black!5},
		stringstyle=\color{black}, 
		showspaces=false,
		showstringspaces=false, 
		showtabs=false, 
		tabsize=2, 
		numbers=none, 
		breaklines=true
}

\lstdefinelanguage{javascript}{
	keywords={typeof, new, true, false, catch, function, return, null, catch, switch, var, if, in, while, do, else, case, break},
	keywordstyle=\color{blue}\bfseries,
	ndkeywords={class, export, boolean, throw, implements, import, this},
	ndkeywordstyle=\color{darkgray}\bfseries,
	identifierstyle=\color{black},
	sensitive=false,
	comment=[l]{//},
	morecomment=[s]{/*}{*/},
	commentstyle=\color{purple}\ttfamily,
	stringstyle=\color{red}\ttfamily,
	morestring=[b]',
	morestring=[b]"
}

% disponi sezioni
\usepackage{titlesec}

\titleformat{\section}
	{\sffamily\Large\bfseries} 
	{\thesection}{1em}{} 
\titleformat{\subsection}
	{\sffamily\large\bfseries}   
	{\thesubsection}{1em}{} 
\titleformat{\subsubsection}
	{\sffamily\normalsize\bfseries} 
	{\thesubsubsection}{1em}{}

% disponi alberi
\usepackage{forest}

\forestset{
	rectstyle/.style={
		for tree={rectangle,draw,font=\large\sffamily}
	},
	roundstyle/.style={
		for tree={circle,draw,font=\large}
	}
}

% disponi algoritmi
\usepackage{algorithm}
\usepackage{algorithmic}
\makeatletter
\renewcommand{\ALG@name}{Algoritmo}
\makeatother

% disponi numeri di pagina
\usepackage{fancyhdr}
\fancyhf{} 
\fancyfoot[L]{\sffamily{\thepage}}

\makeatletter
\fancyhead[L]{\raisebox{1ex}[0pt][0pt]{\sffamily{\@title \ \@date}}} 
\fancyhead[R]{\raisebox{1ex}[0pt][0pt]{\sffamily{\@author}}}
\makeatother

\begin{document}

% sezione (data)
\section{Lezione del 03-04-25}

% stili pagina
\thispagestyle{empty}
\pagestyle{fancy}

% testo
\subsection{Bilancio}
Il \textbf{bilancio} è uno schema diviso in due sezioni:
\begin{itemize}
	\item \textbf{Stato patrimoniale:} riguarda la differenza fra \textit{dare} e \textit{avere} dell'azienda, in relazione al cosiddetto \textbf{capitale netto}.
		Riguarda quindi lo stato del patrimonio dell azienda in un dato momento temporale, tenuto conto dei beni, crediti, ecc... (\textit{avere}) che essa possiede, e dei debiti, obbligazioni, ecc... (\textit{dare}) che essa deve a terzi;
	\item \textbf{Conto economico}: riguarda la sola prestazione economica, quindi la differenza fra ricavi e spese, in un dato periodo in relazione all'attività economica dell'azienda stessa.
\end{itemize}
Il bilancio è corredato da un documento testuale, detto \textbf{nota integrativa}, che specifica informazioni riguardo ai vari campi dello stato patrimoniale e del conto economico.

Dal bilancio ci interessa valutare lo \textbf{stato di salute} dell'azienda, che a priori si può dire essere legato a diversi indici, fra cui:
\begin{itemize}
	\item Stato della cassa;
	\item Crediti e debiti;
	\item Il fatturato;
	\item Il valore di mercato delle azioni; 
\end{itemize}

\subsection{Struttura di un bilancio}
Un bilancio è solitamente sviluppato su due colonne, la colonna di \textit{sinistra} e la colonna di \textit{destra}, che contengono informazioni di natura diversa.
Ad esempio, vedremo che nello stato patrimoniale la colonna sinistra è l'\textit{avere}, e la colonna di destra è il \textit{dare}, con aggiunto il capitale netto, in modo che il bilancio \textit{"quadri"}, cioè la somma algebrica delle due colonne sia uguale.

\begin{table}[h!]
	\center \rowcolors{2}{white}{black!10}
	\begin{tabular} { p{4cm} c | p{4cm} c }
		\bfseries Colonna sinistra &  & \bfseries Colonna destra & \\	
		\hline 
		Entrata 1 sinistra & 100.000 & Entrata 1 destra & 50.000 \\ 
		\vdots & & \vdots & \\
		\bfseries Totale sinstra & 150.000 & \bfseries Totale destra & 150.000 
	\end{tabular}
\end{table}

\subsubsection{Misurazione dello stato di salute}
Vediamo quindi nel dettaglio i \textbf{modelli} per lo stato di salute, fra cui distinguiamo:
\begin{itemize}
	\item Il \textbf{modello contabile}, che rappresenta una \textit{retrospettiva} (quindi dati prevalentemente \textit{storici}).
		Tiene quindi conto della \textbf{capitalizzazione} dei flussi di cassa futuri.
		
		Il modello contabile considera intervalli di tempo pari a 12 mesi, detti \textbf{anni contabili}, che non corrispondono necessariamente agli anni reali.
	\item Il \textbf{modello del valore}, derivante dalla \textit{finanza}, che rappresenta invece una \textit{prospettiva} (quindi è composto anche da \textit{stime}).
		Tiene quindi conto dell'\textbf{attualizzazione} dei flussi di cassa futuri.
\end{itemize}

\subsubsection{Evoluzione dell'azienda}
Possiamo individuare diverse fasi temporali, cioè:
\begin{itemize}
	\item \textbf{Fase di impianto:} dove si effettuano le operazioni di acquisto dei fattori pluriennalie e le operazioni di finanziamento;
	\item \textbf{Fase di regime:} dove si svolge il ciclo di acquisto-produzione-vendita;
	\item \textbf{Fase di cessazione:} dove si effettuano operazioni \textit{straordinarie}, cioè fusione, fallimento, cessione o trasformazione. 
\end{itemize}
Le prime 2 fasi vengono dette \textbf{fasi di funzionamento}, in quanto vi individuiamo le \textbf{ipotesi di funzionamento}.
Durante la fase di funzionamento, si redime quello che è il \textbf{bilancio di esercizio}, o \textit{bilancio ordinario}.

Nella fase di cessazione, invece, si redige il \textbf{bilancio straordinario}.

\subsection{Modello contabile}
Il \textbf{modello contabile}, che è quello secondo il quale è redatto lo \textit{stato patrimoniale}, usa 2 misuratori:
\begin{itemize}
	\item Il \textbf{reddito}, che è un misuratore di flusso;
	\item Il \textbf{capitale}, che è un misuratore di stock.
\end{itemize}
A seconda della fase in cui l'impresa si trova, si hanno poi degli aggettivi che si aggiungono a tali misuratori.
In particolare, nella fase di esercizio si ha:
\begin{itemize}
	\item Il \textbf{reddito di esercizio};
	\item Il \textbf{capitale di funzionamento}.
\end{itemize}
Nella fase di cessazione, invece, e quindi sul bilancio straordinario, si ha il \textbf{capitale di cessione}. 

Il modello contabile viene tenuto nello \textbf{stato patrimoniale}. 

\subsubsection{Reddito}
Dal punto di vista matematico, preso un certo capitale $c_1$ ad un istante $t_1$, ed un capitale $c_2$ ad un istante $t_2$, possiamo definire la variazione di capitale $\Delta c = c_2 - c_1$ come reddito.
Un reddito $\Delta c > 0$ è detto \textbf{utile} di esercizio, mentre un reddito $\Delta c < 0$ è detta \textbf{perdita}.

I flussi in entrata nel capitale $c$ vengono detti \textbf{ricavi} di esercizio, mentre i flussi in uscita \textbf{costi} di esercizi.
Il reddito potrà quindi essere difinito, date $r_i$ e $k_i$ ricavi e costi mensili, come la sommatoria sul periodo di 12 mesi:
$$
\Delta c = \sum_{i = 1}^{12} \left( r_i - k_i \right)
$$

\subsubsection{Capitale}
Veniamo quindi a definire il capitale vero e proprio. 
Questo rappresenta l'insieme dei beni che possono essere impiegati nell'attivitò di impresa, ed è costituito da \textbf{attivi} e \textbf{passivi}  
La sommma algebrica attività e passività dà il cosiddetto \textbf{capitale netto}:
$$
A - P = \text{capitale netto}
$$
Nello stato patrimoniale, il capitale netto si mette a \textbf{destra}, assieme ai passivi, mentre gli attivi si mettono a \textbf{sinistra}, così che la somma delle due colonne sia identica secondo l'identità:
$$
A = P + \text{capitale netto}
$$
In questo, il capitale netto è un valore astratto espresso in moneta di conto, che rappresenta la differenza fra attivi e passivi.

\subsubsection{Valori finanziari}
Potremmo pensare di tenere un bilancio, più semplice, della sola \textbf{cassa}, dei debiti e dei crediti, con le stesse convenzioni sinistra-destra (l'unica differenza è che debiti in \textit{discesa} stanno a \textbf{sinistra}).
In particolare, lo schema riferito a un oggetto viene detto \textbf{mastrino}, il mastrino asociato ad un oggetto e quindi una regola contabile viene detto \textbf{conto}, e un insieme di conti viene detto \textbf{sistema contabile}.

Cassa, banca, crediti, debiti ecc... sono dal punto di vista contabile valori \textbf{finanziari}.
Si può quindi tenere un conto del valore finanziario, dove a sinistra stanno le variazioni in positivo e a destra le variazioni in negativo.
I conti che abbiamo riportato prima di cassa, debiti e crediti farebbero parte di queste variazioni.
Si avrannno quindi due variazioni dei valori finanziari, in positivo e in negagativo, cioè $\Delta F+$ e $\Delta F-$.

\subsubsection{Valori economici}
Infine consideriamo i valori economici.
Questi riguardano capitale e reddito:
\begin{itemize}
	\item Riguardo al \textit{capitale netto}, la \textbf{variazione} di capitale;
	\item Riguardo al \textit{reddito}, i \textbf{ricavi} e i \textbf{costi}.
\end{itemize}

I valori economici vengono tenuti nel \textbf{conto economico}, che è strutturato al contrario: il capitale netto (capitale sociale), assieme ai valori finanziari e ai ricavi, stanno a \textbf{destra}, mentre i costi stanno a \textbf{sinistra}.

\par\medskip

Abbiamo quindi che valori finanziari ed economici sono legati: se il bilancio dei valori finanziari non corrisponde, significa che c'è stata qualche azione da parte dei valori economici (perdite o utili), e quindi una variazione del capitale netto.

\end{document}
