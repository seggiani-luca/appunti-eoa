
\documentclass[a4paper,11pt]{article}
\usepackage[a4paper, margin=8em]{geometry}

% usa i pacchetti per la scrittura in italiano
\usepackage[french,italian]{babel}
\usepackage[T1]{fontenc}
\usepackage[utf8]{inputenc}
\frenchspacing 

% usa i pacchetti per la formattazione matematica
\usepackage{amsmath, amssymb, amsthm, amsfonts}

% usa altri pacchetti
\usepackage{gensymb}
\usepackage{hyperref}
\usepackage{standalone}

% imposta il titolo
\title{Appunti Economia ed Organizzazione Aziendale}
\author{Luca Seggiani}
\date{2025}

% disegni
\usepackage{pgfplots}
\pgfplotsset{width=10cm,compat=1.9}

% imposta lo stile
% usa helvetica
\usepackage[scaled]{helvet}
% usa palatino
\usepackage{palatino}
% usa un font monospazio guardabile
\usepackage{lmodern}

\renewcommand{\rmdefault}{ppl}
\renewcommand{\sfdefault}{phv}
\renewcommand{\ttdefault}{lmtt}

% disponi il titolo
\makeatletter
\renewcommand{\maketitle} {
	\begin{center} 
		\begin{minipage}[t]{.8\textwidth}
			\textsf{\huge\bfseries \@title} 
		\end{minipage}%
		\begin{minipage}[t]{.2\textwidth}
			\raggedleft \vspace{-1.65em}
			\textsf{\small \@author} \vfill
			\textsf{\small \@date}
		\end{minipage}
		\par
	\end{center}

	\thispagestyle{empty}
	\pagestyle{fancy}
}
\makeatother

% disponi teoremi
\usepackage{tcolorbox}
\newtcolorbox[auto counter, number within=section]{theorem}[2][]{%
	colback=blue!10, 
	colframe=blue!40!black, 
	sharp corners=northwest,
	fonttitle=\sffamily\bfseries, 
	title=Teorema~\thetcbcounter: #2, 
	#1
}

% disponi definizioni
\newtcolorbox[auto counter, number within=section]{definition}[2][]{%
	colback=red!10,
	colframe=red!40!black,
	sharp corners=northwest,
	fonttitle=\sffamily\bfseries,
	title=Definizione~\thetcbcounter: #2,
	#1
}

% disponi problemi
\newtcolorbox[auto counter, number within=section]{problem}[2][]{%
	colback=green!10,
	colframe=green!40!black,
	sharp corners=northwest,
	fonttitle=\sffamily\bfseries,
	title=Problema~\thetcbcounter: #2,
	#1
}

% disponi codice
\usepackage{listings}
\usepackage[table]{xcolor}

\lstdefinestyle{codestyle}{
		backgroundcolor=\color{black!5}, 
		commentstyle=\color{codegreen},
		keywordstyle=\bfseries\color{magenta},
		numberstyle=\sffamily\tiny\color{black!60},
		stringstyle=\color{green!50!black},
		basicstyle=\ttfamily\footnotesize,
		breakatwhitespace=false,         
		breaklines=true,                 
		captionpos=b,                    
		keepspaces=true,                 
		numbers=left,                    
		numbersep=5pt,                  
		showspaces=false,                
		showstringspaces=false,
		showtabs=false,                  
		tabsize=2
}

\lstdefinestyle{shellstyle}{
		backgroundcolor=\color{black!5}, 
		basicstyle=\ttfamily\footnotesize\color{black}, 
		commentstyle=\color{black}, 
		keywordstyle=\color{black},
		numberstyle=\color{black!5},
		stringstyle=\color{black}, 
		showspaces=false,
		showstringspaces=false, 
		showtabs=false, 
		tabsize=2, 
		numbers=none, 
		breaklines=true
}

\lstdefinelanguage{javascript}{
	keywords={typeof, new, true, false, catch, function, return, null, catch, switch, var, if, in, while, do, else, case, break},
	keywordstyle=\color{blue}\bfseries,
	ndkeywords={class, export, boolean, throw, implements, import, this},
	ndkeywordstyle=\color{darkgray}\bfseries,
	identifierstyle=\color{black},
	sensitive=false,
	comment=[l]{//},
	morecomment=[s]{/*}{*/},
	commentstyle=\color{purple}\ttfamily,
	stringstyle=\color{red}\ttfamily,
	morestring=[b]',
	morestring=[b]"
}

% disponi sezioni
\usepackage{titlesec}

\titleformat{\section}
	{\sffamily\Large\bfseries} 
	{\thesection}{1em}{} 
\titleformat{\subsection}
	{\sffamily\large\bfseries}   
	{\thesubsection}{1em}{} 
\titleformat{\subsubsection}
	{\sffamily\normalsize\bfseries} 
	{\thesubsubsection}{1em}{}

% disponi alberi
\usepackage{forest}

\forestset{
	rectstyle/.style={
		for tree={rectangle,draw,font=\large\sffamily}
	},
	roundstyle/.style={
		for tree={circle,draw,font=\large}
	}
}

% disponi algoritmi
\usepackage{algorithm}
\usepackage{algorithmic}
\makeatletter
\renewcommand{\ALG@name}{Algoritmo}
\makeatother

% disponi numeri di pagina
\usepackage{fancyhdr}
\fancyhf{} 
\fancyfoot[L]{\sffamily{\thepage}}

\makeatletter
\fancyhead[L]{\raisebox{1ex}[0pt][0pt]{\sffamily{\@title \ \@date}}} 
\fancyhead[R]{\raisebox{1ex}[0pt][0pt]{\sffamily{\@author}}}
\makeatother

\begin{document}

% sezione (data)
\section{Lezione del 13-03-25}

% stili pagina
\thispagestyle{empty}
\pagestyle{fancy}

% testo
Riprendiamo la trattazione dei tipi di società.

\subsubsection{Responsabilità}
E' importante distinguere fra i diversi tipi di responsabilità che possono avere i soci nei diversi tipi di società:
\begin{itemize}
	\item Responsabilità \textbf{illimitata}, tipica delle società di persone, rappresenta responsabilità piena per le obbligazioni sociali, spesso anche \textit{solidale} (obbligazioni pagate in solido da tutti i soci);
	\item Responsabilità \textbf{limitata}, tipica delle società di capitali, rappresenta responsabilità nelle obbligazioni aziendali fino alla quota conferita dal socio.
\end{itemize}

Le società "ibride" (S.a.s. e S.a.p.a.) hanno delle leggere divergenze.
In questo tipo di società esistono due tipi di soci, gli \textbf{accomandatari}, che hanno responsabilità illimitata, e gli \textbf{accomandanti}, che hanno responsabilità limitata alla quota conferita.

La responsabilità di un socio è spesso legata alla sua qualità di \textbf{amministratore}.
Nelle S.n.c. tutti i soci sono illimitatamente responsabili allo stesso modo, e considerati amministratori. 
Nelle S.a.s., chiaramente, i soci accomandatari saranno considerati amministratori, mentre i soci accomandanti no.
Una divisione simile, fra soci amministratori e non, può trovarsi anche nella S.s.

\subsubsection{Organizzazione corporativa}
Nelle società cooperative e nelle società di capitali deve esistere un organizzazione di tipo \textbf{corporativa}, costituita da (ma non limitata a):
\begin{itemize}
	\item L'assemblea dei soci;
	\item Consiglio di amministrazione;
	\item Collegio sindacale.
\end{itemize}

Questo è vero per S.p.A. e S.a.p.a., mentre nelle S.r.l. non è obbligatorio disporre di un collegio sindacale.

Il funzionamento di questi organi è dominato dal principio maggioritario, e il oscio, in quanto tale, non ha alcun potere diretto di amministrazione e controllo.

Il \textbf{management} si distingue dall'amministrazione in quanto è effettivamente formato da \textit{dipendenti} dei soci.
Questo significa che management e amministrazione non spesso condividono le stesse intenzioni (la faccenda si complica oltre se si pensa alle diverse impostazioni delle società, ad esempio si consideri la differenza fra una società per azioni e una società di famiglia rispetto agli interessi dei soci stessi).

\par\smallskip

Nelle società di persone non è invece prevista alcuna organizzazione di tipo corporativo, in quanto abbiamo detto i soci vengono considerati amministratori.

\par\medskip

Iniziamo quindi a vedere nel dettaglio le società che ci interesseranno, quindi le società di capitali.

\subsection{Società per azioni}
La società per azioni è una società di capitali dove i soci vengono rappresentati dagli azionisti (le quote di partecipazione sono azioni).
In quanto società di capitali, alle obbligazioni risponde solo la società con il suo capitale (\textit{personalità giuridica}), e viceversa la società non risponde alle obbligazioni personali dei soci.
E' obbligatoria l'organizzazione cooperativa (in quanto i soci non sono amministratori), ed è previsto un capitale sociale minimo di 50.000 €.

\subsubsection{Costituzione}
Per la costituzione di una S.p.A. servono 2 documenti:
\begin{itemize}
	\item \textbf{Atto costitutivo:} il contratto in cui i soci manifestano la volontà di dare vita al rapporto sociale;
	\item \textbf{Statuto:} riporta le norme di funzionamento della società.
\end{itemize}

Esistono due modalità per la stipulazione dell'atto costitutivo:
\begin{itemize}
	\item \textbf{Simultanea:} tutte le parti costituiscono la società di fronte ad un notaio;
	\item Per \textbf{pubblica sottoscrizione:} la società viene costituita al termine della raccolta iscrizioni.
\end{itemize}

L'atto costitutivo dovrà essere depositato poi entro 20 giorni al registro delle imprese, con iscrizione.

Le condizioni per la costituzione sono:
\begin{itemize}
	\item La sottoscrizione per intero del capitale sociale;
	\item Versamento di almeno il 25\% dei conferimenti in denaro presso un istituto di credito.
\end{itemize}

\subsubsection{Organi sociali}
Vediamo quindi quali sono gli organi sociali della S.p.A.: 
\begin{itemize}
	\item
		\textbf{Assemblea dei soci:} i soci si costituiscono in \textbf{assemblea}, che può essere \textit{ordinaria} o \textit{straordinaria} (ad esempio, è straordinaria l'assemblea che si costituisce in fase di liquidazione, ma non lo è quella che si costituisce in fase di cessione).

I soci dell'assemblea, che si convoca nella sede legale, con presenza del presidente (solitamente il presidente della società), non è considerata valida alla prima convocazione se si presenta il meno del 50\% (al minimo, può essere alzato da statuto) della quota sociale (si guarda alle \textit{quote} e non alle \textit{teste}).

\item \textbf{Consiglio di amministrazione:} gli \textbf{amministratori} possono essere soci o non soci, può esserci un amministratore unico o una loro pluralità (a formare il consiglio di amministrazione), e possono esistere uno o più organi delegati nel consiglio di amministrazione nel caso questo esista, fra cui:
		\begin{itemize}
			\item Comitato esecutivo;
			\item Amministratori delegati.
		\end{itemize}
		L'amministrazione si occupa di gestione ordinaria (la gestione \textit{straordinaria} spetta ai soci).

Nelle società di capitali l'amministrazione, come abbiamo visto, può essere anche affidata ai non soci.
Notiamo però la (forse già riportata) dicotomia fra il comportamento dei soci e il comportamento degli amministratori, se questi sono esterni: se il tornaconto degli amministratori è ad esempio quello di ricavare un vantaggio dalle azioni, o comunque dalla performance sul breve termine della società, questi implementeranno allora politiche che massimizzano il profitto sul breve termine ma magari hanno delle conseguenze negative sul lungo termine.
Di contro i soci vorranno, nell'interesse di mantenere il controllo della società per tempi più lunghi, implementare politiche più conservative, o comunque meno rischiose sul breve termine e atte alla conservazione sul lungo termine.
Questo conflitto è oggi presente in maniera anche abbastanza estesa in molte grandi società.

L'\textbf{amministratore delegato} è colui che viene nominato dal consiglio per detenere poteri decisionali più importanti, e in qualche modo rappresentare il volere del consiglio di amministrazione stesso.
In questo si distingue dall'eventuale \textit{amministratore unico}.

\item \textbf{Collegio sindacale:}
 il collegio sindacale è costuito da membri sia soci che non soci.
Fa da organo di vigilanza, cioè vigila sull'assetto organizzativo, amministrativo e contabile della società, e eventualmente subentra in ruolo amministrativo (in caso di mancanze da parte degli altri organi). 
\end{itemize}

\end{document}
