
\documentclass[a4paper,11pt]{article}
\usepackage[a4paper, margin=8em]{geometry}

% usa i pacchetti per la scrittura in italiano
\usepackage[french,italian]{babel}
\usepackage[T1]{fontenc}
\usepackage[utf8]{inputenc}
\frenchspacing 

% usa i pacchetti per la formattazione matematica
\usepackage{amsmath, amssymb, amsthm, amsfonts}

% usa altri pacchetti
\usepackage{gensymb}
\usepackage{hyperref}
\usepackage{standalone}

% imposta il titolo
\title{Appunti Economia ed Organizzazione Aziendale}
\author{Luca Seggiani}
\date{2025}

% disegni
\usepackage{pgfplots}
\pgfplotsset{width=10cm,compat=1.9}

% imposta lo stile
% usa helvetica
\usepackage[scaled]{helvet}
% usa palatino
\usepackage{palatino}
% usa un font monospazio guardabile
\usepackage{lmodern}

\renewcommand{\rmdefault}{ppl}
\renewcommand{\sfdefault}{phv}
\renewcommand{\ttdefault}{lmtt}

% disponi il titolo
\makeatletter
\renewcommand{\maketitle} {
	\begin{center} 
		\begin{minipage}[t]{.8\textwidth}
			\textsf{\huge\bfseries \@title} 
		\end{minipage}%
		\begin{minipage}[t]{.2\textwidth}
			\raggedleft \vspace{-1.65em}
			\textsf{\small \@author} \vfill
			\textsf{\small \@date}
		\end{minipage}
		\par
	\end{center}

	\thispagestyle{empty}
	\pagestyle{fancy}
}
\makeatother

% disponi teoremi
\usepackage{tcolorbox}
\newtcolorbox[auto counter, number within=section]{theorem}[2][]{%
	colback=blue!10, 
	colframe=blue!40!black, 
	sharp corners=northwest,
	fonttitle=\sffamily\bfseries, 
	title=Teorema~\thetcbcounter: #2, 
	#1
}

% disponi definizioni
\newtcolorbox[auto counter, number within=section]{definition}[2][]{%
	colback=red!10,
	colframe=red!40!black,
	sharp corners=northwest,
	fonttitle=\sffamily\bfseries,
	title=Definizione~\thetcbcounter: #2,
	#1
}

% disponi problemi
\newtcolorbox[auto counter, number within=section]{problem}[2][]{%
	colback=green!10,
	colframe=green!40!black,
	sharp corners=northwest,
	fonttitle=\sffamily\bfseries,
	title=Problema~\thetcbcounter: #2,
	#1
}

% disponi codice
\usepackage{listings}
\usepackage[table]{xcolor}

\lstdefinestyle{codestyle}{
		backgroundcolor=\color{black!5}, 
		commentstyle=\color{codegreen},
		keywordstyle=\bfseries\color{magenta},
		numberstyle=\sffamily\tiny\color{black!60},
		stringstyle=\color{green!50!black},
		basicstyle=\ttfamily\footnotesize,
		breakatwhitespace=false,         
		breaklines=true,                 
		captionpos=b,                    
		keepspaces=true,                 
		numbers=left,                    
		numbersep=5pt,                  
		showspaces=false,                
		showstringspaces=false,
		showtabs=false,                  
		tabsize=2
}

\lstdefinestyle{shellstyle}{
		backgroundcolor=\color{black!5}, 
		basicstyle=\ttfamily\footnotesize\color{black}, 
		commentstyle=\color{black}, 
		keywordstyle=\color{black},
		numberstyle=\color{black!5},
		stringstyle=\color{black}, 
		showspaces=false,
		showstringspaces=false, 
		showtabs=false, 
		tabsize=2, 
		numbers=none, 
		breaklines=true
}

\lstdefinelanguage{javascript}{
	keywords={typeof, new, true, false, catch, function, return, null, catch, switch, var, if, in, while, do, else, case, break},
	keywordstyle=\color{blue}\bfseries,
	ndkeywords={class, export, boolean, throw, implements, import, this},
	ndkeywordstyle=\color{darkgray}\bfseries,
	identifierstyle=\color{black},
	sensitive=false,
	comment=[l]{//},
	morecomment=[s]{/*}{*/},
	commentstyle=\color{purple}\ttfamily,
	stringstyle=\color{red}\ttfamily,
	morestring=[b]',
	morestring=[b]"
}

% disponi sezioni
\usepackage{titlesec}

\titleformat{\section}
	{\sffamily\Large\bfseries} 
	{\thesection}{1em}{} 
\titleformat{\subsection}
	{\sffamily\large\bfseries}   
	{\thesubsection}{1em}{} 
\titleformat{\subsubsection}
	{\sffamily\normalsize\bfseries} 
	{\thesubsubsection}{1em}{}

% disponi alberi
\usepackage{forest}

\forestset{
	rectstyle/.style={
		for tree={rectangle,draw,font=\large\sffamily}
	},
	roundstyle/.style={
		for tree={circle,draw,font=\large}
	}
}

% disponi algoritmi
\usepackage{algorithm}
\usepackage{algorithmic}
\makeatletter
\renewcommand{\ALG@name}{Algoritmo}
\makeatother

% disponi numeri di pagina
\usepackage{fancyhdr}
\fancyhf{} 
\fancyfoot[L]{\sffamily{\thepage}}

\makeatletter
\fancyhead[L]{\raisebox{1ex}[0pt][0pt]{\sffamily{\@title \ \@date}}} 
\fancyhead[R]{\raisebox{1ex}[0pt][0pt]{\sffamily{\@author}}}
\makeatother

\begin{document}

% sezione (data)
\section{Lezione del 06-03-25}

% stili pagina
\thispagestyle{empty}
\pagestyle{fancy}

% testo
\subsection{Diritto delle società}
Il \textbf{diritto} è un termine usato con due accezioni differenti:
\begin{itemize}
	\item La prima accezione intende il diritto come il complesso delle \textbf{norme giuridiche} che regolano la vita dei membri di una comunità, quindi l'\textbf{ordinamento giuridico}. In questo viene detto anche \textbf{diritto oggettivo};
	\item La seconda accezione intende il diritto come la facoltà (o il potere) garantito dall'\textit{ordinamento giuridico} ad un soggetto. In questo viene detto anche \textbf{diritto soggettivo}.
\end{itemize}

L'ordinamento giuridico si divide ulteriormente in:
\begin{itemize}
	\item \textbf{Diritto privato:} è l'insieme delle norme giuridiche che regolano i rapporti fra i \textit{privati}: questi possono essere rappresentati da \textit{atti personali} (\textbf{diritto civile}) o \textit{atti di commercio} (\textbf{diritto commerciale});
	\item \textbf{Diritto pubblico:} ha per oggetto il funzionamento dello stato e degli \textit{enti territoriali}, e i rapporti che questi hanno con i cittadini.
\end{itemize}

Il diritto commerciale si divide poi ulteriormente in innumerevoli categorie, fra cui:
\begin{itemize}
	\item Diritto societario;
	\item Diritto fallimentare;
	\item Antitrust;
	\item Diritto del lavoro;
	\item ecc...
\end{itemize}

A noi in particolare sarà di interesse il diritto oggettivo, privato,  commerciale e societario.

\subsubsection{Norme giuridiche}
Una norma giuridica rappresenta una regola di condotta, o comunque un precetto che stabilisce un comportamento condivisio da una comunità.
Le norme sono contenute solitamente in testi, detti \textbf{testi normativi}.
I testi normativi rappresentano le \textbf{fonti del diritto}.
L'ordine di importanza delle fonti del diritto è determinato dalla \textbf{gerarchia delle fonti}:
\begin{itemize}
	\item Norme di \textbf{primo livello}: la costituzione e le leggi costituzionali, i regolamenti comunitari;
	\item Norme di \textbf{secondo livello}: le leggi dello stato (fatte dal \textit{parlamento}), i decreti legge (fatti dal \textit{governo} e temporanei) e i decreti legislativi (fatti sempre dal \textit{governo}, ma su delega del parlamento), il referendum abrogativo;
	\item Norme di \textbf{terzo livello}: regolamenti governativi e degli enti locali, usi e consuetudini.
\end{itemize}

\subsubsection{Giurisprudenza}
La \textbf{giurisprudenza} è la disciplina che studia il diritto e la sua \textit{interpretazione}.
La \textbf{dottrina giuridica} è invece l'attività di studio scientifico del diritto.

Gli organi giudicanti dello stato (la magistratura) interpreta e applica le leggi, processo da cui risultano le \textbf{sentenze}.
Esistono in questo 2 sistemi:
\begin{itemize}
	\item \textbf{Common law:} le sentenze dei tribunali fanno \textit{precedente} e vengono prese ad esempio;
	\item \textbf{Civil law:} si basa invece principalmente su codici prestabiliti e non sentenze precedenti.
\end{itemize}

Il \textbf{diritto societario} è e definito con leggi e raccolte di leggi (detto \textbf{codice civile}).
Il codice civile distingue diversi tipi di imprese in base a 3 criteri:
\begin{itemize}
	\item \textbf{Oggetto} dell'impresa (imprenditore agricolo, imprenditore commerciale, ecc...);
	\item \textbf{Dimesnione} dell'impresa (piccolo imprenditore, grande imprenditore, ecc...);
	\item \textbf{Natura del soggetto} che esercita l'impresa (impesa individuale, collettiva, ecc...).
\end{itemize}

Non si trova quindi nel codice civile una definizione propria di \textit{impresa}, ma di \textbf{imprenditore}: è imprenditore chi esercita \textbf{professionalmente} un'\textbf{attività economica organizzata} al fine della \textbf{produzione} e dello \textbf{scambio} di \textbf{beni} o \textbf{servizio}.

Dove:
\begin{itemize}
	\item \textbf{Organizzata} rappresenta l'impiego coordinato di fattori produttivi propri o altrui;
	\item L'\textbf{economicità} riguarda il metodo con cui l'attività è svolta;
	\item E \textbf{professionalità} rappresenta l'esercizio abituale dell'attivita produttiva.
\end{itemize}

Questa definizione è quella del codice civile, e differisce quindi da quella che si può trovare nel diritto tributario, ecc... 

Si trova poi una definizione di \textbf{azienda}, cioè l'insieme dei eni organizzati dall'imprenditore perl l'esercizio dell'impresa.

I requisiti \textit{civilistici} dell'imprenditore sono quindi quelli dello svolgimento di un \textbf{attività produttiva}, che come abbiamo visto è alla base dell'impresa.

\end{document}
