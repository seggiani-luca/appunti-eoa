
\documentclass[a4paper,11pt]{article}
\usepackage[a4paper, margin=8em]{geometry}

% usa i pacchetti per la scrittura in italiano
\usepackage[french,italian]{babel}
\usepackage[T1]{fontenc}
\usepackage[utf8]{inputenc}
\frenchspacing 

% usa i pacchetti per la formattazione matematica
\usepackage{amsmath, amssymb, amsthm, amsfonts}

% usa altri pacchetti
\usepackage{gensymb}
\usepackage{hyperref}
\usepackage{standalone}

% cose fluttuanti
\usepackage{float}

% imposta il titolo
\title{Appunti Economia ed Organizzazione Aziendale}
\author{Luca Seggiani}
\date{2025}

% disegni
\usepackage{pgfplots}
\pgfplotsset{width=10cm,compat=1.9}

% imposta lo stile
% usa helvetica
\usepackage[scaled]{helvet}
% usa palatino
\usepackage{palatino}
% usa un font monospazio guardabile
\usepackage{lmodern}

\renewcommand{\rmdefault}{ppl}
\renewcommand{\sfdefault}{phv}
\renewcommand{\ttdefault}{lmtt}

% disponi il titolo
\makeatletter
\renewcommand{\maketitle} {
	\begin{center} 
		\begin{minipage}[t]{.8\textwidth}
			\textsf{\huge\bfseries \@title} 
		\end{minipage}%
		\begin{minipage}[t]{.2\textwidth}
			\raggedleft \vspace{-1.65em}
			\textsf{\small \@author} \vfill
			\textsf{\small \@date}
		\end{minipage}
		\par
	\end{center}

	\thispagestyle{empty}
	\pagestyle{fancy}
}
\makeatother

% disponi teoremi
\usepackage{tcolorbox}
\newtcolorbox[auto counter, number within=section]{theorem}[2][]{%
	colback=blue!10, 
	colframe=blue!40!black, 
	sharp corners=northwest,
	fonttitle=\sffamily\bfseries, 
	title=Teorema~\thetcbcounter: #2, 
	#1
}

% disponi definizioni
\newtcolorbox[auto counter, number within=section]{definition}[2][]{%
	colback=red!10,
	colframe=red!40!black,
	sharp corners=northwest,
	fonttitle=\sffamily\bfseries,
	title=Definizione~\thetcbcounter: #2,
	#1
}

% disponi problemi
\newtcolorbox[auto counter, number within=section]{problem}[2][]{%
	colback=green!10,
	colframe=green!40!black,
	sharp corners=northwest,
	fonttitle=\sffamily\bfseries,
	title=Problema~\thetcbcounter: #2,
	#1
}

% disponi codice
\usepackage{listings}
\usepackage[table]{xcolor}

\lstdefinestyle{codestyle}{
		backgroundcolor=\color{black!5}, 
		commentstyle=\color{codegreen},
		keywordstyle=\bfseries\color{magenta},
		numberstyle=\sffamily\tiny\color{black!60},
		stringstyle=\color{green!50!black},
		basicstyle=\ttfamily\footnotesize,
		breakatwhitespace=false,         
		breaklines=true,                 
		captionpos=b,                    
		keepspaces=true,                 
		numbers=left,                    
		numbersep=5pt,                  
		showspaces=false,                
		showstringspaces=false,
		showtabs=false,                  
		tabsize=2
}

\lstdefinestyle{shellstyle}{
		backgroundcolor=\color{black!5}, 
		basicstyle=\ttfamily\footnotesize\color{black}, 
		commentstyle=\color{black}, 
		keywordstyle=\color{black},
		numberstyle=\color{black!5},
		stringstyle=\color{black}, 
		showspaces=false,
		showstringspaces=false, 
		showtabs=false, 
		tabsize=2, 
		numbers=none, 
		breaklines=true
}

\lstdefinelanguage{javascript}{
	keywords={typeof, new, true, false, catch, function, return, null, catch, switch, var, if, in, while, do, else, case, break},
	keywordstyle=\color{blue}\bfseries,
	ndkeywords={class, export, boolean, throw, implements, import, this},
	ndkeywordstyle=\color{darkgray}\bfseries,
	identifierstyle=\color{black},
	sensitive=false,
	comment=[l]{//},
	morecomment=[s]{/*}{*/},
	commentstyle=\color{purple}\ttfamily,
	stringstyle=\color{red}\ttfamily,
	morestring=[b]',
	morestring=[b]"
}

% disponi sezioni
\usepackage{titlesec}

\titleformat{\section}
	{\sffamily\Large\bfseries} 
	{\thesection}{1em}{} 
\titleformat{\subsection}
	{\sffamily\large\bfseries}   
	{\thesubsection}{1em}{} 
\titleformat{\subsubsection}
	{\sffamily\normalsize\bfseries} 
	{\thesubsubsection}{1em}{}

% disponi alberi
\usepackage{forest}

\forestset{
	rectstyle/.style={
		for tree={rectangle,draw,font=\large\sffamily}
	},
	roundstyle/.style={
		for tree={circle,draw,font=\large}
	}
}

% disponi algoritmi
\usepackage{algorithm}
\usepackage{algorithmic}
\makeatletter
\renewcommand{\ALG@name}{Algoritmo}
\makeatother

% disponi numeri di pagina
\usepackage{fancyhdr}
\fancyhf{} 
\fancyfoot[L]{\sffamily{\thepage}}

\makeatletter
\fancyhead[L]{\raisebox{1ex}[0pt][0pt]{\sffamily{\@title \ \@date}}} 
\fancyhead[R]{\raisebox{1ex}[0pt][0pt]{\sffamily{\@author}}}
\makeatother

\begin{document}

\pagestyle{fancy}
\thispagestyle{empty}
\renewcommand{\thispagestyle}[1]{}

\maketitle

\documentclass[a4paper,11pt]{article}
\usepackage[a4paper, margin=8em]{geometry}

% usa i pacchetti per la scrittura in italiano
\usepackage[french,italian]{babel}
\usepackage[T1]{fontenc}
\usepackage[utf8]{inputenc}
\frenchspacing 

% usa i pacchetti per la formattazione matematica
\usepackage{amsmath, amssymb, amsthm, amsfonts}

% usa altri pacchetti
\usepackage{gensymb}
\usepackage{hyperref}
\usepackage{standalone}

% imposta il titolo
\title{Appunti Economia ed Organizzazione Aziendale}
\author{Luca Seggiani}
\date{2025}

% disegni
\usepackage{pgfplots}
\pgfplotsset{width=10cm,compat=1.9}

% imposta lo stile
% usa helvetica
\usepackage[scaled]{helvet}
% usa palatino
\usepackage{palatino}
% usa un font monospazio guardabile
\usepackage{lmodern}

\renewcommand{\rmdefault}{ppl}
\renewcommand{\sfdefault}{phv}
\renewcommand{\ttdefault}{lmtt}

% disponi il titolo
\makeatletter
\renewcommand{\maketitle} {
	\begin{center} 
		\begin{minipage}[t]{.8\textwidth}
			\textsf{\huge\bfseries \@title} 
		\end{minipage}%
		\begin{minipage}[t]{.2\textwidth}
			\raggedleft \vspace{-1.65em}
			\textsf{\small \@author} \vfill
			\textsf{\small \@date}
		\end{minipage}
		\par
	\end{center}

	\thispagestyle{empty}
	\pagestyle{fancy}
}
\makeatother

% disponi teoremi
\usepackage{tcolorbox}
\newtcolorbox[auto counter, number within=section]{theorem}[2][]{%
	colback=blue!10, 
	colframe=blue!40!black, 
	sharp corners=northwest,
	fonttitle=\sffamily\bfseries, 
	title=Teorema~\thetcbcounter: #2, 
	#1
}

% disponi definizioni
\newtcolorbox[auto counter, number within=section]{definition}[2][]{%
	colback=red!10,
	colframe=red!40!black,
	sharp corners=northwest,
	fonttitle=\sffamily\bfseries,
	title=Definizione~\thetcbcounter: #2,
	#1
}

% disponi problemi
\newtcolorbox[auto counter, number within=section]{problem}[2][]{%
	colback=green!10,
	colframe=green!40!black,
	sharp corners=northwest,
	fonttitle=\sffamily\bfseries,
	title=Problema~\thetcbcounter: #2,
	#1
}

% disponi codice
\usepackage{listings}
\usepackage[table]{xcolor}

\lstdefinestyle{codestyle}{
		backgroundcolor=\color{black!5}, 
		commentstyle=\color{codegreen},
		keywordstyle=\bfseries\color{magenta},
		numberstyle=\sffamily\tiny\color{black!60},
		stringstyle=\color{green!50!black},
		basicstyle=\ttfamily\footnotesize,
		breakatwhitespace=false,         
		breaklines=true,                 
		captionpos=b,                    
		keepspaces=true,                 
		numbers=left,                    
		numbersep=5pt,                  
		showspaces=false,                
		showstringspaces=false,
		showtabs=false,                  
		tabsize=2
}

\lstdefinestyle{shellstyle}{
		backgroundcolor=\color{black!5}, 
		basicstyle=\ttfamily\footnotesize\color{black}, 
		commentstyle=\color{black}, 
		keywordstyle=\color{black},
		numberstyle=\color{black!5},
		stringstyle=\color{black}, 
		showspaces=false,
		showstringspaces=false, 
		showtabs=false, 
		tabsize=2, 
		numbers=none, 
		breaklines=true
}

\lstdefinelanguage{javascript}{
	keywords={typeof, new, true, false, catch, function, return, null, catch, switch, var, if, in, while, do, else, case, break},
	keywordstyle=\color{blue}\bfseries,
	ndkeywords={class, export, boolean, throw, implements, import, this},
	ndkeywordstyle=\color{darkgray}\bfseries,
	identifierstyle=\color{black},
	sensitive=false,
	comment=[l]{//},
	morecomment=[s]{/*}{*/},
	commentstyle=\color{purple}\ttfamily,
	stringstyle=\color{red}\ttfamily,
	morestring=[b]',
	morestring=[b]"
}

% disponi sezioni
\usepackage{titlesec}

\titleformat{\section}
	{\sffamily\Large\bfseries} 
	{\thesection}{1em}{} 
\titleformat{\subsection}
	{\sffamily\large\bfseries}   
	{\thesubsection}{1em}{} 
\titleformat{\subsubsection}
	{\sffamily\normalsize\bfseries} 
	{\thesubsubsection}{1em}{}

% disponi alberi
\usepackage{forest}

\forestset{
	rectstyle/.style={
		for tree={rectangle,draw,font=\large\sffamily}
	},
	roundstyle/.style={
		for tree={circle,draw,font=\large}
	}
}

% disponi algoritmi
\usepackage{algorithm}
\usepackage{algorithmic}
\makeatletter
\renewcommand{\ALG@name}{Algoritmo}
\makeatother

% disponi numeri di pagina
\usepackage{fancyhdr}
\fancyhf{} 
\fancyfoot[L]{\sffamily{\thepage}}

\makeatletter
\fancyhead[L]{\raisebox{1ex}[0pt][0pt]{\sffamily{\@title \ \@date}}} 
\fancyhead[R]{\raisebox{1ex}[0pt][0pt]{\sffamily{\@author}}}
\makeatother

\begin{document}

% sezione (data)
\section{Lezione del 26-02-25}

% stili pagina
\thispagestyle{empty}
\pagestyle{fancy}

% testo
\subsection{Introduzione al corso}
L'economia riguarda il modo in cui gli agenti economici giungono a compiere scelte ottime in presenza di risorse limitate.
Tratta di attività di \textbf{produzione}, \textbf{scambi} e \textbf{consumo} di \textit{beni} atti a \textbf{soddisfare bisogni} della società.
Le variabili su cui agire sono quindi:
\begin{itemize}
	\item Quali bene produrre o consumare;
	\item Come produrli (metodi di produzione, ripartizione delle risorse);
	\item Per chi produrli (mercati).
\end{itemize}

Ci sono due approcci principali al problema:
\begin{itemize}
	\item \textbf{Macro economia:} fenomeni a livello di sistema come sviiluppo, occupazione, inflazione, ecc...
	\item \textbf{Micro economia:} modelli di comportamento per produttori e consumatori, forme di mercato, ecc...
\end{itemize}

L'economia aziendale presenta un approccio dal basso all'economia, trattando la nascita, la struttura e lo sviluppo di un impresa.
\end{document}


\documentclass[a4paper,11pt]{article}
\usepackage[a4paper, margin=8em]{geometry}

% usa i pacchetti per la scrittura in italiano
\usepackage[french,italian]{babel}
\usepackage[T1]{fontenc}
\usepackage[utf8]{inputenc}
\frenchspacing 

% usa i pacchetti per la formattazione matematica
\usepackage{amsmath, amssymb, amsthm, amsfonts}

% usa altri pacchetti
\usepackage{gensymb}
\usepackage{hyperref}
\usepackage{standalone}

% imposta il titolo
\title{Appunti Economia ed Organizzazione Aziendale}
\author{Luca Seggiani}
\date{2025}

% disegni
\usepackage{pgfplots}
\pgfplotsset{width=10cm,compat=1.9}

% imposta lo stile
% usa helvetica
\usepackage[scaled]{helvet}
% usa palatino
\usepackage{palatino}
% usa un font monospazio guardabile
\usepackage{lmodern}

\renewcommand{\rmdefault}{ppl}
\renewcommand{\sfdefault}{phv}
\renewcommand{\ttdefault}{lmtt}

% disponi il titolo
\makeatletter
\renewcommand{\maketitle} {
	\begin{center} 
		\begin{minipage}[t]{.8\textwidth}
			\textsf{\huge\bfseries \@title} 
		\end{minipage}%
		\begin{minipage}[t]{.2\textwidth}
			\raggedleft \vspace{-1.65em}
			\textsf{\small \@author} \vfill
			\textsf{\small \@date}
		\end{minipage}
		\par
	\end{center}

	\thispagestyle{empty}
	\pagestyle{fancy}
}
\makeatother

% disponi teoremi
\usepackage{tcolorbox}
\newtcolorbox[auto counter, number within=section]{theorem}[2][]{%
	colback=blue!10, 
	colframe=blue!40!black, 
	sharp corners=northwest,
	fonttitle=\sffamily\bfseries, 
	title=Teorema~\thetcbcounter: #2, 
	#1
}

% disponi definizioni
\newtcolorbox[auto counter, number within=section]{definition}[2][]{%
	colback=red!10,
	colframe=red!40!black,
	sharp corners=northwest,
	fonttitle=\sffamily\bfseries,
	title=Definizione~\thetcbcounter: #2,
	#1
}

% disponi problemi
\newtcolorbox[auto counter, number within=section]{problem}[2][]{%
	colback=green!10,
	colframe=green!40!black,
	sharp corners=northwest,
	fonttitle=\sffamily\bfseries,
	title=Problema~\thetcbcounter: #2,
	#1
}

% disponi codice
\usepackage{listings}
\usepackage[table]{xcolor}

\lstdefinestyle{codestyle}{
		backgroundcolor=\color{black!5}, 
		commentstyle=\color{codegreen},
		keywordstyle=\bfseries\color{magenta},
		numberstyle=\sffamily\tiny\color{black!60},
		stringstyle=\color{green!50!black},
		basicstyle=\ttfamily\footnotesize,
		breakatwhitespace=false,         
		breaklines=true,                 
		captionpos=b,                    
		keepspaces=true,                 
		numbers=left,                    
		numbersep=5pt,                  
		showspaces=false,                
		showstringspaces=false,
		showtabs=false,                  
		tabsize=2
}

\lstdefinestyle{shellstyle}{
		backgroundcolor=\color{black!5}, 
		basicstyle=\ttfamily\footnotesize\color{black}, 
		commentstyle=\color{black}, 
		keywordstyle=\color{black},
		numberstyle=\color{black!5},
		stringstyle=\color{black}, 
		showspaces=false,
		showstringspaces=false, 
		showtabs=false, 
		tabsize=2, 
		numbers=none, 
		breaklines=true
}

\lstdefinelanguage{javascript}{
	keywords={typeof, new, true, false, catch, function, return, null, catch, switch, var, if, in, while, do, else, case, break},
	keywordstyle=\color{blue}\bfseries,
	ndkeywords={class, export, boolean, throw, implements, import, this},
	ndkeywordstyle=\color{darkgray}\bfseries,
	identifierstyle=\color{black},
	sensitive=false,
	comment=[l]{//},
	morecomment=[s]{/*}{*/},
	commentstyle=\color{purple}\ttfamily,
	stringstyle=\color{red}\ttfamily,
	morestring=[b]',
	morestring=[b]"
}

% disponi sezioni
\usepackage{titlesec}

\titleformat{\section}
	{\sffamily\Large\bfseries} 
	{\thesection}{1em}{} 
\titleformat{\subsection}
	{\sffamily\large\bfseries}   
	{\thesubsection}{1em}{} 
\titleformat{\subsubsection}
	{\sffamily\normalsize\bfseries} 
	{\thesubsubsection}{1em}{}

% disponi alberi
\usepackage{forest}

\forestset{
	rectstyle/.style={
		for tree={rectangle,draw,font=\large\sffamily}
	},
	roundstyle/.style={
		for tree={circle,draw,font=\large}
	}
}

% disponi algoritmi
\usepackage{algorithm}
\usepackage{algorithmic}
\makeatletter
\renewcommand{\ALG@name}{Algoritmo}
\makeatother

% disponi numeri di pagina
\usepackage{fancyhdr}
\fancyhf{} 
\fancyfoot[L]{\sffamily{\thepage}}

\makeatletter
\fancyhead[L]{\raisebox{1ex}[0pt][0pt]{\sffamily{\@title \ \@date}}} 
\fancyhead[R]{\raisebox{1ex}[0pt][0pt]{\sffamily{\@author}}}
\makeatother

\begin{document}

% sezione (data)
\section{Lezione del 27-02-25}

% stili pagina
\thispagestyle{empty}
\pagestyle{fancy}

% testo
\subsection{Introduzione all'economia}
Abbiamo visto come l'economia tratta dell'interazione economica fra diversi \textbf{soggetti} o \textit{enti} all'interno di diversi \textbf{sistemi economici}.
Questi possono essere \textit{privati}, \textit{aziende}, come ancora enti di varia natura se non addirittura interi \textit{stati}.

Il comportamento di ogni soggetto è determinato dal suo particolare modo di vedere e agire la realtà circostante, cioè da dei particolari \textit{assunti} o \textbf{paradigmi}.
I paradigmi dei soggetti influenzano il sistema economico a cui appartengono, e viceversa.
Proprio per questo motivo, storicamente si sono sempre formati innumerevoli sistemi economici, come innumerevoli erano gli assunti dei soggetti che vi appartenevano.

L'obiettivo principale dell'economia è quindi di comprendere il comportamento di questi soggetti all'interno di un dato sistema economico, inteso come la loro azione su determinati \textit{mezzi di produzione}, in presenza di \textbf{beni scarsi} (come li avevamo definiti in ricerca operativa, \textit{risorse limitate}).
Una volta comprese questo tipo di dinamiche, si possono ricavare \textbf{strumenti} che ci aiutino a fare scelte economiche migliori (più efficienti, più informate, ecc...). 

\subsubsection{Economia politica}
Una branca dell'economia che non considereremo in particolare è l'\textbf{economia politica}.
Questa tratta del comportamento di una vasta quantità di individui, a livello statale o oltre.
Per questo, si dice un approccio \textbf{top-down}.

Inoltre, non si premette di osservare il funzionamento interno dei soggetti economici, e quindi ad esempio di osservare la struttura delle imprese.
Per questo viene detto anche un approccio \textbf{black box} (a \textit{scatola nera}).

\subsubsection{Economia aziendale}
L'economia aziendale, come abbiamo detto, sceglie il percorso inverso: arriva alla comprensione di un sistema economico studiando la struttura e l'organizzazione delle imprese che si trovano al suo interno.
Questo lo rende un approccio \textbf{bottom-up}.

\subsubsection{Paradigmi}
Il modo di porsi di fronte a un problema (\textit{mentalità}) è definito, come abbiamo detto, \textit{paradigma}.
La \textit{propensione} è la tendenza di dare a priori un certo peso a diversi aspetti del problema da risolvere.
Possiamo assumere 3 tipi di mentalità rispetto alla gestione dell'impresa:
\begin{itemize}
	\item \textbf{Tecnico-produttiva:} tipica di chi si occupa del lato tecnico (e.g. ingegneri). 
		Unilaterale, tende al mantenimento dello \textit{status quo}: una strategia che si dimostra vincente non viene cambiata. Eccezione sono le \textbf{start-up}, dove una mentalità tecnico-produttiva può portare a cambiamenti e innovazioni.
	\item \textbf{Finanziaria:} ancora unilaterale, entra in gioco in periodi di \textbf{abbondanza} o \textbf{scarsità}, rispettivamente incentivando la tendenza a \textit{spendere} o a \textit{contrarre} le spese;
	\item \textbf{Economica:} cerca di unire gli approcci tecnico-produttivi e finanziari.
\end{itemize}

\subsubsection{Elementi in gioco}
Vediamo quindi quali saranno gli elementi di interesse nel \textbf{modello di business} che tratteremo.
# finisci

\end{document}


\documentclass[a4paper,11pt]{article}
\usepackage[a4paper, margin=8em]{geometry}

% usa i pacchetti per la scrittura in italiano
\usepackage[french,italian]{babel}
\usepackage[T1]{fontenc}
\usepackage[utf8]{inputenc}
\frenchspacing 

% usa i pacchetti per la formattazione matematica
\usepackage{amsmath, amssymb, amsthm, amsfonts}

% usa altri pacchetti
\usepackage{gensymb}
\usepackage{hyperref}
\usepackage{standalone}

% imposta il titolo
\title{Appunti Economia ed Organizzazione Aziendale}
\author{Luca Seggiani}
\date{2025}

% disegni
\usepackage{pgfplots}
\pgfplotsset{width=10cm,compat=1.9}

% imposta lo stile
% usa helvetica
\usepackage[scaled]{helvet}
% usa palatino
\usepackage{palatino}
% usa un font monospazio guardabile
\usepackage{lmodern}

\renewcommand{\rmdefault}{ppl}
\renewcommand{\sfdefault}{phv}
\renewcommand{\ttdefault}{lmtt}

% disponi il titolo
\makeatletter
\renewcommand{\maketitle} {
	\begin{center} 
		\begin{minipage}[t]{.8\textwidth}
			\textsf{\huge\bfseries \@title} 
		\end{minipage}%
		\begin{minipage}[t]{.2\textwidth}
			\raggedleft \vspace{-1.65em}
			\textsf{\small \@author} \vfill
			\textsf{\small \@date}
		\end{minipage}
		\par
	\end{center}

	\thispagestyle{empty}
	\pagestyle{fancy}
}
\makeatother

% disponi teoremi
\usepackage{tcolorbox}
\newtcolorbox[auto counter, number within=section]{theorem}[2][]{%
	colback=blue!10, 
	colframe=blue!40!black, 
	sharp corners=northwest,
	fonttitle=\sffamily\bfseries, 
	title=Teorema~\thetcbcounter: #2, 
	#1
}

% disponi definizioni
\newtcolorbox[auto counter, number within=section]{definition}[2][]{%
	colback=red!10,
	colframe=red!40!black,
	sharp corners=northwest,
	fonttitle=\sffamily\bfseries,
	title=Definizione~\thetcbcounter: #2,
	#1
}

% disponi problemi
\newtcolorbox[auto counter, number within=section]{problem}[2][]{%
	colback=green!10,
	colframe=green!40!black,
	sharp corners=northwest,
	fonttitle=\sffamily\bfseries,
	title=Problema~\thetcbcounter: #2,
	#1
}

% disponi codice
\usepackage{listings}
\usepackage[table]{xcolor}

\lstdefinestyle{codestyle}{
		backgroundcolor=\color{black!5}, 
		commentstyle=\color{codegreen},
		keywordstyle=\bfseries\color{magenta},
		numberstyle=\sffamily\tiny\color{black!60},
		stringstyle=\color{green!50!black},
		basicstyle=\ttfamily\footnotesize,
		breakatwhitespace=false,         
		breaklines=true,                 
		captionpos=b,                    
		keepspaces=true,                 
		numbers=left,                    
		numbersep=5pt,                  
		showspaces=false,                
		showstringspaces=false,
		showtabs=false,                  
		tabsize=2
}

\lstdefinestyle{shellstyle}{
		backgroundcolor=\color{black!5}, 
		basicstyle=\ttfamily\footnotesize\color{black}, 
		commentstyle=\color{black}, 
		keywordstyle=\color{black},
		numberstyle=\color{black!5},
		stringstyle=\color{black}, 
		showspaces=false,
		showstringspaces=false, 
		showtabs=false, 
		tabsize=2, 
		numbers=none, 
		breaklines=true
}

\lstdefinelanguage{javascript}{
	keywords={typeof, new, true, false, catch, function, return, null, catch, switch, var, if, in, while, do, else, case, break},
	keywordstyle=\color{blue}\bfseries,
	ndkeywords={class, export, boolean, throw, implements, import, this},
	ndkeywordstyle=\color{darkgray}\bfseries,
	identifierstyle=\color{black},
	sensitive=false,
	comment=[l]{//},
	morecomment=[s]{/*}{*/},
	commentstyle=\color{purple}\ttfamily,
	stringstyle=\color{red}\ttfamily,
	morestring=[b]',
	morestring=[b]"
}

% disponi sezioni
\usepackage{titlesec}

\titleformat{\section}
	{\sffamily\Large\bfseries} 
	{\thesection}{1em}{} 
\titleformat{\subsection}
	{\sffamily\large\bfseries}   
	{\thesubsection}{1em}{} 
\titleformat{\subsubsection}
	{\sffamily\normalsize\bfseries} 
	{\thesubsubsection}{1em}{}

% disponi alberi
\usepackage{forest}

\forestset{
	rectstyle/.style={
		for tree={rectangle,draw,font=\large\sffamily}
	},
	roundstyle/.style={
		for tree={circle,draw,font=\large}
	}
}

% disponi algoritmi
\usepackage{algorithm}
\usepackage{algorithmic}
\makeatletter
\renewcommand{\ALG@name}{Algoritmo}
\makeatother

% disponi numeri di pagina
\usepackage{fancyhdr}
\fancyhf{} 
\fancyfoot[L]{\sffamily{\thepage}}

\makeatletter
\fancyhead[L]{\raisebox{1ex}[0pt][0pt]{\sffamily{\@title \ \@date}}} 
\fancyhead[R]{\raisebox{1ex}[0pt][0pt]{\sffamily{\@author}}}
\makeatother

\begin{document}

% sezione (data)
\section{Lezione del 05-03-25}

% stili pagina
\thispagestyle{empty}
\pagestyle{fancy}

% testo
\subsection{Modello di business}
L'agente principale che prendiamo in osservazione durante il corso e' l'\textbf{impresa}, nella sua organizzazione e nella sua attivita' commerciale e finanziaria.
L'impresa rappesenta a tutti gli effetti un \textbf{sistema}, che interagisce con l'esterno ottenendo qualcosa, e restituendo da parte loro qualcos'altro.

L'impresa interagisce quindi coi \textbf{mercati} scambiando merci (coi fornitori), prodotti (coi clienti), titoli di vario tipo e in generale denaro.
Avra' bisogno di \textbf{capitale}, che si procurera' tramite \textit{scelte finanziarie} (mezzi propri e finanziamenti).
Sfruttera' il capitale cosi' ottenuto per cimentarsi nella \textbf{produzione} di un bene (un \textbf{prodotto}) da metter in commercio.

Nella \textit{costituzione} dell'impresa sara' necessaria un \textit{idea} di impresa, nonche' decisioni riguardo alla merce prodotta, quindi al \textit{mercato target}, al \textit{sistema di offerta} e alla struttura interna (\textbf{organizzazione aziendale}). 

Dall'esterno agiranno sull'impresa varie \textit{forze macroeconomiche}, nonche' l'attivita' dello \textbf{stato} (imposte, ecc...), le \textit{forze di mercato} e i \textit{trend socio-economici}.
Le imprese saranno poi in \textbf{concorrenza}, sia questa diretta, indiretta o potenziale, fra di loro.

\subsubsection{Mercati finanziari}
Una delle categorie di mercati con cui interagisce l'impresa e' rappresentata dai \textbf{mercati finanziari}.
Questa interazione viene fatta attraverso gli \textbf{strumenti finanziari} (obbligazioni, azioni, mutui, ecc...).
In particolare, azioni e obbligazioni vengono comprate e vendute sul \textbf{mercato mobiliare}, solitamente detto \textit{borsa}.

\subsection{Capitale}
L'obiettivo principale dell'impresa e' quello dell'ottenimento del \textbf{capitale}.
Questo puo' derivare da:
\begin{itemize}
	\item \textbf{Capitale proprio} dell'imprenditore o di eventuali \textbf{soci} disposti ad unirsi al supporto dell'idea di business, viene detto anche \textbf{capitale di rischio}, in quanto non ha alcuna garanzia di essere recuperato nel caso del fallimento dell'impresa. Di contro, i soci hanno \textbf{diritto residuale}, cioe' di suddividersi cio' che residua a soddisfacimento di tutti gli altri debiti.

		In termini contabili un sottoinsieme del capitale proprio viene definito \textbf{capitale sociale}, cioe' l'insieme dei \textit{conferimenti} effettuati dai soci durante la costituzione o in momenti successivi della vita dell'impresa.

		Altre fonti di capitale proprio sono rappresentate anche dall'\textbf{utile}, cioe' dal risultato delle attivita' dell'impresa.
	\item \textbf{Capitale di terzi} dei \textit{finanziatori}, viene detto anche \textbf{capitale di credito}, in quanto i finanziatori hanno diritto alla sua restituzione anche nel caso di fallimento, oltre che ad un \textbf{interesse} che l'impresa paga per il capitale che il finanziatore non spende personalmente, ma gli mette a disposizione.
		I finanziatori rappresentano per noi dei \textbf{creditori}, cioe' abbiamo per loro un \textbf{debito}.
	\item \textbf{Crowdsourcing}.
\end{itemize}

\subsubsection{Obbligazioni e azioni come fonti di capitale}

Le \textbf{obbligazioni} rappresentano quindi un tipo di \textit{capitale di credito}, cioe' dei prestiti che l'impresa si impegna a rimborsare al finanziatore con un certo interesse.
Le obbligazioni sono poi \textbf{titoli di credito} (cambiali, assegni, ecc...), con le loro regole proprie che determinano le modalita' secondo le quali possono essere immesse nel mercato.

Le \textbf{azioni} sono invece un tipo di \textit{capitale di rischio}.
L'azione consiste per gli acquirenti in un versamento da parte dell'impresa di \textbf{dividendi}, ricavati dall'attivita' della stessa, e legati alla relazione dell'azionista con la stessa.

\subsubsection{Note sui tipi di societa'}
Notiamo che solo alcuni tipi di societa' (S.p.A., \textit{Societa' per Azioni} o S.a.p.a. (\textit{Societa' in accomandita per azioni})) possono emanare azioni.
Le S.r.l. \textit{Societa' a responsabilita' limitata} e le \textit{societa' di persone} detengono invece \textbf{quote di capitale}.

In particolare, le S.p.A., le S.a.p.a. e le S.r.l. rappresentano \textbf{societa' di capitali}, mentre le societa' di persone sono ulteriormente divise in S.n.C. (\textit{Societa' in nome Collettivo}), S.s. (\textit{Societa' semplice}) e S.a.s (\textit{Societa' in accomandita semplice}).

Si ha quindi la divisione:
\begin{itemize}
	\item Societa' di capitali:
		\begin{itemize}
			\item S.p.A. - Societa' per Azioni;
			\item S.a.p.a. - Societa' in accomandita per azioni;
			\item S.r.l. - Societa' a responsabilita' limitata.
		\end{itemize}
	\item Societa' di persone:
		\begin{itemize}
			\item S.n.c. - Societa' in nome collettivo;
			\item S.s. - Societa' semplici;
			\item S.a.s. - Societa' in accomandita semplice.
		\end{itemize}
\end{itemize}

\subsection{Produzione}
La societa' in disponibilita' di liquido (capitale proprio e di terzi) dovra' effettuare \textbf{operazioni di acquisto} dei cosiddetti \textbf{fattori produttivi}, che distinguiamo in:
\begin{itemize}
	\item \textbf{Fattori produttivi pluriennali}, stabilimenti, automezzi, strumentazioni, ecc...
	\item \textbf{Manodopera}, cioe' il \textit{lavoro} dei dipendenti;
		\item \textbf{Fattori produttivi di esercizio}, cioe' \textit{merci}, \textit{materia prima}, o ancora \textit{materia sussidiaria}, ecc...
\end{itemize}

Notiamo che i fattori produttivi di esercizio contribuiscono totalmente al valore di un prodotto, mentre i fattori produttivi pluriennali vi contribuiscono solo parzialmente (la stessa fabbrica produce piu' di un prodotto all'anno).
Il processo di ripartizione di un costo pluriennale sul consumo fatto in un periodo di tempo (solitamente un anno) viene detto \textbf{ammortamento}.

Nell'operazione di acquisto l'impresa crea un \textbf{debito} dei fornitori dei fattori produttivi.
Questo debito si distingue dal debito che abbiamo con i finanziatori in quanto rappresenta \textbf{debito di funzionamento} o \textbf{debito commerciale}.
La presenza di debiti commerciali e' fisiologica all'attivita' dell'impresa, in quanto e' necessaria alla produzione e vendita del prodotto.

\par\smallskip

Una volta che si e' in possesso delle liquidita', e quindi dei fattori di produzione, si procede con la produzione vera e proprio del prodotto da mettere in commercio.

Nelle \textbf{imprese manufatturiere} e' tipica una \textit{trasformazione fisica} dei fattori produttivi di esercizio.

Le \textbf{imprese commerciali} effettuano invece \textit{trasformazioni economiche} (e.g. di denaro o merci) nel \textit{tempo} e nello \textit{spazio}, traendo guadagno da fattori come l'interesse o la rivendita di merci in veste di intermediari fra produttori e consumatori.

\subsection{Vendita}
Il prodotto ultimato verra' venduto nei \textbf{mercati di sbocco}, che possono essere:
\begin{itemize}
	\item \textbf{B2B}, \textit{Business-to-Business}: da imprese ad altre imprese, ad esempio componenti (\textbf{prodotti intermedi}) che verranno assemblati da altre imprese;
	\item \textbf{B2C}, \textit{Business-to-Consumer}: dall'impresa al consumatore, di \textbf{prodotti finali}.
\end{itemize}

Durante la fase di vendita si puo' creare un'altro tipo di debito, cio il \textbf{credito commerciale} dei clienti nei confronti dell'impresa.

\subsection{Gestione}
L'insieme di passaggi che abbiamo visto finora (finanziamento, produzione e vendita) formano il \textbf{ciclo operativo della gestione} dell'impresa.
La \textbf{gestione} d'impresa e l'insieme delle operazioni che le persone operanti nell'impresa compiono (sia decisioni che azioni) tramite i fattori produttivi a disposizione per svolgere le attivita' che l'impresa ha definito.

Le operazioni di gestione \textbf{ordinaria} sono cio' che l'impresa fa su base quotidiana: \textbf{acquisto} $\rightarrow$ \textbf{produzione} $\rightarrow$ \textbf{vendita}, supportate da finanziamento \textbf{attinto} (mercati finanziari) e finanziamento \textbf{concesso}.

In particolare, chiamiamo operazioni di gestione ordinaria \textbf{esterna} quelle operazioni che ci mettono in contatto con terzi.
Sono di questo tipo tutte le operazioni tranne la produzione.
La produzione e' quindi l'unica operazione che, dal magazzino di ingresso al magazzino di uscita, non mette l'impresa in contatto con i terzi.
Le operazioni di gestione ordinaria esterna sono quindi quelle che riguardano anche la \textbf{contabilita'} e quindi il \textbf{bilancio}, cioe' la valutazione dei flussi di denaro in entranta e in uscita.

\subsection{Soggetti dell'impresa}
L'impresa coinvolge una vasta gamma di soggetti, primi fra tutti gli \textbf{Stakeholder} (portatori di \textit{interesse}), cioe' coloro che costituiscono il \textbf{soggetto economico} di influenza sull'impresa.
Questi possono essere:
\begin{itemize}
	\item I soci;
	\item I finanziatori;
	\item Manager e dipendenti;
	\item I clienti;
	\item Lo stato;
	\item I concorrenti;
	\item La comunita' sociale.
\end{itemize}

In particolare, gli \textbf{shareholder} sono rappresentati dai soci e dagli azionisti (di maggioranza/minoranza), cioe' da chi ha fornito capitale di rischio all'impresa.

Il \textbf{soggetto giuridico} e' invece rappresentato dalla persona fisica o giuridica che assume la titolarita' dell'impresa.

I soggetti dell'impresa hanno fra di loro responsabilita' diverse per quanto riguarda l'attivita' e le responsabilita' cui deve render conto l'impresa.
Ad esempio, come avevamo detto, i titolari di obbligazioni avranno diritti a restituzioni, mentre gli azionisti non avranno lo stesso diritto.
Di contro, il capitale proprio degli shareholder, essendo loro privato, non potra' essere usato per il rimborso di rischi andati in fallo (mentre lo stato potra' procedere per quanto riguarda titoli di credito regolamentati).

\subsection{Struttura dell'impresa, organigramma}
A livello formale la struttura dell'impresa e' solitamente data dal suo \textbf{organigramma}, che ci da informazioni di livello strutturale e formale, ma non delle interdipendenze informali fra i suoi elementi.
	
Di base, consultando l'organigramma potremo stabilire chi comanda e chi e' sottoposto, ma ad esempio non potremo comprendere relazioni fra elementi costruite all'esterno della struttura propria dell'impresa.

\end{document}


\documentclass[a4paper,11pt]{article}
\usepackage[a4paper, margin=8em]{geometry}

% usa i pacchetti per la scrittura in italiano
\usepackage[french,italian]{babel}
\usepackage[T1]{fontenc}
\usepackage[utf8]{inputenc}
\frenchspacing 

% usa i pacchetti per la formattazione matematica
\usepackage{amsmath, amssymb, amsthm, amsfonts}

% usa altri pacchetti
\usepackage{gensymb}
\usepackage{hyperref}
\usepackage{standalone}

% imposta il titolo
\title{Appunti Economia ed Organizzazione Aziendale}
\author{Luca Seggiani}
\date{2025}

% disegni
\usepackage{pgfplots}
\pgfplotsset{width=10cm,compat=1.9}

% imposta lo stile
% usa helvetica
\usepackage[scaled]{helvet}
% usa palatino
\usepackage{palatino}
% usa un font monospazio guardabile
\usepackage{lmodern}

\renewcommand{\rmdefault}{ppl}
\renewcommand{\sfdefault}{phv}
\renewcommand{\ttdefault}{lmtt}

% disponi il titolo
\makeatletter
\renewcommand{\maketitle} {
	\begin{center} 
		\begin{minipage}[t]{.8\textwidth}
			\textsf{\huge\bfseries \@title} 
		\end{minipage}%
		\begin{minipage}[t]{.2\textwidth}
			\raggedleft \vspace{-1.65em}
			\textsf{\small \@author} \vfill
			\textsf{\small \@date}
		\end{minipage}
		\par
	\end{center}

	\thispagestyle{empty}
	\pagestyle{fancy}
}
\makeatother

% disponi teoremi
\usepackage{tcolorbox}
\newtcolorbox[auto counter, number within=section]{theorem}[2][]{%
	colback=blue!10, 
	colframe=blue!40!black, 
	sharp corners=northwest,
	fonttitle=\sffamily\bfseries, 
	title=Teorema~\thetcbcounter: #2, 
	#1
}

% disponi definizioni
\newtcolorbox[auto counter, number within=section]{definition}[2][]{%
	colback=red!10,
	colframe=red!40!black,
	sharp corners=northwest,
	fonttitle=\sffamily\bfseries,
	title=Definizione~\thetcbcounter: #2,
	#1
}

% disponi problemi
\newtcolorbox[auto counter, number within=section]{problem}[2][]{%
	colback=green!10,
	colframe=green!40!black,
	sharp corners=northwest,
	fonttitle=\sffamily\bfseries,
	title=Problema~\thetcbcounter: #2,
	#1
}

% disponi codice
\usepackage{listings}
\usepackage[table]{xcolor}

\lstdefinestyle{codestyle}{
		backgroundcolor=\color{black!5}, 
		commentstyle=\color{codegreen},
		keywordstyle=\bfseries\color{magenta},
		numberstyle=\sffamily\tiny\color{black!60},
		stringstyle=\color{green!50!black},
		basicstyle=\ttfamily\footnotesize,
		breakatwhitespace=false,         
		breaklines=true,                 
		captionpos=b,                    
		keepspaces=true,                 
		numbers=left,                    
		numbersep=5pt,                  
		showspaces=false,                
		showstringspaces=false,
		showtabs=false,                  
		tabsize=2
}

\lstdefinestyle{shellstyle}{
		backgroundcolor=\color{black!5}, 
		basicstyle=\ttfamily\footnotesize\color{black}, 
		commentstyle=\color{black}, 
		keywordstyle=\color{black},
		numberstyle=\color{black!5},
		stringstyle=\color{black}, 
		showspaces=false,
		showstringspaces=false, 
		showtabs=false, 
		tabsize=2, 
		numbers=none, 
		breaklines=true
}

\lstdefinelanguage{javascript}{
	keywords={typeof, new, true, false, catch, function, return, null, catch, switch, var, if, in, while, do, else, case, break},
	keywordstyle=\color{blue}\bfseries,
	ndkeywords={class, export, boolean, throw, implements, import, this},
	ndkeywordstyle=\color{darkgray}\bfseries,
	identifierstyle=\color{black},
	sensitive=false,
	comment=[l]{//},
	morecomment=[s]{/*}{*/},
	commentstyle=\color{purple}\ttfamily,
	stringstyle=\color{red}\ttfamily,
	morestring=[b]',
	morestring=[b]"
}

% disponi sezioni
\usepackage{titlesec}

\titleformat{\section}
	{\sffamily\Large\bfseries} 
	{\thesection}{1em}{} 
\titleformat{\subsection}
	{\sffamily\large\bfseries}   
	{\thesubsection}{1em}{} 
\titleformat{\subsubsection}
	{\sffamily\normalsize\bfseries} 
	{\thesubsubsection}{1em}{}

% disponi alberi
\usepackage{forest}

\forestset{
	rectstyle/.style={
		for tree={rectangle,draw,font=\large\sffamily}
	},
	roundstyle/.style={
		for tree={circle,draw,font=\large}
	}
}

% disponi algoritmi
\usepackage{algorithm}
\usepackage{algorithmic}
\makeatletter
\renewcommand{\ALG@name}{Algoritmo}
\makeatother

% disponi numeri di pagina
\usepackage{fancyhdr}
\fancyhf{} 
\fancyfoot[L]{\sffamily{\thepage}}

\makeatletter
\fancyhead[L]{\raisebox{1ex}[0pt][0pt]{\sffamily{\@title \ \@date}}} 
\fancyhead[R]{\raisebox{1ex}[0pt][0pt]{\sffamily{\@author}}}
\makeatother

\begin{document}

% sezione (data)
\section{Lezione del 06-03-25}

% stili pagina
\thispagestyle{empty}
\pagestyle{fancy}

% testo
\subsection{Diritto delle società}
Il \textbf{diritto} è un termine usato con due accezioni differenti:
\begin{itemize}
	\item La prima accezione intende il diritto come il complesso delle \textbf{norme giuridiche} che regolano la vita dei membri di una comunità, quindi l'\textbf{ordinamento giuridico}. In questo viene detto anche \textbf{diritto oggettivo};
	\item La seconda accezione intende il diritto come la facoltà (o il potere) garantito dall'\textit{ordinamento giuridico} ad un soggetto. In questo viene detto anche \textbf{diritto soggettivo}.
\end{itemize}

L'ordinamento giuridico si divide ulteriormente in:
\begin{itemize}
	\item \textbf{Diritto privato:} è l'insieme delle norme giuridiche che regolano i rapporti fra i \textit{privati}: questi possono essere rappresentati da \textit{atti personali} (\textbf{diritto civile}) o \textit{atti di commercio} (\textbf{diritto commerciale});
	\item \textbf{Diritto pubblico:} ha per oggetto il funzionamento dello stato e degli \textit{enti territoriali}, e i rapporti che questi hanno con i cittadini.
\end{itemize}

Il diritto commerciale si divide poi ulteriormente in innumerevoli categorie, fra cui:
\begin{itemize}
	\item Diritto societario;
	\item Diritto fallimentare;
	\item Antitrust;
	\item Diritto del lavoro;
	\item ecc...
\end{itemize}

A noi in particolare sarà di interesse il diritto oggettivo, privato,  commerciale e societario.

\subsubsection{Norme giuridiche}
Una norma giuridica rappresenta una regola di condotta, o comunque un precetto che stabilisce un comportamento condivisio da una comunità.
Le norme sono contenute solitamente in testi, detti \textbf{testi normativi}.
I testi normativi rappresentano le \textbf{fonti del diritto}.
L'ordine di importanza delle fonti del diritto è determinato dalla \textbf{gerarchia delle fonti}:
\begin{itemize}
	\item Norme di \textbf{primo livello}: la costituzione e le leggi costituzionali, i regolamenti comunitari;
	\item Norme di \textbf{secondo livello}: le leggi dello stato (fatte dal \textit{parlamento}), i decreti legge (fatti dal \textit{governo} e temporanei) e i decreti legislativi (fatti sempre dal \textit{governo}, ma su delega del parlamento), il referendum abrogativo;
	\item Norme di \textbf{terzo livello}: regolamenti governativi e degli enti locali, usi e consuetudini.
\end{itemize}

\subsubsection{Giurisprudenza}
La \textbf{giurisprudenza} è la disciplina che studia il diritto e la sua \textit{interpretazione}.
La \textbf{dottrina giuridica} è invece l'attività di studio scientifico del diritto.

Gli organi giudicanti dello stato (la magistratura) interpreta e applica le leggi, processo da cui risultano le \textbf{sentenze}.
Esistono in questo 2 sistemi:
\begin{itemize}
	\item \textbf{Common law:} le sentenze dei tribunali fanno \textit{precedente} e vengono prese ad esempio;
	\item \textbf{Civil law:} si basa invece principalmente su codici prestabiliti e non sentenze precedenti.
\end{itemize}

Il \textbf{diritto societario} è e definito con leggi e raccolte di leggi (detto \textbf{codice civile}).
Il codice civile distingue diversi tipi di imprese in base a 3 criteri:
\begin{itemize}
	\item \textbf{Oggetto} dell'impresa (imprenditore agricolo, imprenditore commerciale, ecc...);
	\item \textbf{Dimensione} dell'impresa (piccolo imprenditore, grande imprenditore, ecc...);
	\item \textbf{Natura del soggetto} che esercita l'impresa (impesa individuale, collettiva, ecc...).
\end{itemize}

Non si trova quindi nel codice civile una definizione propria di \textit{impresa}, ma di \textbf{imprenditore}: è imprenditore chi esercita \textbf{professionalmente} un'\textbf{attività economica organizzata} al fine della \textbf{produzione} e dello \textbf{scambio} di \textbf{beni} o \textbf{servizio}.

Dove:
\begin{itemize}
	\item \textbf{Organizzata} rappresenta l'impiego coordinato di fattori produttivi propri o altrui;
	\item L'\textbf{economicità} riguarda il metodo con cui l'attività è svolta;
	\item E \textbf{professionalità} rappresenta l'esercizio abituale dell'attivita produttiva.
\end{itemize}

Questa definizione è quella del codice civile, e differisce quindi da quella che si può trovare nel diritto tributario, ecc... 

Si trova poi una definizione di \textbf{azienda}, cioè l'insieme dei beni organizzati dall'imprenditore per l'esercizio dell'impresa.

I requisiti \textit{civilistici} dell'imprenditore sono quindi quelli dello svolgimento di un \textbf{attività produttiva}, che come abbiamo visto è alla base dell'impresa.

\end{document}


\documentclass[a4paper,11pt]{article}
\usepackage[a4paper, margin=8em]{geometry}

% usa i pacchetti per la scrittura in italiano
\usepackage[french,italian]{babel}
\usepackage[T1]{fontenc}
\usepackage[utf8]{inputenc}
\frenchspacing 

% usa i pacchetti per la formattazione matematica
\usepackage{amsmath, amssymb, amsthm, amsfonts}

% usa altri pacchetti
\usepackage{gensymb}
\usepackage{hyperref}
\usepackage{standalone}

% imposta il titolo
\title{Appunti Economia ed Organizzazione Aziendale}
\author{Luca Seggiani}
\date{2025}

% disegni
\usepackage{pgfplots}
\pgfplotsset{width=10cm,compat=1.9}

% imposta lo stile
% usa helvetica
\usepackage[scaled]{helvet}
% usa palatino
\usepackage{palatino}
% usa un font monospazio guardabile
\usepackage{lmodern}

\renewcommand{\rmdefault}{ppl}
\renewcommand{\sfdefault}{phv}
\renewcommand{\ttdefault}{lmtt}

% disponi il titolo
\makeatletter
\renewcommand{\maketitle} {
	\begin{center} 
		\begin{minipage}[t]{.8\textwidth}
			\textsf{\huge\bfseries \@title} 
		\end{minipage}%
		\begin{minipage}[t]{.2\textwidth}
			\raggedleft \vspace{-1.65em}
			\textsf{\small \@author} \vfill
			\textsf{\small \@date}
		\end{minipage}
		\par
	\end{center}

	\thispagestyle{empty}
	\pagestyle{fancy}
}
\makeatother

% disponi teoremi
\usepackage{tcolorbox}
\newtcolorbox[auto counter, number within=section]{theorem}[2][]{%
	colback=blue!10, 
	colframe=blue!40!black, 
	sharp corners=northwest,
	fonttitle=\sffamily\bfseries, 
	title=Teorema~\thetcbcounter: #2, 
	#1
}

% disponi definizioni
\newtcolorbox[auto counter, number within=section]{definition}[2][]{%
	colback=red!10,
	colframe=red!40!black,
	sharp corners=northwest,
	fonttitle=\sffamily\bfseries,
	title=Definizione~\thetcbcounter: #2,
	#1
}

% disponi problemi
\newtcolorbox[auto counter, number within=section]{problem}[2][]{%
	colback=green!10,
	colframe=green!40!black,
	sharp corners=northwest,
	fonttitle=\sffamily\bfseries,
	title=Problema~\thetcbcounter: #2,
	#1
}

% disponi codice
\usepackage{listings}
\usepackage[table]{xcolor}

\lstdefinestyle{codestyle}{
		backgroundcolor=\color{black!5}, 
		commentstyle=\color{codegreen},
		keywordstyle=\bfseries\color{magenta},
		numberstyle=\sffamily\tiny\color{black!60},
		stringstyle=\color{green!50!black},
		basicstyle=\ttfamily\footnotesize,
		breakatwhitespace=false,         
		breaklines=true,                 
		captionpos=b,                    
		keepspaces=true,                 
		numbers=left,                    
		numbersep=5pt,                  
		showspaces=false,                
		showstringspaces=false,
		showtabs=false,                  
		tabsize=2
}

\lstdefinestyle{shellstyle}{
		backgroundcolor=\color{black!5}, 
		basicstyle=\ttfamily\footnotesize\color{black}, 
		commentstyle=\color{black}, 
		keywordstyle=\color{black},
		numberstyle=\color{black!5},
		stringstyle=\color{black}, 
		showspaces=false,
		showstringspaces=false, 
		showtabs=false, 
		tabsize=2, 
		numbers=none, 
		breaklines=true
}

\lstdefinelanguage{javascript}{
	keywords={typeof, new, true, false, catch, function, return, null, catch, switch, var, if, in, while, do, else, case, break},
	keywordstyle=\color{blue}\bfseries,
	ndkeywords={class, export, boolean, throw, implements, import, this},
	ndkeywordstyle=\color{darkgray}\bfseries,
	identifierstyle=\color{black},
	sensitive=false,
	comment=[l]{//},
	morecomment=[s]{/*}{*/},
	commentstyle=\color{purple}\ttfamily,
	stringstyle=\color{red}\ttfamily,
	morestring=[b]',
	morestring=[b]"
}

% disponi sezioni
\usepackage{titlesec}

\titleformat{\section}
	{\sffamily\Large\bfseries} 
	{\thesection}{1em}{} 
\titleformat{\subsection}
	{\sffamily\large\bfseries}   
	{\thesubsection}{1em}{} 
\titleformat{\subsubsection}
	{\sffamily\normalsize\bfseries} 
	{\thesubsubsection}{1em}{}

% disponi alberi
\usepackage{forest}

\forestset{
	rectstyle/.style={
		for tree={rectangle,draw,font=\large\sffamily}
	},
	roundstyle/.style={
		for tree={circle,draw,font=\large}
	}
}

% disponi algoritmi
\usepackage{algorithm}
\usepackage{algorithmic}
\makeatletter
\renewcommand{\ALG@name}{Algoritmo}
\makeatother

% disponi numeri di pagina
\usepackage{fancyhdr}
\fancyhf{} 
\fancyfoot[L]{\sffamily{\thepage}}

\makeatletter
\fancyhead[L]{\raisebox{1ex}[0pt][0pt]{\sffamily{\@title \ \@date}}} 
\fancyhead[R]{\raisebox{1ex}[0pt][0pt]{\sffamily{\@author}}}
\makeatother

\begin{document}

% sezione (data)
\section{Lezione del 12-03-25}

% stili pagina
\thispagestyle{empty}
\pagestyle{fancy}

% testo
Proseguiamo quindi lo studio dal punto di vista giuridico della figura dell'\textit{imprenditore}, e quindi dell'\textit{impresa}.

\subsection{Statuto dell'imprenditore}
L'obiettivo del testo giuridico che regola l'impresa, cioè lo \textbf{statuto dell'imprenditore}, è quello di definire il \textit{perimetro} all'interno del cui l'impresa può muoversi.

Bisogna innanzitutto distinguere l'\textbf{oggetto} dell'impresa in:
\begin{itemize}
	\item Impresa commerciale (cioè tutto ciò che non è agricolo);
	\item Impresa agricola.
\end{itemize}
Si può poi distinguere la \textbf{dimensione} dell'impresa:
\begin{itemize}
	\item Piccola impresa; 
	\item Medio/grande impresa.
\end{itemize}
Infine, si può distinguere sulla natura del \textbf{soggetto}:
\begin{itemize}
	\item Impresa individuale;
	\item Impresa collettiva, che si divide a sua volta in:
		\begin{itemize}
			\item Impresa societaria;
			\item Impresa pubblica;
			\item Fondazioni e associazioni.
		\end{itemize}
\end{itemize}

\subsubsection{Registro delle imprese}
Il \textbf{registro delle imprese} è uno strumento, istituito dalle camere di commercio (enti pubblici locali non territoriali, dotati di autonomia funzionale), che tiene conto di tutti gli \textit{atti} (costituzione dell'impresa, richieste di finanziamento da terzi, ec...) e i \textit{fatti} riguardanti le imprese iscritte.
Dal registro delle imprese si può attingere ai \textit{prospetti ufficiali} dell'impresa (bilanci, storici, ecc...) sotto versamento di una certa somma.
L'iscrizione è solitamente \textbf{dichiaratava}, anche se in alcuni casi è \textit{costitutiva} (S.p.A.) o solamente \textit{pubblicitaria} (piccoli imprenditori).

L'esistenza di un registro delle imprese ha diversi effetti dal punto di vista legale:
\begin{itemize}
	\item I fatti dichiarati pubblicamente sono assunti noti da terzi;
	\item Di contro, i fatti non dichiarati non sono opponibili a terzi (a meno di non dimostrarne la conoscenza).
\end{itemize}

\subsubsection{Scritture contabili}
Le scritture contabili sono documenti che contengono i singoli atti dell'impresa, che l'impresa è obbligata a tenere (e conservare per una durata di 10 anni).
Queste includono:
\begin{itemize}
	\item Tutte le scritture richieste dalla natura e dalla dimensione dell'impresa (libro mastro, di cassa, di magazzino, ecc...);
	\item \textbf{Libro giornale:} registro cronologico-analitico;
	\item \textbf{Libro degli inventari:} registro periodico-sistematico;
	\item \textbf{Originali} della corrispondenza commerciale ricevuta e \textbf{copie} di quella spedita.
\end{itemize}

Gli effetti legali delle scritture contabili sono considerevoli in quanto queste rappresentano prova, con efficacia \textbf{probatoria}.

\subsubsection{Ausiliari dell'imprenditore}
Gli ausiliari dell'impresa sono divisi in due categorie:
\begin{itemize}
	\item \textbf{Ausiliari subordinati:} cioè subordinati all'imprenditore da un rapporto di lavoro. Questi possono essere:
		\begin{itemize}
			\item \textbf{Institori:} coloro che sono preposti dal titolare all'esercizio dell'impresa commerciale, stanno all'\textit{interno} della gerarchia organizzativa in ruoli abbastanza stabili, e solitamente dotati di un certo potere;
			\item \textbf{Procuratori:} svolgono sempre l'opera di esercizio dell'impresa, ma sono sottoposti agli institori, e legati al titolare da un rapporto di \textbf{procura} (più flessibile rispetto al ruolo gerarchico degli institori). Questo significa che il loro potere non deriva dalla loro posozione gerarchica, ma solo dal rapporto di procura col titolare;
			\item \textbf{Commessi:} compiono gli incarichi gli atti che ordinariamente comporta la specie di operazioni delle quali sono incaricati.
		\end{itemize}
	\item \textbf{Ausiliari autonomi:} chiamati informalmente \textit{partite IVA}, legati all'imprenditore da un rapporto di prestazione d'opera (restano quindi indipendenti).
\end{itemize}

Gli ausiliari possono concludere affari con terzi per conto dell'imprenditore, e quindi si pone il problema della \textbf{rappresentanza}.
In generale si richiede il conferimento della \textbf{procura}, cioè il potere di rappresentanza esiste nei limiti fissati dalla procura.

Una regola speciale è rappresentata dalla \textbf{rappresentanza commerciale}: gli ausiliari subordinati sono automaticamente investiti dal potere di rappresentanza, per la sola natura della loro attività. Chiaramente, ogni figura ha il suo livello di rappresentanza (il commesso avrà meno potere di rappresentanza di un procuratore o un institore).

\subsection{Azienda}
Fino ad ora abbiamo parlato d'impresa, e di riflesso della figura dell'imprenditore.
Vediamo adesso alla definizione dell'\textbf{azienda}, intesa come \textit{il complesso dei beni organizzati dall'imprenditore all'esercizio dell'impresa}.

In questo, ad esempio, i \textbf{liberi professionisti} non possono considerarsi azienda, in quanto il libero professionista stesso non è imprenditore: diventa tale solo nel caso in cui svolga un'attività che di per sé è un'attivita d'impresa, e non la sua sola prestazione intellettuale.

Si possono quindi distinguere 3 segni distintivi dell'azienda:
\begin{itemize}
	\item \textbf{Ditta:} il nome sotto il quale l'imprenditore esercita l'attività d'impresa.
		E' un segno necessario, e deve rispettare 2 principi: \textbf{verità} e \textbf{novità}. 
		
		Si applica in questo caso di sole imprese individuali.
		Nel caso di imprese societarie si parla invece di \textbf{ragione sociale} (società di persone, deve contenere il nome di almeno un socio a responsabilità illimitata) o \textbf{denominazione sociale} (società di capitali, non ha limitazioni riguardanti i nomi dei soci).

		Dovrà quindi contenere il cognome o la sigla dell'imprenditore (altrimenti si considera irregolare e rientra nei segni distintivi atipici).
		La ditta può inoltre rimanere dopo la cessione dell'attività a terzi, in quanto il principio di verità non viene invalidato (questo non significa che la ditta non può comunque essere cambiata);

	\item \textbf{Insegna:} segno distintivo del locale nel quale si svolge l'attività di imprenditore, può corrispondere o non corrispondere alla ditta. Deve avere \textbf{liceità}, \textbf{veridicità} ed \textbf{originalità};

	\item \textbf{Marchio:} segno distintivo del prodotto o del servizio fornito dall'impresa. Qui si può distinguere in:
		\begin{itemize}
			\item \textbf{Marchio di fabbrica:} apposto dal produttore;
			\item \textbf{Marchio di commercio:} apposto da colui che commercializza il prodotto;
			\item \textbf{Marchio di forma:} il marchio può essere rappresentato da diverse qualità, tra cui una sigla, un motto, un immagine, un \textit{jingle} musicale o la \textit{forma} stessa del prodotto distinguiamo infatti marchi \textbf{denominativi}, \textbf{figurativi} o \textbf{misti};
		\end{itemize}
		I requisiti del marchio sono \textbf{originalità}, \textbf{novità}, \textbf{conformità} e la non violazione dei diritti esclusivi di terzi (ad esempio, i \textbf{diritti d'autore}).

		Il \textbf{diritto all'uso} del marchio da parte di un azienda si acquista con:
		\begin{itemize}
			\item \textbf{Registrazione} del marchio, i marchi celebri sono più protetti. Un marchio si può perdere per \textbf{volgarizzazione}, cioè quando diventa una parola di uso comune;
			\item \textbf{Uso di fatto}. 
		\end{itemize}

		Il marchio può infine essere ceduto, concesso in licenza o in merchandising.
		
\end{itemize}

\subsubsection{Diritti di privativa}
I diritti di privativa sono la categoria di cui fanno parte il \textbf{diritto di autore}, il \textbf{diritto di inventore} e il \textbf{brevetto}.
\begin{itemize}
	\item \textbf{Diritto d'autore:} si applica a beni immateriali (opere dell'ingegno di carattere creativo), e si distinguno nel:
		\begin{itemize}
			\item \textbf{Diritto morale} d'autore, cioè la paternità dell'opera;
			\item \textbf{Diritto patrimoniale} d'autore, cioè il diritto di pubblicare l'opera e utilizzarla economicamente.
		\end{itemize}
	\item \textbf{Diritto d'inventore:} si tratta di idee creative che appartengono al campo della tecnica. E' caratterizzato da \textbf{industrialità}, \textbf{liceità}, e infine \textbf{novità intrinseca} e \textbf{novità estrinseca}, cioè rispettivamente la capacità di incrementare il patrimonio tecnico presente (\textit{intrinseca}) e la mancata divulgazione (\textit{estrinseca});
	\item \textbf{Brevetto:} mezzo attraverso il quale si rende \textit{invenzione} cpò che prima era pubblico dominio, e quindi permette all'inventore di capitalizzare sulla sua opera. Decade, può essere espropiato e concesso in licenza, ed è trasferibile \textit{inter vivos} o \textit{mortis causa}.
\end{itemize}

\subsection{Società}
Entriamo quindi nel dettaglio delle società, cioè delle imprese collettive.
In particolare, una società è un contratto, attraverso il quale \textit{due o più persone conferiscono beni o servizi per l'esercizio in comune di un'attività economica allo scopo di dividerne gli utili}.

Il \textit{conferimento} rappresenta le prestazioni in cui le parti della società si obbligano.
Dal punto di vista pratico, questo è rappresentato semplicemente da mezzi finanziari, o mezzi di produzione, immobili, credito, ecc...
Il conferimento del lavoro (\textbf{socio d'opera}) varia invece di società in società.
L'\textit{esercizio in comune} è preordinato alla realizzazione di un risultato unico, nella prospettiva della \textit{divisione degli utili}. 

\subsubsection{Scopo della società}
Si può distinguere in diverse categorie di scopo della società:
\begin{itemize}
	\item \textbf{Lucrativo:} svolgimento dell'attività d'impresa per \textit{produrre utile} (\textbf{lucro oggettivo}) destinato ad essere diviso fra i soci (\textbf{lucro soggettivo});
	\item \textbf{Mutualistico:} tipico delle \textit{società cooperative}, atto a fornire beni o servizi od occasioni di lavoro, direttamente ai soci, a condizioni più vantaggiose di quelle che otterrebbero sul mercato;
	\item \textbf{Consortile:} vantaggio patrimoniale diretto, riguarda i consorzi fra due o più imprese.
\end{itemize}

\subsubsection{Tipi di società}
Ripercorriamo quindi i vari tipi di società:
\begin{itemize}
	\item \textbf{Lucrative:}
		\begin{itemize}
			\item \textbf{Società di persone:}
				\begin{itemize}
					\item Società semplice (S.s.) (solo agricole);
					\item Società in nome collettivo (S.n.c.);
					\item Soietà in accomandita semplice (S.a.s.).
				\end{itemize}
			\item \textbf{Società di capitali:}
				\begin{itemize}
					\item Società in accomandita per azioni (S.a.p.a.);
					\item Società per azioni (S.p.A.);
					\item Società a responsabilità limitata (S.r.l).
				\end{itemize}
		\end{itemize}
	\item \textbf{Mutualistiche:}
		\begin{itemize}
			\item \textbf{Società cooperative:}
				\begin{itemize}
					\item Società cooperativa a responsabilità limitata;
					\item Società cooperativa a responsabilità illimitata;
				\end{itemize}
			\item Mutue assicuratrici.
		\end{itemize}
	\item \textbf{Consortile:} ne possono far parte tutti i tipi tranne la società semplice.
\end{itemize}

Inoltre possiamo distinguere tra società \textbf{for profit} e \textbf{not for profit}, nonché la categoria ibrida delle \textit{società benefit}.

Nelle società in \textit{accomandita} (S.a.s. e S.a.p.a.), si può distinguere fra soci \textbf{accomandanti} e soci \textbf{accomandatari}.
I \textit{soci accomandanti} sono del tutto identici ai soci delle \textbf{società di persone}, mentre i \textit{soci accomandatari} sono del tutto identici ai soci delle \textbf{società di capitali}.

Una delle differenze fra le società di persone e di capitali è rappresentata dalla \textbf{personalità giuridica}.
Le società di capitali hanno personalità giuridica: risponde la società con il proprio capitale, mentre nelle società di persone rispondono i soci con il \textit{loro} capitale.

Fondamentalmente il patrimonio dei soci e il patrimonio della società nelle società di capitali sono fra di loro completamente separati: i creditori personali non possono aggredire il patrimonio sociale, e viceversa i creditori sociali non possono aggredire il patrimonio personale.

Nelle società di persone vale invece l'\textbf{autonomia patrimoniale}, che possiamo intendere come una versione più debole della personalità giuridica.
In questo caso i creditori della società non possono comunque aggredire \textit{direttamente} il patrimonio personale dei soci, quando questi sono illimitatamente responsabili (\textit{beneficio di escussione}).
I creditori personali possono invece, solo nelle società semplici, chiedere la liquidazione della quota in società dei loro debitori, cioè che questa venga venduta.
Negli altri tipi di società, ai soci a responsabilità limitata si può chiedere soltanto il \textit{sequestro conservativo} in caso di liquidazione (per altri motivi) della società.

I patrimoni dei soci e della società nel caso delle società di persone sono quindi \textit{relativamente} separati fra di loro: esistono modalità secondo le quali possono finire a mescolarsi.

\end{document}

\end{document}