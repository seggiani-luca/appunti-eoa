
\documentclass[a4paper,11pt]{article}
\usepackage[a4paper, margin=8em]{geometry}

% usa i pacchetti per la scrittura in italiano
\usepackage[french,italian]{babel}
\usepackage[T1]{fontenc}
\usepackage[utf8]{inputenc}
\frenchspacing 

% usa i pacchetti per la formattazione matematica
\usepackage{amsmath, amssymb, amsthm, amsfonts}

% usa altri pacchetti
\usepackage{gensymb}
\usepackage{hyperref}
\usepackage{standalone}

% cose fluttuanti
\usepackage{float}

% imposta il titolo
\title{Appunti Economia ed Organizzazione Aziendale}
\author{Luca Seggiani}
\date{2025}

% disegni
\usepackage{pgfplots}
\pgfplotsset{width=10cm,compat=1.9}

% imposta lo stile
% usa helvetica
\usepackage[scaled]{helvet}
% usa palatino
\usepackage{palatino}
% usa un font monospazio guardabile
\usepackage{lmodern}

\renewcommand{\rmdefault}{ppl}
\renewcommand{\sfdefault}{phv}
\renewcommand{\ttdefault}{lmtt}

% disponi il titolo
\makeatletter
\renewcommand{\maketitle} {
	\begin{center} 
		\begin{minipage}[t]{.8\textwidth}
			\textsf{\huge\bfseries \@title} 
		\end{minipage}%
		\begin{minipage}[t]{.2\textwidth}
			\raggedleft \vspace{-1.65em}
			\textsf{\small \@author} \vfill
			\textsf{\small \@date}
		\end{minipage}
		\par
	\end{center}

	\thispagestyle{empty}
	\pagestyle{fancy}
}
\makeatother

% disponi teoremi
\usepackage{tcolorbox}
\newtcolorbox[auto counter, number within=section]{theorem}[2][]{%
	colback=blue!10, 
	colframe=blue!40!black, 
	sharp corners=northwest,
	fonttitle=\sffamily\bfseries, 
	title=Teorema~\thetcbcounter: #2, 
	#1
}

% disponi definizioni
\newtcolorbox[auto counter, number within=section]{definition}[2][]{%
	colback=red!10,
	colframe=red!40!black,
	sharp corners=northwest,
	fonttitle=\sffamily\bfseries,
	title=Definizione~\thetcbcounter: #2,
	#1
}

% disponi problemi
\newtcolorbox[auto counter, number within=section]{problem}[2][]{%
	colback=green!10,
	colframe=green!40!black,
	sharp corners=northwest,
	fonttitle=\sffamily\bfseries,
	title=Problema~\thetcbcounter: #2,
	#1
}

% disponi codice
\usepackage{listings}
\usepackage[table]{xcolor}

\lstdefinestyle{codestyle}{
		backgroundcolor=\color{black!5}, 
		commentstyle=\color{codegreen},
		keywordstyle=\bfseries\color{magenta},
		numberstyle=\sffamily\tiny\color{black!60},
		stringstyle=\color{green!50!black},
		basicstyle=\ttfamily\footnotesize,
		breakatwhitespace=false,         
		breaklines=true,                 
		captionpos=b,                    
		keepspaces=true,                 
		numbers=left,                    
		numbersep=5pt,                  
		showspaces=false,                
		showstringspaces=false,
		showtabs=false,                  
		tabsize=2
}

\lstdefinestyle{shellstyle}{
		backgroundcolor=\color{black!5}, 
		basicstyle=\ttfamily\footnotesize\color{black}, 
		commentstyle=\color{black}, 
		keywordstyle=\color{black},
		numberstyle=\color{black!5},
		stringstyle=\color{black}, 
		showspaces=false,
		showstringspaces=false, 
		showtabs=false, 
		tabsize=2, 
		numbers=none, 
		breaklines=true
}

\lstdefinelanguage{javascript}{
	keywords={typeof, new, true, false, catch, function, return, null, catch, switch, var, if, in, while, do, else, case, break},
	keywordstyle=\color{blue}\bfseries,
	ndkeywords={class, export, boolean, throw, implements, import, this},
	ndkeywordstyle=\color{darkgray}\bfseries,
	identifierstyle=\color{black},
	sensitive=false,
	comment=[l]{//},
	morecomment=[s]{/*}{*/},
	commentstyle=\color{purple}\ttfamily,
	stringstyle=\color{red}\ttfamily,
	morestring=[b]',
	morestring=[b]"
}

% disponi sezioni
\usepackage{titlesec}

\titleformat{\section}
	{\sffamily\Large\bfseries} 
	{\thesection}{1em}{} 
\titleformat{\subsection}
	{\sffamily\large\bfseries}   
	{\thesubsection}{1em}{} 
\titleformat{\subsubsection}
	{\sffamily\normalsize\bfseries} 
	{\thesubsubsection}{1em}{}

% disponi alberi
\usepackage{forest}

\forestset{
	rectstyle/.style={
		for tree={rectangle,draw,font=\large\sffamily}
	},
	roundstyle/.style={
		for tree={circle,draw,font=\large}
	}
}

% disponi algoritmi
\usepackage{algorithm}
\usepackage{algorithmic}
\makeatletter
\renewcommand{\ALG@name}{Algoritmo}
\makeatother

% disponi numeri di pagina
\usepackage{fancyhdr}
\fancyhf{} 
\fancyfoot[L]{\sffamily{\thepage}}

\makeatletter
\fancyhead[L]{\raisebox{1ex}[0pt][0pt]{\sffamily{\@title \ \@date}}} 
\fancyhead[R]{\raisebox{1ex}[0pt][0pt]{\sffamily{\@author}}}
\makeatother

\begin{document}

\pagestyle{fancy}
\thispagestyle{empty}
\renewcommand{\thispagestyle}[1]{}

\maketitle

\documentclass[a4paper,11pt]{article}
\usepackage[a4paper, margin=8em]{geometry}

% usa i pacchetti per la scrittura in italiano
\usepackage[french,italian]{babel}
\usepackage[T1]{fontenc}
\usepackage[utf8]{inputenc}
\frenchspacing 

% usa i pacchetti per la formattazione matematica
\usepackage{amsmath, amssymb, amsthm, amsfonts}

% usa altri pacchetti
\usepackage{gensymb}
\usepackage{hyperref}
\usepackage{standalone}

% imposta il titolo
\title{Appunti Economia ed Organizzazione Aziendale}
\author{Luca Seggiani}
\date{2025}

% disegni
\usepackage{pgfplots}
\pgfplotsset{width=10cm,compat=1.9}

% imposta lo stile
% usa helvetica
\usepackage[scaled]{helvet}
% usa palatino
\usepackage{palatino}
% usa un font monospazio guardabile
\usepackage{lmodern}

\renewcommand{\rmdefault}{ppl}
\renewcommand{\sfdefault}{phv}
\renewcommand{\ttdefault}{lmtt}

% disponi il titolo
\makeatletter
\renewcommand{\maketitle} {
	\begin{center} 
		\begin{minipage}[t]{.8\textwidth}
			\textsf{\huge\bfseries \@title} 
		\end{minipage}%
		\begin{minipage}[t]{.2\textwidth}
			\raggedleft \vspace{-1.65em}
			\textsf{\small \@author} \vfill
			\textsf{\small \@date}
		\end{minipage}
		\par
	\end{center}

	\thispagestyle{empty}
	\pagestyle{fancy}
}
\makeatother

% disponi teoremi
\usepackage{tcolorbox}
\newtcolorbox[auto counter, number within=section]{theorem}[2][]{%
	colback=blue!10, 
	colframe=blue!40!black, 
	sharp corners=northwest,
	fonttitle=\sffamily\bfseries, 
	title=Teorema~\thetcbcounter: #2, 
	#1
}

% disponi definizioni
\newtcolorbox[auto counter, number within=section]{definition}[2][]{%
	colback=red!10,
	colframe=red!40!black,
	sharp corners=northwest,
	fonttitle=\sffamily\bfseries,
	title=Definizione~\thetcbcounter: #2,
	#1
}

% disponi problemi
\newtcolorbox[auto counter, number within=section]{problem}[2][]{%
	colback=green!10,
	colframe=green!40!black,
	sharp corners=northwest,
	fonttitle=\sffamily\bfseries,
	title=Problema~\thetcbcounter: #2,
	#1
}

% disponi codice
\usepackage{listings}
\usepackage[table]{xcolor}

\lstdefinestyle{codestyle}{
		backgroundcolor=\color{black!5}, 
		commentstyle=\color{codegreen},
		keywordstyle=\bfseries\color{magenta},
		numberstyle=\sffamily\tiny\color{black!60},
		stringstyle=\color{green!50!black},
		basicstyle=\ttfamily\footnotesize,
		breakatwhitespace=false,         
		breaklines=true,                 
		captionpos=b,                    
		keepspaces=true,                 
		numbers=left,                    
		numbersep=5pt,                  
		showspaces=false,                
		showstringspaces=false,
		showtabs=false,                  
		tabsize=2
}

\lstdefinestyle{shellstyle}{
		backgroundcolor=\color{black!5}, 
		basicstyle=\ttfamily\footnotesize\color{black}, 
		commentstyle=\color{black}, 
		keywordstyle=\color{black},
		numberstyle=\color{black!5},
		stringstyle=\color{black}, 
		showspaces=false,
		showstringspaces=false, 
		showtabs=false, 
		tabsize=2, 
		numbers=none, 
		breaklines=true
}

\lstdefinelanguage{javascript}{
	keywords={typeof, new, true, false, catch, function, return, null, catch, switch, var, if, in, while, do, else, case, break},
	keywordstyle=\color{blue}\bfseries,
	ndkeywords={class, export, boolean, throw, implements, import, this},
	ndkeywordstyle=\color{darkgray}\bfseries,
	identifierstyle=\color{black},
	sensitive=false,
	comment=[l]{//},
	morecomment=[s]{/*}{*/},
	commentstyle=\color{purple}\ttfamily,
	stringstyle=\color{red}\ttfamily,
	morestring=[b]',
	morestring=[b]"
}

% disponi sezioni
\usepackage{titlesec}

\titleformat{\section}
	{\sffamily\Large\bfseries} 
	{\thesection}{1em}{} 
\titleformat{\subsection}
	{\sffamily\large\bfseries}   
	{\thesubsection}{1em}{} 
\titleformat{\subsubsection}
	{\sffamily\normalsize\bfseries} 
	{\thesubsubsection}{1em}{}

% disponi alberi
\usepackage{forest}

\forestset{
	rectstyle/.style={
		for tree={rectangle,draw,font=\large\sffamily}
	},
	roundstyle/.style={
		for tree={circle,draw,font=\large}
	}
}

% disponi algoritmi
\usepackage{algorithm}
\usepackage{algorithmic}
\makeatletter
\renewcommand{\ALG@name}{Algoritmo}
\makeatother

% disponi numeri di pagina
\usepackage{fancyhdr}
\fancyhf{} 
\fancyfoot[L]{\sffamily{\thepage}}

\makeatletter
\fancyhead[L]{\raisebox{1ex}[0pt][0pt]{\sffamily{\@title \ \@date}}} 
\fancyhead[R]{\raisebox{1ex}[0pt][0pt]{\sffamily{\@author}}}
\makeatother

\begin{document}

% sezione (data)
\section{Lezione del 26-02-25}

% stili pagina
\thispagestyle{empty}
\pagestyle{fancy}

% testo
\subsection{Introduzione al corso}
L'economia riguarda il modo in cui gli agenti economici giungono a compiere scelte ottime in presenza di risorse limitate.
Tratta di attività di \textbf{produzione}, \textbf{scambi} e \textbf{consumo} di \textit{beni} atti a \textbf{soddisfare bisogni} della società.
Le variabili su cui agire sono quindi:
\begin{itemize}
	\item Quali bene produrre o consumare;
	\item Come produrli (metodi di produzione, ripartizione delle risorse);
	\item Per chi produrli (mercati).
\end{itemize}

Ci sono due approcci principali al problema:
\begin{itemize}
	\item \textbf{Macro economia:} fenomeni a livello di sistema come sviiluppo, occupazione, inflazione, ecc...
	\item \textbf{Micro economia:} modelli di comportamento per produttori e consumatori, forme di mercato, ecc...
\end{itemize}

L'economia aziendale presenta un approccio dal basso all'economia, trattando la nascita, la struttura e lo sviluppo di un impresa.
\end{document}


\documentclass[a4paper,11pt]{article}
\usepackage[a4paper, margin=8em]{geometry}

% usa i pacchetti per la scrittura in italiano
\usepackage[french,italian]{babel}
\usepackage[T1]{fontenc}
\usepackage[utf8]{inputenc}
\frenchspacing 

% usa i pacchetti per la formattazione matematica
\usepackage{amsmath, amssymb, amsthm, amsfonts}

% usa altri pacchetti
\usepackage{gensymb}
\usepackage{hyperref}
\usepackage{standalone}

% imposta il titolo
\title{Appunti Economia ed Organizzazione Aziendale}
\author{Luca Seggiani}
\date{2025}

% disegni
\usepackage{pgfplots}
\pgfplotsset{width=10cm,compat=1.9}

% imposta lo stile
% usa helvetica
\usepackage[scaled]{helvet}
% usa palatino
\usepackage{palatino}
% usa un font monospazio guardabile
\usepackage{lmodern}

\renewcommand{\rmdefault}{ppl}
\renewcommand{\sfdefault}{phv}
\renewcommand{\ttdefault}{lmtt}

% disponi il titolo
\makeatletter
\renewcommand{\maketitle} {
	\begin{center} 
		\begin{minipage}[t]{.8\textwidth}
			\textsf{\huge\bfseries \@title} 
		\end{minipage}%
		\begin{minipage}[t]{.2\textwidth}
			\raggedleft \vspace{-1.65em}
			\textsf{\small \@author} \vfill
			\textsf{\small \@date}
		\end{minipage}
		\par
	\end{center}

	\thispagestyle{empty}
	\pagestyle{fancy}
}
\makeatother

% disponi teoremi
\usepackage{tcolorbox}
\newtcolorbox[auto counter, number within=section]{theorem}[2][]{%
	colback=blue!10, 
	colframe=blue!40!black, 
	sharp corners=northwest,
	fonttitle=\sffamily\bfseries, 
	title=Teorema~\thetcbcounter: #2, 
	#1
}

% disponi definizioni
\newtcolorbox[auto counter, number within=section]{definition}[2][]{%
	colback=red!10,
	colframe=red!40!black,
	sharp corners=northwest,
	fonttitle=\sffamily\bfseries,
	title=Definizione~\thetcbcounter: #2,
	#1
}

% disponi problemi
\newtcolorbox[auto counter, number within=section]{problem}[2][]{%
	colback=green!10,
	colframe=green!40!black,
	sharp corners=northwest,
	fonttitle=\sffamily\bfseries,
	title=Problema~\thetcbcounter: #2,
	#1
}

% disponi codice
\usepackage{listings}
\usepackage[table]{xcolor}

\lstdefinestyle{codestyle}{
		backgroundcolor=\color{black!5}, 
		commentstyle=\color{codegreen},
		keywordstyle=\bfseries\color{magenta},
		numberstyle=\sffamily\tiny\color{black!60},
		stringstyle=\color{green!50!black},
		basicstyle=\ttfamily\footnotesize,
		breakatwhitespace=false,         
		breaklines=true,                 
		captionpos=b,                    
		keepspaces=true,                 
		numbers=left,                    
		numbersep=5pt,                  
		showspaces=false,                
		showstringspaces=false,
		showtabs=false,                  
		tabsize=2
}

\lstdefinestyle{shellstyle}{
		backgroundcolor=\color{black!5}, 
		basicstyle=\ttfamily\footnotesize\color{black}, 
		commentstyle=\color{black}, 
		keywordstyle=\color{black},
		numberstyle=\color{black!5},
		stringstyle=\color{black}, 
		showspaces=false,
		showstringspaces=false, 
		showtabs=false, 
		tabsize=2, 
		numbers=none, 
		breaklines=true
}

\lstdefinelanguage{javascript}{
	keywords={typeof, new, true, false, catch, function, return, null, catch, switch, var, if, in, while, do, else, case, break},
	keywordstyle=\color{blue}\bfseries,
	ndkeywords={class, export, boolean, throw, implements, import, this},
	ndkeywordstyle=\color{darkgray}\bfseries,
	identifierstyle=\color{black},
	sensitive=false,
	comment=[l]{//},
	morecomment=[s]{/*}{*/},
	commentstyle=\color{purple}\ttfamily,
	stringstyle=\color{red}\ttfamily,
	morestring=[b]',
	morestring=[b]"
}

% disponi sezioni
\usepackage{titlesec}

\titleformat{\section}
	{\sffamily\Large\bfseries} 
	{\thesection}{1em}{} 
\titleformat{\subsection}
	{\sffamily\large\bfseries}   
	{\thesubsection}{1em}{} 
\titleformat{\subsubsection}
	{\sffamily\normalsize\bfseries} 
	{\thesubsubsection}{1em}{}

% disponi alberi
\usepackage{forest}

\forestset{
	rectstyle/.style={
		for tree={rectangle,draw,font=\large\sffamily}
	},
	roundstyle/.style={
		for tree={circle,draw,font=\large}
	}
}

% disponi algoritmi
\usepackage{algorithm}
\usepackage{algorithmic}
\makeatletter
\renewcommand{\ALG@name}{Algoritmo}
\makeatother

% disponi numeri di pagina
\usepackage{fancyhdr}
\fancyhf{} 
\fancyfoot[L]{\sffamily{\thepage}}

\makeatletter
\fancyhead[L]{\raisebox{1ex}[0pt][0pt]{\sffamily{\@title \ \@date}}} 
\fancyhead[R]{\raisebox{1ex}[0pt][0pt]{\sffamily{\@author}}}
\makeatother

\begin{document}

% sezione (data)
\section{Lezione del 27-02-25}

% stili pagina
\thispagestyle{empty}
\pagestyle{fancy}

% testo
\subsection{Introduzione all'economia}
Abbiamo visto come l'economia tratta dell'interazione economica fra diversi \textbf{soggetti} o \textit{enti} all'interno di diversi \textbf{sistemi economici}.
Questi possono essere \textit{privati}, \textit{aziende}, come ancora enti di varia natura se non addirittura interi \textit{stati}.

Il comportamento di ogni soggetto è determinato dal suo particolare modo di vedere e agire la realtà circostante, cioè da dei particolari \textit{assunti} o \textbf{paradigmi}.
I paradigmi dei soggetti influenzano il sistema economico a cui appartengono, e viceversa.
Proprio per questo motivo, storicamente si sono sempre formati innumerevoli sistemi economici, come innumerevoli erano gli assunti dei soggetti che vi appartenevano.

L'obiettivo principale dell'economia è quindi di comprendere il comportamento di questi soggetti all'interno di un dato sistema economico, inteso come la loro azione su determinati \textit{mezzi di produzione}, in presenza di \textbf{beni scarsi} (come li avevamo definiti in ricerca operativa, \textit{risorse limitate}).
Una volta comprese questo tipo di dinamiche, si possono ricavare \textbf{strumenti} che ci aiutino a fare scelte economiche migliori (più efficienti, più informate, ecc...). 

\subsubsection{Economia politica}
Una branca dell'economia che non considereremo in particolare è l'\textbf{economia politica}.
Questa tratta del comportamento di una vasta quantità di individui, a livello statale o oltre.
Per questo, si dice un approccio \textbf{top-down}.

Inoltre, non si premette di osservare il funzionamento interno dei soggetti economici, e quindi ad esempio di osservare la struttura delle imprese.
Per questo viene detto anche un approccio \textbf{black box} (a \textit{scatola nera}).

\subsubsection{Economia aziendale}
L'economia aziendale, come abbiamo detto, sceglie il percorso inverso: arriva alla comprensione di un sistema economico studiando la struttura e l'organizzazione delle imprese che si trovano al suo interno.
Questo lo rende un approccio \textbf{bottom-up}.

\subsubsection{Paradigmi}
Il modo di porsi di fronte a un problema (\textit{mentalità}) è definito, come abbiamo detto, \textit{paradigma}.
La \textit{propensione} è la tendenza di dare a priori un certo peso a diversi aspetti del problema da risolvere.
Possiamo assumere 3 tipi di mentalità rispetto alla gestione dell'impresa:
\begin{itemize}
	\item \textbf{Tecnico-produttiva:} tipica di chi si occupa del lato tecnico (e.g. ingegneri). 
		Unilaterale, tende al mantenimento dello \textit{status quo}: una strategia che si dimostra vincente non viene cambiata. Eccezione sono le \textbf{start-up}, dove una mentalità tecnico-produttiva può portare a cambiamenti e innovazioni.
	\item \textbf{Finanziaria:} ancora unilaterale, entra in gioco in periodi di \textbf{abbondanza} o \textbf{scarsità}, rispettivamente incentivando la tendenza a \textit{spendere} o a \textit{contrarre} le spese;
	\item \textbf{Economica:} cerca di unire gli approcci tecnico-produttivi e finanziari.
\end{itemize}

\subsubsection{Elementi in gioco}
Vediamo quindi quali saranno gli elementi di interesse nel \textbf{modello di business} che tratteremo.
# finisci

\end{document}


\documentclass[a4paper,11pt]{article}
\usepackage[a4paper, margin=8em]{geometry}

% usa i pacchetti per la scrittura in italiano
\usepackage[french,italian]{babel}
\usepackage[T1]{fontenc}
\usepackage[utf8]{inputenc}
\frenchspacing 

% usa i pacchetti per la formattazione matematica
\usepackage{amsmath, amssymb, amsthm, amsfonts}

% usa altri pacchetti
\usepackage{gensymb}
\usepackage{hyperref}
\usepackage{standalone}

% imposta il titolo
\title{Appunti Economia ed Organizzazione Aziendale}
\author{Luca Seggiani}
\date{2025}

% disegni
\usepackage{pgfplots}
\pgfplotsset{width=10cm,compat=1.9}

% imposta lo stile
% usa helvetica
\usepackage[scaled]{helvet}
% usa palatino
\usepackage{palatino}
% usa un font monospazio guardabile
\usepackage{lmodern}

\renewcommand{\rmdefault}{ppl}
\renewcommand{\sfdefault}{phv}
\renewcommand{\ttdefault}{lmtt}

% disponi il titolo
\makeatletter
\renewcommand{\maketitle} {
	\begin{center} 
		\begin{minipage}[t]{.8\textwidth}
			\textsf{\huge\bfseries \@title} 
		\end{minipage}%
		\begin{minipage}[t]{.2\textwidth}
			\raggedleft \vspace{-1.65em}
			\textsf{\small \@author} \vfill
			\textsf{\small \@date}
		\end{minipage}
		\par
	\end{center}

	\thispagestyle{empty}
	\pagestyle{fancy}
}
\makeatother

% disponi teoremi
\usepackage{tcolorbox}
\newtcolorbox[auto counter, number within=section]{theorem}[2][]{%
	colback=blue!10, 
	colframe=blue!40!black, 
	sharp corners=northwest,
	fonttitle=\sffamily\bfseries, 
	title=Teorema~\thetcbcounter: #2, 
	#1
}

% disponi definizioni
\newtcolorbox[auto counter, number within=section]{definition}[2][]{%
	colback=red!10,
	colframe=red!40!black,
	sharp corners=northwest,
	fonttitle=\sffamily\bfseries,
	title=Definizione~\thetcbcounter: #2,
	#1
}

% disponi problemi
\newtcolorbox[auto counter, number within=section]{problem}[2][]{%
	colback=green!10,
	colframe=green!40!black,
	sharp corners=northwest,
	fonttitle=\sffamily\bfseries,
	title=Problema~\thetcbcounter: #2,
	#1
}

% disponi codice
\usepackage{listings}
\usepackage[table]{xcolor}

\lstdefinestyle{codestyle}{
		backgroundcolor=\color{black!5}, 
		commentstyle=\color{codegreen},
		keywordstyle=\bfseries\color{magenta},
		numberstyle=\sffamily\tiny\color{black!60},
		stringstyle=\color{green!50!black},
		basicstyle=\ttfamily\footnotesize,
		breakatwhitespace=false,         
		breaklines=true,                 
		captionpos=b,                    
		keepspaces=true,                 
		numbers=left,                    
		numbersep=5pt,                  
		showspaces=false,                
		showstringspaces=false,
		showtabs=false,                  
		tabsize=2
}

\lstdefinestyle{shellstyle}{
		backgroundcolor=\color{black!5}, 
		basicstyle=\ttfamily\footnotesize\color{black}, 
		commentstyle=\color{black}, 
		keywordstyle=\color{black},
		numberstyle=\color{black!5},
		stringstyle=\color{black}, 
		showspaces=false,
		showstringspaces=false, 
		showtabs=false, 
		tabsize=2, 
		numbers=none, 
		breaklines=true
}

\lstdefinelanguage{javascript}{
	keywords={typeof, new, true, false, catch, function, return, null, catch, switch, var, if, in, while, do, else, case, break},
	keywordstyle=\color{blue}\bfseries,
	ndkeywords={class, export, boolean, throw, implements, import, this},
	ndkeywordstyle=\color{darkgray}\bfseries,
	identifierstyle=\color{black},
	sensitive=false,
	comment=[l]{//},
	morecomment=[s]{/*}{*/},
	commentstyle=\color{purple}\ttfamily,
	stringstyle=\color{red}\ttfamily,
	morestring=[b]',
	morestring=[b]"
}

% disponi sezioni
\usepackage{titlesec}

\titleformat{\section}
	{\sffamily\Large\bfseries} 
	{\thesection}{1em}{} 
\titleformat{\subsection}
	{\sffamily\large\bfseries}   
	{\thesubsection}{1em}{} 
\titleformat{\subsubsection}
	{\sffamily\normalsize\bfseries} 
	{\thesubsubsection}{1em}{}

% disponi alberi
\usepackage{forest}

\forestset{
	rectstyle/.style={
		for tree={rectangle,draw,font=\large\sffamily}
	},
	roundstyle/.style={
		for tree={circle,draw,font=\large}
	}
}

% disponi algoritmi
\usepackage{algorithm}
\usepackage{algorithmic}
\makeatletter
\renewcommand{\ALG@name}{Algoritmo}
\makeatother

% disponi numeri di pagina
\usepackage{fancyhdr}
\fancyhf{} 
\fancyfoot[L]{\sffamily{\thepage}}

\makeatletter
\fancyhead[L]{\raisebox{1ex}[0pt][0pt]{\sffamily{\@title \ \@date}}} 
\fancyhead[R]{\raisebox{1ex}[0pt][0pt]{\sffamily{\@author}}}
\makeatother

\begin{document}

% sezione (data)
\section{Lezione del 05-03-25}

% stili pagina
\thispagestyle{empty}
\pagestyle{fancy}

% testo
\subsection{Modello di business}
L'agente principale che prendiamo in osservazione durante il corso e' l'\textbf{impresa}, nella sua organizzazione e nella sua attivita' commerciale e finanziaria.
L'impresa rappesenta a tutti gli effetti un \textbf{sistema}, che interagisce con l'esterno ottenendo qualcosa, e restituendo da parte loro qualcos'altro.

L'impresa interagisce quindi coi \textbf{mercati} scambiando merci (coi fornitori), prodotti (coi clienti), titoli di vario tipo e in generale denaro.
Avra' bisogno di \textbf{capitale}, che si procurera' tramite \textit{scelte finanziarie} (mezzi propri e finanziamenti).
Sfruttera' il capitale cosi' ottenuto per cimentarsi nella \textbf{produzione} di un bene (un \textbf{prodotto}) da metter in commercio.

Nella \textit{costituzione} dell'impresa sara' necessaria un \textit{idea} di impresa, nonche' decisioni riguardo alla merce prodotta, quindi al \textit{mercato target}, al \textit{sistema di offerta} e alla struttura interna (\textbf{organizzazione aziendale}). 

Dall'esterno agiranno sull'impresa varie \textit{forze macroeconomiche}, nonche' l'attivita' dello \textbf{stato} (imposte, ecc...), le \textit{forze di mercato} e i \textit{trend socio-economici}.
Le imprese saranno poi in \textbf{concorrenza}, sia questa diretta, indiretta o potenziale, fra di loro.

\subsubsection{Mercati finanziari}
Una delle categorie di mercati con cui interagisce l'impresa e' rappresentata dai \textbf{mercati finanziari}.
Questa interazione viene fatta attraverso gli \textbf{strumenti finanziari} (obbligazioni, azioni, mutui, ecc...).
In particolare, azioni e obbligazioni vengono comprate e vendute sul \textbf{mercato mobiliare}, solitamente detto \textit{borsa}.

\subsection{Capitale}
L'obiettivo principale dell'impresa e' quello dell'ottenimento del \textbf{capitale}.
Questo puo' derivare da:
\begin{itemize}
	\item \textbf{Capitale proprio} dell'imprenditore o di eventuali \textbf{soci} disposti ad unirsi al supporto dell'idea di business, viene detto anche \textbf{capitale di rischio}, in quanto non ha alcuna garanzia di essere recuperato nel caso del fallimento dell'impresa. Di contro, i soci hanno \textbf{diritto residuale}, cioe' di suddividersi cio' che residua a soddisfacimento di tutti gli altri debiti.

		In termini contabili un sottoinsieme del capitale proprio viene definito \textbf{capitale sociale}, cioe' l'insieme dei \textit{conferimenti} effettuati dai soci durante la costituzione o in momenti successivi della vita dell'impresa.

		Altre fonti di capitale proprio sono rappresentate anche dall'\textbf{utile}, cioe' dal risultato delle attivita' dell'impresa.
	\item \textbf{Capitale di terzi} dei \textit{finanziatori}, viene detto anche \textbf{capitale di credito}, in quanto i finanziatori hanno diritto alla sua restituzione anche nel caso di fallimento, oltre che ad un \textbf{interesse} che l'impresa paga per il capitale che il finanziatore non spende personalmente, ma gli mette a disposizione.
		I finanziatori rappresentano per noi dei \textbf{creditori}, cioe' abbiamo per loro un \textbf{debito}.
	\item \textbf{Crowdsourcing}.
\end{itemize}

\subsubsection{Obbligazioni e azioni come fonti di capitale}

Le \textbf{obbligazioni} rappresentano quindi un tipo di \textit{capitale di credito}, cioe' dei prestiti che l'impresa si impegna a rimborsare al finanziatore con un certo interesse.
Le obbligazioni sono poi \textbf{titoli di credito} (cambiali, assegni, ecc...), con le loro regole proprie che determinano le modalita' secondo le quali possono essere immesse nel mercato.

Le \textbf{azioni} sono invece un tipo di \textit{capitale di rischio}.
L'azione consiste per gli acquirenti in un versamento da parte dell'impresa di \textbf{dividendi}, ricavati dall'attivita' della stessa, e legati alla relazione dell'azionista con la stessa.

\subsubsection{Note sui tipi di societa'}
Notiamo che solo alcuni tipi di societa' (S.p.A., \textit{Societa' per Azioni} o S.a.p.a. (\textit{Societa' in accomandita per azioni})) possono emanare azioni.
Le S.r.l. \textit{Societa' a responsabilita' limitata} e le \textit{societa' di persone} detengono invece \textbf{quote di capitale}.

In particolare, le S.p.A., le S.a.p.a. e le S.r.l. rappresentano \textbf{societa' di capitali}, mentre le societa' di persone sono ulteriormente divise in S.n.C. (\textit{Societa' in nome Collettivo}), S.s. (\textit{Societa' semplice}) e S.a.s (\textit{Societa' in accomandita semplice}).

Si ha quindi la divisione:
\begin{itemize}
	\item Societa' di capitali:
		\begin{itemize}
			\item S.p.A. - Societa' per Azioni;
			\item S.a.p.a. - Societa' in accomandita per azioni;
			\item S.r.l. - Societa' a responsabilita' limitata.
		\end{itemize}
	\item Societa' di persone:
		\begin{itemize}
			\item S.n.c. - Societa' in nome collettivo;
			\item S.s. - Societa' semplici;
			\item S.a.s. - Societa' in accomandita semplice.
		\end{itemize}
\end{itemize}

\subsection{Produzione}
La societa' in disponibilita' di liquido (capitale proprio e di terzi) dovra' effettuare \textbf{operazioni di acquisto} dei cosiddetti \textbf{fattori produttivi}, che distinguiamo in:
\begin{itemize}
	\item \textbf{Fattori produttivi pluriennali}, stabilimenti, automezzi, strumentazioni, ecc...
	\item \textbf{Manodopera}, cioe' il \textit{lavoro} dei dipendenti;
		\item \textbf{Fattori produttivi di esercizio}, cioe' \textit{merci}, \textit{materia prima}, o ancora \textit{materia sussidiaria}, ecc...
\end{itemize}

Notiamo che i fattori produttivi di esercizio contribuiscono totalmente al valore di un prodotto, mentre i fattori produttivi pluriennali vi contribuiscono solo parzialmente (la stessa fabbrica produce piu' di un prodotto all'anno).
Il processo di ripartizione di un costo pluriennale sul consumo fatto in un periodo di tempo (solitamente un anno) viene detto \textbf{ammortamento}.

Nell'operazione di acquisto l'impresa crea un \textbf{debito} dei fornitori dei fattori produttivi.
Questo debito si distingue dal debito che abbiamo con i finanziatori in quanto rappresenta \textbf{debito di funzionamento} o \textbf{debito commerciale}.
La presenza di debiti commerciali e' fisiologica all'attivita' dell'impresa, in quanto e' necessaria alla produzione e vendita del prodotto.

\par\smallskip

Una volta che si e' in possesso delle liquidita', e quindi dei fattori di produzione, si procede con la produzione vera e proprio del prodotto da mettere in commercio.

Nelle \textbf{imprese manufatturiere} e' tipica una \textit{trasformazione fisica} dei fattori produttivi di esercizio.

Le \textbf{imprese commerciali} effettuano invece \textit{trasformazioni economiche} (e.g. di denaro o merci) nel \textit{tempo} e nello \textit{spazio}, traendo guadagno da fattori come l'interesse o la rivendita di merci in veste di intermediari fra produttori e consumatori.

\subsection{Vendita}
Il prodotto ultimato verra' venduto nei \textbf{mercati di sbocco}, che possono essere:
\begin{itemize}
	\item \textbf{B2B}, \textit{Business-to-Business}: da imprese ad altre imprese, ad esempio componenti (\textbf{prodotti intermedi}) che verranno assemblati da altre imprese;
	\item \textbf{B2C}, \textit{Business-to-Consumer}: dall'impresa al consumatore, di \textbf{prodotti finali}.
\end{itemize}

Durante la fase di vendita si puo' creare un'altro tipo di debito, cio il \textbf{credito commerciale} dei clienti nei confronti dell'impresa.

\subsection{Gestione}
L'insieme di passaggi che abbiamo visto finora (finanziamento, produzione e vendita) formano il \textbf{ciclo operativo della gestione} dell'impresa.
La \textbf{gestione} d'impresa e l'insieme delle operazioni che le persone operanti nell'impresa compiono (sia decisioni che azioni) tramite i fattori produttivi a disposizione per svolgere le attivita' che l'impresa ha definito.

Le operazioni di gestione \textbf{ordinaria} sono cio' che l'impresa fa su base quotidiana: \textbf{acquisto} $\rightarrow$ \textbf{produzione} $\rightarrow$ \textbf{vendita}, supportate da finanziamento \textbf{attinto} (mercati finanziari) e finanziamento \textbf{concesso}.

In particolare, chiamiamo operazioni di gestione ordinaria \textbf{esterna} quelle operazioni che ci mettono in contatto con terzi.
Sono di questo tipo tutte le operazioni tranne la produzione.
La produzione e' quindi l'unica operazione che, dal magazzino di ingresso al magazzino di uscita, non mette l'impresa in contatto con i terzi.
Le operazioni di gestione ordinaria esterna sono quindi quelle che riguardano anche la \textbf{contabilita'} e quindi il \textbf{bilancio}, cioe' la valutazione dei flussi di denaro in entranta e in uscita.

\subsection{Soggetti dell'impresa}
L'impresa coinvolge una vasta gamma di soggetti, primi fra tutti gli \textbf{Stakeholder} (portatori di \textit{interesse}), cioe' coloro che costituiscono il \textbf{soggetto economico} di influenza sull'impresa.
Questi possono essere:
\begin{itemize}
	\item I soci;
	\item I finanziatori;
	\item Manager e dipendenti;
	\item I clienti;
	\item Lo stato;
	\item I concorrenti;
	\item La comunita' sociale.
\end{itemize}

In particolare, gli \textbf{shareholder} sono rappresentati dai soci e dagli azionisti (di maggioranza/minoranza), cioe' da chi ha fornito capitale di rischio all'impresa.

Il \textbf{soggetto giuridico} e' invece rappresentato dalla persona fisica o giuridica che assume la titolarita' dell'impresa.

I soggetti dell'impresa hanno fra di loro responsabilita' diverse per quanto riguarda l'attivita' e le responsabilita' cui deve render conto l'impresa.
Ad esempio, come avevamo detto, i titolari di obbligazioni avranno diritti a restituzioni, mentre gli azionisti non avranno lo stesso diritto.
Di contro, il capitale proprio degli shareholder, essendo loro privato, non potra' essere usato per il rimborso di rischi andati in fallo (mentre lo stato potra' procedere per quanto riguarda titoli di credito regolamentati).

\subsection{Struttura dell'impresa, organigramma}
A livello formale la struttura dell'impresa e' solitamente data dal suo \textbf{organigramma}, che ci da informazioni di livello strutturale e formale, ma non delle interdipendenze informali fra i suoi elementi.
	
Di base, consultando l'organigramma potremo stabilire chi comanda e chi e' sottoposto, ma ad esempio non potremo comprendere relazioni fra elementi costruite all'esterno della struttura propria dell'impresa.

\end{document}


\documentclass[a4paper,11pt]{article}
\usepackage[a4paper, margin=8em]{geometry}

% usa i pacchetti per la scrittura in italiano
\usepackage[french,italian]{babel}
\usepackage[T1]{fontenc}
\usepackage[utf8]{inputenc}
\frenchspacing 

% usa i pacchetti per la formattazione matematica
\usepackage{amsmath, amssymb, amsthm, amsfonts}

% usa altri pacchetti
\usepackage{gensymb}
\usepackage{hyperref}
\usepackage{standalone}

% imposta il titolo
\title{Appunti Economia ed Organizzazione Aziendale}
\author{Luca Seggiani}
\date{2025}

% disegni
\usepackage{pgfplots}
\pgfplotsset{width=10cm,compat=1.9}

% imposta lo stile
% usa helvetica
\usepackage[scaled]{helvet}
% usa palatino
\usepackage{palatino}
% usa un font monospazio guardabile
\usepackage{lmodern}

\renewcommand{\rmdefault}{ppl}
\renewcommand{\sfdefault}{phv}
\renewcommand{\ttdefault}{lmtt}

% disponi il titolo
\makeatletter
\renewcommand{\maketitle} {
	\begin{center} 
		\begin{minipage}[t]{.8\textwidth}
			\textsf{\huge\bfseries \@title} 
		\end{minipage}%
		\begin{minipage}[t]{.2\textwidth}
			\raggedleft \vspace{-1.65em}
			\textsf{\small \@author} \vfill
			\textsf{\small \@date}
		\end{minipage}
		\par
	\end{center}

	\thispagestyle{empty}
	\pagestyle{fancy}
}
\makeatother

% disponi teoremi
\usepackage{tcolorbox}
\newtcolorbox[auto counter, number within=section]{theorem}[2][]{%
	colback=blue!10, 
	colframe=blue!40!black, 
	sharp corners=northwest,
	fonttitle=\sffamily\bfseries, 
	title=Teorema~\thetcbcounter: #2, 
	#1
}

% disponi definizioni
\newtcolorbox[auto counter, number within=section]{definition}[2][]{%
	colback=red!10,
	colframe=red!40!black,
	sharp corners=northwest,
	fonttitle=\sffamily\bfseries,
	title=Definizione~\thetcbcounter: #2,
	#1
}

% disponi problemi
\newtcolorbox[auto counter, number within=section]{problem}[2][]{%
	colback=green!10,
	colframe=green!40!black,
	sharp corners=northwest,
	fonttitle=\sffamily\bfseries,
	title=Problema~\thetcbcounter: #2,
	#1
}

% disponi codice
\usepackage{listings}
\usepackage[table]{xcolor}

\lstdefinestyle{codestyle}{
		backgroundcolor=\color{black!5}, 
		commentstyle=\color{codegreen},
		keywordstyle=\bfseries\color{magenta},
		numberstyle=\sffamily\tiny\color{black!60},
		stringstyle=\color{green!50!black},
		basicstyle=\ttfamily\footnotesize,
		breakatwhitespace=false,         
		breaklines=true,                 
		captionpos=b,                    
		keepspaces=true,                 
		numbers=left,                    
		numbersep=5pt,                  
		showspaces=false,                
		showstringspaces=false,
		showtabs=false,                  
		tabsize=2
}

\lstdefinestyle{shellstyle}{
		backgroundcolor=\color{black!5}, 
		basicstyle=\ttfamily\footnotesize\color{black}, 
		commentstyle=\color{black}, 
		keywordstyle=\color{black},
		numberstyle=\color{black!5},
		stringstyle=\color{black}, 
		showspaces=false,
		showstringspaces=false, 
		showtabs=false, 
		tabsize=2, 
		numbers=none, 
		breaklines=true
}

\lstdefinelanguage{javascript}{
	keywords={typeof, new, true, false, catch, function, return, null, catch, switch, var, if, in, while, do, else, case, break},
	keywordstyle=\color{blue}\bfseries,
	ndkeywords={class, export, boolean, throw, implements, import, this},
	ndkeywordstyle=\color{darkgray}\bfseries,
	identifierstyle=\color{black},
	sensitive=false,
	comment=[l]{//},
	morecomment=[s]{/*}{*/},
	commentstyle=\color{purple}\ttfamily,
	stringstyle=\color{red}\ttfamily,
	morestring=[b]',
	morestring=[b]"
}

% disponi sezioni
\usepackage{titlesec}

\titleformat{\section}
	{\sffamily\Large\bfseries} 
	{\thesection}{1em}{} 
\titleformat{\subsection}
	{\sffamily\large\bfseries}   
	{\thesubsection}{1em}{} 
\titleformat{\subsubsection}
	{\sffamily\normalsize\bfseries} 
	{\thesubsubsection}{1em}{}

% disponi alberi
\usepackage{forest}

\forestset{
	rectstyle/.style={
		for tree={rectangle,draw,font=\large\sffamily}
	},
	roundstyle/.style={
		for tree={circle,draw,font=\large}
	}
}

% disponi algoritmi
\usepackage{algorithm}
\usepackage{algorithmic}
\makeatletter
\renewcommand{\ALG@name}{Algoritmo}
\makeatother

% disponi numeri di pagina
\usepackage{fancyhdr}
\fancyhf{} 
\fancyfoot[L]{\sffamily{\thepage}}

\makeatletter
\fancyhead[L]{\raisebox{1ex}[0pt][0pt]{\sffamily{\@title \ \@date}}} 
\fancyhead[R]{\raisebox{1ex}[0pt][0pt]{\sffamily{\@author}}}
\makeatother

\begin{document}

% sezione (data)
\section{Lezione del 06-03-25}

% stili pagina
\thispagestyle{empty}
\pagestyle{fancy}

% testo
\subsection{Diritto delle società}
Il \textbf{diritto} è un termine usato con due accezioni differenti:
\begin{itemize}
	\item La prima accezione intende il diritto come il complesso delle \textbf{norme giuridiche} che regolano la vita dei membri di una comunità, quindi l'\textbf{ordinamento giuridico}. In questo viene detto anche \textbf{diritto oggettivo};
	\item La seconda accezione intende il diritto come la facoltà (o il potere) garantito dall'\textit{ordinamento giuridico} ad un soggetto. In questo viene detto anche \textbf{diritto soggettivo}.
\end{itemize}

L'ordinamento giuridico si divide ulteriormente in:
\begin{itemize}
	\item \textbf{Diritto privato:} è l'insieme delle norme giuridiche che regolano i rapporti fra i \textit{privati}: questi possono essere rappresentati da \textit{atti personali} (\textbf{diritto civile}) o \textit{atti di commercio} (\textbf{diritto commerciale});
	\item \textbf{Diritto pubblico:} ha per oggetto il funzionamento dello stato e degli \textit{enti territoriali}, e i rapporti che questi hanno con i cittadini.
\end{itemize}

Il diritto commerciale si divide poi ulteriormente in innumerevoli categorie, fra cui:
\begin{itemize}
	\item Diritto societario;
	\item Diritto fallimentare;
	\item Antitrust;
	\item Diritto del lavoro;
	\item ecc...
\end{itemize}

A noi in particolare sarà di interesse il diritto oggettivo, privato,  commerciale e societario.

\subsubsection{Norme giuridiche}
Una norma giuridica rappresenta una regola di condotta, o comunque un precetto che stabilisce un comportamento condivisio da una comunità.
Le norme sono contenute solitamente in testi, detti \textbf{testi normativi}.
I testi normativi rappresentano le \textbf{fonti del diritto}.
L'ordine di importanza delle fonti del diritto è determinato dalla \textbf{gerarchia delle fonti}:
\begin{itemize}
	\item Norme di \textbf{primo livello}: la costituzione e le leggi costituzionali, i regolamenti comunitari;
	\item Norme di \textbf{secondo livello}: le leggi dello stato (fatte dal \textit{parlamento}), i decreti legge (fatti dal \textit{governo} e temporanei) e i decreti legislativi (fatti sempre dal \textit{governo}, ma su delega del parlamento), il referendum abrogativo;
	\item Norme di \textbf{terzo livello}: regolamenti governativi e degli enti locali, usi e consuetudini.
\end{itemize}

\subsubsection{Giurisprudenza}
La \textbf{giurisprudenza} è la disciplina che studia il diritto e la sua \textit{interpretazione}.
La \textbf{dottrina giuridica} è invece l'attività di studio scientifico del diritto.

Gli organi giudicanti dello stato (la magistratura) interpreta e applica le leggi, processo da cui risultano le \textbf{sentenze}.
Esistono in questo 2 sistemi:
\begin{itemize}
	\item \textbf{Common law:} le sentenze dei tribunali fanno \textit{precedente} e vengono prese ad esempio;
	\item \textbf{Civil law:} si basa invece principalmente su codici prestabiliti e non sentenze precedenti.
\end{itemize}

Il \textbf{diritto societario} è e definito con leggi e raccolte di leggi (detto \textbf{codice civile}).
Il codice civile distingue diversi tipi di imprese in base a 3 criteri:
\begin{itemize}
	\item \textbf{Oggetto} dell'impresa (imprenditore agricolo, imprenditore commerciale, ecc...);
	\item \textbf{Dimensione} dell'impresa (piccolo imprenditore, grande imprenditore, ecc...);
	\item \textbf{Natura del soggetto} che esercita l'impresa (impesa individuale, collettiva, ecc...).
\end{itemize}

Non si trova quindi nel codice civile una definizione propria di \textit{impresa}, ma di \textbf{imprenditore}: è imprenditore chi esercita \textbf{professionalmente} un'\textbf{attività economica organizzata} al fine della \textbf{produzione} e dello \textbf{scambio} di \textbf{beni} o \textbf{servizio}.

Dove:
\begin{itemize}
	\item \textbf{Organizzata} rappresenta l'impiego coordinato di fattori produttivi propri o altrui;
	\item L'\textbf{economicità} riguarda il metodo con cui l'attività è svolta;
	\item E \textbf{professionalità} rappresenta l'esercizio abituale dell'attivita produttiva.
\end{itemize}

Questa definizione è quella del codice civile, e differisce quindi da quella che si può trovare nel diritto tributario, ecc... 

Si trova poi una definizione di \textbf{azienda}, cioè l'insieme dei beni organizzati dall'imprenditore per l'esercizio dell'impresa.

I requisiti \textit{civilistici} dell'imprenditore sono quindi quelli dello svolgimento di un \textbf{attività produttiva}, che come abbiamo visto è alla base dell'impresa.

\end{document}


\documentclass[a4paper,11pt]{article}
\usepackage[a4paper, margin=8em]{geometry}

% usa i pacchetti per la scrittura in italiano
\usepackage[french,italian]{babel}
\usepackage[T1]{fontenc}
\usepackage[utf8]{inputenc}
\frenchspacing 

% usa i pacchetti per la formattazione matematica
\usepackage{amsmath, amssymb, amsthm, amsfonts}

% usa altri pacchetti
\usepackage{gensymb}
\usepackage{hyperref}
\usepackage{standalone}

% imposta il titolo
\title{Appunti Economia ed Organizzazione Aziendale}
\author{Luca Seggiani}
\date{2025}

% disegni
\usepackage{pgfplots}
\pgfplotsset{width=10cm,compat=1.9}

% imposta lo stile
% usa helvetica
\usepackage[scaled]{helvet}
% usa palatino
\usepackage{palatino}
% usa un font monospazio guardabile
\usepackage{lmodern}

\renewcommand{\rmdefault}{ppl}
\renewcommand{\sfdefault}{phv}
\renewcommand{\ttdefault}{lmtt}

% disponi il titolo
\makeatletter
\renewcommand{\maketitle} {
	\begin{center} 
		\begin{minipage}[t]{.8\textwidth}
			\textsf{\huge\bfseries \@title} 
		\end{minipage}%
		\begin{minipage}[t]{.2\textwidth}
			\raggedleft \vspace{-1.65em}
			\textsf{\small \@author} \vfill
			\textsf{\small \@date}
		\end{minipage}
		\par
	\end{center}

	\thispagestyle{empty}
	\pagestyle{fancy}
}
\makeatother

% disponi teoremi
\usepackage{tcolorbox}
\newtcolorbox[auto counter, number within=section]{theorem}[2][]{%
	colback=blue!10, 
	colframe=blue!40!black, 
	sharp corners=northwest,
	fonttitle=\sffamily\bfseries, 
	title=Teorema~\thetcbcounter: #2, 
	#1
}

% disponi definizioni
\newtcolorbox[auto counter, number within=section]{definition}[2][]{%
	colback=red!10,
	colframe=red!40!black,
	sharp corners=northwest,
	fonttitle=\sffamily\bfseries,
	title=Definizione~\thetcbcounter: #2,
	#1
}

% disponi problemi
\newtcolorbox[auto counter, number within=section]{problem}[2][]{%
	colback=green!10,
	colframe=green!40!black,
	sharp corners=northwest,
	fonttitle=\sffamily\bfseries,
	title=Problema~\thetcbcounter: #2,
	#1
}

% disponi codice
\usepackage{listings}
\usepackage[table]{xcolor}

\lstdefinestyle{codestyle}{
		backgroundcolor=\color{black!5}, 
		commentstyle=\color{codegreen},
		keywordstyle=\bfseries\color{magenta},
		numberstyle=\sffamily\tiny\color{black!60},
		stringstyle=\color{green!50!black},
		basicstyle=\ttfamily\footnotesize,
		breakatwhitespace=false,         
		breaklines=true,                 
		captionpos=b,                    
		keepspaces=true,                 
		numbers=left,                    
		numbersep=5pt,                  
		showspaces=false,                
		showstringspaces=false,
		showtabs=false,                  
		tabsize=2
}

\lstdefinestyle{shellstyle}{
		backgroundcolor=\color{black!5}, 
		basicstyle=\ttfamily\footnotesize\color{black}, 
		commentstyle=\color{black}, 
		keywordstyle=\color{black},
		numberstyle=\color{black!5},
		stringstyle=\color{black}, 
		showspaces=false,
		showstringspaces=false, 
		showtabs=false, 
		tabsize=2, 
		numbers=none, 
		breaklines=true
}

\lstdefinelanguage{javascript}{
	keywords={typeof, new, true, false, catch, function, return, null, catch, switch, var, if, in, while, do, else, case, break},
	keywordstyle=\color{blue}\bfseries,
	ndkeywords={class, export, boolean, throw, implements, import, this},
	ndkeywordstyle=\color{darkgray}\bfseries,
	identifierstyle=\color{black},
	sensitive=false,
	comment=[l]{//},
	morecomment=[s]{/*}{*/},
	commentstyle=\color{purple}\ttfamily,
	stringstyle=\color{red}\ttfamily,
	morestring=[b]',
	morestring=[b]"
}

% disponi sezioni
\usepackage{titlesec}

\titleformat{\section}
	{\sffamily\Large\bfseries} 
	{\thesection}{1em}{} 
\titleformat{\subsection}
	{\sffamily\large\bfseries}   
	{\thesubsection}{1em}{} 
\titleformat{\subsubsection}
	{\sffamily\normalsize\bfseries} 
	{\thesubsubsection}{1em}{}

% disponi alberi
\usepackage{forest}

\forestset{
	rectstyle/.style={
		for tree={rectangle,draw,font=\large\sffamily}
	},
	roundstyle/.style={
		for tree={circle,draw,font=\large}
	}
}

% disponi algoritmi
\usepackage{algorithm}
\usepackage{algorithmic}
\makeatletter
\renewcommand{\ALG@name}{Algoritmo}
\makeatother

% disponi numeri di pagina
\usepackage{fancyhdr}
\fancyhf{} 
\fancyfoot[L]{\sffamily{\thepage}}

\makeatletter
\fancyhead[L]{\raisebox{1ex}[0pt][0pt]{\sffamily{\@title \ \@date}}} 
\fancyhead[R]{\raisebox{1ex}[0pt][0pt]{\sffamily{\@author}}}
\makeatother

\begin{document}

% sezione (data)
\section{Lezione del 12-03-25}

% stili pagina
\thispagestyle{empty}
\pagestyle{fancy}

% testo
Proseguiamo quindi lo studio dal punto di vista giuridico della figura dell'\textit{imprenditore}, e quindi dell'\textit{impresa}.

\subsection{Statuto dell'imprenditore}
L'obiettivo del testo giuridico che regola l'impresa, cioè lo \textbf{statuto dell'imprenditore}, è quello di definire il \textit{perimetro} all'interno del cui l'impresa può muoversi.

Bisogna innanzitutto distinguere l'\textbf{oggetto} dell'impresa in:
\begin{itemize}
	\item Impresa commerciale (cioè tutto ciò che non è agricolo);
	\item Impresa agricola.
\end{itemize}
Si può poi distinguere la \textbf{dimensione} dell'impresa:
\begin{itemize}
	\item Piccola impresa; 
	\item Medio/grande impresa.
\end{itemize}
Infine, si può distinguere sulla natura del \textbf{soggetto}:
\begin{itemize}
	\item Impresa individuale;
	\item Impresa collettiva, che si divide a sua volta in:
		\begin{itemize}
			\item Impresa societaria;
			\item Impresa pubblica;
			\item Fondazioni e associazioni.
		\end{itemize}
\end{itemize}

\subsubsection{Registro delle imprese}
Il \textbf{registro delle imprese} è uno strumento, istituito dalle camere di commercio (enti pubblici locali non territoriali, dotati di autonomia funzionale), che tiene conto di tutti gli \textit{atti} (costituzione dell'impresa, richieste di finanziamento da terzi, ec...) e i \textit{fatti} riguardanti le imprese iscritte.
Dal registro delle imprese si può attingere ai \textit{prospetti ufficiali} dell'impresa (bilanci, storici, ecc...) sotto versamento di una certa somma.
L'iscrizione è solitamente \textbf{dichiaratava}, anche se in alcuni casi è \textit{costitutiva} (S.p.A.) o solamente \textit{pubblicitaria} (piccoli imprenditori).

L'esistenza di un registro delle imprese ha diversi effetti dal punto di vista legale:
\begin{itemize}
	\item I fatti dichiarati pubblicamente sono assunti noti da terzi;
	\item Di contro, i fatti non dichiarati non sono opponibili a terzi (a meno di non dimostrarne la conoscenza).
\end{itemize}

\subsubsection{Scritture contabili}
Le scritture contabili sono documenti che contengono i singoli atti dell'impresa, che l'impresa è obbligata a tenere (e conservare per una durata di 10 anni).
Queste includono:
\begin{itemize}
	\item Tutte le scritture richieste dalla natura e dalla dimensione dell'impresa (libro mastro, di cassa, di magazzino, ecc...);
	\item \textbf{Libro giornale:} registro cronologico-analitico;
	\item \textbf{Libro degli inventari:} registro periodico-sistematico;
	\item \textbf{Originali} della corrispondenza commerciale ricevuta e \textbf{copie} di quella spedita.
\end{itemize}

Gli effetti legali delle scritture contabili sono considerevoli in quanto queste rappresentano prova, con efficacia \textbf{probatoria}.

\subsubsection{Ausiliari dell'imprenditore}
Gli ausiliari dell'impresa sono divisi in due categorie:
\begin{itemize}
	\item \textbf{Ausiliari subordinati:} cioè subordinati all'imprenditore da un rapporto di lavoro. Questi possono essere:
		\begin{itemize}
			\item \textbf{Institori:} coloro che sono preposti dal titolare all'esercizio dell'impresa commerciale, stanno all'\textit{interno} della gerarchia organizzativa in ruoli abbastanza stabili, e solitamente dotati di un certo potere;
			\item \textbf{Procuratori:} svolgono sempre l'opera di esercizio dell'impresa, ma sono sottoposti agli institori, e legati al titolare da un rapporto di \textbf{procura} (più flessibile rispetto al ruolo gerarchico degli institori). Questo significa che il loro potere non deriva dalla loro posozione gerarchica, ma solo dal rapporto di procura col titolare;
			\item \textbf{Commessi:} compiono gli incarichi gli atti che ordinariamente comporta la specie di operazioni delle quali sono incaricati.
		\end{itemize}
	\item \textbf{Ausiliari autonomi:} chiamati informalmente \textit{partite IVA}, legati all'imprenditore da un rapporto di prestazione d'opera (restano quindi indipendenti).
\end{itemize}

Gli ausiliari possono concludere affari con terzi per conto dell'imprenditore, e quindi si pone il problema della \textbf{rappresentanza}.
In generale si richiede il conferimento della \textbf{procura}, cioè il potere di rappresentanza esiste nei limiti fissati dalla procura.

Una regola speciale è rappresentata dalla \textbf{rappresentanza commerciale}: gli ausiliari subordinati sono automaticamente investiti dal potere di rappresentanza, per la sola natura della loro attività. Chiaramente, ogni figura ha il suo livello di rappresentanza (il commesso avrà meno potere di rappresentanza di un procuratore o un institore).

\subsection{Azienda}
Fino ad ora abbiamo parlato d'impresa, e di riflesso della figura dell'imprenditore.
Vediamo adesso alla definizione dell'\textbf{azienda}, intesa come \textit{il complesso dei beni organizzati dall'imprenditore all'esercizio dell'impresa}.

In questo, ad esempio, i \textbf{liberi professionisti} non possono considerarsi azienda, in quanto il libero professionista stesso non è imprenditore: diventa tale solo nel caso in cui svolga un'attività che di per sé è un'attivita d'impresa, e non la sua sola prestazione intellettuale.

Si possono quindi distinguere 3 segni distintivi dell'azienda:
\begin{itemize}
	\item \textbf{Ditta:} il nome sotto il quale l'imprenditore esercita l'attività d'impresa.
		E' un segno necessario, e deve rispettare 2 principi: \textbf{verità} e \textbf{novità}. 
		
		Si applica in questo caso di sole imprese individuali.
		Nel caso di imprese societarie si parla invece di \textbf{ragione sociale} (società di persone, deve contenere il nome di almeno un socio a responsabilità illimitata) o \textbf{denominazione sociale} (società di capitali, non ha limitazioni riguardanti i nomi dei soci).

		Dovrà quindi contenere il cognome o la sigla dell'imprenditore (altrimenti si considera irregolare e rientra nei segni distintivi atipici).
		La ditta può inoltre rimanere dopo la cessione dell'attività a terzi, in quanto il principio di verità non viene invalidato (questo non significa che la ditta non può comunque essere cambiata);

	\item \textbf{Insegna:} segno distintivo del locale nel quale si svolge l'attività di imprenditore, può corrispondere o non corrispondere alla ditta. Deve avere \textbf{liceità}, \textbf{veridicità} ed \textbf{originalità};

	\item \textbf{Marchio:} segno distintivo del prodotto o del servizio fornito dall'impresa. Qui si può distinguere in:
		\begin{itemize}
			\item \textbf{Marchio di fabbrica:} apposto dal produttore;
			\item \textbf{Marchio di commercio:} apposto da colui che commercializza il prodotto;
			\item \textbf{Marchio di forma:} il marchio può essere rappresentato da diverse qualità, tra cui una sigla, un motto, un immagine, un \textit{jingle} musicale o la \textit{forma} stessa del prodotto distinguiamo infatti marchi \textbf{denominativi}, \textbf{figurativi} o \textbf{misti};
		\end{itemize}
		I requisiti del marchio sono \textbf{originalità}, \textbf{novità}, \textbf{conformità} e la non violazione dei diritti esclusivi di terzi (ad esempio, i \textbf{diritti d'autore}).

		Il \textbf{diritto all'uso} del marchio da parte di un azienda si acquista con:
		\begin{itemize}
			\item \textbf{Registrazione} del marchio, i marchi celebri sono più protetti. Un marchio si può perdere per \textbf{volgarizzazione}, cioè quando diventa una parola di uso comune;
			\item \textbf{Uso di fatto}. 
		\end{itemize}

		Il marchio può infine essere ceduto, concesso in licenza o in merchandising.
		
\end{itemize}

\subsubsection{Diritti di privativa}
I diritti di privativa sono la categoria di cui fanno parte il \textbf{diritto di autore}, il \textbf{diritto di inventore} e il \textbf{brevetto}.
\begin{itemize}
	\item \textbf{Diritto d'autore:} si applica a beni immateriali (opere dell'ingegno di carattere creativo), e si distinguno nel:
		\begin{itemize}
			\item \textbf{Diritto morale} d'autore, cioè la paternità dell'opera;
			\item \textbf{Diritto patrimoniale} d'autore, cioè il diritto di pubblicare l'opera e utilizzarla economicamente.
		\end{itemize}
	\item \textbf{Diritto d'inventore:} si tratta di idee creative che appartengono al campo della tecnica. E' caratterizzato da \textbf{industrialità}, \textbf{liceità}, e infine \textbf{novità intrinseca} e \textbf{novità estrinseca}, cioè rispettivamente la capacità di incrementare il patrimonio tecnico presente (\textit{intrinseca}) e la mancata divulgazione (\textit{estrinseca});
	\item \textbf{Brevetto:} mezzo attraverso il quale si rende \textit{invenzione} cpò che prima era pubblico dominio, e quindi permette all'inventore di capitalizzare sulla sua opera. Decade, può essere espropiato e concesso in licenza, ed è trasferibile \textit{inter vivos} o \textit{mortis causa}.
\end{itemize}

\subsection{Società}
Entriamo quindi nel dettaglio delle società, cioè delle imprese collettive.
In particolare, una società è un contratto, attraverso il quale \textit{due o più persone conferiscono beni o servizi per l'esercizio in comune di un'attività economica allo scopo di dividerne gli utili}.

Il \textit{conferimento} rappresenta le prestazioni in cui le parti della società si obbligano.
Dal punto di vista pratico, questo è rappresentato semplicemente da mezzi finanziari, o mezzi di produzione, immobili, credito, ecc...
Il conferimento del lavoro (\textbf{socio d'opera}) varia invece di società in società.
L'\textit{esercizio in comune} è preordinato alla realizzazione di un risultato unico, nella prospettiva della \textit{divisione degli utili}. 

\subsubsection{Scopo della società}
Si può distinguere in diverse categorie di scopo della società:
\begin{itemize}
	\item \textbf{Lucrativo:} svolgimento dell'attività d'impresa per \textit{produrre utile} (\textbf{lucro oggettivo}) destinato ad essere diviso fra i soci (\textbf{lucro soggettivo});
	\item \textbf{Mutualistico:} tipico delle \textit{società cooperative}, atto a fornire beni o servizi od occasioni di lavoro, direttamente ai soci, a condizioni più vantaggiose di quelle che otterrebbero sul mercato;
	\item \textbf{Consortile:} vantaggio patrimoniale diretto, riguarda i consorzi fra due o più imprese.
\end{itemize}

\subsubsection{Tipi di società}
Ripercorriamo quindi i vari tipi di società:
\begin{itemize}
	\item \textbf{Lucrative:}
		\begin{itemize}
			\item \textbf{Società di persone:}
				\begin{itemize}
					\item Società semplice (S.s.) (solo agricole);
					\item Società in nome collettivo (S.n.c.);
					\item Soietà in accomandita semplice (S.a.s.).
				\end{itemize}
			\item \textbf{Società di capitali:}
				\begin{itemize}
					\item Società in accomandita per azioni (S.a.p.a.);
					\item Società per azioni (S.p.A.);
					\item Società a responsabilità limitata (S.r.l).
				\end{itemize}
		\end{itemize}
	\item \textbf{Mutualistiche:}
		\begin{itemize}
			\item \textbf{Società cooperative:}
				\begin{itemize}
					\item Società cooperativa a responsabilità limitata;
					\item Società cooperativa a responsabilità illimitata;
				\end{itemize}
			\item Mutue assicuratrici.
		\end{itemize}
	\item \textbf{Consortile:} ne possono far parte tutti i tipi tranne la società semplice.
\end{itemize}

Inoltre possiamo distinguere tra società \textbf{for profit} e \textbf{not for profit}, nonché la categoria ibrida delle \textit{società benefit}.

Nelle società in \textit{accomandita} (S.a.s. e S.a.p.a.), si può distinguere fra soci \textbf{accomandanti} e soci \textbf{accomandatari}.
I \textit{soci accomandanti} sono del tutto identici ai soci delle \textbf{società di persone}, mentre i \textit{soci accomandatari} sono del tutto identici ai soci delle \textbf{società di capitali}.

Una delle differenze fra le società di persone e di capitali è rappresentata dalla \textbf{personalità giuridica}.
Le società di capitali hanno personalità giuridica: risponde la società con il proprio capitale, mentre nelle società di persone rispondono i soci con il \textit{loro} capitale.

Fondamentalmente il patrimonio dei soci e il patrimonio della società nelle società di capitali sono fra di loro completamente separati: i creditori personali non possono aggredire il patrimonio sociale, e viceversa i creditori sociali non possono aggredire il patrimonio personale.

Nelle società di persone vale invece l'\textbf{autonomia patrimoniale}, che possiamo intendere come una versione più debole della personalità giuridica.
In questo caso i creditori della società non possono comunque aggredire \textit{direttamente} il patrimonio personale dei soci, quando questi sono illimitatamente responsabili (\textit{beneficio di escussione}).
I creditori personali possono invece, solo nelle società semplici, chiedere la liquidazione della quota in società dei loro debitori, cioè che questa venga venduta.
Negli altri tipi di società, ai soci a responsabilità limitata si può chiedere soltanto il \textit{sequestro conservativo} in caso di liquidazione (per altri motivi) della società.

I patrimoni dei soci e della società nel caso delle società di persone sono quindi \textit{relativamente} separati fra di loro: esistono modalità secondo le quali possono finire a mescolarsi.

\end{document}


\documentclass[a4paper,11pt]{article}
\usepackage[a4paper, margin=8em]{geometry}

% usa i pacchetti per la scrittura in italiano
\usepackage[french,italian]{babel}
\usepackage[T1]{fontenc}
\usepackage[utf8]{inputenc}
\frenchspacing 

% usa i pacchetti per la formattazione matematica
\usepackage{amsmath, amssymb, amsthm, amsfonts}

% usa altri pacchetti
\usepackage{gensymb}
\usepackage{hyperref}
\usepackage{standalone}

% imposta il titolo
\title{Appunti Economia ed Organizzazione Aziendale}
\author{Luca Seggiani}
\date{2025}

% disegni
\usepackage{pgfplots}
\pgfplotsset{width=10cm,compat=1.9}

% imposta lo stile
% usa helvetica
\usepackage[scaled]{helvet}
% usa palatino
\usepackage{palatino}
% usa un font monospazio guardabile
\usepackage{lmodern}

\renewcommand{\rmdefault}{ppl}
\renewcommand{\sfdefault}{phv}
\renewcommand{\ttdefault}{lmtt}

% disponi il titolo
\makeatletter
\renewcommand{\maketitle} {
	\begin{center} 
		\begin{minipage}[t]{.8\textwidth}
			\textsf{\huge\bfseries \@title} 
		\end{minipage}%
		\begin{minipage}[t]{.2\textwidth}
			\raggedleft \vspace{-1.65em}
			\textsf{\small \@author} \vfill
			\textsf{\small \@date}
		\end{minipage}
		\par
	\end{center}

	\thispagestyle{empty}
	\pagestyle{fancy}
}
\makeatother

% disponi teoremi
\usepackage{tcolorbox}
\newtcolorbox[auto counter, number within=section]{theorem}[2][]{%
	colback=blue!10, 
	colframe=blue!40!black, 
	sharp corners=northwest,
	fonttitle=\sffamily\bfseries, 
	title=Teorema~\thetcbcounter: #2, 
	#1
}

% disponi definizioni
\newtcolorbox[auto counter, number within=section]{definition}[2][]{%
	colback=red!10,
	colframe=red!40!black,
	sharp corners=northwest,
	fonttitle=\sffamily\bfseries,
	title=Definizione~\thetcbcounter: #2,
	#1
}

% disponi problemi
\newtcolorbox[auto counter, number within=section]{problem}[2][]{%
	colback=green!10,
	colframe=green!40!black,
	sharp corners=northwest,
	fonttitle=\sffamily\bfseries,
	title=Problema~\thetcbcounter: #2,
	#1
}

% disponi codice
\usepackage{listings}
\usepackage[table]{xcolor}

\lstdefinestyle{codestyle}{
		backgroundcolor=\color{black!5}, 
		commentstyle=\color{codegreen},
		keywordstyle=\bfseries\color{magenta},
		numberstyle=\sffamily\tiny\color{black!60},
		stringstyle=\color{green!50!black},
		basicstyle=\ttfamily\footnotesize,
		breakatwhitespace=false,         
		breaklines=true,                 
		captionpos=b,                    
		keepspaces=true,                 
		numbers=left,                    
		numbersep=5pt,                  
		showspaces=false,                
		showstringspaces=false,
		showtabs=false,                  
		tabsize=2
}

\lstdefinestyle{shellstyle}{
		backgroundcolor=\color{black!5}, 
		basicstyle=\ttfamily\footnotesize\color{black}, 
		commentstyle=\color{black}, 
		keywordstyle=\color{black},
		numberstyle=\color{black!5},
		stringstyle=\color{black}, 
		showspaces=false,
		showstringspaces=false, 
		showtabs=false, 
		tabsize=2, 
		numbers=none, 
		breaklines=true
}

\lstdefinelanguage{javascript}{
	keywords={typeof, new, true, false, catch, function, return, null, catch, switch, var, if, in, while, do, else, case, break},
	keywordstyle=\color{blue}\bfseries,
	ndkeywords={class, export, boolean, throw, implements, import, this},
	ndkeywordstyle=\color{darkgray}\bfseries,
	identifierstyle=\color{black},
	sensitive=false,
	comment=[l]{//},
	morecomment=[s]{/*}{*/},
	commentstyle=\color{purple}\ttfamily,
	stringstyle=\color{red}\ttfamily,
	morestring=[b]',
	morestring=[b]"
}

% disponi sezioni
\usepackage{titlesec}

\titleformat{\section}
	{\sffamily\Large\bfseries} 
	{\thesection}{1em}{} 
\titleformat{\subsection}
	{\sffamily\large\bfseries}   
	{\thesubsection}{1em}{} 
\titleformat{\subsubsection}
	{\sffamily\normalsize\bfseries} 
	{\thesubsubsection}{1em}{}

% disponi alberi
\usepackage{forest}

\forestset{
	rectstyle/.style={
		for tree={rectangle,draw,font=\large\sffamily}
	},
	roundstyle/.style={
		for tree={circle,draw,font=\large}
	}
}

% disponi algoritmi
\usepackage{algorithm}
\usepackage{algorithmic}
\makeatletter
\renewcommand{\ALG@name}{Algoritmo}
\makeatother

% disponi numeri di pagina
\usepackage{fancyhdr}
\fancyhf{} 
\fancyfoot[L]{\sffamily{\thepage}}

\makeatletter
\fancyhead[L]{\raisebox{1ex}[0pt][0pt]{\sffamily{\@title \ \@date}}} 
\fancyhead[R]{\raisebox{1ex}[0pt][0pt]{\sffamily{\@author}}}
\makeatother

\begin{document}

% sezione (data)
\section{Lezione del 13-03-25}

% stili pagina
\thispagestyle{empty}
\pagestyle{fancy}

% testo
Riprendiamo la trattazione dei tipi di società.

\subsubsection{Responsabilità}
E' importante distinguere fra i diversi tipi di responsabilità che possono avere i soci nei diversi tipi di società:
\begin{itemize}
	\item Responsabilità \textbf{illimitata}, tipica delle società di persone, rappresenta responsabilità piena per le obbligazioni sociali, spesso anche \textit{solidale} (obbligazioni pagate in solido da tutti i soci);
	\item Responsabilità \textbf{limitata}, tipica delle società di capitali, rappresenta responsabilità nelle obbligazioni aziendali fino alla quota conferita dal socio.
\end{itemize}

Le società "ibride" (S.a.s. e S.a.p.a.) hanno delle leggere divergenze.
In questo tipo di società esistono due tipi di soci, gli \textbf{accomandatari}, che hanno responsabilità illimitata, e gli \textbf{accomandanti}, che hanno responsabilità limitata alla quota conferita.

La responsabilità di un socio è spesso legata alla sua qualità di \textbf{amministratore}.
Nelle S.n.c. tutti i soci sono illimitatamente responsabili allo stesso modo, e considerati amministratori. 
Nelle S.a.s., chiaramente, i soci accomandatari saranno considerati amministratori, mentre i soci accomandanti no.
Una divisione simile, fra soci amministratori e non, può trovarsi anche nella S.s.

\subsubsection{Organizzazione corporativa}
Nelle società cooperative e nelle società di capitali deve esistere un organizzazione di tipo \textbf{corporativa}, costituita da (ma non limitata a):
\begin{itemize}
	\item L'assemblea dei soci;
	\item Consiglio di amministrazione;
	\item Collegio sindacale.
\end{itemize}

Questo è vero per S.p.A. e S.a.p.a., mentre nelle S.r.l. non è obbligatorio disporre di un collegio sindacale.

Il funzionamento di questi organi è dominato dal principio maggioritario, e il oscio, in quanto tale, non ha alcun potere diretto di amministrazione e controllo.

Il \textbf{management} si distingue dall'amministrazione in quanto è effettivamente formato da \textit{dipendenti} dei soci.
Questo significa che management e amministrazione non spesso condividono le stesse intenzioni (la faccenda si complica oltre se si pensa alle diverse impostazioni delle società, ad esempio si consideri la differenza fra una società per azioni e una società di famiglia rispetto agli interessi dei soci stessi).

\par\smallskip

Nelle società di persone non è invece prevista alcuna organizzazione di tipo corporativo, in quanto abbiamo detto i soci vengono considerati amministratori.

\par\medskip

Iniziamo quindi a vedere nel dettaglio le società che ci interesseranno, quindi le società di capitali.

\subsection{Società per azioni}
La società per azioni è una società di capitali dove i soci vengono rappresentati dagli azionisti (le quote di partecipazione sono azioni).
In quanto società di capitali, alle obbligazioni risponde solo la società con il suo capitale (\textit{personalità giuridica}), e viceversa la società non risponde alle obbligazioni personali dei soci.
E' obbligatoria l'organizzazione cooperativa (in quanto i soci non sono amministratori), ed è previsto un capitale sociale minimo di 50.000 €.

\subsubsection{Costituzione}
Per la costituzione di una S.p.A. servono 2 documenti:
\begin{itemize}
	\item \textbf{Atto costitutivo:} il contratto in cui i soci manifestano la volontà di dare vita al rapporto sociale;
	\item \textbf{Statuto:} riporta le norme di funzionamento della società.
\end{itemize}

Esistono due modalità per la stipulazione dell'atto costitutivo:
\begin{itemize}
	\item \textbf{Simultanea:} tutte le parti costituiscono la società di fronte ad un notaio;
	\item Per \textbf{pubblica sottoscrizione:} la società viene costituita al termine della raccolta iscrizioni.
\end{itemize}

L'atto costitutivo dovrà essere depositato poi entro 20 giorni al registro delle imprese, con iscrizione.

Le condizioni per la costituzione sono:
\begin{itemize}
	\item La sottoscrizione per intero del capitale sociale;
	\item Versamento di almeno il 25\% dei conferimenti in denaro presso un istituto di credito.
\end{itemize}

\subsubsection{Organi sociali}
Vediamo quindi quali sono gli organi sociali della S.p.A. 
\begin{itemize}
	\item Gli \textbf{amministratori} possono essere soci o non soci, può esserci un amministratore unico o una loro pluralità (a formare il consiglio di amministrazione), e possono esistere uno o più organi delegati nel consiglio di amministrazione nel caso questo esista, fra cui:
		\begin{itemize}
			\item Comitato esecutivo;
			\item Amministratori delegati.
		\end{itemize}

		L'amministrazione si occupa di gestione ordinaria (la gestione \textit{straordinaria} spetta ai soci).
\end{itemize}

\end{document}


\documentclass[a4paper,11pt]{article}
\usepackage[a4paper, margin=8em]{geometry}

% usa i pacchetti per la scrittura in italiano
\usepackage[french,italian]{babel}
\usepackage[T1]{fontenc}
\usepackage[utf8]{inputenc}
\frenchspacing 

% usa i pacchetti per la formattazione matematica
\usepackage{amsmath, amssymb, amsthm, amsfonts}

% usa altri pacchetti
\usepackage{gensymb}
\usepackage{hyperref}
\usepackage{standalone}

% imposta il titolo
\title{Appunti Economia ed Organizzazione Aziendale}
\author{Luca Seggiani}
\date{2025}

% disegni
\usepackage{pgfplots}
\pgfplotsset{width=10cm,compat=1.9}

% imposta lo stile
% usa helvetica
\usepackage[scaled]{helvet}
% usa palatino
\usepackage{palatino}
% usa un font monospazio guardabile
\usepackage{lmodern}

\renewcommand{\rmdefault}{ppl}
\renewcommand{\sfdefault}{phv}
\renewcommand{\ttdefault}{lmtt}

% disponi il titolo
\makeatletter
\renewcommand{\maketitle} {
	\begin{center} 
		\begin{minipage}[t]{.8\textwidth}
			\textsf{\huge\bfseries \@title} 
		\end{minipage}%
		\begin{minipage}[t]{.2\textwidth}
			\raggedleft \vspace{-1.65em}
			\textsf{\small \@author} \vfill
			\textsf{\small \@date}
		\end{minipage}
		\par
	\end{center}

	\thispagestyle{empty}
	\pagestyle{fancy}
}
\makeatother

% disponi teoremi
\usepackage{tcolorbox}
\newtcolorbox[auto counter, number within=section]{theorem}[2][]{%
	colback=blue!10, 
	colframe=blue!40!black, 
	sharp corners=northwest,
	fonttitle=\sffamily\bfseries, 
	title=Teorema~\thetcbcounter: #2, 
	#1
}

% disponi definizioni
\newtcolorbox[auto counter, number within=section]{definition}[2][]{%
	colback=red!10,
	colframe=red!40!black,
	sharp corners=northwest,
	fonttitle=\sffamily\bfseries,
	title=Definizione~\thetcbcounter: #2,
	#1
}

% disponi problemi
\newtcolorbox[auto counter, number within=section]{problem}[2][]{%
	colback=green!10,
	colframe=green!40!black,
	sharp corners=northwest,
	fonttitle=\sffamily\bfseries,
	title=Problema~\thetcbcounter: #2,
	#1
}

% disponi codice
\usepackage{listings}
\usepackage[table]{xcolor}

\lstdefinestyle{codestyle}{
		backgroundcolor=\color{black!5}, 
		commentstyle=\color{codegreen},
		keywordstyle=\bfseries\color{magenta},
		numberstyle=\sffamily\tiny\color{black!60},
		stringstyle=\color{green!50!black},
		basicstyle=\ttfamily\footnotesize,
		breakatwhitespace=false,         
		breaklines=true,                 
		captionpos=b,                    
		keepspaces=true,                 
		numbers=left,                    
		numbersep=5pt,                  
		showspaces=false,                
		showstringspaces=false,
		showtabs=false,                  
		tabsize=2
}

\lstdefinestyle{shellstyle}{
		backgroundcolor=\color{black!5}, 
		basicstyle=\ttfamily\footnotesize\color{black}, 
		commentstyle=\color{black}, 
		keywordstyle=\color{black},
		numberstyle=\color{black!5},
		stringstyle=\color{black}, 
		showspaces=false,
		showstringspaces=false, 
		showtabs=false, 
		tabsize=2, 
		numbers=none, 
		breaklines=true
}

\lstdefinelanguage{javascript}{
	keywords={typeof, new, true, false, catch, function, return, null, catch, switch, var, if, in, while, do, else, case, break},
	keywordstyle=\color{blue}\bfseries,
	ndkeywords={class, export, boolean, throw, implements, import, this},
	ndkeywordstyle=\color{darkgray}\bfseries,
	identifierstyle=\color{black},
	sensitive=false,
	comment=[l]{//},
	morecomment=[s]{/*}{*/},
	commentstyle=\color{purple}\ttfamily,
	stringstyle=\color{red}\ttfamily,
	morestring=[b]',
	morestring=[b]"
}

% disponi sezioni
\usepackage{titlesec}

\titleformat{\section}
	{\sffamily\Large\bfseries} 
	{\thesection}{1em}{} 
\titleformat{\subsection}
	{\sffamily\large\bfseries}   
	{\thesubsection}{1em}{} 
\titleformat{\subsubsection}
	{\sffamily\normalsize\bfseries} 
	{\thesubsubsection}{1em}{}

% disponi alberi
\usepackage{forest}

\forestset{
	rectstyle/.style={
		for tree={rectangle,draw,font=\large\sffamily}
	},
	roundstyle/.style={
		for tree={circle,draw,font=\large}
	}
}

% disponi algoritmi
\usepackage{algorithm}
\usepackage{algorithmic}
\makeatletter
\renewcommand{\ALG@name}{Algoritmo}
\makeatother

% disponi numeri di pagina
\usepackage{fancyhdr}
\fancyhf{} 
\fancyfoot[L]{\sffamily{\thepage}}

\makeatletter
\fancyhead[L]{\raisebox{1ex}[0pt][0pt]{\sffamily{\@title \ \@date}}} 
\fancyhead[R]{\raisebox{1ex}[0pt][0pt]{\sffamily{\@author}}}
\makeatother

\begin{document}

% sezione (data)
\section{Lezione del 19-03-25}

% stili pagina
\thispagestyle{empty}
\pagestyle{fancy}

% testo
Riprendiamo la discussione delle società di capitali, e in praticolare delle S.p.A.

\subsubsection{Assemblea}
I soci si costituiscono in \textbf{assemblea}, che può essere \textit{ordinaria} o \textit{straordinaria} (ad esempio, è straordinaria l'assemblea che si costituisce in fase di liquidazione, ma non lo è quella che si costituisce in fase di cessione).

I soci dell'assemblea, che si convoca nella sede legale, con presenza del presidente (solitamente il presidente della società), non è considerata valida alla prima convocazione se si presenta il meno del 50\% (al minimo, può essere alzato da statuto) della quota sociale (si guarda alle \textit{quote} e non alle \textit{teste}).

\subsubsection{Consiglio di amministrazione}
Nelle società di capitali l'amministrazione, come abbiamo visto, può essere anche affidata ai non soci.

Notiamo però la (forse già riportata) dicotomia fra il comportamento dei soci e il comportamento degli amministratori, se questi sono esterni: se il tornaconto degli amministratori è ad esempio quello di ricavare un vantaggio dalle azioni, o comunque dalla performance sul breve termine della società, questi implementeranno allora politiche che massimizzano il profitto sul breve termine ma magari hanno delle conseguenze negative sul lungo termine.
Di contro i soci vorranno, nell'interesse di mantenere il controllo della società per tempi più lunghi, implementare politiche più conservative, o comunque meno rischiose sul breve termine e atte alla conservazione sul lungo termine.
Questo conflitto è oggi presente in maniera anche abbastanza estesa in molte grandi società.

L'\textbf{amministratore delegato} è colui che viene nominato dal consiglio per detenere poteri decisionali più importanti, e in qualche modo rappresentare il volere del consiglio di amministrazione stesso.
In questo si distingue dall'eventuale \textit{amministratore unico}.

\subsubsection{Collegio sindacale}
Il collegio sindacale è costuito da membri sia soci che non soci.
Fa da organo di vigilanza, cioè vigila sull'assetto organizzativo, amministrativo e contabile della società, e eventualmente subentra in ruolo amministrativo (in caso di mancanze da parte degli altri organi). 

\subsection{Società a responsabilità limitata}
Abbiamo visto come nelle S.r.l. non parliamo di azioni, ma di \textbf{quote di partecipazione} (non sono previste azioni).

Il capitale sociale minimo è di 10.000 €, e non è permessa l'emissione di obbligazioni.

Non è obbligatoria la presenza di un collegio sindacale, e le regole di funzionamento dell'assemblea sono semplificate.
Gli amministratori, inoltre, devono essere obbligatoriamente soci.

Esistono una variante dell'S.r.l., l \textit{S.r.l. a un euro}, disciplina di riferimento \textbf{S.r.l. semplificata} (S.r.l.s.).
Questa richiede minori costi di costituzione ed è stata pensata per favorire l'imprenditorialità giovanile.

\subsection{Società in accomandita}
Le società in accomandita rappresentano la via di mezzo fra società di persone e società di capitali.
Sono rappresentate dalle S.a.s. e dalle S.a.p.a.

Abbiamo visto come si distinguono per la differenza fra soci \textit{accomandatari} e soci \textit{accomandati}, che dispongono rispettivamente di responsabilità illimitata e limitata.

\subsection{Titoli di credito}
Iniziamo a parlare dei \textbf{titoli di credito} partendo dalle azioni.

\subsubsection{Azioni}
I \textit{titoli azionari}, comunemente \textbf{azioni}, sono documenti che rappresentano le quote di partecipazione nelle S.p.a. (S.a.s. e S.r.l. non prevedono azioni).

Le azioni possono essere di 2 tipi rispetto alla proprietà:
\begin{itemize}
	\item \textbf{Azioni nominative:} intestate a nome di una persona fisica o giuridica, il cui nome è riportato sull'azione, e su un registro tenuto dalla società emittente (il cosiddetto \textbf{libro soci});
	\item \textbf{Azioni al portatore:} il trasferimento di questo tipo di azione avviene mediante la semplice consegna del titolo all'acquirente. In questo caso il possessore del titolo è legittimato all'esercizio dei suoi diritti di socio previa la sola presentazione del titolo della società. 
\end{itemize}

Si può poi distinguere fra diverse categorie di azione (quasi tutte nominative tranne le \textit{azioni di risparmio}), divise in 2 macrocategorie:
\begin{itemize}
	\item \textbf{Azioni ordinarie}, caratterizzate dai:
	\begin{itemize}
		\item \textit{Diritti di partecipazione alla vita della società}: di partecipazione alle assemblee (ordinarie e straordinarie) e contestualmente di voto nelle assemblee;
		\item \textit{Diritti patrimoniali}: diritto al \textbf{dividendo}, cioè alla ricezione di una parte di utile della società (\textit{se} l'assemblea che approva il bilancio, cioè l'ordinaria, approva la distribuzione utili), e alla restituzione del capitale in caso di scioglimento della società o di riduzione del capitale sociale.
			Sempre dal punto di vista patrimoniale, chi detiene azioni ordinarie ha l'obbligo di concorrere alle perdite della società. 
	\end{itemize}
	\item \textbf{Azioni speciali}, che si dividono in diverse categorie:
		\begin{itemize}
			\item \textbf{Azioni privilegiate:} disposte al di sotto delle azioni di risparmio in termini di priorità sulla distribuzione dell'utile (le ordinarie stanno sotto, e tutte le altre categorie di azione stanno ancora sotto); 
			\item \textbf{Azioni di godimento:} come le azioni di risparmio, ma senza i privilegi in termini di distribuzione dell'utile;
			\item \textbf{Azioni assegnate ai prestatori di lavoro};
			\item \textbf{Azioni con prestazioni accessorie};
			\item \textbf{Azioni a voto limitato};
			\item \textbf{Azioni di risparmio:} sono le sole che possono essere intestate \textit{al portatore}, e quindi che non sono nominative.
				Hanno il diritto di partecipazione e intervento in assemblea, ma non il diritto di voto.
				Il diritto di partecipazione alla vita della società è quindi limitato, mentre è rafforzato quello patrimoniale: infatti assicurano un dividendo annuo minimo pari al 5\% del valore nominale dell'azione, e l'eventuale distribuzione degli utili residui deve essere effettuata in modo che allezioni di risparmio corrisponda il 2\% in più rispetto alle azioni ordinarie.

				Riguardo al \textbf{valore} dell'azione, possiamo fare una parentesi.
				\begin{enumerate}
					\item Un azione ha un valore \textbf{nominale} che rappresenta la frazione di capitale sociale che questa rappresenta (cioè il numero di \textit{azioni} in senso stretto che sono erogate in un azione-documento);
					\item C'è poi il valore di \textbf{emissione}, cioè il prezzo di emisione dell'azione stessa dalla società al momento dell'emissione nel mercato mobiliare primario;
					\item Infine, c'è il valore di \textbf{mercato}, cioè il prezzo che l'azione ha sul mercato mobiliare secondario, cioè la quotazione del giorno in borsa, determinata dai meccanismi della domanda/offerta.
				\end{enumerate}

				Tornando alle azioni di risparmio, queste hanno altri due privilegi: sono privilegiate (sopra alle privilegiate stesse) nella distribuzione del'utile e nella restituzione del capitale, e vengono dopo alle azioni ordinarie per la concorrenza alle perdite.

				Le azioni di risparmio sono quelle che si comprano e vendono più spesso nel mercato della borsa, in quanto l'interesse dei possessori è principalmente quello patrimoniale (per cui potrebbero decidere di lucrare sulla differenza in caso di crescita del valore di mercato).
				Di contro, chi acquista azioni ordinarie cerca solitamente una qualche partecipazione, e quindi \textit{controllo}, sulla società. 
		\end{itemize}
\end{itemize}

\subsection{Obbligazioni}
Le obbligazioni sono titoli di credito nominativi o al portatore che rappresentano frazioni di uguale valore nominale e con uguali diritti di un operazione di finanziamento unitaria a titolo di mutuo.
In questo, chi detiene le obbligazioni è per le società un creditore, e non un socio.

La differenza fra azioni e obbigazioni si può schematizzare nella seguente tabella:

\begin{table}[H]
	\center \rowcolors{2}{white}{black!10}
	\begin{tabular} {p{7cm} | p{7cm}}
		\bfseries Azione & \bfseries Obbligazione \\
		\hline 
		Qualità di socio & Qualità di creditore della società \\
		Diritto di compartecipare ai risultati & Remunerazione periodica fissa (\textit{interesse}) svincolata dai risultati \\
		Rimborso del capitale conferito solo in sede di liquidazione, e residuale (cioè se ne rimane un attivo netto). Inoltre la quota di liquidazione è svincolata al valore nominale di conferimento. & Diritto al rimborso del valore nominale del capitale prestato alla scadenza, nella sua totalità. \\
	\end{tabular}
\end{table}

Facciamo una nota sulle \textbf{obbligazione convertibili}, cioè obbligazioni che possono essere convertite in azione.

\end{document}


\documentclass[a4paper,11pt]{article}
\usepackage[a4paper, margin=8em]{geometry}

% usa i pacchetti per la scrittura in italiano
\usepackage[french,italian]{babel}
\usepackage[T1]{fontenc}
\usepackage[utf8]{inputenc}
\frenchspacing 

% usa i pacchetti per la formattazione matematica
\usepackage{amsmath, amssymb, amsthm, amsfonts}

% usa altri pacchetti
\usepackage{gensymb}
\usepackage{hyperref}
\usepackage{standalone}

% imposta il titolo
\title{Appunti Economia ed Organizzazione Aziendale}
\author{Luca Seggiani}
\date{2025}

% disegni
\usepackage{pgfplots}
\pgfplotsset{width=10cm,compat=1.9}

% imposta lo stile
% usa helvetica
\usepackage[scaled]{helvet}
% usa palatino
\usepackage{palatino}
% usa un font monospazio guardabile
\usepackage{lmodern}

\renewcommand{\rmdefault}{ppl}
\renewcommand{\sfdefault}{phv}
\renewcommand{\ttdefault}{lmtt}

% disponi il titolo
\makeatletter
\renewcommand{\maketitle} {
	\begin{center} 
		\begin{minipage}[t]{.8\textwidth}
			\textsf{\huge\bfseries \@title} 
		\end{minipage}%
		\begin{minipage}[t]{.2\textwidth}
			\raggedleft \vspace{-1.65em}
			\textsf{\small \@author} \vfill
			\textsf{\small \@date}
		\end{minipage}
		\par
	\end{center}

	\thispagestyle{empty}
	\pagestyle{fancy}
}
\makeatother

% disponi teoremi
\usepackage{tcolorbox}
\newtcolorbox[auto counter, number within=section]{theorem}[2][]{%
	colback=blue!10, 
	colframe=blue!40!black, 
	sharp corners=northwest,
	fonttitle=\sffamily\bfseries, 
	title=Teorema~\thetcbcounter: #2, 
	#1
}

% disponi definizioni
\newtcolorbox[auto counter, number within=section]{definition}[2][]{%
	colback=red!10,
	colframe=red!40!black,
	sharp corners=northwest,
	fonttitle=\sffamily\bfseries,
	title=Definizione~\thetcbcounter: #2,
	#1
}

% disponi problemi
\newtcolorbox[auto counter, number within=section]{problem}[2][]{%
	colback=green!10,
	colframe=green!40!black,
	sharp corners=northwest,
	fonttitle=\sffamily\bfseries,
	title=Problema~\thetcbcounter: #2,
	#1
}

% disponi codice
\usepackage{listings}
\usepackage[table]{xcolor}

\lstdefinestyle{codestyle}{
		backgroundcolor=\color{black!5}, 
		commentstyle=\color{codegreen},
		keywordstyle=\bfseries\color{magenta},
		numberstyle=\sffamily\tiny\color{black!60},
		stringstyle=\color{green!50!black},
		basicstyle=\ttfamily\footnotesize,
		breakatwhitespace=false,         
		breaklines=true,                 
		captionpos=b,                    
		keepspaces=true,                 
		numbers=left,                    
		numbersep=5pt,                  
		showspaces=false,                
		showstringspaces=false,
		showtabs=false,                  
		tabsize=2
}

\lstdefinestyle{shellstyle}{
		backgroundcolor=\color{black!5}, 
		basicstyle=\ttfamily\footnotesize\color{black}, 
		commentstyle=\color{black}, 
		keywordstyle=\color{black},
		numberstyle=\color{black!5},
		stringstyle=\color{black}, 
		showspaces=false,
		showstringspaces=false, 
		showtabs=false, 
		tabsize=2, 
		numbers=none, 
		breaklines=true
}

\lstdefinelanguage{javascript}{
	keywords={typeof, new, true, false, catch, function, return, null, catch, switch, var, if, in, while, do, else, case, break},
	keywordstyle=\color{blue}\bfseries,
	ndkeywords={class, export, boolean, throw, implements, import, this},
	ndkeywordstyle=\color{darkgray}\bfseries,
	identifierstyle=\color{black},
	sensitive=false,
	comment=[l]{//},
	morecomment=[s]{/*}{*/},
	commentstyle=\color{purple}\ttfamily,
	stringstyle=\color{red}\ttfamily,
	morestring=[b]',
	morestring=[b]"
}

% disponi sezioni
\usepackage{titlesec}

\titleformat{\section}
	{\sffamily\Large\bfseries} 
	{\thesection}{1em}{} 
\titleformat{\subsection}
	{\sffamily\large\bfseries}   
	{\thesubsection}{1em}{} 
\titleformat{\subsubsection}
	{\sffamily\normalsize\bfseries} 
	{\thesubsubsection}{1em}{}

% disponi alberi
\usepackage{forest}

\forestset{
	rectstyle/.style={
		for tree={rectangle,draw,font=\large\sffamily}
	},
	roundstyle/.style={
		for tree={circle,draw,font=\large}
	}
}

% disponi algoritmi
\usepackage{algorithm}
\usepackage{algorithmic}
\makeatletter
\renewcommand{\ALG@name}{Algoritmo}
\makeatother

% disponi numeri di pagina
\usepackage{fancyhdr}
\fancyhf{} 
\fancyfoot[L]{\sffamily{\thepage}}

\makeatletter
\fancyhead[L]{\raisebox{1ex}[0pt][0pt]{\sffamily{\@title \ \@date}}} 
\fancyhead[R]{\raisebox{1ex}[0pt][0pt]{\sffamily{\@author}}}
\makeatother

\begin{document}

% sezione (data)
\section{Lezione del 20-03-25}

% stili pagina
\thispagestyle{empty}
\pagestyle{fancy}

% testo
\subsection{Il not for profit}
Nell'ordinamento corrente la società parte di default come \textit{for profit}, cioè a scopo di lucro.
Abbiamo però accennato all'esistenza di società \textit{not for profit}, e delle più recenti \textbf{società benefit}.
Notiamo che queste si distinguono poi da altri enti non a scopo di lucro, come ad esempio quelli che si occupano di \textit{volontariato}.

\end{document}


\documentclass[a4paper,11pt]{article}
\usepackage[a4paper, margin=8em]{geometry}

% usa i pacchetti per la scrittura in italiano
\usepackage[french,italian]{babel}
\usepackage[T1]{fontenc}
\usepackage[utf8]{inputenc}
\frenchspacing 

% usa i pacchetti per la formattazione matematica
\usepackage{amsmath, amssymb, amsthm, amsfonts}

% usa altri pacchetti
\usepackage{gensymb}
\usepackage{hyperref}
\usepackage{standalone}

% imposta il titolo
\title{Appunti Economia ed Organizzazione Aziendale}
\author{Luca Seggiani}
\date{2025}

% disegni
\usepackage{pgfplots}
\pgfplotsset{width=10cm,compat=1.9}

% imposta lo stile
% usa helvetica
\usepackage[scaled]{helvet}
% usa palatino
\usepackage{palatino}
% usa un font monospazio guardabile
\usepackage{lmodern}

\renewcommand{\rmdefault}{ppl}
\renewcommand{\sfdefault}{phv}
\renewcommand{\ttdefault}{lmtt}

% disponi il titolo
\makeatletter
\renewcommand{\maketitle} {
	\begin{center} 
		\begin{minipage}[t]{.8\textwidth}
			\textsf{\huge\bfseries \@title} 
		\end{minipage}%
		\begin{minipage}[t]{.2\textwidth}
			\raggedleft \vspace{-1.65em}
			\textsf{\small \@author} \vfill
			\textsf{\small \@date}
		\end{minipage}
		\par
	\end{center}

	\thispagestyle{empty}
	\pagestyle{fancy}
}
\makeatother

% disponi teoremi
\usepackage{tcolorbox}
\newtcolorbox[auto counter, number within=section]{theorem}[2][]{%
	colback=blue!10, 
	colframe=blue!40!black, 
	sharp corners=northwest,
	fonttitle=\sffamily\bfseries, 
	title=Teorema~\thetcbcounter: #2, 
	#1
}

% disponi definizioni
\newtcolorbox[auto counter, number within=section]{definition}[2][]{%
	colback=red!10,
	colframe=red!40!black,
	sharp corners=northwest,
	fonttitle=\sffamily\bfseries,
	title=Definizione~\thetcbcounter: #2,
	#1
}

% disponi problemi
\newtcolorbox[auto counter, number within=section]{problem}[2][]{%
	colback=green!10,
	colframe=green!40!black,
	sharp corners=northwest,
	fonttitle=\sffamily\bfseries,
	title=Problema~\thetcbcounter: #2,
	#1
}

% disponi codice
\usepackage{listings}
\usepackage[table]{xcolor}

\lstdefinestyle{codestyle}{
		backgroundcolor=\color{black!5}, 
		commentstyle=\color{codegreen},
		keywordstyle=\bfseries\color{magenta},
		numberstyle=\sffamily\tiny\color{black!60},
		stringstyle=\color{green!50!black},
		basicstyle=\ttfamily\footnotesize,
		breakatwhitespace=false,         
		breaklines=true,                 
		captionpos=b,                    
		keepspaces=true,                 
		numbers=left,                    
		numbersep=5pt,                  
		showspaces=false,                
		showstringspaces=false,
		showtabs=false,                  
		tabsize=2
}

\lstdefinestyle{shellstyle}{
		backgroundcolor=\color{black!5}, 
		basicstyle=\ttfamily\footnotesize\color{black}, 
		commentstyle=\color{black}, 
		keywordstyle=\color{black},
		numberstyle=\color{black!5},
		stringstyle=\color{black}, 
		showspaces=false,
		showstringspaces=false, 
		showtabs=false, 
		tabsize=2, 
		numbers=none, 
		breaklines=true
}

\lstdefinelanguage{javascript}{
	keywords={typeof, new, true, false, catch, function, return, null, catch, switch, var, if, in, while, do, else, case, break},
	keywordstyle=\color{blue}\bfseries,
	ndkeywords={class, export, boolean, throw, implements, import, this},
	ndkeywordstyle=\color{darkgray}\bfseries,
	identifierstyle=\color{black},
	sensitive=false,
	comment=[l]{//},
	morecomment=[s]{/*}{*/},
	commentstyle=\color{purple}\ttfamily,
	stringstyle=\color{red}\ttfamily,
	morestring=[b]',
	morestring=[b]"
}

% disponi sezioni
\usepackage{titlesec}

\titleformat{\section}
	{\sffamily\Large\bfseries} 
	{\thesection}{1em}{} 
\titleformat{\subsection}
	{\sffamily\large\bfseries}   
	{\thesubsection}{1em}{} 
\titleformat{\subsubsection}
	{\sffamily\normalsize\bfseries} 
	{\thesubsubsection}{1em}{}

% disponi alberi
\usepackage{forest}

\forestset{
	rectstyle/.style={
		for tree={rectangle,draw,font=\large\sffamily}
	},
	roundstyle/.style={
		for tree={circle,draw,font=\large}
	}
}

% disponi algoritmi
\usepackage{algorithm}
\usepackage{algorithmic}
\makeatletter
\renewcommand{\ALG@name}{Algoritmo}
\makeatother

% disponi numeri di pagina
\usepackage{fancyhdr}
\fancyhf{} 
\fancyfoot[L]{\sffamily{\thepage}}

\makeatletter
\fancyhead[L]{\raisebox{1ex}[0pt][0pt]{\sffamily{\@title \ \@date}}} 
\fancyhead[R]{\raisebox{1ex}[0pt][0pt]{\sffamily{\@author}}}
\makeatother

\begin{document}

% sezione (data)
\section{Lezione del 27-03-25}

% stili pagina
\thispagestyle{empty}
\pagestyle{fancy}

% testo
Riprendiamo la discussione delle strutture organizzative.

\subsubsection{Diversifizazione e differenziazione}
I termini \textbf{diversificazione} e \textbf{differenziazione} fanno riferimento alla \textbf{gamma} di prodotti di un'impresa.
La gamma di prodotti di un'impresa è definita dalle \textbf{linee} di prodotto.
Ad esempio, nel caso di un impresa che si occupa di abbigliamento, si potrebbe avere una linea di jeans, una linea di camicie, una linea di intimo, ecc... fra di loro distinte.
Questo tipo di sviluppo in \textbf{verticale} (aumento del numero di linee) rappresenta una \textit{diversificazione}, quindi più linee di prodotto sono disponibili, più l'impresa diversifica.

Chiaramente, aumentando le linee di prodotto le imprese entrano in nuove \textit{linee di business}.
Ad esempio, l'impresa di abbigliamento di prima potrebbe diversificare entrando nell'arredo casa, ecc...

Guardando alle singole linee di prodotto, invece, ci rendiamo conto che questa può essere più o meno \textbf{profonda}: per \textit{profondità} ci riferiamo al numero di \textbf{varianti} di prodotto all'interno di quella linea.
Quindi nel caso dei jeans si potrebbe dividere in più tipologie (regular, skinny, ecc...).
A loro volta, queste tipologie si distinguono in taglie, colore, ecc...
Si ha quindi che più una linea si divide in \textbf{varianti} di prodotto, più questa è \textit{differenziata}.

\par\smallskip

Tornando al discorso delle \textit{strutture organizzative}, direzioni tipiche di sviluppo delle imprese, man di mano che crescono in dimensioni e complessità, sono quelle di diversificazione e differenziazione. 

\subsubsection{Progettazione della microstruttura}
Avevamo parlato di \textbf{macro} e \textbf{micro}, e riguardo alla \textit{micro}-struttura, che si occupa di stabilire i meccanismi di coordinazione fra individui e gruppi e di definirne la \textit{specializzazione}, avevamo parlato di specializzazione \textbf{orizzontale} e \textbf{verticale}.

Iniziamo quindi a vedere più nel dettaglio i concetti fondamentali di \textbf{compito}, \textbf{mansione} e \textbf{ruolo}.
\begin{itemize}
	\item \textbf{Compito:} l'elemento più piccolo di lavoro che andiamo a considerare, inteso come un insieme di attività o operazioni necessariamente collegate in funzione di proprietà/capacità del lavoro umano;
	\item \textbf{Mansione:} l'insieme di compiti assegnabili ad una certa \textbf{posizione};
	\item \textbf{Ruolo:} termine che deriva dal mondo del teatro, che identifica l'insieme delle aspettative sul \textit{comportamento} di una persona in riferimento agli obiettivi dell'intera organizzazione.

		Si parla quindi non solo della \textit{mansione} che un individuo deve svolgere, ma anche sul tipo di \textit{comportamento} che questo deve mantenere nello svolgimento di tale mansione.
		In questo potremmo dire che la mansione è fondamentalmente \textit{asettica}, mentre il ruolo copre anche le aspettative di comportamento che si hanno sull'individuo che la svolge.
\end{itemize}

La diversificazione dei compiti assegnabili ad una posizione (la mansione) rappresentava quella che avevamo definito \textbf{specializzazione orizzontale}.
Veniamo quindi alla \textbf{specializzazione verticale}.
Questa riguarda la quantità di autonomia che un individuo ha nello svolgimento della sua mansione, e quindi alla quantità di \textit{controllo} che ha sul \textit{come} viene svolta la mansione.

Possiamo individuare la seguente tabella che distingue fra specializzazioni orizzontali e verticali, alte e basse:
\begin{table}[h!]
	\center 
	\begin{tabular} { c | c c }
		& \bfseries Orizzontale alta & \bfseries Orizzontale bassa \\
		\hline
		\bfseries Verticale alta & Mansioni non qualificate & Manager di basso livello \\
		\bfseries Verticale bassa & Mansioni professionali & Manager \\
	\end{tabular}
\end{table}

Abbiamo quindi che i \textbf{manager} hanno bassa specializzazione sia verticale che orizzontale: si occupano quindi di una vasta gamma di compiti con una grande autonomia, con il caso specifico dei manager \textit{di basso livello}, che hanno minore controllo (specializzazione verticale).

Le \textbf{mansioni professionali} (cioè che riguardano i \textit{professionisti}), invece, hanno alta specializzazione orizzontale e bassa specializzazione verticale, cioè si occupano di ambiti specifici ma con un grande livello di autonomia.
Le mansioni \textit{non qualificate} invece, hanno alta specializzazione sia orizzontale che verticale, cioè compiti specifici con bassa autonomia (in questo non richiedono \textit{qualifiche}).

\subsubsection{Riprogettazione delle mansioni}
Abbiamo diverse modalità di riprogettazione delle mansioni durante lo sviluppo dell'impresa:
\begin{itemize}
	\item \textbf{Job enlargement:} l'ampliamento dei compiti elementari assegnati a ciascuna posizione;
	\item \textbf{Job enrichement:} aumento della discrezionalità riaccorpando compiti di programmazione e controllo dei risultati, cioè aumento del grado di autonomia;
	\item \textbf{Job rotation:} rotazione dei compiti ftra gli individui per ottenere minore ripetitività e rigidità.
\end{itemize}

Abbiamo quindi che se partiamo da mansioni \textit{tayloristiche} (ad alta sepcializzazione orizzontale e verticale), il job enlargement e la job rotation abbassano la specializzazione orizzontale (\textit{mansioni polivalenti}); il job enlargement e il job enrichment abbassano la specializzazione orizzontale e verticale, e infine il job enrichement di per sé abbassa la specializzazione verticale (maggiore autonomia).

\subsubsection{Formalizzazione della mansione}
Il \textbf{mansionario} è la descrizione verbale dei compiti assegnati ad una \textbf{posizione} od \textbf{unità organizzativo}.
IN questo il mansionario garantisce il prestatore d'opera ed esplicita nero su bianco le dipendenze e i compiti assegnate alle diverse mansioni.
Di contro, può rappresentare uno strumento molto rigido, con basso livello di libertà nella diversifiazione delle mansioni. 

\subsubsection{Matrice di responsabilità}
Un'altro strumento per la formalizzazione delle mansioni è la \textbf{matrice delle responsabilità}, che rappresenta le responsabilità (più propriamente i tipi di coinvolgimento) dei diversi attori riguardo ai diversi compiti.
La struttura di una matrice di responsabilità è la seguente:
\begin{table}[h!]
	\center \rowcolors{2}{white}{black!10}
	\begin{tabular} {  c | c c }
		\bfseries Attori / Compiti & A & B \\
		\hline 
		1 & Modalità A-1 & Modalità B-1 \\ 
		2 & Modalità A-2 & Modalià B-2
	\end{tabular}
\end{table}

\subsection{Progettazione della macrostruttura}
La \textbf{macrostruttura} tiene conto delle \textbf{unità organizzative}, cioè l'integrazione delle posizioni e delle unità stesse.
Quindi, in breve, un unità organizzativa si può intendere come un insieme di posizioni, o di altre unità.

Possiamo definire riguardo alle unità tre parametri:
\begin{itemize}
	\item I \textbf{criteri di raggruppamento};
	\item I \textbf{meccanismi} e i \textbf{ruoli di collegamento};
	\item La \textbf{formalizzazione} della struttura (\textbf{organigramma}).
\end{itemize}

\subsubsection{Criteri di raggruppamento}
I criteri possono essere \textbf{funzionali} o \textbf{divisionali}.
\begin{itemize}
	\item \textbf{Criteri funzionali:} in questi la base di raggruppamento per la formazione dell'unita è orientata agli \textit{input}: quindi fa riferimento  alle competenze necessarie, o al tipo di processo di cui si occupa l'unità;
	\item \textbf{Criteri divisionali:} in questi la base di raggruppamento è invece l'\textit{output}: si guarda al prodotto, al cliente e all'area geografic.
\end{itemize}
Notiamo che il problema della scelta dei criteri di raggruppamento si ripete a ciascun livello della struttura.



\end{document}


\documentclass[a4paper,11pt]{article}
\usepackage[a4paper, margin=8em]{geometry}

% usa i pacchetti per la scrittura in italiano
\usepackage[french,italian]{babel}
\usepackage[T1]{fontenc}
\usepackage[utf8]{inputenc}
\frenchspacing 

% usa i pacchetti per la formattazione matematica
\usepackage{amsmath, amssymb, amsthm, amsfonts}

% usa altri pacchetti
\usepackage{gensymb}
\usepackage{hyperref}
\usepackage{standalone}

% imposta il titolo
\title{Appunti Economia ed Organizzazione Aziendale}
\author{Luca Seggiani}
\date{2025}

% disegni
\usepackage{pgfplots}
\pgfplotsset{width=10cm,compat=1.9}

% imposta lo stile
% usa helvetica
\usepackage[scaled]{helvet}
% usa palatino
\usepackage{palatino}
% usa un font monospazio guardabile
\usepackage{lmodern}

\renewcommand{\rmdefault}{ppl}
\renewcommand{\sfdefault}{phv}
\renewcommand{\ttdefault}{lmtt}

% disponi il titolo
\makeatletter
\renewcommand{\maketitle} {
	\begin{center} 
		\begin{minipage}[t]{.8\textwidth}
			\textsf{\huge\bfseries \@title} 
		\end{minipage}%
		\begin{minipage}[t]{.2\textwidth}
			\raggedleft \vspace{-1.65em}
			\textsf{\small \@author} \vfill
			\textsf{\small \@date}
		\end{minipage}
		\par
	\end{center}

	\thispagestyle{empty}
	\pagestyle{fancy}
}
\makeatother

% disponi teoremi
\usepackage{tcolorbox}
\newtcolorbox[auto counter, number within=section]{theorem}[2][]{%
	colback=blue!10, 
	colframe=blue!40!black, 
	sharp corners=northwest,
	fonttitle=\sffamily\bfseries, 
	title=Teorema~\thetcbcounter: #2, 
	#1
}

% disponi definizioni
\newtcolorbox[auto counter, number within=section]{definition}[2][]{%
	colback=red!10,
	colframe=red!40!black,
	sharp corners=northwest,
	fonttitle=\sffamily\bfseries,
	title=Definizione~\thetcbcounter: #2,
	#1
}

% disponi problemi
\newtcolorbox[auto counter, number within=section]{problem}[2][]{%
	colback=green!10,
	colframe=green!40!black,
	sharp corners=northwest,
	fonttitle=\sffamily\bfseries,
	title=Problema~\thetcbcounter: #2,
	#1
}

% disponi codice
\usepackage{listings}
\usepackage[table]{xcolor}

\lstdefinestyle{codestyle}{
		backgroundcolor=\color{black!5}, 
		commentstyle=\color{codegreen},
		keywordstyle=\bfseries\color{magenta},
		numberstyle=\sffamily\tiny\color{black!60},
		stringstyle=\color{green!50!black},
		basicstyle=\ttfamily\footnotesize,
		breakatwhitespace=false,         
		breaklines=true,                 
		captionpos=b,                    
		keepspaces=true,                 
		numbers=left,                    
		numbersep=5pt,                  
		showspaces=false,                
		showstringspaces=false,
		showtabs=false,                  
		tabsize=2
}

\lstdefinestyle{shellstyle}{
		backgroundcolor=\color{black!5}, 
		basicstyle=\ttfamily\footnotesize\color{black}, 
		commentstyle=\color{black}, 
		keywordstyle=\color{black},
		numberstyle=\color{black!5},
		stringstyle=\color{black}, 
		showspaces=false,
		showstringspaces=false, 
		showtabs=false, 
		tabsize=2, 
		numbers=none, 
		breaklines=true
}

\lstdefinelanguage{javascript}{
	keywords={typeof, new, true, false, catch, function, return, null, catch, switch, var, if, in, while, do, else, case, break},
	keywordstyle=\color{blue}\bfseries,
	ndkeywords={class, export, boolean, throw, implements, import, this},
	ndkeywordstyle=\color{darkgray}\bfseries,
	identifierstyle=\color{black},
	sensitive=false,
	comment=[l]{//},
	morecomment=[s]{/*}{*/},
	commentstyle=\color{purple}\ttfamily,
	stringstyle=\color{red}\ttfamily,
	morestring=[b]',
	morestring=[b]"
}

% disponi sezioni
\usepackage{titlesec}

\titleformat{\section}
	{\sffamily\Large\bfseries} 
	{\thesection}{1em}{} 
\titleformat{\subsection}
	{\sffamily\large\bfseries}   
	{\thesubsection}{1em}{} 
\titleformat{\subsubsection}
	{\sffamily\normalsize\bfseries} 
	{\thesubsubsection}{1em}{}

% disponi alberi
\usepackage{forest}

\forestset{
	rectstyle/.style={
		for tree={rectangle,draw,font=\large\sffamily}
	},
	roundstyle/.style={
		for tree={circle,draw,font=\large}
	}
}

% disponi algoritmi
\usepackage{algorithm}
\usepackage{algorithmic}
\makeatletter
\renewcommand{\ALG@name}{Algoritmo}
\makeatother

% disponi numeri di pagina
\usepackage{fancyhdr}
\fancyhf{} 
\fancyfoot[L]{\sffamily{\thepage}}

\makeatletter
\fancyhead[L]{\raisebox{1ex}[0pt][0pt]{\sffamily{\@title \ \@date}}} 
\fancyhead[R]{\raisebox{1ex}[0pt][0pt]{\sffamily{\@author}}}
\makeatother

\begin{document}

% sezione (data)
\section{Lezione del 02-04-25}

% stili pagina
\thispagestyle{empty}
\pagestyle{fancy}

% testo
\subsection{Configurazioni organizzative}
Iniziamo a vedere gli organigrammi delle configurazioni organizzative più popolari:
\begin{itemize}
	\item \textbf{Struttura semplice:} l'\textit{alta direzione} controlla più unità fra di loro separate, ma allo stesso livello.
		Questa è la struttura tipica di imprese giovani o di piccole dimensioni.
		I meccanismi di coordinamento principali sono l'adattamento reciproco, la supervisione diretta e la standardizzazione delle capacità.
		\begin{itemize}
			\item \textbf{Punti di forza:} flessibilità;
			\item \textbf{Punti di debolezza:} problemi di conflitti o congestione del vertice.
		\end{itemize}

	\item \textbf{Struttura funzionale:} una struttura più articolata, dove l'\textit{alta direzione} dirige una serie di \textit{unità funzionali}, che non sono più quello che probabilmente nella struttura semplice erano individui, ma unità di più persone atte ad un unico scopo.
		E' quindi necessario un criterio di divisione del lavoro basato sul tipo di processo o tecnica, e sulle conoscenze.
		Questa è la struttura tipica di società di dimensioni piccole-medie.
		Rimangono dalla struttura semplice, poi, la supervisione diretta e la standardizzazione dei processi di lavoro.
		\begin{itemize}
			\item \textbf{Punti di forza:} permette di ottimizzare l'impiego delle risorse umane e tecnologiche concentrando risorse simili e favorendo la specializzazione, ottenendo così \textit{economie di scala}; 
			\item \textbf{Punti di debolezza:} potrebbe non esserci collaborazione fra i diversi "blocchi" funzionali, cosa che impatta la produttività; inoltre è meno flessibile.
		\end{itemize}
		
	\item \textbf{Struttura divisionale:} una struttura ancora più articolata, dove le unità organizzative sono costituite su un criterio basato sugli \textit{output}.
		Le unità sono quindi orientate a diversi clienti, prodotti o aree geografiche.
		Ogni unità divisionale è a sua volta divisa in maniera funzionale, nelle diverse funzioni che essa ha.
		Si avrà quindi, ad esempio, la divisione orientata al prodotto A, con il \textit{reparto vendite} del prodotto A, il \textit{reparto produzione} del prodotto A, ecc...
		Questa struttura è tipica di società grandi e sviluppate in diversi settori.
		\begin{itemize}
			\item \textbf{Punti di forza:} si recuperà la velocità di risposta al mercato (singole divisioni possono reagire a variazioni riguardo al loro prodotto, mercato, area geografica ecc...);
			\item \textbf{Punti di debolezza:} si rinuncia in parte all'economia di scala, e in genere alla specializzazione, con efficienze ed incoerenze date dalla duplicazione delle risorse fra le divisioni.
		\end{itemize}
\end{itemize}

\subsection{Meccanismi di collegamento}
Possiamo individuare diversi meccanismi di collegamento, ad esempio:
\begin{itemize}
	\item \textbf{Posizioni di collegamento:} posizioni, perlopià informali, che si trovano a "metà" fra più blocchi funzionali; 
	\item \textbf{Team interfunzionali:} gruppi di lavoro istituzionalizzati generalmente interdisciplinari.
		Possiamo distingurli ulteriormente rispetto all'\textbf{orizzonte} della collaborazione e alla sua \textbf{intensità}:
		\begin{table}[h!]
			\center \rowcolors{2}{white}{black!10}
			\begin{tabular} { c | c | c }
				& \bfseries Collaborazione continua & \bfseries Collaborazione discontinua \\ 
				\hline
				\bfseries Orizzonte permanente & // & Comitato \\ 
				\bfseries Orizzonte temporaneao & Task force & Riunione
			\end{tabular}
		\end{table}
	\item \textbf{Manager integratori:} si sovrappongono alla struttura organizzativa esistente, con l'obiettivo di occuparsi di un ambito specifico senza detenere tutte le leve di potere tipiche di un manager di livello superiore.
\end{itemize}

\subsection{Bilancio}
Il \textbf{bilancio} è uno schema diviso in due sezioni:
\begin{itemize}
	\item \textbf{Stato patrimoniale};
	\item \textbf{Conto economico}.
\end{itemize}
Il bilancio è corredato da un documento testuale, detto \textbf{nota integrativa}.

Lo \textbf{stato di salute} del bilancio può essere valutato secondo diversi indici:
\begin{itemize}
	\item Stato della cassa;
	\item Crediti e debiti;
	\item Il fatturato;
	\item Il valore di mercato delle azioni; 
\end{itemize}
\end{document}


\documentclass[a4paper,11pt]{article}
\usepackage[a4paper, margin=8em]{geometry}

% usa i pacchetti per la scrittura in italiano
\usepackage[french,italian]{babel}
\usepackage[T1]{fontenc}
\usepackage[utf8]{inputenc}
\frenchspacing 

% usa i pacchetti per la formattazione matematica
\usepackage{amsmath, amssymb, amsthm, amsfonts}

% usa altri pacchetti
\usepackage{gensymb}
\usepackage{hyperref}
\usepackage{standalone}

% imposta il titolo
\title{Appunti Economia ed Organizzazione Aziendale}
\author{Luca Seggiani}
\date{2025}

% disegni
\usepackage{pgfplots}
\pgfplotsset{width=10cm,compat=1.9}

% imposta lo stile
% usa helvetica
\usepackage[scaled]{helvet}
% usa palatino
\usepackage{palatino}
% usa un font monospazio guardabile
\usepackage{lmodern}

\renewcommand{\rmdefault}{ppl}
\renewcommand{\sfdefault}{phv}
\renewcommand{\ttdefault}{lmtt}

% disponi il titolo
\makeatletter
\renewcommand{\maketitle} {
	\begin{center} 
		\begin{minipage}[t]{.8\textwidth}
			\textsf{\huge\bfseries \@title} 
		\end{minipage}%
		\begin{minipage}[t]{.2\textwidth}
			\raggedleft \vspace{-1.65em}
			\textsf{\small \@author} \vfill
			\textsf{\small \@date}
		\end{minipage}
		\par
	\end{center}

	\thispagestyle{empty}
	\pagestyle{fancy}
}
\makeatother

% disponi teoremi
\usepackage{tcolorbox}
\newtcolorbox[auto counter, number within=section]{theorem}[2][]{%
	colback=blue!10, 
	colframe=blue!40!black, 
	sharp corners=northwest,
	fonttitle=\sffamily\bfseries, 
	title=Teorema~\thetcbcounter: #2, 
	#1
}

% disponi definizioni
\newtcolorbox[auto counter, number within=section]{definition}[2][]{%
	colback=red!10,
	colframe=red!40!black,
	sharp corners=northwest,
	fonttitle=\sffamily\bfseries,
	title=Definizione~\thetcbcounter: #2,
	#1
}

% disponi problemi
\newtcolorbox[auto counter, number within=section]{problem}[2][]{%
	colback=green!10,
	colframe=green!40!black,
	sharp corners=northwest,
	fonttitle=\sffamily\bfseries,
	title=Problema~\thetcbcounter: #2,
	#1
}

% disponi codice
\usepackage{listings}
\usepackage[table]{xcolor}

\lstdefinestyle{codestyle}{
		backgroundcolor=\color{black!5}, 
		commentstyle=\color{codegreen},
		keywordstyle=\bfseries\color{magenta},
		numberstyle=\sffamily\tiny\color{black!60},
		stringstyle=\color{green!50!black},
		basicstyle=\ttfamily\footnotesize,
		breakatwhitespace=false,         
		breaklines=true,                 
		captionpos=b,                    
		keepspaces=true,                 
		numbers=left,                    
		numbersep=5pt,                  
		showspaces=false,                
		showstringspaces=false,
		showtabs=false,                  
		tabsize=2
}

\lstdefinestyle{shellstyle}{
		backgroundcolor=\color{black!5}, 
		basicstyle=\ttfamily\footnotesize\color{black}, 
		commentstyle=\color{black}, 
		keywordstyle=\color{black},
		numberstyle=\color{black!5},
		stringstyle=\color{black}, 
		showspaces=false,
		showstringspaces=false, 
		showtabs=false, 
		tabsize=2, 
		numbers=none, 
		breaklines=true
}

\lstdefinelanguage{javascript}{
	keywords={typeof, new, true, false, catch, function, return, null, catch, switch, var, if, in, while, do, else, case, break},
	keywordstyle=\color{blue}\bfseries,
	ndkeywords={class, export, boolean, throw, implements, import, this},
	ndkeywordstyle=\color{darkgray}\bfseries,
	identifierstyle=\color{black},
	sensitive=false,
	comment=[l]{//},
	morecomment=[s]{/*}{*/},
	commentstyle=\color{purple}\ttfamily,
	stringstyle=\color{red}\ttfamily,
	morestring=[b]',
	morestring=[b]"
}

% disponi sezioni
\usepackage{titlesec}

\titleformat{\section}
	{\sffamily\Large\bfseries} 
	{\thesection}{1em}{} 
\titleformat{\subsection}
	{\sffamily\large\bfseries}   
	{\thesubsection}{1em}{} 
\titleformat{\subsubsection}
	{\sffamily\normalsize\bfseries} 
	{\thesubsubsection}{1em}{}

% disponi alberi
\usepackage{forest}

\forestset{
	rectstyle/.style={
		for tree={rectangle,draw,font=\large\sffamily}
	},
	roundstyle/.style={
		for tree={circle,draw,font=\large}
	}
}

% disponi algoritmi
\usepackage{algorithm}
\usepackage{algorithmic}
\makeatletter
\renewcommand{\ALG@name}{Algoritmo}
\makeatother

% disponi numeri di pagina
\usepackage{fancyhdr}
\fancyhf{} 
\fancyfoot[L]{\sffamily{\thepage}}

\makeatletter
\fancyhead[L]{\raisebox{1ex}[0pt][0pt]{\sffamily{\@title \ \@date}}} 
\fancyhead[R]{\raisebox{1ex}[0pt][0pt]{\sffamily{\@author}}}
\makeatother

\begin{document}

% sezione (data)
\section{Lezione del 03-04-25}

% stili pagina
\thispagestyle{empty}
\pagestyle{fancy}

% testo
\subsection{Bilancio}
Il \textbf{bilancio} è uno schema diviso in due sezioni:
\begin{itemize}
	\item \textbf{Stato patrimoniale:} riguarda la differenza fra \textit{dare} e \textit{avere} dell'azienda, in relazione al cosiddetto \textbf{capitale netto}.
		Riguarda quindi lo stato del patrimonio dell azienda in un dato momento temporale, tenuto conto dei beni, crediti, ecc... (\textit{avere}) che essa possiede, e dei debiti, obbligazioni, ecc... (\textit{dare}) che essa deve a terzi;
	\item \textbf{Conto economico}: riguarda la sola prestazione economica, quindi la differenza fra ricavi e spese, in un dato periodo in relazione all'attività economica dell'azienda stessa.
\end{itemize}
Il bilancio è corredato da un documento testuale, detto \textbf{nota integrativa}, che specifica informazioni riguardo ai vari campi dello stato patrimoniale e del conto economico.

Dal bilancio ci interessa valutare lo \textbf{stato di salute} dell'azienda, che a priori si può dire essere legato a diversi indici, fra cui:
\begin{itemize}
	\item Stato della cassa;
	\item Crediti e debiti;
	\item Il fatturato;
	\item Il valore di mercato delle azioni; 
\end{itemize}

\subsection{Struttura di un bilancio}
Un bilancio è solitamente sviluppato su due colonne, la colonna di \textit{sinistra} e la colonna di \textit{destra}, che contengono informazioni di natura diversa.
Ad esempio, vedremo che nello stato patrimoniale la colonna sinistra è l'\textit{avere}, e la colonna di destra è il \textit{dare}, con aggiunto il capitale netto, in modo che il bilancio \textit{"quadri"}, cioè la somma algebrica delle due colonne sia uguale.

\begin{table}[h!]
	\center \rowcolors{2}{white}{black!10}
	\begin{tabular} { p{4cm} c | p{4cm} c }
		\bfseries Colonna sinistra &  & \bfseries Colonna destra & \\	
		\hline 
		Entrata 1 sinistra & 100.000 & Entrata 1 destra & 50.000 \\ 
		\vdots & & \vdots & \\
		\bfseries Totale sinstra & 150.000 & \bfseries Totale destra & 150.000 
	\end{tabular}
\end{table}

\subsubsection{Misurazione dello stato di salute}
Vediamo quindi nel dettaglio i \textbf{modelli} per lo stato di salute, fra cui distinguiamo:
\begin{itemize}
	\item Il \textbf{modello contabile}, che rappresenta una \textit{retrospettiva} (quindi dati prevalentemente \textit{storici}).
		Tiene quindi conto della \textbf{capitalizzazione} dei flussi di cassa futuri.
		
		Il modello contabile considera intervalli di tempo pari a 12 mesi, detti \textbf{anni contabili}, che non corrispondono necessariamente agli anni reali.
	\item Il \textbf{modello del valore}, derivante dalla \textit{finanza}, che rappresenta invece una \textit{prospettiva} (quindi è composto anche da \textit{stime}).
		Tiene quindi conto dell'\textbf{attualizzazione} dei flussi di cassa futuri.
\end{itemize}

\subsubsection{Evoluzione dell'azienda}
Possiamo individuare diverse fasi temporali, cioè:
\begin{itemize}
	\item \textbf{Fase di impianto:} dove si effettuano le operazioni di acquisto dei fattori pluriennalie e le operazioni di finanziamento;
	\item \textbf{Fase di regime:} dove si svolge il ciclo di acquisto-produzione-vendita;
	\item \textbf{Fase di cessazione:} dove si effettuano operazioni \textit{straordinarie}, cioè fusione, fallimento, cessione o trasformazione. 
\end{itemize}
Le prime 2 fasi vengono dette \textbf{fasi di funzionamento}, in quanto vi individuiamo le \textbf{ipotesi di funzionamento}.
Durante la fase di funzionamento, si redime quello che è il \textbf{bilancio di esercizio}, o \textit{bilancio ordinario}.

Nella fase di cessazione, invece, si redige il \textbf{bilancio straordinario}.

\subsection{Modello contabile}
Il \textbf{modello contabile}, che è quello secondo il quale è redatto lo \textit{stato patrimoniale}, usa 2 misuratori:
\begin{itemize}
	\item Il \textbf{reddito}, che è un misuratore di flusso;
	\item Il \textbf{capitale}, che è un misuratore di stock.
\end{itemize}
A seconda della fase in cui l'impresa si trova, si hanno poi degli aggettivi che si aggiungono a tali misuratori.
In particolare, nella fase di esercizio si ha:
\begin{itemize}
	\item Il \textbf{reddito di esercizio};
	\item Il \textbf{capitale di funzionamento}.
\end{itemize}
Nella fase di cessazione, invece, e quindi sul bilancio straordinario, si ha il \textbf{capitale di cessione}. 

Il modello contabile viene tenuto nello \textbf{stato patrimoniale}. 

\subsubsection{Reddito}
Dal punto di vista matematico, preso un certo capitale $c_1$ ad un istante $t_1$, ed un capitale $c_2$ ad un istante $t_2$, possiamo definire la variazione di capitale $\Delta c = c_2 - c_1$ come reddito.
Un reddito $\Delta c > 0$ è detto \textbf{utile} di esercizio, mentre un reddito $\Delta c < 0$ è detta \textbf{perdita}.

I flussi in entrata nel capitale $c$ vengono detti \textbf{ricavi} di esercizio, mentre i flussi in uscita \textbf{costi} di esercizi.
Il reddito potrà quindi essere difinito, date $r_i$ e $k_i$ ricavi e costi mensili, come la sommatoria sul periodo di 12 mesi:
$$
\Delta c = \sum_{i = 1}^{12} \left( r_i - k_i \right)
$$

\subsubsection{Capitale}
Veniamo quindi a definire il capitale vero e proprio. 
Questo rappresenta l'insieme dei beni che possono essere impiegati nell'attivitò di impresa, ed è costituito da \textbf{attivi} e \textbf{passivi}  
La sommma algebrica attività e passività dà il cosiddetto \textbf{capitale netto}:
$$
A - P = \text{capitale netto}
$$
Nello stato patrimoniale, il capitale netto si mette a \textbf{destra}, assieme ai passivi, mentre gli attivi si mettono a \textbf{sinistra}, così che la somma delle due colonne sia identica secondo l'identità:
$$
A = P + \text{capitale netto}
$$
In questo, il capitale netto è un valore astratto espresso in moneta di conto, che rappresenta la differenza fra attivi e passivi.

\subsubsection{Valori finanziari}
Potremmo pensare di tenere un bilancio, più semplice, della sola \textbf{cassa}, dei debiti e dei crediti, con le stesse convenzioni sinistra-destra (l'unica differenza è che debiti in \textit{discesa} stanno a \textbf{sinistra}).
In particolare, lo schema riferito a un oggetto viene detto \textbf{mastrino}, il mastrino asociato ad un oggetto e quindi una regola contabile viene detto \textbf{conto}, e un insieme di conti viene detto \textbf{sistema contabile}.

Cassa, banca, crediti, debiti ecc... sono dal punto di vista contabile valori \textbf{finanziari}.
Si può quindi tenere un conto del valore finanziario, dove a sinistra stanno le variazioni in positivo e a destra le variazioni in negativo.
I conti che abbiamo riportato prima di cassa, debiti e crediti farebbero parte di queste variazioni.
Si avrannno quindi due variazioni dei valori finanziari, in positivo e in negativo, cioè $\Delta F+$ e $\Delta F-$.

\subsubsection{Valori economici}
Infine consideriamo i valori economici.
Questi riguardano capitale e reddito:
\begin{itemize}
	\item Riguardo al \textit{capitale netto}, la \textbf{variazione} di capitale;
	\item Riguardo al \textit{reddito}, i \textbf{ricavi} e i \textbf{costi}.
\end{itemize}

I valori economici vengono tenuti nel \textbf{conto economico}, che è strutturato al contrario: il capitale netto (capitale sociale), assieme ai valori finanziari e ai ricavi, stanno a \textbf{destra}, mentre i costi stanno a \textbf{sinistra}.

\par\medskip

Abbiamo quindi che valori finanziari ed economici sono legati: se il bilancio dei valori finanziari non corrisponde, significa che c'è stata qualche azione da parte dei valori economici (perdite o utili), e quindi una variazione del capitale netto.

La variazione del capitale netto sul capitale proprio, abbiamo detto, è quella che viene chiamata \textbf{utile} ($\Delta > 0$) o \textbf{perdita} ($\Delta < 0$).

\end{document}


\documentclass[a4paper,11pt]{article}
\usepackage[a4paper, margin=8em]{geometry}

% usa i pacchetti per la scrittura in italiano
\usepackage[french,italian]{babel}
\usepackage[T1]{fontenc}
\usepackage[utf8]{inputenc}
\frenchspacing 

% usa i pacchetti per la formattazione matematica
\usepackage{amsmath, amssymb, amsthm, amsfonts}

% usa altri pacchetti
\usepackage{gensymb}
\usepackage{hyperref}
\usepackage{standalone}

% imposta il titolo
\title{Appunti Economia ed Organizzazione Aziendale}
\author{Luca Seggiani}
\date{2025}

% disegni
\usepackage{pgfplots}
\pgfplotsset{width=10cm,compat=1.9}

% imposta lo stile
% usa helvetica
\usepackage[scaled]{helvet}
% usa palatino
\usepackage{palatino}
% usa un font monospazio guardabile
\usepackage{lmodern}

\renewcommand{\rmdefault}{ppl}
\renewcommand{\sfdefault}{phv}
\renewcommand{\ttdefault}{lmtt}

% disponi il titolo
\makeatletter
\renewcommand{\maketitle} {
	\begin{center} 
		\begin{minipage}[t]{.8\textwidth}
			\textsf{\huge\bfseries \@title} 
		\end{minipage}%
		\begin{minipage}[t]{.2\textwidth}
			\raggedleft \vspace{-1.65em}
			\textsf{\small \@author} \vfill
			\textsf{\small \@date}
		\end{minipage}
		\par
	\end{center}

	\thispagestyle{empty}
	\pagestyle{fancy}
}
\makeatother

% disponi teoremi
\usepackage{tcolorbox}
\newtcolorbox[auto counter, number within=section]{theorem}[2][]{%
	colback=blue!10, 
	colframe=blue!40!black, 
	sharp corners=northwest,
	fonttitle=\sffamily\bfseries, 
	title=Teorema~\thetcbcounter: #2, 
	#1
}

% disponi definizioni
\newtcolorbox[auto counter, number within=section]{definition}[2][]{%
	colback=red!10,
	colframe=red!40!black,
	sharp corners=northwest,
	fonttitle=\sffamily\bfseries,
	title=Definizione~\thetcbcounter: #2,
	#1
}

% disponi problemi
\newtcolorbox[auto counter, number within=section]{problem}[2][]{%
	colback=green!10,
	colframe=green!40!black,
	sharp corners=northwest,
	fonttitle=\sffamily\bfseries,
	title=Problema~\thetcbcounter: #2,
	#1
}

% disponi codice
\usepackage{listings}
\usepackage[table]{xcolor}

\lstdefinestyle{codestyle}{
	backgroundcolor=\color{black!5}, 
	commentstyle=\color{codegreen},
	keywordstyle=\bfseries\color{magenta},
	numberstyle=\sffamily\tiny\color{black!60},
	stringstyle=\color{green!50!black},
	basicstyle=\ttfamily\footnotesize,
	breakatwhitespace=false,         
	breaklines=true,                 
	captionpos=b,                    
	keepspaces=true,                 
	numbers=left,                    
	numbersep=5pt,                  
	showspaces=false,                
	showstringspaces=false,
	showtabs=false,                  
	tabsize=2
}

\lstdefinestyle{shellstyle}{
	backgroundcolor=\color{black!5}, 
	basicstyle=\ttfamily\footnotesize\color{black}, 
	commentstyle=\color{black}, 
	keywordstyle=\color{black},
	numberstyle=\color{black!5},
	stringstyle=\color{black}, 
	showspaces=false,
	showstringspaces=false, 
	showtabs=false, 
	tabsize=2, 
	numbers=none, 
	breaklines=true
}

\lstdefinelanguage{javascript}{
	keywords={typeof, new, true, false, catch, function, return, null, catch, switch, var, if, in, while, do, else, case, break},
	keywordstyle=\color{blue}\bfseries,
	ndkeywords={class, export, boolean, throw, implements, import, this},
	ndkeywordstyle=\color{darkgray}\bfseries,
	identifierstyle=\color{black},
	sensitive=false,
	comment=[l]{//},
	morecomment=[s]{/*}{*/},
	commentstyle=\color{purple}\ttfamily,
	stringstyle=\color{red}\ttfamily,
	morestring=[b]',
	morestring=[b]"
}

% disponi sezioni
\usepackage{titlesec}

\titleformat{\section}
{\sffamily\Large\bfseries} 
{\thesection}{1em}{} 
\titleformat{\subsection}
{\sffamily\large\bfseries}   
{\thesubsection}{1em}{} 
\titleformat{\subsubsection}
{\sffamily\normalsize\bfseries} 
{\thesubsubsection}{1em}{}

% disponi alberi
\usepackage{forest}

\forestset{
	rectstyle/.style={
		for tree={rectangle,draw,font=\large\sffamily}
	},
	roundstyle/.style={
		for tree={circle,draw,font=\large}
	}
}

% disponi algoritmi
\usepackage{algorithm}
\usepackage{algorithmic}
\makeatletter
\renewcommand{\ALG@name}{Algoritmo}
\makeatother

% disponi numeri di pagina
\usepackage{fancyhdr}
\fancyhf{} 
\fancyfoot[L]{\sffamily{\thepage}}

\makeatletter
\fancyhead[L]{\raisebox{1ex}[0pt][0pt]{\sffamily{\@title \ \@date}}} 
\fancyhead[R]{\raisebox{1ex}[0pt][0pt]{\sffamily{\@author}}}
\makeatother

\begin{document}

% sezione (data)
\section{Lezione del 09-04-25}

% stili pagina
\thispagestyle{empty}
\pagestyle{fancy}

% testo
Concludiamo il discorso sul bilancio parlando dell'\textit{esercizio}.

\subsubsection{Esercizio}
Alla scorsa lezione avevamo parlato di \textbf{capitale} e \textbf{reddito}, e di come questi avevano importanza sotto l'ipotesi di funzionamento, dando quindi vita al \textbf{bilancio di esercizio}.
Il bilancio di esercizio viene tenuto lungo il periodo di tempo detto \textbf{esercizio}, corrispondente a l'\textit{anno contabile} (dal 1/1 al 31/12).
Per l'interesse degli investitori si possono tenere anche bilanci \textit{infra}-annuali, magari a scadenza trimestrale o semestrale.
Questi hanno principalmente lo scopo di informare gli inverstitori e quindi il mercato finanziario.

\subsection{Operazioni di gestione}
Iniziamo a vedere nel dettaglio le operazioni di gestione delle società, partendo da quelle \textit{straordinarie}.

\subsubsection{Operazioni di gestione straordinaria}
Le operazioni di gestione \textbf{straordinaria} possono essere le seguenti:
\begin{itemize}
	\item \textbf{Cessione:} vendita in blocco dell'azienda, al termine della quale questa può continuare a vivere in capo ad un nuovo soggetto (\textit{cessazione relativa}) o essere liquidata (\textit{cessazione assoluta}).
		Queste casistiche si distinguono nella tabella:
		\begin{table}[h!]
			\center \rowcolors{2}{white}{black!10}
			\begin{tabular} { c | c | c }
				& \bfseries Cessazione assoluta & \bfseries Cessazione relativa \\
				\hline
				\bfseries Cessazione volontaria & Liquidazione & Cessione \\ 
				\bfseries Cessazione coatta & Liquidazione & // \\
			\end{tabular}
		\end{table}

		La \textit{liquidazione} può essere \textit{coatta} (\textbf{fallimento}), cioè imposta dalla legge, o volontaria.

		Nel caso della \textit{cessione}, invece, si fa un bilancio apposito, detto \textbf{bilancio di cessazione}, appunto per valutare l'ipotesi di vendita (e dall'altra parte, di acquisto) dell'azienda.
		Qui, ad esempio, si tiene conto dell'\textit{avviamento} dell'azienda.
		L'ipotesi di cessione coatta, invece, non è prevista dalla legge.

	\item \textbf{Fusione:} si riduce all'unità il patrimonio di due o più società e si fanno confluire i soci in un unica struttura organizzata (che è una società preesistente o fondata \textit{ex novo}).

		La fusione può in particolare essere:
		\begin{itemize}
			\item \textbf{Fusione ropriamente detta:} più società si \textit{estinguono}, cioè cessano di esistere, per dar vita ad una nuova società creata da zero;
			\item \textbf{Fusione per incorporazione:} più società vengono \textit{incorporate} in una società preesistente.
		\end{itemize}
	\item \textbf{Scissione:} trasferimento del patrimonio di una società preesistente a più società, che può essere totale (\textbf{scissione integrale}, o \textit{split-up}) o parziale (\textbf{scissione parziale} o \textit{split-off}).
		Solo nel caso di scissione parziale, ovviamente, si può scindere ad una sola società (in caso contrario si parlerebbe di qualche tipo di trasformazione).

		La scissione di tipo \textit{split-off} è tipica di società che si specializzano al punto di dover dividere attività e passitvità in altre società.
		In questo caso, le azioni delle società create vengono assegnate ai soci della società madre, che le scambiano con le azioni della società madre da loro possedute.
		Chiaramente, se questa scissione lascia la società madre senza attività o passività e una scissione integrale o \textit{split-up}.

	\item \textbf{Scorporo:} sembra simile alla scissione parziale, ma porta alla formazione di \textit{spin-off}.
		In questo caso la società madre cede la sua azienda o rami di essa ad altre società, formando quello che è effettivamente un \textbf{gruppo aziendale}.
		In questo caso le azioni delle società scisse restano nella società madre, che diventa \textbf{caopgruppo} di quel gruppo aziendale, cioè una società il cui unico interesse e gestire le altre società del gruppo.

	\item \textbf{Trasformazione:} la società (non s.s.) cambia forma modificando il suo atto costitutivo (magari passa da S.r.l. a S.p.A., ecc...). 
		Consiste quindi in un cambio di veste giuridica di un soggetto giuridico presistente.
\end{itemize}

\subsubsection{Operazioni di gestione ordinaria}
Vediamo quindi nel dettaglio quali sono e in cosa consistono le operazioni di gestione ordinaria della società.
Abbiamo visto il discorso dello stato patrimoniale, e quindi del capitale e del reddito.

Possiamo pensare ad un esempio che espliciti il funzionamento del meccanismo dello stato patrimoniale.
Abbiamo che questo reagisce solamente a operazioni esterne, che influenzano:
\begin{itemize}
	\item \textbf{Cassa}, \textbf{conti bancari} o \textbf{posta};
	\item \textbf{Crediti};
	\item \textbf{Debiti}.
\end{itemize}

Ad esempio l'estinguimento del debito con i soldi provenienti dalla cassa di un'azienda debitrice rappresenta una \textbf{permutazione finanziaria} che \textit{chiude}, cioè ha somma zero da entrambi i lati del bilancio: i soldi che escono dalla cassa ($\Delta F < 0$) vanno ad estinguere il debito ($\Delta F > 0$), cioè non variano il capitale sociale.

Un'altro movimento finanziario è quello che riguarda il \textbf{capitale Netto}, (\textit{Capitale Sociale} $+ \, \Delta$): ad esempio, un aumento del capitale sociale comporta una variazione economica, che è positiva ($\Delta F > 0$) se si aumenta il capitale (creando nuove azioni) e negativa ($\Delta F < 0$) se si diminusice (annullando azioni esistenti senza aumentare il valore nominale).

Potremmo considerare allora le operazioni di \textbf{finanziamento attinto}, cioè l'operazione mediante la quale l’azienda so dota dei mezzi monetari necessari per lo svolgimento della sua attività, ottenendo appunto un finanzamento da terzi.
In questo caso si crea un aumento della cassa, ma anche un debito (tra l'altro con possibile interesse) da restituire, ergo il capitale netto resta costante (se non addirittura diminuisce, $\Delta F < 0$, all'estinzione del debito).

\par\smallskip

Vediamo allora se le operazioni \textbf{economiche} $\Delta E$, cioè quelle che riguardano \textit{costi} e \textit{ricavi}, possono cambiare il capitale netto, introducendo variazioni finanziarie $\Delta F$ minori o maggiori di 0.

Queste riguardano quelli che sono i \textbf{valori economici} dell'azienda, che abbiamo già visto essere legati al \textbf{reddito.} 
Da qqui si possono avere risultati economici:
\begin{itemize}
	\item $\Delta E > 0$, si parla di \textbf{ricavi};
	\item $\Delta E < 0$, si parla di \textbf{costi} che possono essere:
		\begin{itemize}
			\item Costi \textbf{di esercizio};
			\item Costi \textbf{pluriennali}.
		\end{itemize}
\end{itemize}
Notiamo che questa operazioni non provocano variazioni di capitale netto, in quanto il costo dell'\textbf{acquisto} di merci abbatte il valore delle merci, mentre il ricavo dalla \textbf{vendita} di beni è abbattuto dalla perdita di tali beni.

A far variare la ricchezza nel corso dell'esercizio è allora la differenza fra ricavi e costi \textit{di esercizio} alla fine di un certo periodo, cioè:
$$
\sum_{i = 1}^n R - \sum_{i = 1}^n C = \Delta E
$$
Come abbiamo già visto, se questa $\Delta E$ è positiva, si parla di \textbf{utile}, altrimenti si parla di \textbf{perdita}.

\subsubsection{Operazioni di gestione esterna}
Le operazioni di cui abbiamo parlato finora soono operazioni di \textbf{gestione esterna}, cioè che guardano a quello che la società scambia con l'esterno.
Si ha che ogni oeprazione di gestione esterna ha sempre almeno 2 movimenti:
\begin{enumerate}
	\item Il primo movimento è di tipo finanziario (movimentazione di cassa, estinzione di debiti, ecc...), cioè un $\Delta F$;
	\item Il secondo movimento è di tipo finanziario o di tipo economico.
		\begin{itemize}
			\item Se il secondo movimento è di tipo finanziario, può \textit{chiudere} (portando quindi a nessuna variazione di capitale netto), quindi rappresentare una \textbf{permutazione finanziaria}, oppure non chiudere, portando ad una variazione economica $\Delta E$ dello stesso segno;
		\end{itemize}
\end{enumerate}

# qui c'è un grafico boh

\subsection{Conto economico}
Il \textbf{conto economico} fa parte assieme allo \textit{stato patrimoniale}, che abbiamo visto finora, del \textbf{bilancio}.

Questo, al contrario dello stato patrimoniale, riporta il valore delle risorse che sono state consumate in un certo periodo (\textit{costi}) per ottenere dei \textit{ricavi}.

# eh poi boh qui finì

\subsection{IVA}
L'\textbf{IVA}, \textit{Imposta sul Valore Aggiunto}, è un \textbf{imposta} applicata sul valore aggiunto in ogni fase di produzione di un bene.

\par\smallskip

La differenza fra \textit{imposta} e \textit{tassa} è il principio della \textit{controprestazione}: a una tassa corrisponde un bene o servizio che viene erogato dallo stato o dall'ente pubblico a cui la si paga, mentre l'imposta si traduce in un finanziamento dei servizi pubblici generali (pubblici nel senso di forniti anche a chi \textit{non} paga le imposte). 

\par\smallskip

L'IVA viene pagata dal consumatore finale, e si accumula in tutte le fasi di produzione di un certo bene.
Ad esempio, poniamo che il produttore principale produce un prodotto ad un certo valore, detto \textbf{base imponibile}.
Su questo valore si calcolerà una percentuale (22\% all'ordinamento corrente) di \textbf{valore aggiunto}, che il produttore deve versare allo stato.
Questa percentuale dovrà essere pagata, assieme alla base imponibile, da tutte quelle aziende che elaborano tale prodotto, ricalcolata ad ogni passaggio della catena di produzione.
In verità, vediamo che queste aziende non pagano di tasca propria il valore dell'IVA, ma lo \textit{"anticipano"} al consumatore (che può essere un'altra azienda produttrice o il consumatore finale), aggiungendolo al prezzo del prodotto venduto alla fine della produzione (come aveva fatto il produttore principale stesso).
Alla fine l'IVA accumulata viene quindi pagata dall'ultimo che entra in possesso del prodotto, cioè il consumatore finale.


\end{document}


\documentclass[a4paper,11pt]{article}
\usepackage[a4paper, margin=8em]{geometry}

% usa i pacchetti per la scrittura in italiano
\usepackage[french,italian]{babel}
\usepackage[T1]{fontenc}
\usepackage[utf8]{inputenc}
\frenchspacing 

% usa i pacchetti per la formattazione matematica
\usepackage{amsmath, amssymb, amsthm, amsfonts}

% usa altri pacchetti
\usepackage{gensymb}
\usepackage{hyperref}
\usepackage{standalone}

% imposta il titolo
\title{Appunti Economia ed Organizzazione Aziendale}
\author{Luca Seggiani}
\date{2025}

% disegni
\usepackage{pgfplots}
\pgfplotsset{width=10cm,compat=1.9}

% imposta lo stile
% usa helvetica
\usepackage[scaled]{helvet}
% usa palatino
\usepackage{palatino}
% usa un font monospazio guardabile
\usepackage{lmodern}

\renewcommand{\rmdefault}{ppl}
\renewcommand{\sfdefault}{phv}
\renewcommand{\ttdefault}{lmtt}

% disponi il titolo
\makeatletter
\renewcommand{\maketitle} {
	\begin{center} 
		\begin{minipage}[t]{.8\textwidth}
			\textsf{\huge\bfseries \@title} 
		\end{minipage}%
		\begin{minipage}[t]{.2\textwidth}
			\raggedleft \vspace{-1.65em}
			\textsf{\small \@author} \vfill
			\textsf{\small \@date}
		\end{minipage}
		\par
	\end{center}

	\thispagestyle{empty}
	\pagestyle{fancy}
}
\makeatother

% disponi teoremi
\usepackage{tcolorbox}
\newtcolorbox[auto counter, number within=section]{theorem}[2][]{%
	colback=blue!10, 
	colframe=blue!40!black, 
	sharp corners=northwest,
	fonttitle=\sffamily\bfseries, 
	title=Teorema~\thetcbcounter: #2, 
	#1
}

% disponi definizioni
\newtcolorbox[auto counter, number within=section]{definition}[2][]{%
	colback=red!10,
	colframe=red!40!black,
	sharp corners=northwest,
	fonttitle=\sffamily\bfseries,
	title=Definizione~\thetcbcounter: #2,
	#1
}

% disponi problemi
\newtcolorbox[auto counter, number within=section]{problem}[2][]{%
	colback=green!10,
	colframe=green!40!black,
	sharp corners=northwest,
	fonttitle=\sffamily\bfseries,
	title=Problema~\thetcbcounter: #2,
	#1
}

% disponi codice
\usepackage{listings}
\usepackage[table]{xcolor}

\lstdefinestyle{codestyle}{
		backgroundcolor=\color{black!5}, 
		commentstyle=\color{codegreen},
		keywordstyle=\bfseries\color{magenta},
		numberstyle=\sffamily\tiny\color{black!60},
		stringstyle=\color{green!50!black},
		basicstyle=\ttfamily\footnotesize,
		breakatwhitespace=false,         
		breaklines=true,                 
		captionpos=b,                    
		keepspaces=true,                 
		numbers=left,                    
		numbersep=5pt,                  
		showspaces=false,                
		showstringspaces=false,
		showtabs=false,                  
		tabsize=2
}

\lstdefinestyle{shellstyle}{
		backgroundcolor=\color{black!5}, 
		basicstyle=\ttfamily\footnotesize\color{black}, 
		commentstyle=\color{black}, 
		keywordstyle=\color{black},
		numberstyle=\color{black!5},
		stringstyle=\color{black}, 
		showspaces=false,
		showstringspaces=false, 
		showtabs=false, 
		tabsize=2, 
		numbers=none, 
		breaklines=true
}

\lstdefinelanguage{javascript}{
	keywords={typeof, new, true, false, catch, function, return, null, catch, switch, var, if, in, while, do, else, case, break},
	keywordstyle=\color{blue}\bfseries,
	ndkeywords={class, export, boolean, throw, implements, import, this},
	ndkeywordstyle=\color{darkgray}\bfseries,
	identifierstyle=\color{black},
	sensitive=false,
	comment=[l]{//},
	morecomment=[s]{/*}{*/},
	commentstyle=\color{purple}\ttfamily,
	stringstyle=\color{red}\ttfamily,
	morestring=[b]',
	morestring=[b]"
}

% disponi sezioni
\usepackage{titlesec}

\titleformat{\section}
	{\sffamily\Large\bfseries} 
	{\thesection}{1em}{} 
\titleformat{\subsection}
	{\sffamily\large\bfseries}   
	{\thesubsection}{1em}{} 
\titleformat{\subsubsection}
	{\sffamily\normalsize\bfseries} 
	{\thesubsubsection}{1em}{}

% disponi alberi
\usepackage{forest}

\forestset{
	rectstyle/.style={
		for tree={rectangle,draw,font=\large\sffamily}
	},
	roundstyle/.style={
		for tree={circle,draw,font=\large}
	}
}

% disponi algoritmi
\usepackage{algorithm}
\usepackage{algorithmic}
\makeatletter
\renewcommand{\ALG@name}{Algoritmo}
\makeatother

% disponi numeri di pagina
\usepackage{fancyhdr}
\fancyhf{} 
\fancyfoot[L]{\sffamily{\thepage}}

\makeatletter
\fancyhead[L]{\raisebox{1ex}[0pt][0pt]{\sffamily{\@title \ \@date}}} 
\fancyhead[R]{\raisebox{1ex}[0pt][0pt]{\sffamily{\@author}}}
\makeatother

\begin{document}

% sezione (data)
\section{Lezione del 10-04-25}

% stili pagina 
\thispagestyle{empty}
\pagestyle{fancy}

% testo
\subsection{Nota integrativa}
La nota integrativa è un documento di testo che funge in qualche modo da \textbf{legenda} dello \textit{stato patrimoniale} e del \textit{conto economico}, e contiene quindi informazioni quali la data di acquisto di immobili, il modo in cui è valutato l'ammortamento, verso chi abbiamo crediti o debiti, la possibilità che i debitori hanno di pagarci o meno, ecc... 

\subsection{Prudenza}
Uno dei principi fondamentali del bilancio è il \textbf{principio di prudenza}: i \textbf{ricavi} di esercizio vanno nel conto economico \textit{solo} se sono certi, mentre i \textbf{costi} ci vanno anche se \textit{presunti}.

Questo principio ha lo scopo di salvaguardare gli \textbf{stakeholder} (clienti, dipendenti, stato, ecc...), in quanto impedisce all'impresa di \textit{"gonfiare"} il proprio capitale. 

\subsubsection{Fondi spesa e rischio}
A proposito dei crediti, distinguiamo il \textbf{valore nominale dei crediti} dal prezzo dopo la \textbf{svalutazione crediti}.
Questo va a finire nel cosiddetto \textbf{fondo svalutazione crediti}, che è una voce fra le passività dello stato patrimoniale, di valore uguale a cioè che viene accantonato del credito nominale.
Visto che non c'è un'altro movimento finanziario, la svalutazione crediti rappresenta un costo dal punto di vista del conto economico d'esercizio.

Il fondo svalutazione crediti fa parte della categoria più ampia dei \textbf{fondi rischi}: esistono es esempio:
\begin{itemize}
	\item \textbf{Fondi titoli}, riguardanti i titoli finanziari più o meno recuperabili;
	\item \textbf{Fondi cambi}, riguardanti i crediti in valute straniere, il cui valore è variabile nel tempo.
\end{itemize}

Il fondo è quindi uno strumento che ci permette di ammortizzare, nell'anno contabile, i rischi \textit{previsti} d'esercizio: al momento del pagamento \textit{parziale} del credito, ad esempio, si può usare il fondo svalutazione crediti per pareggiare la differenza non pagata, quindi non registra un costo.
Il costo invece c'è, chiaramente, quando il fondo è stato valutato in difetto, cioè il rischio è di entità maggiore di quanto previsto, anche se in questo caso si paga solo la differenza rispetto alla valutazione fatta e non il valore assoluto del costo.

Nel caso il fondo svalutazione crediti sia stato invece valutato in eccesso, questo in una normale situazione di esercizio andrà a coprire altri crediti arrivati nel frattempo.
Se invece ci si trova in una situazione di cessazione di esercizio, e non esiste più l'ipotesi di rischio, si potrà rmiuovere valore dal fondo, portando ad una variazione economica di esercizio in positivo detta \textbf{sopravvenienza attiva}, che è un ricavo di natura \textit{straordinaria}.

A far parte di un altro tipo di fondo, che è il \textbf{fondo spese}, è il \textbf{fondo TFR}.
Il fondo spese si riferisce ad eventi \textit{certi} di cui però è incerta la data e l'importo.
In questo si contrappone ai fondi rischi, che si riferisce ad eventi fondamentalmente incerti, oltre che incerti in data e in importo.
Il fondo TFR fa quindi riferimento al \textbf{TFR} (\textit{Trattamento Fine Rapporto}), cioè al \textbf{licenziamento}, sia da parte del datore di lavoro che da parte del lavoratore, o al \textbf{pensionamento}.
Il fondo TFR matura quindi nel corso del rapporto del lavoratore, assieme allo stipendio.
Quest'ultimo viene pagato periodicamente, mentre il primo viene rilasciato al termine del rapporto lavorativo.
Visto che questo finanziamento rappresenterebbe una grande uscita di cassa al momento della fine del rapporto, si dedica un fondo apposito, appunto il fondo TFR, per coprirlo. 
Questo viene accumulato, su base annuale, come un debito nei confronti di ogni dipendente, di ammontare pari alla cosiddetta \textbf{quota TFR}, che tendenzialmente varia ogni anno su base dello stipendio e del costo della vita.
Chiaramente questo è un \textit{costo d'esercizio} che va nel \textit{conto economico}.
Al versamento del TFR, quindi, non si hanno costi in quanto il TFR è stato contabilizzato anno per anno per ogni dipendente.

\subsection{Competenza}
Un'altro fattore da considerare nel bilancio è la \textbf{competenza}.
Costi e ricavi \textit{di competanza} sono quei costi e ricavi che vengono registrati prima di eventuali movimenti di cassa, cioè flussi di denaro da o nella cassa.


\subsection{Motivazione di bilancio}
Il bilancio viene redatto per più motivi, che possono essere interni od esterni:
\begin{itemize}
	\item \textbf{Motivi interni:} sicuramente il bilancio aiuta a migliorare il funzionamento dell'impresa, anche se oggi conviene avere informazioni su base ancora più frequente di quella fornita dal bilancio;

	\item \textbf{Motivi esterni:} questi si possono dividere in:
		\begin{itemize}
			\item \textit{Civilistici}: il legislatore \textit{civilistico} vuole assicurarsi che tutti i soggetti (gli \textit{stakeholder}) possano avere rapporti con l'impresa affinché questi non siano tratti in inganno da incrementi patrimoniali o reddituali fittizi;

			\item \textit{Fiscali:} il legislatore \textit{fiscale} vuole invece evitare che l'impresa attui manovre elusive in modo tale da ridurre i versamenti tributari (il pagamento delle imposte);
		\end{itemize}
\end{itemize}

Non esiste un unicio bilancio, allo stesso modo in cui non esiste un unico risultato d'esercizio.
Motivazioni diverse alla base della redazione danno vita a bilanci diversi: ad esempio, il bilancio che possimo vedere noi in qualità di \textit{stakeholder} è il bilancio \textit{civilistico}, mentre al legislatore fiscale interesserà il bilancio \textit{fiscale}, ecc...

\subsection{Fattori di esercizio}
 Modellizziamo la situazione con due magazzini, \textbf{IN} e \textbf{OUT}, che stanno agli estremi della catena di produzione.
Nel magazzino IN entrano le materie prime, e nel magazzino OUT vanno a finire i prodotti finiti.

Vediamo cosa accade nei due magazzini nel corso dell'anno contabile.

\begin{itemize}
	\item \textbf{Magazzino IN:}
nel corso dell'anno contabile si effettuano \textbf{acquisti di materie}, che vanno a finire nel magazzino IN.
Al 1/1, inoltre, il magazzino IN potrebbe contenere \textbf{esistenze iniziali} di materie, comprate in anni contabili precedenti.
Le esistenze iniziali più gli acquisti di materie ci danno le \textbf{materie disponibili} alla produzione.

Si ha quindi:
$$
\text{esistenze iniziali} + \text{acquisti di materie} = \text{materie disponibili}
$$

Al 31/12 il magazzino potrebbe non essere vuoto.
Quelle che rimangono vengono dette \textbf{rimanenze finali}.
La presenza di rimanenze finali rappresenta il fatto che una parte della materie prime non è andata in produzione, e quindi l'esistenza di risorse non consumate.

Si ha quindi:
$$
\text{materie disponibili} - \text{rimanenze finali} = \text{materie consumate}
$$

Le \textbf{materie consumate} vanno nei costi di esercizio, mentre le risorse non consumate vanno nelle attività dello stato patrimoniale: in genere, nell'esercizio si considerano quelle materie effettivamente ottenute e consumate agli scopi della produzione nel corso del periodo, e tutto il resto viene contabilizzato nello stato patrimoniale.

\item \textbf{Magazzino OUT:}
	dal punto di vista della produzione, si fanno le stesse considerazioni per il magazzino OUT: se l'1/1 il magazzino OUT non è vuoto, si hanno delle \textbf{esistenze iniziali} che non sono state prodotte quest'anno.
$$
\text{esistenze iniziali} + \text{produzione di esercizio} = \text{prodotti disponibili}
$$

	Allo stesso modo, quello che resta nel magazzino OUT alla fine dell'anno contabile, cioè le \textbf{rimanenze finali}, rappresentano qualcosa che si è prodotto nella \textit{durata} dell'anno contabile.
$$
\text{prodotti disponibili} - \text{rimanenze finali} = \text{prodotti venduti}
$$

Il prodotti disponibili invenduti, come per sopra, restano nello stato patrimoniale.

\end{itemize}

\end{document}


\documentclass[a4paper,11pt]{article}
\usepackage[a4paper, margin=8em]{geometry}

% usa i pacchetti per la scrittura in italiano
\usepackage[french,italian]{babel}
\usepackage[T1]{fontenc}
\usepackage[utf8]{inputenc}
\frenchspacing 

% usa i pacchetti per la formattazione matematica
\usepackage{amsmath, amssymb, amsthm, amsfonts}

% usa altri pacchetti
\usepackage{gensymb}
\usepackage{hyperref}
\usepackage{standalone}

% imposta il titolo
\title{Appunti Economia ed Organizzazione Aziendale}
\author{Luca Seggiani}
\date{2025}

% disegni
\usepackage{pgfplots}
\pgfplotsset{width=10cm,compat=1.9}

% imposta lo stile
% usa helvetica
\usepackage[scaled]{helvet}
% usa palatino
\usepackage{palatino}
% usa un font monospazio guardabile
\usepackage{lmodern}

\renewcommand{\rmdefault}{ppl}
\renewcommand{\sfdefault}{phv}
\renewcommand{\ttdefault}{lmtt}

% disponi il titolo
\makeatletter
\renewcommand{\maketitle} {
	\begin{center} 
		\begin{minipage}[t]{.8\textwidth}
			\textsf{\huge\bfseries \@title} 
		\end{minipage}%
		\begin{minipage}[t]{.2\textwidth}
			\raggedleft \vspace{-1.65em}
			\textsf{\small \@author} \vfill
			\textsf{\small \@date}
		\end{minipage}
		\par
	\end{center}

	\thispagestyle{empty}
	\pagestyle{fancy}
}
\makeatother

% disponi teoremi
\usepackage{tcolorbox}
\newtcolorbox[auto counter, number within=section]{theorem}[2][]{%
	colback=blue!10, 
	colframe=blue!40!black, 
	sharp corners=northwest,
	fonttitle=\sffamily\bfseries, 
	title=Teorema~\thetcbcounter: #2, 
	#1
}

% disponi definizioni
\newtcolorbox[auto counter, number within=section]{definition}[2][]{%
	colback=red!10,
	colframe=red!40!black,
	sharp corners=northwest,
	fonttitle=\sffamily\bfseries,
	title=Definizione~\thetcbcounter: #2,
	#1
}

% disponi problemi
\newtcolorbox[auto counter, number within=section]{problem}[2][]{%
	colback=green!10,
	colframe=green!40!black,
	sharp corners=northwest,
	fonttitle=\sffamily\bfseries,
	title=Problema~\thetcbcounter: #2,
	#1
}

% disponi codice
\usepackage{listings}
\usepackage[table]{xcolor}

\lstdefinestyle{codestyle}{
		backgroundcolor=\color{black!5}, 
		commentstyle=\color{codegreen},
		keywordstyle=\bfseries\color{magenta},
		numberstyle=\sffamily\tiny\color{black!60},
		stringstyle=\color{green!50!black},
		basicstyle=\ttfamily\footnotesize,
		breakatwhitespace=false,         
		breaklines=true,                 
		captionpos=b,                    
		keepspaces=true,                 
		numbers=left,                    
		numbersep=5pt,                  
		showspaces=false,                
		showstringspaces=false,
		showtabs=false,                  
		tabsize=2
}

\lstdefinestyle{shellstyle}{
		backgroundcolor=\color{black!5}, 
		basicstyle=\ttfamily\footnotesize\color{black}, 
		commentstyle=\color{black}, 
		keywordstyle=\color{black},
		numberstyle=\color{black!5},
		stringstyle=\color{black}, 
		showspaces=false,
		showstringspaces=false, 
		showtabs=false, 
		tabsize=2, 
		numbers=none, 
		breaklines=true
}

\lstdefinelanguage{javascript}{
	keywords={typeof, new, true, false, catch, function, return, null, catch, switch, var, if, in, while, do, else, case, break},
	keywordstyle=\color{blue}\bfseries,
	ndkeywords={class, export, boolean, throw, implements, import, this},
	ndkeywordstyle=\color{darkgray}\bfseries,
	identifierstyle=\color{black},
	sensitive=false,
	comment=[l]{//},
	morecomment=[s]{/*}{*/},
	commentstyle=\color{purple}\ttfamily,
	stringstyle=\color{red}\ttfamily,
	morestring=[b]',
	morestring=[b]"
}

% disponi sezioni
\usepackage{titlesec}

\titleformat{\section}
	{\sffamily\Large\bfseries} 
	{\thesection}{1em}{} 
\titleformat{\subsection}
	{\sffamily\large\bfseries}   
	{\thesubsection}{1em}{} 
\titleformat{\subsubsection}
	{\sffamily\normalsize\bfseries} 
	{\thesubsubsection}{1em}{}

% disponi alberi
\usepackage{forest}

\forestset{
	rectstyle/.style={
		for tree={rectangle,draw,font=\large\sffamily}
	},
	roundstyle/.style={
		for tree={circle,draw,font=\large}
	}
}

% disponi algoritmi
\usepackage{algorithm}
\usepackage{algorithmic}
\makeatletter
\renewcommand{\ALG@name}{Algoritmo}
\makeatother

% disponi numeri di pagina
\usepackage{fancyhdr}
\fancyhf{} 
\fancyfoot[L]{\sffamily{\thepage}}

\makeatletter
\fancyhead[L]{\raisebox{1ex}[0pt][0pt]{\sffamily{\@title \ \@date}}} 
\fancyhead[R]{\raisebox{1ex}[0pt][0pt]{\sffamily{\@author}}}
\makeatother

\begin{document}

% sezione (data)
\section{Lezione del 16-04-25}

% stili pagina
\thispagestyle{empty}
\pagestyle{fancy}

% testo
\subsection{Fattori pluriennali}
I fattori pluriennali possono essere \textbf{materiali} o \textbf{immateriali}, ad esempio possiamo distinguere:
\begin{itemize}
	\item Fattori pluriennali \textbf{materiali}:
		\begin{itemize}
			\item Impianto;
			\item Macchinari;
			\item Attrezzature;
			\item Immobili.
		\end{itemize}
	\item Fattori pluriennali \textbf{immateriali}:
		\begin{itemize}
			\item Brevetti;
			\item Costi di impianto;
			\item Avviamento;
			\item Ricerca e sviluppo;
			\item Costi di pubblicità.
		\end{itemize}
\end{itemize}

\subsubsection{Avviamento}
L'\textbf{avviamento} merita qualche parola a sé.
L'avviamento di un'azienda rappresenta un valore utile al momento della vendita (e quindi l'acquisto) di quell'azienda.
In questo caso si calcola infatti un \textbf{capitale netto di cessione}, cioè dato sì dalla somma algebrica fra attività e passività, ma in prospettiva di cessione, quindi tenendo conto appunto dell'avviamento, cioè il fatto che l'azienda è \textit{avviata}, ha un suo nome, una sua clientela, ecc... e quindi un valore aggiunto rispetto ai soli beni che essa detiene.

\subsubsection{Pubblicità}
Parlamo anche di \textbf{pubblicità}.
Si possono fare diversi tipi di pubblicità:
\begin{itemize}
	\item Pubblicità per un \textbf{prodotto nuovo} (per quanto sia difficile definire "nuovo", non visto dal mercato, tecnologicamente avanzato, ecc...);
	\item Pubblicità per prodotti esistenti/migliorati.
\end{itemize}

I costi dovuti alle spese pubblicitarie vengono recuperati con le vendite dei prodotti pubblicizzati.
Ora, per prodotti esistenti/migliorati, si può assumere che i costi pubblicitari vengano soddisfatti, nel corso del periodo.
Per i prodotti nuovi, si deve invece immaginare che la pubblicità diventerà un "innesco" per un prodotto ancora di nicchia, che quindi si ripagherà magari nel corso di diversi anni.
Questo costo (relativo ai soli prodotti nuovi) viene quindi \textbf{capitalizzato} (portato a stato patrimoniale), e allora considerato come pluriennale.
Il costo della pubblicità di esercizio resta invece nel reddito di esercizio.

\subsection{Ammortamento}
L'\textbf{ammortamento} si applica ai \textbf{fattori pluriennali} a \textit{sinistra} dello stato patrimoniale, e ai \textbf{fattori di esercizio} a \textit{destra}, e consiste nel determinare la parte di costo pluriennale (o di fattore di esercizio) consumata nel periodo.

Nel dettaglio del costo pluriennale, quindi, possiamo dire una risorsa avrà un \textbf{costo storico}, dato ad esempio da:
\begin{itemize}
	\item Il prezzo della risora;
	\item Il trasporto;
	\item L'installazione;
	\item L'(eventuale) collaudo.
\end{itemize}

Questo costo storico non viene fattorizzato totalmente al momento dell'ottenimento della risora, ma quindi \textbf{ammortato} nel corso dell'opera.
Si tiene quindi conto del \textit{valore di consumo} della risorsa nel corso del periodo, dove per ogni periodo si riduce il prezzo storico di una \textbf{quota di ammortamento}.
Il valore che rimane della risorsa viene detto \textit{valore contabile} o \textbf{valore di libro}.

Si può fare una precisazione sul \textit{corso d'opera} che abbiamo considerato prima.
Avremo infatti una \textbf{vita utile}, che non corrisponde necessariamente alla \textbf{vita tecnica}, che è il periodo in cui la risorsa è effettivamente utilizzabile (si pensi ad un macchinario funzionante ma obsoleto).
La vita utile è quindi il periodo in cui la risorsa è \textit{economicamente} funzionante, cioè crea più ricavi che costi.

Quello che si va ad ammortizzare è quindi il costo storico del fattore, a cui si sottrae il \textbf{valore di recupero}, cioè il prezzo a cui si suppone di poter vendere il fattore al termine della sua vita utile (definito chiaramente \textit{ex ante}).

Infine, bisogna tenere conto delle \textbf{modalità di ripartizione} della risorsa, che può essere:
\begin{itemize}
	\item A quote lineari;
	\item A quote descrescenti, il più diffuso (tanto che l'ordinamento corrente prevede l'\textbf{ammortamento accelerato});
	\item A quote crescenti.
\end{itemize}
Di queste quote si tiene conto attraverso il \textbf{fondo ammortamento}.

Al momento della vendita della risorsa si tiene conto del valore di libro di tale risorsa.
Un eventuale guadagno sul valore di libro è detto \textbf{plusvalenza}, mentre una perdita viene detta \textbf{minusvalenza}.
La plusvalenza è un valore economico d'esercizio positivo, che quindi va a finre nel conto economico.

Vediamo quindi che il conto economico è stratificato: i ricavi e i costi che vi troviamo possono derivare da diverse \textbf{aree}.
\begin{itemize}
	\item Area \textbf{operativa}, divisa in:
	\begin{itemize}
	\item Area \textbf{caratteristica} (\textit{core business}), cioè quella che riguarda l'attività propria dell'impresa;
	\item Area \textbf{accessoria}, cioè quella che contiene ad esempio gli interessi attivi sul denaro che temporaneamente sta nella cassa (investimenti nel mercato finanziario, ecc...).
	\end{itemize}

	L'area operativa contribuisce all'\textbf{EBIT}, o \textit{risultato operativo}.

	\item Area \textbf{finanziaria}: contiene costi e ricavi di operazioni e finanziamento attinto (capitale di terzi) o concesso.
		Sono di questo tipo ad esempio gli \textbf{oneri} e i \textbf{proventi} finanziari;
	\item Area \textbf{fiscale}: riguarda le imposte;
	\item Area \textbf{straordinaria}: appunto, quella che contiene \textbf{plusvalenze} e \textbf{minusvalenze} di operazioni di vendita (cioè ciò di cui stavamo parlando adesso).
\end{itemize}
\end{document}


\documentclass[a4paper,11pt]{article}
\usepackage[a4paper, margin=8em]{geometry}

% usa i pacchetti per la scrittura in italiano
\usepackage[french,italian]{babel}
\usepackage[T1]{fontenc}
\usepackage[utf8]{inputenc}
\frenchspacing 

% usa i pacchetti per la formattazione matematica
\usepackage{amsmath, amssymb, amsthm, amsfonts}

% usa altri pacchetti
\usepackage{gensymb}
\usepackage{hyperref}
\usepackage{standalone}

% imposta il titolo
\title{Appunti Economia ed Organizzazione Aziendale}
\author{Luca Seggiani}
\date{2025}

% disegni
\usepackage{pgfplots}
\pgfplotsset{width=10cm,compat=1.9}

% imposta lo stile
% usa helvetica
\usepackage[scaled]{helvet}
% usa palatino
\usepackage{palatino}
% usa un font monospazio guardabile
\usepackage{lmodern}

\renewcommand{\rmdefault}{ppl}
\renewcommand{\sfdefault}{phv}
\renewcommand{\ttdefault}{lmtt}

% disponi il titolo
\makeatletter
\renewcommand{\maketitle} {
	\begin{center} 
		\begin{minipage}[t]{.8\textwidth}
			\textsf{\huge\bfseries \@title} 
		\end{minipage}%
		\begin{minipage}[t]{.2\textwidth}
			\raggedleft \vspace{-1.65em}
			\textsf{\small \@author} \vfill
			\textsf{\small \@date}
		\end{minipage}
		\par
	\end{center}

	\thispagestyle{empty}
	\pagestyle{fancy}
}
\makeatother

% disponi teoremi
\usepackage{tcolorbox}
\newtcolorbox[auto counter, number within=section]{theorem}[2][]{%
	colback=blue!10, 
	colframe=blue!40!black, 
	sharp corners=northwest,
	fonttitle=\sffamily\bfseries, 
	title=Teorema~\thetcbcounter: #2, 
	#1
}

% disponi definizioni
\newtcolorbox[auto counter, number within=section]{definition}[2][]{%
	colback=red!10,
	colframe=red!40!black,
	sharp corners=northwest,
	fonttitle=\sffamily\bfseries,
	title=Definizione~\thetcbcounter: #2,
	#1
}

% disponi problemi
\newtcolorbox[auto counter, number within=section]{problem}[2][]{%
	colback=green!10,
	colframe=green!40!black,
	sharp corners=northwest,
	fonttitle=\sffamily\bfseries,
	title=Problema~\thetcbcounter: #2,
	#1
}

% disponi codice
\usepackage{listings}
\usepackage[table]{xcolor}

\lstdefinestyle{codestyle}{
	backgroundcolor=\color{black!5}, 
	commentstyle=\color{codegreen},
	keywordstyle=\bfseries\color{magenta},
	numberstyle=\sffamily\tiny\color{black!60},
	stringstyle=\color{green!50!black},
	basicstyle=\ttfamily\footnotesize,
	breakatwhitespace=false,         
	breaklines=true,                 
	captionpos=b,                    
	keepspaces=true,                 
	numbers=left,                    
	numbersep=5pt,                  
	showspaces=false,                
	showstringspaces=false,
	showtabs=false,                  
	tabsize=2
}

\lstdefinestyle{shellstyle}{
	backgroundcolor=\color{black!5}, 
	basicstyle=\ttfamily\footnotesize\color{black}, 
	commentstyle=\color{black}, 
	keywordstyle=\color{black},
	numberstyle=\color{black!5},
	stringstyle=\color{black}, 
	showspaces=false,
	showstringspaces=false, 
	showtabs=false, 
	tabsize=2, 
	numbers=none, 
	breaklines=true
}

\lstdefinelanguage{javascript}{
	keywords={typeof, new, true, false, catch, function, return, null, catch, switch, var, if, in, while, do, else, case, break},
	keywordstyle=\color{blue}\bfseries,
	ndkeywords={class, export, boolean, throw, implements, import, this},
	ndkeywordstyle=\color{darkgray}\bfseries,
	identifierstyle=\color{black},
	sensitive=false,
	comment=[l]{//},
	morecomment=[s]{/*}{*/},
	commentstyle=\color{purple}\ttfamily,
	stringstyle=\color{red}\ttfamily,
	morestring=[b]',
	morestring=[b]"
}

% disponi sezioni
\usepackage{titlesec}

\titleformat{\section}
{\sffamily\Large\bfseries} 
{\thesection}{1em}{} 
\titleformat{\subsection}
{\sffamily\large\bfseries}   
{\thesubsection}{1em}{} 
\titleformat{\subsubsection}
{\sffamily\normalsize\bfseries} 
{\thesubsubsection}{1em}{}

% disponi alberi
\usepackage{forest}

\forestset{
	rectstyle/.style={
		for tree={rectangle,draw,font=\large\sffamily}
	},
	roundstyle/.style={
		for tree={circle,draw,font=\large}
	}
}

% disponi algoritmi
\usepackage{algorithm}
\usepackage{algorithmic}
\makeatletter
\renewcommand{\ALG@name}{Algoritmo}
\makeatother

% disponi numeri di pagina
\usepackage{fancyhdr}
\fancyhf{} 
\fancyfoot[L]{\sffamily{\thepage}}

\makeatletter
\fancyhead[L]{\raisebox{1ex}[0pt][0pt]{\sffamily{\@title \ \@date}}} 
\fancyhead[R]{\raisebox{1ex}[0pt][0pt]{\sffamily{\@author}}}
\makeatother

\begin{document}

% sezione (data)
\section{Lezione del 30-04-25}

% stili pagina
\thispagestyle{empty}
\pagestyle{fancy}

% testo
\subsection{Modelli di stato patrimoniale}
Abbiamo quindi visto il modello contabile dello stato patrimoniale che le aziende devono tenere.
Vediamo adesso due modelli per la redazione dello stato patrimoniale, quello descritto dallo standard internazionale \textbf{IAS-IFRS} (\textit{International Accounting Standards} e \textit{International Financial Reporting Standards}), e quello descritto dal codice civile (schema art. 2424 c.c.).

\subsubsection{Codice civile}
Nello schema del codice civile abbiamo sezioni individuate da lettere nel conto attivo e passivo, che sono:
\begin{itemize}
	\item \textbf{Attivo:}
		\begin{itemize}
			\item[A)] Crediti verso soci; 
			\item[B)] Immobilizzazioni:
				\begin{enumerate}
					\item Immobilizzazioni immateriali;
					\item Immobilizzazioni materiali;
					\item Immobilizzazioni finanziarie.
				\end{enumerate}
			\item[C)] Attivo circolante:
				\begin{enumerate}
					\item Rimanenze;
					\item Crediti;
					\item Attività finanziarie;
					\item Disponibilità liquide.
				\end{enumerate}
			\item[D)] Ratei e risconti attivi.
		\end{itemize}
	\item \textbf{Passivo:}
		\begin{itemize}
			\item[A)] Patrimonio netto;
			\item[B)] Fondi per rischi ed oneri;
			\item[C)] Trattamento di fine rapporto;
			\item[D)] Debiti;
			\item[E)] Ratei e risconti passivi.
		\end{itemize}
\end{itemize}

\subsubsection{IAS-IFRS}
Nel modello IAS-IFRS l'attivo fra asset \textit{correnti} e \textit{non correnti}, cioè:
\begin{itemize}
	\item Attività \textbf{correnti}:
		\begin{itemize}
			\item Crediti commerciali e altri;
			\item Rimanenze;
			\item Lavori in corso su ordinazione;
			\item Attività finanziarie correnti;
			\item Disponibiltà liquide.
		\end{itemize}
	\item Attività \textbf{non correnti}:
		\begin{itemize}
			\item Immobili, impianti e macchinari;
			\item Investimenti immobiliari;
			\item Avviamento e attività immateriali a vita non definita;
			\item Altre attività immateriali;
			\item Partecipazioni;
			\item Altre attività finanziarie;
			\item Imposte differite attive.
		\end{itemize}
\end{itemize}

Anche per il passivo si distingue fra corrente e non corrente, come:
\begin{itemize}
	\item Passività \textbf{correnti}:
		\begin{itemize}
			\item Passività finanziare correnti;
			\item Debiti commerciali;
			\item Debiti per imposte;
			\item Debiti vari e altre passività correnti.
		\end{itemize}
	\item Passività \textbf{non correnti}:
		\begin{itemize}
			\item Passività finanziarie non correnti;
			\item TFR e altri fondi relativi al personale;
			\item Fondo imposte differite;
			\item Fondo per rischi e oneri futuri;
			\item Debiti vari e altre passività non correnti.
		\end{itemize}
\end{itemize}

Infine, nel passivo si riporta anche il patrimonio netto:
\begin{itemize}
	\item Capitale emesso;
	\item Riserve;
	\item Utili/perdite d'esercizio;
	\item Utili/perdite portate a nuovo.
\end{itemize}

\subsection{Modelli di conto economico}
Anche per quanto riguarda il conto economico esiste uno schema \textbf{IAS} e uno schema dettato dal \textbf{codice civile}. 

Riprendiamo innanzitutti le aree di stratificazione che avevamo già visto per valutare il \textbf{valore della produzione}:
\begin{itemize}
	\item Area \textbf{caratteristica}:
		\begin{itemize}
			\item Valore della produzione:
				\begin{itemize}
					\item Ricavi di vendita;
					\item \textbf{Costi in economia}, cioè di macchinari/immobili che non vengono acquistati ma alla cui costruzione si delega la macchina produttiva, che non sono di competenza alla produzione e vanno quindi messi qui per annullare i costi di produzioni che vi si impiegano;
					\item Incremento magazzino OUT.
				\end{itemize}
			\item Costo della produzione:
				\begin{itemize}
					\item Materie prime;
					\item Immobilizzazioni ammortate;
					\item Salari e stipendi, più quote TFR (cioè il \textit{costo lavoro});
					\item Decremento magazzino IN.
				\end{itemize}
		\end{itemize}
		$\rightarrow$ \textbf{Margine Operativo Lordo}.
	\item Area \textbf{accessoria}:
		\begin{itemize}
			\item Proventi ed oneri accessori.
		\end{itemize}
		$\rightarrow$ \textbf{Margine Operativo Netto} (\textbf{EBIT}).
	\item Area \textbf{finanziaria}:
		\begin{itemize}
			\item Proventi ed oneri finanziari;
			\item Rettifiche di valore da attività finanziarie.
		\end{itemize}
		$\rightarrow$ risultato lordo da attività in funzionamento.
	\item Area \textbf{fiscale}:
		\begin{itemize}
			\item Imposte sul reddito di esercizio.
		\end{itemize}
		$\rightarrow$ Utile o perdita di esercizio.
	\item Area \textbf{straordinaria}:
		\begin{itemize}
			\item Utile o perdita su attività destinate a cessare.
		\end{itemize}
\end{itemize}
ricordando anche che area caratteristica e accessoria formano insieme l'area \textbf{operativa}.

Questo è il modello adottato dal codice civile.

Vediamo quindi il conto fatto sul \textbf{costo del venduto}:
\begin{itemize}
	\item Area \textbf{caratteristica}:
		\begin{itemize}
			\item Ricavi di vendita;
			\item Costo del venduto:
				\begin{itemize}
					\item Costi per acquisti di produzione;
					\item Costo del personale di produzione;
					\item Costi per investimenti produttivi;
					\item La variazione delle rimanenze.
				\end{itemize}
		\end{itemize}
		$\rightarrow$ \textbf{Margine Lordo Industriale}, che equivale al margine operativo lordo depurato delle aree non industriali (amministrazione, ricerca e sviluppo, vendite ecc...).
		\begin{itemize}
			\item Altri costi di area caratteristica.
		\end{itemize}
		$\rightarrow$ \textbf{Margine Operativo Lordo}.

\end{itemize}

Questo è il modello adottato dallo standard IAS.

\subsection{Modello di business}
Un \textbf{modello di business} definisce la logica secondo la quale un'organizzazione \textbf{crea}, \textbf{distribuisce} e \textbf{cattura} valore.

Nel business model possiamo individuare diversi elementi:
\begin{itemize}
	\item I \textbf{segmenti di clientela};
	\item La \textbf{proposta di valore};
	\item I \textbf{canali};
	\item Le \textbf{relazioni coi clienti};
	\item I \textbf{flussi di ricavi};
	\item Le \textbf{risorse chiave};
	\item Le \textbf{attività chiave};
	\item Le \textbf{partnership chiave};
	\item La \textbf{struttura dei costi}.
\end{itemize}

\subsubsection{Segmenti di clientela}
Prima di tutto potremmo chiederci per \textit{chi} stiamo creando valore.
Possiamo individuare alcune macrocategorie:
\begin{itemize}
	\item Mercato di massa;
	\item Mercato di nicchia;
	\item Mercato segmentato;
	\item Mercato diversificato;
	\item Piattaforme (o mercati) multi-sided.
\end{itemize}

\subsubsection{Proposta di valore}
La proposta di valore è ciò che l'azienda offre in \textit{più} rispetto a un altra, quindi il motivo per cui i potenziali clienti dovrebbero acquistarne i prodotti.

\subsubsection{Canali}
I canali sono quelli attraverso i quali l'azienda fornisce i suoi prodotti/servizi al cliente.
Possiamo individuare ad esempio:
\begin{itemize}
	\item Forza di vendita;
	\item Vendita su web;
	\item Negozi propri;
	\item Negozi di partner;
	\item Grossisti.
\end{itemize}

\subsubsection{Relazioni coi clienti}
Le relazioni coi clienti riguardano:
\begin{itemize}
	\item L'\textit{acquisizione} di clienti;
	\item La \textit{fidelizzazione} di clienti;
	\item L'\textit{upselling}, cioè l'aumento della frequenza di acquisto dei clienti.
\end{itemize}

Possiamo individuare diversi meccanismi:
\begin{itemize}
	\item Assistenza personale;
	\item Assistenza personale dedicata;
	\item Self-service;
	\item Servizi automatici;
	\item Community;
	\item Co-creazione.
\end{itemize}

Di base, la relazione può distinguersi su \textbf{intensità}, \textbf{frequenza} e \textbf{intimità}:
\begin{itemize}
	\item \textbf{Intensità:}
		\begin{itemize}
			\item Una relazione a bassa intensità (o \textit{indiretta}) è quella dove il cliente non interagisce con l'azienda, ma con intermediari;
			\item Di contro una relazione ad altà intensità (o \textit{diretta}) è quella dove il cliente interagisce direttamente con l'azienda.
		\end{itemize}
	\item \textbf{Frequenza:}
		\begin{itemize}
			\item In una relazione a bassa frequenza si parla principalmente di \textit{transazioni}, cioè acquisti che non creano nessun legame futuro cliente;
			\item In una relazione ad alta frequenza si svilupppono invece relazioni di \textit{lungo termine} che perdurano nel tempo (ne sono un esempio aziende che producono beni pluriennali, che magari richiedono manutenzione, ecc...).
		\end{itemize}
	\item \textbf{Intimità:}
		\begin{itemize}
			\item L'intimità riguarda la natura della relazione: alta intimità (o intimità \textit{personale}) significa relazionarsi col personale, quindi con esseri umani;
			\item Bassa intimità significa invece relazioni automatiche, quindi con operatori automatici, computer, ecc... (ne sono esempi i servizi digitali).
		\end{itemize}
\end{itemize}

\subsubsection{Flussi di ricavi}
Un modello di business può prevedere due tipi di ricavi:
\begin{itemize}
	\item Ricavi correnti;
	\item Ricavi continui.
\end{itemize}

Questi possono essere:
\begin{itemize}
	\item Vendita di beni;
	\item Canone d'uso;
	\item Abbonamenti, che non corrispondono ai canoni d'uso (questi ultimi hanno solitamente natura occasionale, contro la continuità di un abbonamento);
	\item Quote di iscrizione;
	\item Prestito/noleggio/leasing;
	\item Licenza;
	\item Commissioni di intermediazione;
	\item Pubblicità;
	\item Servizi gratuiti: si pensi ad esempio al motore di ricerca Google.
\end{itemize}

\subsubsection{Risorse chiave}
Le risorse chiave sono i beni più importanti affinché un modello di business funzioni, e possono includere:
\begin{itemize}
	\item Materie prime;
	\item Particolari servizi;
	\item Personale specializzato;
	\item Ecc...
\end{itemize}

\subsubsection{Attività chiave}
Le attività chiave sono quelle attività di progettazione, creazione e distribuzione essenziali al funzionamento dell'azienda.
Ad esempio possiamo notare:
\begin{itemize}
	\item Le attività di \textbf{produzione}, fondamentali alle aziende manufatturiere;
	\item Le attività di \textbf{problem solving}, orientate alla risoluzione di problemi per i singoli clienti, tipiche di attività di consulenza e di servizi;
	\item Le attività riguardanti la \textbf{piattaforma}, per aziende che forniscono appunto \textit{piattaforme}: si pensi ai servizi online che richiedono costante manutenzione e moderazione. 
\end{itemize}

\subsubsection{Partnership chiave}
Le partnership chiave sono quelle relazioni con fornitori, intermediari se non addirittura concorrenti fondamentali all'attività dell'azienda. 

Possiamo distinguere:
\begin{itemize}
	\item Partnership fra concorrenti, le cosiddette \textbf{coopetition};
	\item Partnership per creare nuove proposte di valore, ad esempio fra due aziende di settori diversi che creano un prodotto combinato;
	\item Partnership \textit{buyer-supplier}, fra fornitori di lunga data;
	\item Partnership per ridurre i costi o il rischio.
\end{itemize}

\subsubsection{Struttura dei costi}
La sruttura dei costi riguarda i costi che l'azienda deve sopportare nell'ipotesi di funzionamento.

Questi possono essere:
\begin{itemize}
	\item \textbf{Variabili} sulla base della produzione;
	\item \textbf{Fissi}, cioè slegati dalla produzione.
\end{itemize}

Potremo dire che la parte di \textit{destra} del modello di business, cioè quella contenente:
\begin{itemize}
	\item I egmenti di clientela;
	\item La proposta di valore;
	\item I canali;
	\item Le relazioni coi clienti;
	\item I flussi di ricavi.
\end{itemize}
è detta parte del \textbf{valore}, mentre la parte si \textit{sinistra}, cioè quella contenente:
\begin{itemize}
	\item Le risorse chiave;
	\item Le attività chiave;
	\item Le partnership chiave;
	\item La struttura dei costi.
\end{itemize}
è detta parte dell'\textbf{efficienza}.

Di aziende incentrate sulla parte del valore si dice che sono \textbf{value-driven}, mentre di aziende incentrate sulla parte dell'efficienza si dice che sono \textbf{cost-driven}.

\end{document}


\documentclass[a4paper,11pt]{article}
\usepackage[a4paper, margin=8em]{geometry}

% usa i pacchetti per la scrittura in italiano
\usepackage[french,italian]{babel}
\usepackage[T1]{fontenc}
\usepackage[utf8]{inputenc}
\frenchspacing 

% usa i pacchetti per la formattazione matematica
\usepackage{amsmath, amssymb, amsthm, amsfonts}

% usa altri pacchetti
\usepackage{gensymb}
\usepackage{hyperref}
\usepackage{standalone}

% imposta il titolo
\title{Appunti Economia ed Organizzazione Aziendale}
\author{Luca Seggiani}
\date{2025}

% disegni
\usepackage{pgfplots}
\pgfplotsset{width=10cm,compat=1.9}

% imposta lo stile
% usa helvetica
\usepackage[scaled]{helvet}
% usa palatino
\usepackage{palatino}
% usa un font monospazio guardabile
\usepackage{lmodern}

\renewcommand{\rmdefault}{ppl}
\renewcommand{\sfdefault}{phv}
\renewcommand{\ttdefault}{lmtt}

% disponi il titolo
\makeatletter
\renewcommand{\maketitle} {
	\begin{center} 
		\begin{minipage}[t]{.8\textwidth}
			\textsf{\huge\bfseries \@title} 
		\end{minipage}%
		\begin{minipage}[t]{.2\textwidth}
			\raggedleft \vspace{-1.65em}
			\textsf{\small \@author} \vfill
			\textsf{\small \@date}
		\end{minipage}
		\par
	\end{center}

	\thispagestyle{empty}
	\pagestyle{fancy}
}
\makeatother

% disponi teoremi
\usepackage{tcolorbox}
\newtcolorbox[auto counter, number within=section]{theorem}[2][]{%
	colback=blue!10, 
	colframe=blue!40!black, 
	sharp corners=northwest,
	fonttitle=\sffamily\bfseries, 
	title=Teorema~\thetcbcounter: #2, 
	#1
}

% disponi definizioni
\newtcolorbox[auto counter, number within=section]{definition}[2][]{%
	colback=red!10,
	colframe=red!40!black,
	sharp corners=northwest,
	fonttitle=\sffamily\bfseries,
	title=Definizione~\thetcbcounter: #2,
	#1
}

% disponi problemi
\newtcolorbox[auto counter, number within=section]{problem}[2][]{%
	colback=green!10,
	colframe=green!40!black,
	sharp corners=northwest,
	fonttitle=\sffamily\bfseries,
	title=Problema~\thetcbcounter: #2,
	#1
}

% disponi codice
\usepackage{listings}
\usepackage[table]{xcolor}

\lstdefinestyle{codestyle}{
		backgroundcolor=\color{black!5}, 
		commentstyle=\color{codegreen},
		keywordstyle=\bfseries\color{magenta},
		numberstyle=\sffamily\tiny\color{black!60},
		stringstyle=\color{green!50!black},
		basicstyle=\ttfamily\footnotesize,
		breakatwhitespace=false,         
		breaklines=true,                 
		captionpos=b,                    
		keepspaces=true,                 
		numbers=left,                    
		numbersep=5pt,                  
		showspaces=false,                
		showstringspaces=false,
		showtabs=false,                  
		tabsize=2
}

\lstdefinestyle{shellstyle}{
		backgroundcolor=\color{black!5}, 
		basicstyle=\ttfamily\footnotesize\color{black}, 
		commentstyle=\color{black}, 
		keywordstyle=\color{black},
		numberstyle=\color{black!5},
		stringstyle=\color{black}, 
		showspaces=false,
		showstringspaces=false, 
		showtabs=false, 
		tabsize=2, 
		numbers=none, 
		breaklines=true
}

\lstdefinelanguage{javascript}{
	keywords={typeof, new, true, false, catch, function, return, null, catch, switch, var, if, in, while, do, else, case, break},
	keywordstyle=\color{blue}\bfseries,
	ndkeywords={class, export, boolean, throw, implements, import, this},
	ndkeywordstyle=\color{darkgray}\bfseries,
	identifierstyle=\color{black},
	sensitive=false,
	comment=[l]{//},
	morecomment=[s]{/*}{*/},
	commentstyle=\color{purple}\ttfamily,
	stringstyle=\color{red}\ttfamily,
	morestring=[b]',
	morestring=[b]"
}

% disponi sezioni
\usepackage{titlesec}

\titleformat{\section}
	{\sffamily\Large\bfseries} 
	{\thesection}{1em}{} 
\titleformat{\subsection}
	{\sffamily\large\bfseries}   
	{\thesubsection}{1em}{} 
\titleformat{\subsubsection}
	{\sffamily\normalsize\bfseries} 
	{\thesubsubsection}{1em}{}

% disponi alberi
\usepackage{forest}

\forestset{
	rectstyle/.style={
		for tree={rectangle,draw,font=\large\sffamily}
	},
	roundstyle/.style={
		for tree={circle,draw,font=\large}
	}
}

% disponi algoritmi
\usepackage{algorithm}
\usepackage{algorithmic}
\makeatletter
\renewcommand{\ALG@name}{Algoritmo}
\makeatother

% disponi numeri di pagina
\usepackage{fancyhdr}
\fancyhf{} 
\fancyfoot[L]{\sffamily{\thepage}}

\makeatletter
\fancyhead[L]{\raisebox{1ex}[0pt][0pt]{\sffamily{\@title \ \@date}}} 
\fancyhead[R]{\raisebox{1ex}[0pt][0pt]{\sffamily{\@author}}}
\makeatother

\begin{document}

% sezione (data)
\section{Lezione del 08-05-25}

% stili pagina
\thispagestyle{empty}
\pagestyle{fancy}

% testo
\subsection{Analisi dei costi}
Il costo è una variazione economica negativa non corrisposta da una variazione finanziaria positiva.

Possiamo classificare il costo in base alla direzione da cui lo vediamo:
\begin{itemize}
	\item Dal punto di vista della \textit{contabilità esterna/generale}, abbiamo che il costo si intende come misura del valore di una \textbf{transazione} con \textit{economie terze}. Chiamiamo questa classificazione \textbf{per natura};
		\item Dal punto di vista della \textit{contabilità direzionale}, abbiamo che il costo si intende come la misura monetaria delle \textit{risorse} usate nei \textit{processi} che perseguono un qualche \textit{scopo}. Chiamiamo questa classificazione \textbf{per destinazione}.
\end{itemize}

\end{document}


\documentclass[a4paper,11pt]{article}
\usepackage[a4paper, margin=8em]{geometry}

% usa i pacchetti per la scrittura in italiano
\usepackage[french,italian]{babel}
\usepackage[T1]{fontenc}
\usepackage[utf8]{inputenc}
\frenchspacing 

% usa i pacchetti per la formattazione matematica
\usepackage{amsmath, amssymb, amsthm, amsfonts}

% usa altri pacchetti
\usepackage{gensymb}
\usepackage{hyperref}
\usepackage{standalone}

% imposta il titolo
\title{Appunti Economia ed Organizzazione Aziendale}
\author{Luca Seggiani}
\date{2025}

% disegni
\usepackage{pgfplots}
\pgfplotsset{width=10cm,compat=1.9}

% imposta lo stile
% usa helvetica
\usepackage[scaled]{helvet}
% usa palatino
\usepackage{palatino}
% usa un font monospazio guardabile
\usepackage{lmodern}

\renewcommand{\rmdefault}{ppl}
\renewcommand{\sfdefault}{phv}
\renewcommand{\ttdefault}{lmtt}

% disponi il titolo
\makeatletter
\renewcommand{\maketitle} {
	\begin{center} 
		\begin{minipage}[t]{.8\textwidth}
			\textsf{\huge\bfseries \@title} 
		\end{minipage}%
		\begin{minipage}[t]{.2\textwidth}
			\raggedleft \vspace{-1.65em}
			\textsf{\small \@author} \vfill
			\textsf{\small \@date}
		\end{minipage}
		\par
	\end{center}

	\thispagestyle{empty}
	\pagestyle{fancy}
}
\makeatother

% disponi teoremi
\usepackage{tcolorbox}
\newtcolorbox[auto counter, number within=section]{theorem}[2][]{%
	colback=blue!10, 
	colframe=blue!40!black, 
	sharp corners=northwest,
	fonttitle=\sffamily\bfseries, 
	title=Teorema~\thetcbcounter: #2, 
	#1
}

% disponi definizioni
\newtcolorbox[auto counter, number within=section]{definition}[2][]{%
	colback=red!10,
	colframe=red!40!black,
	sharp corners=northwest,
	fonttitle=\sffamily\bfseries,
	title=Definizione~\thetcbcounter: #2,
	#1
}

% disponi problemi
\newtcolorbox[auto counter, number within=section]{problem}[2][]{%
	colback=green!10,
	colframe=green!40!black,
	sharp corners=northwest,
	fonttitle=\sffamily\bfseries,
	title=Problema~\thetcbcounter: #2,
	#1
}

% disponi codice
\usepackage{listings}
\usepackage[table]{xcolor}

\lstdefinestyle{codestyle}{
		backgroundcolor=\color{black!5}, 
		commentstyle=\color{codegreen},
		keywordstyle=\bfseries\color{magenta},
		numberstyle=\sffamily\tiny\color{black!60},
		stringstyle=\color{green!50!black},
		basicstyle=\ttfamily\footnotesize,
		breakatwhitespace=false,         
		breaklines=true,                 
		captionpos=b,                    
		keepspaces=true,                 
		numbers=left,                    
		numbersep=5pt,                  
		showspaces=false,                
		showstringspaces=false,
		showtabs=false,                  
		tabsize=2
}

\lstdefinestyle{shellstyle}{
		backgroundcolor=\color{black!5}, 
		basicstyle=\ttfamily\footnotesize\color{black}, 
		commentstyle=\color{black}, 
		keywordstyle=\color{black},
		numberstyle=\color{black!5},
		stringstyle=\color{black}, 
		showspaces=false,
		showstringspaces=false, 
		showtabs=false, 
		tabsize=2, 
		numbers=none, 
		breaklines=true
}

\lstdefinelanguage{javascript}{
	keywords={typeof, new, true, false, catch, function, return, null, catch, switch, var, if, in, while, do, else, case, break},
	keywordstyle=\color{blue}\bfseries,
	ndkeywords={class, export, boolean, throw, implements, import, this},
	ndkeywordstyle=\color{darkgray}\bfseries,
	identifierstyle=\color{black},
	sensitive=false,
	comment=[l]{//},
	morecomment=[s]{/*}{*/},
	commentstyle=\color{purple}\ttfamily,
	stringstyle=\color{red}\ttfamily,
	morestring=[b]',
	morestring=[b]"
}

% disponi sezioni
\usepackage{titlesec}

\titleformat{\section}
	{\sffamily\Large\bfseries} 
	{\thesection}{1em}{} 
\titleformat{\subsection}
	{\sffamily\large\bfseries}   
	{\thesubsection}{1em}{} 
\titleformat{\subsubsection}
	{\sffamily\normalsize\bfseries} 
	{\thesubsubsection}{1em}{}

% disponi alberi
\usepackage{forest}

\forestset{
	rectstyle/.style={
		for tree={rectangle,draw,font=\large\sffamily}
	},
	roundstyle/.style={
		for tree={circle,draw,font=\large}
	}
}

% disponi algoritmi
\usepackage{algorithm}
\usepackage{algorithmic}
\makeatletter
\renewcommand{\ALG@name}{Algoritmo}
\makeatother

% disponi numeri di pagina
\usepackage{fancyhdr}
\fancyhf{} 
\fancyfoot[L]{\sffamily{\thepage}}

\makeatletter
\fancyhead[L]{\raisebox{1ex}[0pt][0pt]{\sffamily{\@title \ \@date}}} 
\fancyhead[R]{\raisebox{1ex}[0pt][0pt]{\sffamily{\@author}}}
\makeatother

\begin{document}

% sezione (data)
\section{Lezione del 14-05-25}

% stili pagina
\thispagestyle{empty}
\pagestyle{fancy}

% testo
Riprendiamo la trattazione dei costi per destinazione, introducendo il concetto di \textbf{oggetto di costo}.

\subsubsection{Oggetto di costo}
L'oggetto di costo è ciò in relazione a cui valutiamo i costi, cioè lo scopo per il quale i costi sono misurati.
Nel caso più semplice, questo è un \textbf{prodotto}.
Il \textbf{costo pieno} (\textit{full cost}) di un oggetto di costo dipende da tutte le risorse utilizzate per tale oggetto di costo.

Il costo può essere:
\begin{itemize}
	\item \textbf{Diretto:} riconducibile a un singolo oggetto di costo;
	\item \textbf{Indiretto:} sempre legato alla produzione, ma non necessariamente riconducibile a un singolo oggetto di costo.
\end{itemize}

Abbiamo quindi che la produzione genera chiaramente dei \textbf{costi di produzione}, che possono essere diretti o indiretti, fra sostanzialmente 3 componenti:
\begin{itemize}
	\item Materiali diretti;
	\item Manodopera diretta;
	\item Costi generali di produzione: questi sono indiretti, in quanto non riguardano direttamente il singolo prodotto (ma sono connessi al funzionamento della fabbrica, cioè ad eesmpio riscaldamento, ammortamento strutture produttive, manutenzione macchinari, ecc...).
		Possono classificare questi in:
		\begin{itemize}
			\item Materiali indiretti;
			\item Manodopera indiretta;
			\item Altre risorse consumate in produzione non legate al prodotto.
		\end{itemize}
\end{itemize}

Restano quindi fuori i \textbf{costi non di produzione}, che classifichiamo in:
\begin{itemize}
	\item Costi di marketing/vendita: costi necessari per ottenere l'ordine e consegnare il prodotto:
	\item Costi amministrativi: costi generali per mantenere gli uffici amministrativi;
	\item Costi generali, ad esempio interessi, ricerca e sviluppo, eccetera.
\end{itemize}

\par\smallskip

Vediamo che l'idea di oggetto di costo si può estendere oltre i singoli prodotti: ad esempio possiamo prendere come oggetti di costo:
\begin{itemize}
	\item Prodotti (il caso visto adesso);
	\item Linee di prodotti;
	\item Marchi;
	\item Agenti (venditori);
	\item Canali di distribuzione;
	\item Servizi;
	\item Progetti;
	\item Attività;
	\item Unità organizzative.
\end{itemize}

\subsubsection{Classificazione dei costi}

Abbiamo quindi che il costo si classifica come:
\begin{itemize}
	\begin{itemize}
		\item[] Materie prime;
		\item[$+$] Costo del lavoro diretto;
	\end{itemize}
	$=$ \textbf{Costo primo} (o diretto) di prodotto;
	\begin{itemize}
		\item[$+$] Costi di produzione indiretti;
	\end{itemize}
	$=$ Costo pieno industriale;
	\begin{itemize}
		\item[$+$] costi non di produzione;
	\end{itemize}
	$=$ Costo pieno aziendale.
\end{itemize}

Dove notiamo che:
\begin{itemize}
	\item Il \textbf{costo pieno industriale}, o \textit{costo inventariabile} o \textit{costo di prodotto} è il valore delle risorse associabili, in modo diretto o indiretto, a un prodotto, perciò valorizza le rimanenze;
	\item Il \textbf{costo di periodo} (costi non di produzione) comprende attività non sostenute allo scopo diretto di produzione, cioè associabili alla realizzazione di un prodotto (sono fra queste amministrazione, ricerca e sviluppo, ecc...).
\end{itemize}

\subsubsection{Classificazione quantitativa dei costi}

Abbiamo quindi che vale, in linea generale, la seguente classificazione dal punto di vista quantitativo:
\begin{itemize}
	\item \textbf{Costi diretti:} sono ricondotti specificamente all'oggetto di costo in quanto sono da questo causati:
		\begin{itemize}
			\item Quantità del fattore effettivamente impiegata dall'oggetto $\times$ il suo prezzo;
			\item Valore di fattori produttivi i cui servizi sono impiegati in modo esclusivo dall'oggetto di costo.
		\end{itemize}
	\item \textbf{Costi indiretti:} sono causati da 2 o più oggetti di costo, e quindi non sono direttamente riconducibili a nessun oggetto di costo singolo.
		In questo caso la quantificazione rispetto ai singoli è impossibile, o economicamente non conveniente.
\end{itemize}

\subsubsection{Cost driver}
Riguardo alle variazioni di livello di attività, i costi possono essere:
\begin{itemize}
	\item \textbf{Fissi}, cioè che rimangono inalterati in un intervallo significativo di variazione del livello di attività.
		Questi possono essere:
		\begin{itemize}
			\item Costi fissi \textbf{impegnati}: importanti sul lungo termine, non si possono tagliare sul breve termine senza danneggiare gravemente la redditività;
			\item Costi fissi \textbf{discrezionali}: importanti sul breve termine, possono essere tagliati per brevi periodi con danni minimi alla redditività.
		\end{itemize}
	\item \textbf{Variabili}, cioè che si modificano assieme al livello di attività.
		Fra questi possiamo distinguere:
		\begin{itemize}
			\item Costi variabili \textbf{lineari}: che scalano come il cost driver;
			\item Costi variabili \textbf{progressivi}: che scalano sempre più al crescere del cost driver, ad esempio manodopera (grazie a straordinari, ecc...);
			\item Costi variabili \textbf{degressivi}: che scalano sempre meno al crescere del cost driver, ad esempio materia prima (grazie a sconti di quantità, acquisto all'ingrosso, ecc...).
		\end{itemize}
	\item \textbf{Misti}, cioè semivariabili o a scalini (semifissi o variabili al gradino, insomma dati da combinazioni di costi variabili e costi fissi);
\end{itemize}

A spiegare tale variazioni di livello sono i cosiddetti \textbf{cost driver}, presi all'interno di un intervallo di variazione (\textbf{area di rilevanza}), su un certo \textbf{periodo} di tempo.

\end{document}


\documentclass[a4paper,11pt]{article}
\usepackage[a4paper, margin=8em]{geometry}

% usa i pacchetti per la scrittura in italiano
\usepackage[french,italian]{babel}
\usepackage[T1]{fontenc}
\usepackage[utf8]{inputenc}
\frenchspacing 

% usa i pacchetti per la formattazione matematica
\usepackage{amsmath, amssymb, amsthm, amsfonts}

% usa altri pacchetti
\usepackage{gensymb}
\usepackage{hyperref}
\usepackage{standalone}

% imposta il titolo
\title{Appunti Economia ed Organizzazione Aziendale}
\author{Luca Seggiani}
\date{2025}

% disegni
\usepackage{pgfplots}
\pgfplotsset{width=10cm,compat=1.9}

% imposta lo stile
% usa helvetica
\usepackage[scaled]{helvet}
% usa palatino
\usepackage{palatino}
% usa un font monospazio guardabile
\usepackage{lmodern}

\renewcommand{\rmdefault}{ppl}
\renewcommand{\sfdefault}{phv}
\renewcommand{\ttdefault}{lmtt}

% disponi il titolo
\makeatletter
\renewcommand{\maketitle} {
	\begin{center} 
		\begin{minipage}[t]{.8\textwidth}
			\textsf{\huge\bfseries \@title} 
		\end{minipage}%
		\begin{minipage}[t]{.2\textwidth}
			\raggedleft \vspace{-1.65em}
			\textsf{\small \@author} \vfill
			\textsf{\small \@date}
		\end{minipage}
		\par
	\end{center}

	\thispagestyle{empty}
	\pagestyle{fancy}
}
\makeatother

% disponi teoremi
\usepackage{tcolorbox}
\newtcolorbox[auto counter, number within=section]{theorem}[2][]{%
	colback=blue!10, 
	colframe=blue!40!black, 
	sharp corners=northwest,
	fonttitle=\sffamily\bfseries, 
	title=Teorema~\thetcbcounter: #2, 
	#1
}

% disponi definizioni
\newtcolorbox[auto counter, number within=section]{definition}[2][]{%
	colback=red!10,
	colframe=red!40!black,
	sharp corners=northwest,
	fonttitle=\sffamily\bfseries,
	title=Definizione~\thetcbcounter: #2,
	#1
}

% disponi problemi
\newtcolorbox[auto counter, number within=section]{problem}[2][]{%
	colback=green!10,
	colframe=green!40!black,
	sharp corners=northwest,
	fonttitle=\sffamily\bfseries,
	title=Problema~\thetcbcounter: #2,
	#1
}

% disponi codice
\usepackage{listings}
\usepackage[table]{xcolor}

\lstdefinestyle{codestyle}{
		backgroundcolor=\color{black!5}, 
		commentstyle=\color{codegreen},
		keywordstyle=\bfseries\color{magenta},
		numberstyle=\sffamily\tiny\color{black!60},
		stringstyle=\color{green!50!black},
		basicstyle=\ttfamily\footnotesize,
		breakatwhitespace=false,         
		breaklines=true,                 
		captionpos=b,                    
		keepspaces=true,                 
		numbers=left,                    
		numbersep=5pt,                  
		showspaces=false,                
		showstringspaces=false,
		showtabs=false,                  
		tabsize=2
}

\lstdefinestyle{shellstyle}{
		backgroundcolor=\color{black!5}, 
		basicstyle=\ttfamily\footnotesize\color{black}, 
		commentstyle=\color{black}, 
		keywordstyle=\color{black},
		numberstyle=\color{black!5},
		stringstyle=\color{black}, 
		showspaces=false,
		showstringspaces=false, 
		showtabs=false, 
		tabsize=2, 
		numbers=none, 
		breaklines=true
}

\lstdefinelanguage{javascript}{
	keywords={typeof, new, true, false, catch, function, return, null, catch, switch, var, if, in, while, do, else, case, break},
	keywordstyle=\color{blue}\bfseries,
	ndkeywords={class, export, boolean, throw, implements, import, this},
	ndkeywordstyle=\color{darkgray}\bfseries,
	identifierstyle=\color{black},
	sensitive=false,
	comment=[l]{//},
	morecomment=[s]{/*}{*/},
	commentstyle=\color{purple}\ttfamily,
	stringstyle=\color{red}\ttfamily,
	morestring=[b]',
	morestring=[b]"
}

% disponi sezioni
\usepackage{titlesec}

\titleformat{\section}
	{\sffamily\Large\bfseries} 
	{\thesection}{1em}{} 
\titleformat{\subsection}
	{\sffamily\large\bfseries}   
	{\thesubsection}{1em}{} 
\titleformat{\subsubsection}
	{\sffamily\normalsize\bfseries} 
	{\thesubsubsection}{1em}{}

% disponi alberi
\usepackage{forest}

\forestset{
	rectstyle/.style={
		for tree={rectangle,draw,font=\large\sffamily}
	},
	roundstyle/.style={
		for tree={circle,draw,font=\large}
	}
}

% disponi algoritmi
\usepackage{algorithm}
\usepackage{algorithmic}
\makeatletter
\renewcommand{\ALG@name}{Algoritmo}
\makeatother

% disponi numeri di pagina
\usepackage{fancyhdr}
\fancyhf{} 
\fancyfoot[L]{\sffamily{\thepage}}

\makeatletter
\fancyhead[L]{\raisebox{1ex}[0pt][0pt]{\sffamily{\@title \ \@date}}} 
\fancyhead[R]{\raisebox{1ex}[0pt][0pt]{\sffamily{\@author}}}
\makeatother

\begin{document}

% sezione (data)
\section{Lezione del 15-05-25}

% stili pagina
\thispagestyle{empty}
\pagestyle{fancy}

% testo
Riprendiamo la trattazione dei costi.

\subsubsection{Costi totali e unitari}
Facciamo una distinzione fra \textbf{costi totali} $CT$ e \textbf{costi unitari} $CTu$.
Assunta una relazione lineare:
\begin{itemize}
	\item All'aumentare della quantità $Q$ (o \textit{volume}, di produzione), il costo unitario resta costante;
	\item Il costo totale risulta uguale ai costi unitari per la quantità $CTu \times Q$, più i costi fissi $CF$.
\end{itemize}

Lo sfruttamento di costi unitari decrescenti su grandi volumi di produzione $Q$ viene detto \textbf{economia di volume}.

\subsubsection{Analisi Costi-Volume-Risultato}
L'analisi costi-volume-risultato (\textbf{CVR}) serve a valutare il cosiddetto \textit{break-even point}, cioè il punto in cui valutando i costi per un certo volume di produzione, e il risultato ottenuto dalla vendita, si ritorna in pari.

In particolare, quindi, vogliamo valutare come deve modificarsi il livello di output (prodotto) in modo che:
\begin{itemize}
	\item Si raggiunga il pareggio fra costi e ricavi;
	\item Si ottengano certi obiettivi prefissati di reddito operativo.
\end{itemize}

Il modello fa le seguenti ipotesi:
\begin{itemize}
	\item $CT = CVu \times Q + CF$;
	\item $RT = p \times Q$;
	\item Si vende tutto il prodotto (no rimanenze);
	\item Si considera l'azienda come \textit{mono-prodotto} o si considera solo un certo mix di vendite costante;
	\item Si fa un analisi di breve periodo;
	\item Si trascurano gli effetti fiscali e della gestione straordinaria.
\end{itemize}

Il modello si risolve facilmente come:
$$
CT = RT \implies CV u \times Q + CF = p \times Q
$$
$$
Q(p - CV u) = CF: lmplies Q_{bep} = \frac{CF}{p - CV u}
$$
dove $Q_{bep}$ è appunto il punto di break even, e $p - CVu$ rappresenta il \textbf{margine di contribuzione}, cioè il ricavo sul singolo prodotto.

\subsection{Analisi di bilancio}
Valutare il solo \textbf{utile} è troppo limitante nell'analisi di un'azienda: l'\textbf{analisi di bilancio} serve a contestualizzare fra di loro diversi valori di bilancio. 
In particolare, parleremo di analisi di bilancio \textit{per indici}, cioè che calcola \textbf{indici} o \textit{rapporti} fra diverse grandezze, secondo una certa logica, che danno informazioni sullo \textbf{stato di salute} dell'azienda.

\end{document}


\documentclass[a4paper,11pt]{article}
\usepackage[a4paper, margin=8em]{geometry}

% usa i pacchetti per la scrittura in italiano
\usepackage[french,italian]{babel}
\usepackage[T1]{fontenc}
\usepackage[utf8]{inputenc}
\frenchspacing 

% usa i pacchetti per la formattazione matematica
\usepackage{amsmath, amssymb, amsthm, amsfonts}

% usa altri pacchetti
\usepackage{gensymb}
\usepackage{hyperref}
\usepackage{standalone}

% imposta il titolo
\title{Appunti Economia ed Organizzazione Aziendale}
\author{Luca Seggiani}
\date{2025}

% disegni
\usepackage{pgfplots}
\pgfplotsset{width=10cm,compat=1.9}

% imposta lo stile
% usa helvetica
\usepackage[scaled]{helvet}
% usa palatino
\usepackage{palatino}
% usa un font monospazio guardabile
\usepackage{lmodern}

\renewcommand{\rmdefault}{ppl}
\renewcommand{\sfdefault}{phv}
\renewcommand{\ttdefault}{lmtt}

% disponi il titolo
\makeatletter
\renewcommand{\maketitle} {
	\begin{center} 
		\begin{minipage}[t]{.8\textwidth}
			\textsf{\huge\bfseries \@title} 
		\end{minipage}%
		\begin{minipage}[t]{.2\textwidth}
			\raggedleft \vspace{-1.65em}
			\textsf{\small \@author} \vfill
			\textsf{\small \@date}
		\end{minipage}
		\par
	\end{center}

	\thispagestyle{empty}
	\pagestyle{fancy}
}
\makeatother

% disponi teoremi
\usepackage{tcolorbox}
\newtcolorbox[auto counter, number within=section]{theorem}[2][]{%
	colback=blue!10, 
	colframe=blue!40!black, 
	sharp corners=northwest,
	fonttitle=\sffamily\bfseries, 
	title=Teorema~\thetcbcounter: #2, 
	#1
}

% disponi definizioni
\newtcolorbox[auto counter, number within=section]{definition}[2][]{%
	colback=red!10,
	colframe=red!40!black,
	sharp corners=northwest,
	fonttitle=\sffamily\bfseries,
	title=Definizione~\thetcbcounter: #2,
	#1
}

% disponi problemi
\newtcolorbox[auto counter, number within=section]{problem}[2][]{%
	colback=green!10,
	colframe=green!40!black,
	sharp corners=northwest,
	fonttitle=\sffamily\bfseries,
	title=Problema~\thetcbcounter: #2,
	#1
}

% disponi codice
\usepackage{listings}
\usepackage[table]{xcolor}

\lstdefinestyle{codestyle}{
		backgroundcolor=\color{black!5}, 
		commentstyle=\color{codegreen},
		keywordstyle=\bfseries\color{magenta},
		numberstyle=\sffamily\tiny\color{black!60},
		stringstyle=\color{green!50!black},
		basicstyle=\ttfamily\footnotesize,
		breakatwhitespace=false,         
		breaklines=true,                 
		captionpos=b,                    
		keepspaces=true,                 
		numbers=left,                    
		numbersep=5pt,                  
		showspaces=false,                
		showstringspaces=false,
		showtabs=false,                  
		tabsize=2
}

\lstdefinestyle{shellstyle}{
		backgroundcolor=\color{black!5}, 
		basicstyle=\ttfamily\footnotesize\color{black}, 
		commentstyle=\color{black}, 
		keywordstyle=\color{black},
		numberstyle=\color{black!5},
		stringstyle=\color{black}, 
		showspaces=false,
		showstringspaces=false, 
		showtabs=false, 
		tabsize=2, 
		numbers=none, 
		breaklines=true
}

\lstdefinelanguage{javascript}{
	keywords={typeof, new, true, false, catch, function, return, null, catch, switch, var, if, in, while, do, else, case, break},
	keywordstyle=\color{blue}\bfseries,
	ndkeywords={class, export, boolean, throw, implements, import, this},
	ndkeywordstyle=\color{darkgray}\bfseries,
	identifierstyle=\color{black},
	sensitive=false,
	comment=[l]{//},
	morecomment=[s]{/*}{*/},
	commentstyle=\color{purple}\ttfamily,
	stringstyle=\color{red}\ttfamily,
	morestring=[b]',
	morestring=[b]"
}

% disponi sezioni
\usepackage{titlesec}

\titleformat{\section}
	{\sffamily\Large\bfseries} 
	{\thesection}{1em}{} 
\titleformat{\subsection}
	{\sffamily\large\bfseries}   
	{\thesubsection}{1em}{} 
\titleformat{\subsubsection}
	{\sffamily\normalsize\bfseries} 
	{\thesubsubsection}{1em}{}

% disponi alberi
\usepackage{forest}

\forestset{
	rectstyle/.style={
		for tree={rectangle,draw,font=\large\sffamily}
	},
	roundstyle/.style={
		for tree={circle,draw,font=\large}
	}
}

% disponi algoritmi
\usepackage{algorithm}
\usepackage{algorithmic}
\makeatletter
\renewcommand{\ALG@name}{Algoritmo}
\makeatother

% disponi numeri di pagina
\usepackage{fancyhdr}
\fancyhf{} 
\fancyfoot[L]{\sffamily{\thepage}}

\makeatletter
\fancyhead[L]{\raisebox{1ex}[0pt][0pt]{\sffamily{\@title \ \@date}}} 
\fancyhead[R]{\raisebox{1ex}[0pt][0pt]{\sffamily{\@author}}}
\makeatother

\begin{document}

% sezione (data)
\section{Lezione del 21-05-25}

% stili pagina
\thispagestyle{empty}
\pagestyle{fancy}

% testo
\subsection{Analisi di bilancio}
L'analisi di bilancio si può svolgere secondo due ottiche, \textbf{esterna} e \textbf{interna}.
Inoltre, nell'ottica scelta si può fare un bilancio \textbf{consuntivo} o \textbf{prospettico}, cioè che riguarda rispettivamente la strategia corrente o i piani futuri (quindi le considerazioni di budget).

A noi interesserà fare un'analisi di bilancio esterna consuntiva, sfruttando come abbiamo detto lo strumento degli \textbf{indici} (che non sono l'unico strumento per l'analisi di bilancio).

In un analisi di bilancio si individuano tipicamente 3 fasi:
\begin{enumerate}
	\item Ricerca dei dati;
	\item Elaborazione del bilancio, cioè:
		\begin{itemize}
			\item Riclassificazione;
			\item Calcolo indici;
			\item Calcolo flussi.
		\end{itemize}
	\item Valutazione del bilancio, da dove si ricavano informazioni su:
		\begin{itemize}
			\item Redditività;
			\item Liquidità e solidità;
			\item Leva finanziaria.
		\end{itemize}
\end{enumerate}


\end{document}

\end{document}