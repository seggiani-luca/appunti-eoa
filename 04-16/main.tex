
\documentclass[a4paper,11pt]{article}
\usepackage[a4paper, margin=8em]{geometry}

% usa i pacchetti per la scrittura in italiano
\usepackage[french,italian]{babel}
\usepackage[T1]{fontenc}
\usepackage[utf8]{inputenc}
\frenchspacing 

% usa i pacchetti per la formattazione matematica
\usepackage{amsmath, amssymb, amsthm, amsfonts}

% usa altri pacchetti
\usepackage{gensymb}
\usepackage{hyperref}
\usepackage{standalone}

% imposta il titolo
\title{Appunti Economia ed Organizzazione Aziendale}
\author{Luca Seggiani}
\date{2025}

% disegni
\usepackage{pgfplots}
\pgfplotsset{width=10cm,compat=1.9}

% imposta lo stile
% usa helvetica
\usepackage[scaled]{helvet}
% usa palatino
\usepackage{palatino}
% usa un font monospazio guardabile
\usepackage{lmodern}

\renewcommand{\rmdefault}{ppl}
\renewcommand{\sfdefault}{phv}
\renewcommand{\ttdefault}{lmtt}

% disponi il titolo
\makeatletter
\renewcommand{\maketitle} {
	\begin{center} 
		\begin{minipage}[t]{.8\textwidth}
			\textsf{\huge\bfseries \@title} 
		\end{minipage}%
		\begin{minipage}[t]{.2\textwidth}
			\raggedleft \vspace{-1.65em}
			\textsf{\small \@author} \vfill
			\textsf{\small \@date}
		\end{minipage}
		\par
	\end{center}

	\thispagestyle{empty}
	\pagestyle{fancy}
}
\makeatother

% disponi teoremi
\usepackage{tcolorbox}
\newtcolorbox[auto counter, number within=section]{theorem}[2][]{%
	colback=blue!10, 
	colframe=blue!40!black, 
	sharp corners=northwest,
	fonttitle=\sffamily\bfseries, 
	title=Teorema~\thetcbcounter: #2, 
	#1
}

% disponi definizioni
\newtcolorbox[auto counter, number within=section]{definition}[2][]{%
	colback=red!10,
	colframe=red!40!black,
	sharp corners=northwest,
	fonttitle=\sffamily\bfseries,
	title=Definizione~\thetcbcounter: #2,
	#1
}

% disponi problemi
\newtcolorbox[auto counter, number within=section]{problem}[2][]{%
	colback=green!10,
	colframe=green!40!black,
	sharp corners=northwest,
	fonttitle=\sffamily\bfseries,
	title=Problema~\thetcbcounter: #2,
	#1
}

% disponi codice
\usepackage{listings}
\usepackage[table]{xcolor}

\lstdefinestyle{codestyle}{
		backgroundcolor=\color{black!5}, 
		commentstyle=\color{codegreen},
		keywordstyle=\bfseries\color{magenta},
		numberstyle=\sffamily\tiny\color{black!60},
		stringstyle=\color{green!50!black},
		basicstyle=\ttfamily\footnotesize,
		breakatwhitespace=false,         
		breaklines=true,                 
		captionpos=b,                    
		keepspaces=true,                 
		numbers=left,                    
		numbersep=5pt,                  
		showspaces=false,                
		showstringspaces=false,
		showtabs=false,                  
		tabsize=2
}

\lstdefinestyle{shellstyle}{
		backgroundcolor=\color{black!5}, 
		basicstyle=\ttfamily\footnotesize\color{black}, 
		commentstyle=\color{black}, 
		keywordstyle=\color{black},
		numberstyle=\color{black!5},
		stringstyle=\color{black}, 
		showspaces=false,
		showstringspaces=false, 
		showtabs=false, 
		tabsize=2, 
		numbers=none, 
		breaklines=true
}

\lstdefinelanguage{javascript}{
	keywords={typeof, new, true, false, catch, function, return, null, catch, switch, var, if, in, while, do, else, case, break},
	keywordstyle=\color{blue}\bfseries,
	ndkeywords={class, export, boolean, throw, implements, import, this},
	ndkeywordstyle=\color{darkgray}\bfseries,
	identifierstyle=\color{black},
	sensitive=false,
	comment=[l]{//},
	morecomment=[s]{/*}{*/},
	commentstyle=\color{purple}\ttfamily,
	stringstyle=\color{red}\ttfamily,
	morestring=[b]',
	morestring=[b]"
}

% disponi sezioni
\usepackage{titlesec}

\titleformat{\section}
	{\sffamily\Large\bfseries} 
	{\thesection}{1em}{} 
\titleformat{\subsection}
	{\sffamily\large\bfseries}   
	{\thesubsection}{1em}{} 
\titleformat{\subsubsection}
	{\sffamily\normalsize\bfseries} 
	{\thesubsubsection}{1em}{}

% disponi alberi
\usepackage{forest}

\forestset{
	rectstyle/.style={
		for tree={rectangle,draw,font=\large\sffamily}
	},
	roundstyle/.style={
		for tree={circle,draw,font=\large}
	}
}

% disponi algoritmi
\usepackage{algorithm}
\usepackage{algorithmic}
\makeatletter
\renewcommand{\ALG@name}{Algoritmo}
\makeatother

% disponi numeri di pagina
\usepackage{fancyhdr}
\fancyhf{} 
\fancyfoot[L]{\sffamily{\thepage}}

\makeatletter
\fancyhead[L]{\raisebox{1ex}[0pt][0pt]{\sffamily{\@title \ \@date}}} 
\fancyhead[R]{\raisebox{1ex}[0pt][0pt]{\sffamily{\@author}}}
\makeatother

\begin{document}

% sezione (data)
\section{Lezione del 16-04-25}

% stili pagina
\thispagestyle{empty}
\pagestyle{fancy}

% testo
\subsection{Fattori pluriennali}
I fattori pluriennali possono essere \textbf{materiali} o \textbf{immateriali}, ad esempio possiamo distinguere:
\begin{itemize}
	\item Fattori pluriennali \textbf{materiali}:
		\begin{itemize}
			\item Impianto;
			\item Macchinari;
			\item Attrezzature;
			\item Immobili.
		\end{itemize}
	\item Fattori pluriennali \textbf{immateriali}:
		\begin{itemize}
			\item Brevetti;
			\item Costi di impianto;
			\item Avviamento;
			\item Ricerca e sviluppo;
			\item Costi di pubblicità.
		\end{itemize}
\end{itemize}

\subsubsection{Avviamento}
L'\textbf{avviamento} merita qualche parola a sé.
L'avviamento di un'azienda rappresenta un valore utile al momento della vendita (e quindi l'acquisto) di quell'azienda.
In questo caso si calcola infatti un \textbf{capitale netto di cessione}, cioè dato sì dalla somma algebrica fra attività e passività, ma in prospettiva di cessione, quindi tenendo conto appunto dell'avviamento, cioè il fatto che l'azienda è \textit{avviata}, ha un suo nome, una sua clientela, ecc... e quindi un valore aggiunto rispetto ai soli beni che essa detiene.

\subsubsection{Pubblicità}
Parlamo anche di \textbf{pubblicità}.
Si possono fare diversi tipi di pubblicità:
\begin{itemize}
	\item Pubblicità per un \textbf{prodotto nuovo} (per quanto sia difficile definire "nuovo", non visto dal mercato, tecnologicamente avanzato, ecc...);
	\item Pubblicità per prodotti esistenti/migliorati.
\end{itemize}

I costi dovuti alle spese pubblicitarie vengono recuperati con le vendite dei prodotti pubblicizzati.
Ora, per prodotti esistenti/migliorati, si può assumere che i costi pubblicitari vengano soddisfatti, nel corso del periodo.
Per i prodotti nuovi, si deve invece immaginare che la pubblicità diventerà un "innesco" per un prodotto ancora di nicchia, che quindi si ripagherà magari nel corso di diversi anni.
Questo costo (relativo ai soli prodotti nuovi) viene quindi \textbf{capitalizzato} (portato a stato patrimoniale), e allora considerato come pluriennale.
Il costo della pubblicità di esercizio resta invece nel reddito di esercizio.

\subsection{Ammortamento}
L'\textbf{ammortamento} si applica ai \textbf{fattori pluriennali} a \textit{sinistra} dello stato patrimoniale, e ai \textbf{fattori di esercizio} a \textit{destra}, e consiste nel determinare la parte di costo pluriennale (o di fattore di esercizio) consumata nel periodo.

Nel dettaglio del costo pluriennale, quindi, possiamo dire una risorsa avrà un \textbf{costo storico}, dato ad esempio da:
\begin{itemize}
	\item Il prezzo della risora;
	\item Il trasporto;
	\item L'installazione;
	\item L'(eventuale) collaudo.
\end{itemize}

Questo costo storico non viene fattorizzato totalmente al momento dell'ottenimento della risora, ma quindi \textbf{ammortato} nel corso dell'opera.
Si tiene quindi conto del \textit{valore di consumo} della risorsa nel corso del periodo, dove per ogni periodo si riduce il prezzo storico di una \textbf{quota di ammortamento}.
Il valore che rimane della risorsa viene detto \textit{valore contabile} o \textbf{valore di libro}.

Si può fare una precisazione sul \textit{corso d'opera} che abbiamo considerato prima.
Avremo infatti una \textbf{vita utile}, che non corrisponde necessariamente alla \textbf{vita tecnica}, che è il periodo in cui la risorsa è effettivamente utilizzabile (si pensi ad un macchinario funzionante ma obsoleto).
La vita utile è quindi il periodo in cui la risorsa è \textit{economicamente} funzionante, cioè crea più ricavi che costi.

Quello che si va ad ammortizzare è quindi il costo storico del fattore, a cui si sottrae il \textbf{valore di recupero}, cioè il prezzo a cui si suppone di poter vendere il fattore al termine della sua vita utile (definito chiaramente \textit{ex ante}).

Infine, bisogna tenere conto delle \textbf{modalità di ripartizione} della risorsa, che può essere:
\begin{itemize}
	\item A quote lineari;
	\item A quote descrescenti, il più diffuso (tanto che l'ordinamento corrente prevede l'\textbf{ammortamento accelerato});
	\item A quote crescenti.
\end{itemize}
Di queste quote si tiene conto attraverso il \textbf{fondo ammortamento}.

Al momento della vendita della risorsa si tiene conto del valore di libro di tale risorsa.
Un eventuale guadagno sul valore di libro è detto \textbf{plusvalenza}, mentre una perdita viene detta \textbf{minusvalenza}.
La plusvalenza è un valore economico d'esercizio positivo, che quindi va a finre nel conto economico.

Vediamo quindi che il conto economico è stratificato: i ricavi e i costi che vi troviamo possono derivare da diverse \textbf{aree}.
\begin{itemize}
	\item Area \textbf{operativa}, divisa in:
	\begin{itemize}
	\item Area \textbf{caratteristica} (\textit{core business}), cioè quella che riguarda l'attività propria dell'impresa;
	\item Area \textbf{accessoria}, cioè quella che contiene ad esempio gli interessi attivi sul denaro che temporaneamente sta nella cassa (investimenti nel mercato finanziario, ecc...).
	\end{itemize}

	L'area operativa contribuisce all'\textbf{EBIT}, o \textit{risultato operativo}.

	\item Area \textbf{finanziaria}: contiene costi e ricavi di operazioni e finanziamento attinto (capitale di terzi) o concesso.
		Sono di questo tipo ad esempio gli \textbf{oneri} e i \textbf{proventi} finanziari;
	\item Area \textbf{fiscale}: riguarda le imposte;
	\item Area \textbf{straordinaria}: appunto, quella che contiene \textbf{plusvalenze} e \textbf{minusvalenze} di operazioni di vendita (cioè ciò di cui stavamo parlando adesso).
\end{itemize}
\end{document}
