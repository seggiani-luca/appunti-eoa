
\documentclass[a4paper,11pt]{article}
\usepackage[a4paper, margin=8em]{geometry}

% usa i pacchetti per la scrittura in italiano
\usepackage[french,italian]{babel}
\usepackage[T1]{fontenc}
\usepackage[utf8]{inputenc}
\frenchspacing 

% usa i pacchetti per la formattazione matematica
\usepackage{amsmath, amssymb, amsthm, amsfonts}

% usa altri pacchetti
\usepackage{gensymb}
\usepackage{hyperref}
\usepackage{standalone}

% imposta il titolo
\title{Appunti Economia ed Organizzazione Aziendale}
\author{Luca Seggiani}
\date{2025}

% disegni
\usepackage{pgfplots}
\pgfplotsset{width=10cm,compat=1.9}

% imposta lo stile
% usa helvetica
\usepackage[scaled]{helvet}
% usa palatino
\usepackage{palatino}
% usa un font monospazio guardabile
\usepackage{lmodern}

\renewcommand{\rmdefault}{ppl}
\renewcommand{\sfdefault}{phv}
\renewcommand{\ttdefault}{lmtt}

% disponi il titolo
\makeatletter
\renewcommand{\maketitle} {
	\begin{center} 
		\begin{minipage}[t]{.8\textwidth}
			\textsf{\huge\bfseries \@title} 
		\end{minipage}%
		\begin{minipage}[t]{.2\textwidth}
			\raggedleft \vspace{-1.65em}
			\textsf{\small \@author} \vfill
			\textsf{\small \@date}
		\end{minipage}
		\par
	\end{center}

	\thispagestyle{empty}
	\pagestyle{fancy}
}
\makeatother

% disponi teoremi
\usepackage{tcolorbox}
\newtcolorbox[auto counter, number within=section]{theorem}[2][]{%
	colback=blue!10, 
	colframe=blue!40!black, 
	sharp corners=northwest,
	fonttitle=\sffamily\bfseries, 
	title=Teorema~\thetcbcounter: #2, 
	#1
}

% disponi definizioni
\newtcolorbox[auto counter, number within=section]{definition}[2][]{%
	colback=red!10,
	colframe=red!40!black,
	sharp corners=northwest,
	fonttitle=\sffamily\bfseries,
	title=Definizione~\thetcbcounter: #2,
	#1
}

% disponi problemi
\newtcolorbox[auto counter, number within=section]{problem}[2][]{%
	colback=green!10,
	colframe=green!40!black,
	sharp corners=northwest,
	fonttitle=\sffamily\bfseries,
	title=Problema~\thetcbcounter: #2,
	#1
}

% disponi codice
\usepackage{listings}
\usepackage[table]{xcolor}

\lstdefinestyle{codestyle}{
		backgroundcolor=\color{black!5}, 
		commentstyle=\color{codegreen},
		keywordstyle=\bfseries\color{magenta},
		numberstyle=\sffamily\tiny\color{black!60},
		stringstyle=\color{green!50!black},
		basicstyle=\ttfamily\footnotesize,
		breakatwhitespace=false,         
		breaklines=true,                 
		captionpos=b,                    
		keepspaces=true,                 
		numbers=left,                    
		numbersep=5pt,                  
		showspaces=false,                
		showstringspaces=false,
		showtabs=false,                  
		tabsize=2
}

\lstdefinestyle{shellstyle}{
		backgroundcolor=\color{black!5}, 
		basicstyle=\ttfamily\footnotesize\color{black}, 
		commentstyle=\color{black}, 
		keywordstyle=\color{black},
		numberstyle=\color{black!5},
		stringstyle=\color{black}, 
		showspaces=false,
		showstringspaces=false, 
		showtabs=false, 
		tabsize=2, 
		numbers=none, 
		breaklines=true
}

\lstdefinelanguage{javascript}{
	keywords={typeof, new, true, false, catch, function, return, null, catch, switch, var, if, in, while, do, else, case, break},
	keywordstyle=\color{blue}\bfseries,
	ndkeywords={class, export, boolean, throw, implements, import, this},
	ndkeywordstyle=\color{darkgray}\bfseries,
	identifierstyle=\color{black},
	sensitive=false,
	comment=[l]{//},
	morecomment=[s]{/*}{*/},
	commentstyle=\color{purple}\ttfamily,
	stringstyle=\color{red}\ttfamily,
	morestring=[b]',
	morestring=[b]"
}

% disponi sezioni
\usepackage{titlesec}

\titleformat{\section}
	{\sffamily\Large\bfseries} 
	{\thesection}{1em}{} 
\titleformat{\subsection}
	{\sffamily\large\bfseries}   
	{\thesubsection}{1em}{} 
\titleformat{\subsubsection}
	{\sffamily\normalsize\bfseries} 
	{\thesubsubsection}{1em}{}

% disponi alberi
\usepackage{forest}

\forestset{
	rectstyle/.style={
		for tree={rectangle,draw,font=\large\sffamily}
	},
	roundstyle/.style={
		for tree={circle,draw,font=\large}
	}
}

% disponi algoritmi
\usepackage{algorithm}
\usepackage{algorithmic}
\makeatletter
\renewcommand{\ALG@name}{Algoritmo}
\makeatother

% disponi numeri di pagina
\usepackage{fancyhdr}
\fancyhf{} 
\fancyfoot[L]{\sffamily{\thepage}}

\makeatletter
\fancyhead[L]{\raisebox{1ex}[0pt][0pt]{\sffamily{\@title \ \@date}}} 
\fancyhead[R]{\raisebox{1ex}[0pt][0pt]{\sffamily{\@author}}}
\makeatother

\begin{document}

% sezione (data)
\section{Lezione del 27-02-25}

% stili pagina
\thispagestyle{empty}
\pagestyle{fancy}

% testo
\subsection{Introduzione all'economia}
Abbiamo visto come l'economia tratta dell'interazione economica fra diversi \textbf{soggetti} o \textit{enti} all'interno di diversi \textbf{sistemi economici}.
Questi possono essere \textit{privati}, \textit{aziende}, come ancora enti di varia natura se non addirittura interi \textit{stati}.

Il comportamento di ogni soggetto è determinato dal suo particolare modo di vedere e agire la realtà circostante, cioè da dei particolari \textit{assunti} o \textbf{paradigmi}.
I paradigmi dei soggetti influenzano il sistema economico a cui appartengono, e viceversa.
Proprio per questo motivo, storicamente si sono sempre formati innumerevoli sistemi economici, come innumerevoli erano gli assunti dei soggetti che vi appartenevano.

L'obiettivo principale dell'economia è quindi di comprendere il comportamento di questi soggetti all'interno di un dato sistema economico, inteso come la loro azione su determinati \textit{mezzi di produzione}, in presenza di \textbf{beni scarsi} (come li avevamo definiti in ricerca operativa, \textit{risorse limitate}).
Una volta comprese questo tipo di dinamiche, si possono ricavare \textbf{strumenti} che ci aiutino a fare scelte economiche migliori (più efficienti, più informate, ecc...). 

\subsubsection{Economia politica}
Una branca dell'economia che non considereremo in particolare è l'\textbf{economia politica}.
Questa tratta del comportamento di una vasta quantità di individui, a livello statale o oltre.
Per questo, si dice un approccio \textbf{top-down}.

Inoltre, non si premette di osservare il funzionamento interno dei soggetti economici, e quindi ad esempio di osservare la struttura delle imprese.
Per questo viene detto anche un approccio \textbf{black box} (a \textit{scatola nera}).

\subsubsection{Economia aziendale}
L'economia aziendale, come abbiamo detto, sceglie il percorso inverso: arriva alla comprensione di un sistema economico studiando la struttura e l'organizzazione delle imprese che si trovano al suo interno.
Questo lo rende un approccio \textbf{bottom-up}.

\subsubsection{Paradigmi}
Il modo di porsi di fronte a un problema (\textit{mentalità}) è definito, come abbiamo detto, \textit{paradigma}.
La \textit{propensione} è la tendenza di dare a priori un certo peso a diversi aspetti del problema da risolvere.
Possiamo assumere 3 tipi di mentalità rispetto alla gestione dell'impresa:
\begin{itemize}
	\item \textbf{Tecnico-produttiva:} tipica di chi si occupa del lato tecnico (e.g. ingegneri). 
		Unilaterale, tende al mantenimento dello \textit{status quo}: una strategia che si dimostra vincente non viene cambiata. Eccezione sono le \textbf{start-up}, dove una mentalità tecnico-produttiva può portare a cambiamenti e innovazioni.
	\item \textbf{Finanziaria:} ancora unilaterale, entra in gioco in periodi di \textbf{abbondanza} o \textbf{scarsità}, rispettivamente incentivando la tendenza a \textit{spendere} o a \textit{contrarre} le spese;
	\item \textbf{Economica:} cerca di unire gli approcci tecnico-produttivi e finanziari.
\end{itemize}

\subsubsection{Elementi in gioco}
Vediamo quindi quali saranno gli elementi di interesse nel \textbf{modello di business} che tratteremo.
# finisci

\end{document}
