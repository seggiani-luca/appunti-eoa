
\documentclass[a4paper,11pt]{article}
\usepackage[a4paper, margin=8em]{geometry}

% usa i pacchetti per la scrittura in italiano
\usepackage[french,italian]{babel}
\usepackage[T1]{fontenc}
\usepackage[utf8]{inputenc}
\frenchspacing 

% usa i pacchetti per la formattazione matematica
\usepackage{amsmath, amssymb, amsthm, amsfonts}

% usa altri pacchetti
\usepackage{gensymb}
\usepackage{hyperref}
\usepackage{standalone}

% imposta il titolo
\title{Appunti Economia ed Organizzazione Aziendale}
\author{Luca Seggiani}
\date{2025}

% disegni
\usepackage{pgfplots}
\pgfplotsset{width=10cm,compat=1.9}

% imposta lo stile
% usa helvetica
\usepackage[scaled]{helvet}
% usa palatino
\usepackage{palatino}
% usa un font monospazio guardabile
\usepackage{lmodern}

\renewcommand{\rmdefault}{ppl}
\renewcommand{\sfdefault}{phv}
\renewcommand{\ttdefault}{lmtt}

% disponi il titolo
\makeatletter
\renewcommand{\maketitle} {
	\begin{center} 
		\begin{minipage}[t]{.8\textwidth}
			\textsf{\huge\bfseries \@title} 
		\end{minipage}%
		\begin{minipage}[t]{.2\textwidth}
			\raggedleft \vspace{-1.65em}
			\textsf{\small \@author} \vfill
			\textsf{\small \@date}
		\end{minipage}
		\par
	\end{center}

	\thispagestyle{empty}
	\pagestyle{fancy}
}
\makeatother

% disponi teoremi
\usepackage{tcolorbox}
\newtcolorbox[auto counter, number within=section]{theorem}[2][]{%
	colback=blue!10, 
	colframe=blue!40!black, 
	sharp corners=northwest,
	fonttitle=\sffamily\bfseries, 
	title=Teorema~\thetcbcounter: #2, 
	#1
}

% disponi definizioni
\newtcolorbox[auto counter, number within=section]{definition}[2][]{%
	colback=red!10,
	colframe=red!40!black,
	sharp corners=northwest,
	fonttitle=\sffamily\bfseries,
	title=Definizione~\thetcbcounter: #2,
	#1
}

% disponi problemi
\newtcolorbox[auto counter, number within=section]{problem}[2][]{%
	colback=green!10,
	colframe=green!40!black,
	sharp corners=northwest,
	fonttitle=\sffamily\bfseries,
	title=Problema~\thetcbcounter: #2,
	#1
}

% disponi codice
\usepackage{listings}
\usepackage[table]{xcolor}

\lstdefinestyle{codestyle}{
	backgroundcolor=\color{black!5}, 
	commentstyle=\color{codegreen},
	keywordstyle=\bfseries\color{magenta},
	numberstyle=\sffamily\tiny\color{black!60},
	stringstyle=\color{green!50!black},
	basicstyle=\ttfamily\footnotesize,
	breakatwhitespace=false,         
	breaklines=true,                 
	captionpos=b,                    
	keepspaces=true,                 
	numbers=left,                    
	numbersep=5pt,                  
	showspaces=false,                
	showstringspaces=false,
	showtabs=false,                  
	tabsize=2
}

\lstdefinestyle{shellstyle}{
	backgroundcolor=\color{black!5}, 
	basicstyle=\ttfamily\footnotesize\color{black}, 
	commentstyle=\color{black}, 
	keywordstyle=\color{black},
	numberstyle=\color{black!5},
	stringstyle=\color{black}, 
	showspaces=false,
	showstringspaces=false, 
	showtabs=false, 
	tabsize=2, 
	numbers=none, 
	breaklines=true
}

\lstdefinelanguage{javascript}{
	keywords={typeof, new, true, false, catch, function, return, null, catch, switch, var, if, in, while, do, else, case, break},
	keywordstyle=\color{blue}\bfseries,
	ndkeywords={class, export, boolean, throw, implements, import, this},
	ndkeywordstyle=\color{darkgray}\bfseries,
	identifierstyle=\color{black},
	sensitive=false,
	comment=[l]{//},
	morecomment=[s]{/*}{*/},
	commentstyle=\color{purple}\ttfamily,
	stringstyle=\color{red}\ttfamily,
	morestring=[b]',
	morestring=[b]"
}

% disponi sezioni
\usepackage{titlesec}

\titleformat{\section}
{\sffamily\Large\bfseries} 
{\thesection}{1em}{} 
\titleformat{\subsection}
{\sffamily\large\bfseries}   
{\thesubsection}{1em}{} 
\titleformat{\subsubsection}
{\sffamily\normalsize\bfseries} 
{\thesubsubsection}{1em}{}

% disponi alberi
\usepackage{forest}

\forestset{
	rectstyle/.style={
		for tree={rectangle,draw,font=\large\sffamily}
	},
	roundstyle/.style={
		for tree={circle,draw,font=\large}
	}
}

% disponi algoritmi
\usepackage{algorithm}
\usepackage{algorithmic}
\makeatletter
\renewcommand{\ALG@name}{Algoritmo}
\makeatother

% disponi numeri di pagina
\usepackage{fancyhdr}
\fancyhf{} 
\fancyfoot[L]{\sffamily{\thepage}}

\makeatletter
\fancyhead[L]{\raisebox{1ex}[0pt][0pt]{\sffamily{\@title \ \@date}}} 
\fancyhead[R]{\raisebox{1ex}[0pt][0pt]{\sffamily{\@author}}}
\makeatother

\begin{document}

% sezione (data)
\section{Lezione del 09-04-25}

% stili pagina
\thispagestyle{empty}
\pagestyle{fancy}

% testo
Concludiamo il discorso sul bilancio parlando dell'\textit{esercizio}.

\subsubsection{Esercizio}
Alla scorsa lezione avevamo parlato di \textbf{capitale} e \textbf{reddito}, e di come questi avevano importanza sotto l'ipotesi di funzionamento, dando quindi vita al \textbf{bilancio di esercizio}.
Il bilancio di esercizio viene tenuto lungo il periodo di tempo detto \textbf{esercizio}, corrispondente a l'\textit{anno contabile} (dal 1/1 al 31/12).
Per l'interesse degli investitori si possono tenere anche bilanci \textit{infra}-annuali, magari a scadenza trimestrale o semestrale.
Questi hanno principalmente lo scopo di informare gli inverstitori e quindi il mercato finanziario.

\subsection{Operazioni di gestione}
Iniziamo a vedere nel dettaglio le operazioni di gestione delle società, partendo da quelle \textit{straordinarie}.

\subsubsection{Operazioni di gestione straordinaria}
Le operazioni di gestione \textbf{straordinaria} possono essere le seguenti:
\begin{itemize}
	\item \textbf{Cessione:} vendita in blocco dell'azienda, al termine della quale questa può continuare a vivere in capo ad un nuovo soggetto (\textit{cessazione relativa}) o essere liquidata (\textit{cessazione assoluta}).
		Queste casistiche si distinguono nella tabella:
		\begin{table}[h!]
			\center \rowcolors{2}{white}{black!10}
			\begin{tabular} { c | c | c }
				& \bfseries Cessazione assoluta & \bfseries Cessazione relativa \\
				\hline
				\bfseries Cessazione volontaria & Liquidazione & Cessione \\ 
				\bfseries Cessazione coatta & Liquidazione & // \\
			\end{tabular}
		\end{table}

		La \textit{liquidazione} può essere \textit{coatta} (\textbf{fallimento}), cioè imposta dalla legge, o volontaria.

		Nel caso della \textit{cessione}, invece, si fa un bilancio apposito, detto \textbf{bilancio di cessazione}, appunto per valutare l'ipotesi di vendita (e dall'altra parte, di acquisto) dell'azienda.
		Qui, ad esempio, si tiene conto dell'\textit{avviamento} dell'azienda.
		L'ipotesi di cessione coatta, invece, non è prevista dalla legge.

	\item \textbf{Fusione:} si riduce all'unità il patrimonio di due o più società e si fanno confluire i soci in un unica struttura organizzata (che è una società preesistente o fondata \textit{ex novo}).

		La fusione può in particolare essere:
		\begin{itemize}
			\item \textbf{Fusione ropriamente detta:} più società si \textit{estinguono}, cioè cessano di esistere, per dar vita ad una nuova società creata da zero;
			\item \textbf{Fusione per incorporazione:} più società vengono \textit{incorporate} in una società preesistente.
		\end{itemize}
	\item \textbf{Scissione:} trasferimento del patrimonio di una società preesistente a più società, che può essere totale (\textbf{scissione integrale}, o \textit{split-up}) o parziale (\textbf{scissione parziale} o \textit{split-off}).
		Solo nel caso di scissione parziale, ovviamente, si può scindere ad una sola società (in caso contrario si parlerebbe di qualche tipo di trasformazione).

		La scissione di tipo \textit{split-off} è tipica di società che si specializzano al punto di dover dividere attività e passitvità in altre società.
		In questo caso, le azioni delle società create vengono assegnate ai soci della società madre, che le scambiano con le azioni della società madre da loro possedute.
		Chiaramente, se questa scissione lascia la società madre senza attività o passività e una scissione integrale o \textit{split-up}.

	\item \textbf{Scorporo:} sembra simile alla scissione parziale, ma porta alla formazione di \textit{spin-off}.
		In questo caso la società madre cede la sua azienda o rami di essa ad altre società, formando quello che è effettivamente un \textbf{gruppo aziendale}.
		In questo caso le azioni delle società scisse restano nella società madre, che diventa \textbf{caopgruppo} di quel gruppo aziendale, cioè una società il cui unico interesse e gestire le altre società del gruppo.

	\item \textbf{Trasformazione:} la società (non s.s.) cambia forma modificando il suo atto costitutivo (magari passa da S.r.l. a S.p.A., ecc...). 
		Consiste quindi in un cambio di veste giuridica di un soggetto giuridico presistente.
\end{itemize}

\subsubsection{Operazioni di gestione ordinaria}
Vediamo quindi nel dettaglio quali sono e in cosa consistono le operazioni di gestione ordinaria della società.
Abbiamo visto il discorso dello stato patrimoniale, e quindi del capitale e del reddito.

Possiamo pensare ad un esempio che espliciti il funzionamento del meccanismo dello stato patrimoniale.
Abbiamo che questo reagisce solamente a operazioni esterne, che influenzano:
\begin{itemize}
	\item \textbf{Cassa}, \textbf{conti bancari} o \textbf{posta};
	\item \textbf{Crediti};
	\item \textbf{Debiti}.
\end{itemize}

Ad esempio l'estinguimento del debito con i soldi provenienti dalla cassa di un'azienda debitrice rappresenta una \textbf{permutazione finanziaria} che \textit{chiude}, cioè ha somma zero da entrambi i lati del bilancio: i soldi che escono dalla cassa ($\Delta F < 0$) vanno ad estinguere il debito ($\Delta F > 0$), cioè non variano il capitale sociale.

Un'altro movimento finanziario è quello che riguarda il \textbf{capitale Netto}, (\textit{Capitale Sociale} $+ \, \Delta$): ad esempio, un aumento del capitale sociale comporta una variazione economica, che è positiva ($\Delta F > 0$) se si aumenta il capitale (creando nuove azioni) e negativa ($\Delta F < 0$) se si diminusice (annullando azioni esistenti senza aumentare il valore nominale).

Potremmo considerare allora le operazioni di \textbf{finanziamento attinto}, cioè l'operazione mediante la quale l’azienda so dota dei mezzi monetari necessari per lo svolgimento della sua attività, ottenendo appunto un finanzamento da terzi.
In questo caso si crea un aumento della cassa, ma anche un debito (tra l'altro con possibile interesse) da restituire, ergo il capitale netto resta costante (se non addirittura diminuisce, $\Delta F < 0$, all'estinzione del debito).

\par\smallskip

Vediamo allora se le operazioni \textbf{economiche} $\Delta E$, cioè quelle che riguardano \textit{costi} e \textit{ricavi}, possono cambiare il capitale netto, introducendo variazioni finanziarie $\Delta F$ minori o maggiori di 0.

Queste riguardano quelli che sono i \textbf{valori economici} dell'azienda, che abbiamo già visto essere legati al \textbf{reddito.} 
Da qqui si possono avere risultati economici:
\begin{itemize}
	\item $\Delta E > 0$, si parla di \textbf{ricavi};
	\item $\Delta E < 0$, si parla di \textbf{costi} che possono essere:
		\begin{itemize}
			\item Costi \textbf{di esercizio};
			\item Costi \textbf{pluriennali}.
		\end{itemize}
\end{itemize}
Notiamo che questa operazioni non provocano variazioni di capitale netto, in quanto il costo dell'\textbf{acquisto} di merci abbatte il valore delle merci, mentre il ricavo dalla \textbf{vendita} di beni è abbattuto dalla perdita di tali beni.

A far variare la ricchezza nel corso dell'esercizio è allora la differenza fra ricavi e costi \textit{di esercizio} alla fine di un certo periodo, cioè:
$$
\sum_{i = 1}^n R - \sum_{i = 1}^n C = \Delta E
$$
Come abbiamo già visto, se questa $\Delta E$ è positiva, si parla di \textbf{utile}, altrimenti si parla di \textbf{perdita}.

\subsubsection{Operazioni di gestione esterna}
Le operazioni di cui abbiamo parlato finora soono operazioni di \textbf{gestione esterna}, cioè che guardano a quello che la società scambia con l'esterno.
Si ha che ogni oeprazione di gestione esterna ha sempre almeno 2 movimenti:
\begin{enumerate}
	\item Il primo movimento è di tipo finanziario (movimentazione di cassa, estinzione di debiti, ecc...), cioè un $\Delta F$;
	\item Il secondo movimento è di tipo finanziario o di tipo economico.
		\begin{itemize}
			\item Se il secondo movimento è di tipo finanziario, può \textit{chiudere} (portando quindi a nessuna variazione di capitale netto), quindi rappresentare una \textbf{permutazione finanziaria}, oppure non chiudere, portando ad una variazione economica $\Delta E$ dello stesso segno;
		\end{itemize}
\end{enumerate}

# qui c'è un grafico boh

\subsection{Conto economico}
Il \textbf{conto economico} fa parte assieme allo \textit{stato patrimoniale}, che abbiamo visto finora, del \textbf{bilancio}.

Questo, al contrario dello stato patrimoniale, riporta il valore delle risorse che sono state consumate in un certo periodo (\textit{costi}) per ottenere dei \textit{ricavi}.

# eh poi boh qui finì

\subsection{IVA}
L'\textbf{IVA}, \textit{Imposta sul Valore Aggiunto}, è un \textbf{imposta} applicata sul valore aggiunto in ogni fase di produzione di un bene.

\par\smallskip

La differenza fra \textit{imposta} e \textit{tassa} è il principio della \textit{controprestazione}: a una tassa corrisponde un bene o servizio che viene erogato dallo stato o dall'ente pubblico a cui la si paga, mentre l'imposta si traduce in un finanziamento dei servizi pubblici generali (pubblici nel senso di forniti anche a chi \textit{non} paga le imposte). 

\par\smallskip

L'IVA viene pagata dal consumatore finale, e si accumula in tutte le fasi di produzione di un certo bene.
Ad esempio, poniamo che il produttore principale produce un prodotto ad un certo valore, detto \textbf{base imponibile}.
Su questo valore si calcolerà una percentuale (22\% all'ordinamento corrente) di \textbf{valore aggiunto}, che il produttore deve versare allo stato.
Questa percentuale dovrà essere pagata, assieme alla base imponibile, da tutte quelle aziende che elaborano tale prodotto, ricalcolata ad ogni passaggio della catena di produzione.
In verità, vediamo che queste aziende non pagano di tasca propria il valore dell'IVA, ma lo \textit{"anticipano"} al consumatore (che può essere un'altra azienda produttrice o il consumatore finale), aggiungendolo al prezzo del prodotto venduto alla fine della produzione (come aveva fatto il produttore principale stesso).
Alla fine l'IVA accumulata viene quindi pagata dall'ultimo che entra in possesso del prodotto, cioè il consumatore finale.


\end{document}
