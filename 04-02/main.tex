
\documentclass[a4paper,11pt]{article}
\usepackage[a4paper, margin=8em]{geometry}

% usa i pacchetti per la scrittura in italiano
\usepackage[french,italian]{babel}
\usepackage[T1]{fontenc}
\usepackage[utf8]{inputenc}
\frenchspacing 

% usa i pacchetti per la formattazione matematica
\usepackage{amsmath, amssymb, amsthm, amsfonts}

% usa altri pacchetti
\usepackage{gensymb}
\usepackage{hyperref}
\usepackage{standalone}

% imposta il titolo
\title{Appunti Economia ed Organizzazione Aziendale}
\author{Luca Seggiani}
\date{2025}

% disegni
\usepackage{pgfplots}
\pgfplotsset{width=10cm,compat=1.9}

% imposta lo stile
% usa helvetica
\usepackage[scaled]{helvet}
% usa palatino
\usepackage{palatino}
% usa un font monospazio guardabile
\usepackage{lmodern}

\renewcommand{\rmdefault}{ppl}
\renewcommand{\sfdefault}{phv}
\renewcommand{\ttdefault}{lmtt}

% disponi il titolo
\makeatletter
\renewcommand{\maketitle} {
	\begin{center} 
		\begin{minipage}[t]{.8\textwidth}
			\textsf{\huge\bfseries \@title} 
		\end{minipage}%
		\begin{minipage}[t]{.2\textwidth}
			\raggedleft \vspace{-1.65em}
			\textsf{\small \@author} \vfill
			\textsf{\small \@date}
		\end{minipage}
		\par
	\end{center}

	\thispagestyle{empty}
	\pagestyle{fancy}
}
\makeatother

% disponi teoremi
\usepackage{tcolorbox}
\newtcolorbox[auto counter, number within=section]{theorem}[2][]{%
	colback=blue!10, 
	colframe=blue!40!black, 
	sharp corners=northwest,
	fonttitle=\sffamily\bfseries, 
	title=Teorema~\thetcbcounter: #2, 
	#1
}

% disponi definizioni
\newtcolorbox[auto counter, number within=section]{definition}[2][]{%
	colback=red!10,
	colframe=red!40!black,
	sharp corners=northwest,
	fonttitle=\sffamily\bfseries,
	title=Definizione~\thetcbcounter: #2,
	#1
}

% disponi problemi
\newtcolorbox[auto counter, number within=section]{problem}[2][]{%
	colback=green!10,
	colframe=green!40!black,
	sharp corners=northwest,
	fonttitle=\sffamily\bfseries,
	title=Problema~\thetcbcounter: #2,
	#1
}

% disponi codice
\usepackage{listings}
\usepackage[table]{xcolor}

\lstdefinestyle{codestyle}{
		backgroundcolor=\color{black!5}, 
		commentstyle=\color{codegreen},
		keywordstyle=\bfseries\color{magenta},
		numberstyle=\sffamily\tiny\color{black!60},
		stringstyle=\color{green!50!black},
		basicstyle=\ttfamily\footnotesize,
		breakatwhitespace=false,         
		breaklines=true,                 
		captionpos=b,                    
		keepspaces=true,                 
		numbers=left,                    
		numbersep=5pt,                  
		showspaces=false,                
		showstringspaces=false,
		showtabs=false,                  
		tabsize=2
}

\lstdefinestyle{shellstyle}{
		backgroundcolor=\color{black!5}, 
		basicstyle=\ttfamily\footnotesize\color{black}, 
		commentstyle=\color{black}, 
		keywordstyle=\color{black},
		numberstyle=\color{black!5},
		stringstyle=\color{black}, 
		showspaces=false,
		showstringspaces=false, 
		showtabs=false, 
		tabsize=2, 
		numbers=none, 
		breaklines=true
}

\lstdefinelanguage{javascript}{
	keywords={typeof, new, true, false, catch, function, return, null, catch, switch, var, if, in, while, do, else, case, break},
	keywordstyle=\color{blue}\bfseries,
	ndkeywords={class, export, boolean, throw, implements, import, this},
	ndkeywordstyle=\color{darkgray}\bfseries,
	identifierstyle=\color{black},
	sensitive=false,
	comment=[l]{//},
	morecomment=[s]{/*}{*/},
	commentstyle=\color{purple}\ttfamily,
	stringstyle=\color{red}\ttfamily,
	morestring=[b]',
	morestring=[b]"
}

% disponi sezioni
\usepackage{titlesec}

\titleformat{\section}
	{\sffamily\Large\bfseries} 
	{\thesection}{1em}{} 
\titleformat{\subsection}
	{\sffamily\large\bfseries}   
	{\thesubsection}{1em}{} 
\titleformat{\subsubsection}
	{\sffamily\normalsize\bfseries} 
	{\thesubsubsection}{1em}{}

% disponi alberi
\usepackage{forest}

\forestset{
	rectstyle/.style={
		for tree={rectangle,draw,font=\large\sffamily}
	},
	roundstyle/.style={
		for tree={circle,draw,font=\large}
	}
}

% disponi algoritmi
\usepackage{algorithm}
\usepackage{algorithmic}
\makeatletter
\renewcommand{\ALG@name}{Algoritmo}
\makeatother

% disponi numeri di pagina
\usepackage{fancyhdr}
\fancyhf{} 
\fancyfoot[L]{\sffamily{\thepage}}

\makeatletter
\fancyhead[L]{\raisebox{1ex}[0pt][0pt]{\sffamily{\@title \ \@date}}} 
\fancyhead[R]{\raisebox{1ex}[0pt][0pt]{\sffamily{\@author}}}
\makeatother

\begin{document}

% sezione (data)
\section{Lezione del 02-04-25}

% stili pagina
\thispagestyle{empty}
\pagestyle{fancy}

% testo
\subsection{Configurazioni organizzative}
Iniziamo a vedere gli organigrammi delle configurazioni organizzative più popolari:
\begin{itemize}
	\item \textbf{Struttura semplice:} l'\textit{alta direzione} controlla più unità fra di loro separate, ma allo stesso livello.
		Questa è la struttura tipica di imprese giovani o di piccole dimensioni.
		I meccanismi di coordinamento principali sono l'adattamento reciproco, la supervisione diretta e la standardizzazione delle capacità.
		\begin{itemize}
			\item \textbf{Punti di forza:} flessibilità;
			\item \textbf{Punti di debolezza:} problemi di conflitti o congestione del vertice.
		\end{itemize}

	\item \textbf{Struttura funzionale:} una struttura più articolata, dove l'\textit{alta direzione} dirige una serie di \textit{unità funzionali}, che non sono più quello che probabilmente nella struttura semplice erano individui, ma unità di più persone atte ad un unico scopo.
		E' quindi necessario un criterio di divisione del lavoro basato sul tipo di processo o tecnica, e sulle conoscenze.
		Questa è la struttura tipica di società di dimensioni piccole-medie.
		Rimangono dalla struttura semplice, poi, la supervisione diretta e la standardizzazione dei processi di lavoro.
		\begin{itemize}
			\item \textbf{Punti di forza:} permette di ottimizzare l'impiego delle risorse umane e tecnologiche concentrando risorse simili e favorendo la specializzazione, ottenendo così \textit{economie di scala}; 
			\item \textbf{Punti di debolezza:} potrebbe non esserci collaborazione fra i diversi "blocchi" funzionali, cosa che impatta la produttività; inoltre è meno flessibile.
		\end{itemize}
		
	\item \textbf{Struttura divisionale:} una struttura ancora più articolata, dove le unità organizzative sono costituite su un criterio basato sugli \textit{output}.
		Le unità sono quindi orientate a diversi clienti, prodotti o aree geografiche.
		Ogni unità divisionale è a sua volta divisa in maniera funzionale, nelle diverse funzioni che essa ha.
		Si avrà quindi, ad esempio, la divisione orientata al prodotto A, con il \textit{reparto vendite} del prodotto A, il \textit{reparto produzione} del prodotto A, ecc...
		Questa struttura è tipica di società grandi e sviluppate in diversi settori.
		\begin{itemize}
			\item \textbf{Punti di forza:} si recuperà la velocità di risposta al mercato (singole divisioni possono reagire a variazioni riguardo al loro prodotto, mercato, area geografica ecc...);
			\item \textbf{Punti di debolezza:} si rinuncia in parte all'economia di scala, e in genere alla specializzazione, con efficienze ed incoerenze date dalla duplicazione delle risorse fra le divisioni.
		\end{itemize}
\end{itemize}

\subsection{Meccanismi di collegamento}
Possiamo individuare diversi meccanismi di collegamento, ad esempio:
\begin{itemize}
	\item \textbf{Posizioni di collegamento:} posizioni, perlopià informali, che si trovano a "metà" fra più blocchi funzionali; 
	\item \textbf{Team interfunzionali:} gruppi di lavoro istituzionalizzati generalmente interdisciplinari.
		Possiamo distingurli ulteriormente rispetto all'\textbf{orizzonte} della collaborazione e alla sua \textbf{intensità}:
		\begin{table}[h!]
			\center \rowcolors{2}{white}{black!10}
			\begin{tabular} { c | c | c }
				& \bfseries Collaborazione continua & \bfseries Collaborazione discontinua \\ 
				\hline
				\bfseries Orizzonte permanente & // & Comitato \\ 
				\bfseries Orizzonte temporaneao & Task force & Riunione
			\end{tabular}
		\end{table}
	\item \textbf{Manager integratori:} si sovrappongono alla struttura organizzativa esistente, con l'obiettivo di occuparsi di un ambito specifico senza detenere tutte le leve di potere tipiche di un manager di livello superiore.
\end{itemize}

\end{document}
